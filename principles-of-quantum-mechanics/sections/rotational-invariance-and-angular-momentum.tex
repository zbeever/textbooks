\documentclass[../principles-of-quantum-mechanics.tex]{subfiles}

\begin{document}
	\printanswers
	
	\setcounter{section}{11}
	\section{Rotational Invariance and Angular Momentum}
	
	\begin{questions}
	\setcounter{subsection}{0}
	\subsection{Translations in Two Dimensions}
	
	\question Verify that $\hat{a}\cdot\mathbf{P}$ is the generator of infinitesimal translations along $\mathbf{a}$ by considering the relation
	$$\langle x, y|I - \frac{i}{\hbar}\boldsymbol{\delta}a\cdot\mathbf{P}|\psi\rangle = \psi(x - \delta a_x, y - \delta a_y)$$
	
	\begin{solution}
		Noting that $\boldsymbol{\delta}a$ is of order $\varepsilon$, we have
		\begin{align*}
			\langle x, y|I - \frac{i}{\hbar}\boldsymbol{\delta}a\cdot\mathbf{P}|\psi\rangle &= \langle x, y|I - \frac{i}{\hbar}\delta a_x P_x + I - \frac{i}{\hbar}\delta a_y P_y + I - I|\psi\rangle \\
			&= \langle x, y|I - \frac{i}{\hbar}\delta a_xP_x|\psi\rangle + \langle x, y|I - \frac{i}{\hbar}\delta a_yP_y|\psi\rangle - \langle x, y|\psi\rangle \\
			&= \psi(x - \delta a_x, y) + \psi(x, y - \delta a_y) - \psi(x, y) \\
			&\approx \psi(x, y) - \frac{\partial \psi}{\partial x}\Big|_{x, y}\delta a_x + \psi(x ,y) - \frac{\partial \psi}{\partial y}\Big|_{x,y}\delta a_y - \psi(x, y) \\
			&= \psi(x, y) - \frac{\partial \psi}{\partial x}\Big|_{x, y}\delta a_x - \frac{\partial \psi}{\partial y}\Big|_{x,y}\delta a_y \\
			&\approx \psi(x - \delta a_x, y - \delta a_y)
		\end{align*}
		where each approximation becomes an equality as ${\delta a\to 0}$.
	\end{solution}

	\setcounter{subsection}{1}
	\setcounter{question}{0}
	\subsection{Rotations in Two Dimensions}
	
	\question Provide the steps linking Eq. (12.2.8) to Eq. (12.2.9). [Hint: Recall the derivation of Eq. (11.2.8) from Eq. (11.2.6).]
	\begin{solution}
		Working backwards, we see
		\begin{align*}
			\langle x, y|I - \frac{i\varepsilon_z L_z}{\hbar}|\psi\rangle &= \int\!\!\!\!\int\langle x, y|U[R]|x', y'\rangle\langle x', y'|\psi\rangle \,\mathrm{d}x'\mathrm{d}y' \\
			&= \int\!\!\!\!\int \langle x, y|x' - y'\varepsilon_z, x'\varepsilon_z + y'\rangle \psi(x', y')\,\mathrm{d}x'\mathrm{d}y' \\
			&= \int\!\!\!\!\int \langle x, y|x', y'\rangle \psi(x' + y'\varepsilon_z, y' - x'\varepsilon_z)\,\mathrm{d}x'\mathrm{d}y' \\
			&= \int\!\!\!\!\int\delta(x - x')\delta(y - y')\psi(x' + y'\varepsilon_z, y' - x'\varepsilon_z)\,\mathrm{d}x'\mathrm{d}y' \\
			&= \psi(x + y\varepsilon_z, y - x\varepsilon_z)
		\end{align*}
		In the above, we were able to make the changes
		\begin{align*}
			x' &\to x' + y'\varepsilon_z \\
			y' &\to y' - x'\varepsilon_z
		\end{align*}
		because both $\mathrm{d}y'\varepsilon_z$ and $\mathrm{d}x'\varepsilon_z$ vanish to first order.
	\end{solution}

	\question Using these commutation relations (and your keen hindsight) derive $L_z = XP_y - YP_x$. At least show that Eqs. (12.2.16) and (12.2.17) are consistent with $L_z = XP_y - YP_x$.
	
	\begin{solution}
		It is unclear to me how one can fix $L_z$ using only these commutation relations, as it could also depend on $Z$ and $P_z$ while satisfying (12.2.16) and (12.2.17). Still, we can complete the second part of the problem:
		\begin{align*}
			[X, L_z] &= [X, XP_y - YP_x] \\ 
			&= [X, XP_y] - [X, YP_x] \\
			&= [X, X]P_y + X[X, P_y] - [X, Y]P_x - Y[X, P_x] \\
			&= {-i\hbar}Y \\
			[Y, L_z] &= [Y, XP_y - YP_x] \\
			&= [Y, XP_y] - [Y, YP_x] \\
			&= [Y, X]P_y + X[Y, P_y] - [Y, Y]P_x - Y[Y, P_x] \\
			&= i\hbar X \\
			[P_x, L_z] &= [P_x, XP_y - YP_x] \\
			&= [P_x, XP_y] - [P_x, YP_x] \\
			&= [P_x, X]P_y + X[P_x, P_y] - [P_x, Y]P_x - Y[P_x, P_x] \\
			&= {-i\hbar}P_y \\
			[P_y, L_z] &= [P_y, XP_y - YP_x] \\
			&= [P_y, XP_y] - [P_y, YP_x] \\
			&= [P_y, X]P_y + X[P_y, P_y] - [P_y, Y]P_x - Y[P_y, P_x] \\
			&= i\hbar P_x
		\end{align*}
	\end{solution}
	
	\question Derive Eq. (12.2.19) by doing a coordinate transformation on Eq (12.2.10), and also by the direct method mentioned above.
	\begin{solution}
		We must transform both $\partial/\partial_x$ and $\partial/\partial_y$, for which we will need the relations
		\begin{align*}
			\rho &= (x^2 + y^2)^{1/2} \\
			\phi &= \tan^{-1}(\tfrac{y}{x}) \\
			x &= \rho\cos\phi \\
			y &= \rho\sin\phi
		\end{align*}
		Using these and the chain rule applied to $f(\rho(x, y), \phi(x, y))$, we find
		\begin{align*}
			\frac{\partial f}{\partial x} &= \frac{\partial f}{\partial \rho}\frac{\partial \rho}{\partial x} + \frac{\partial f}{\partial \phi}\frac{\partial \phi}{\partial x} \\
			&= \frac{1}{2}\frac{2x}{(x^2 + y^2)^{1/2}}\frac{\partial f}{\partial \rho} + \frac{1}{1 + (\tfrac{y}{x})^2}\Big({-\frac{y}{x^2}}\Big)\frac{\partial f}{\partial \phi} \\
			&= \frac{x}{(x^2 + y^2)^{1/2}}\frac{\partial f}{\partial \rho} - \frac{y}{x^2 + y^2}\frac{\partial f}{\partial \phi} \\
			&= \frac{\rho\cos\phi}{\rho}\frac{\partial f}{\partial \rho} - \frac{\rho\sin\phi}{\rho^2}\frac{\partial f}{\partial \phi} \\
			&= \cos\phi\frac{\partial f}{\partial \rho} - \frac{\sin\phi}{\rho}\frac{\partial f}{\partial \phi} \\
			\frac{\partial f}{\partial y} &= \frac{\partial f}{\partial \rho}\frac{\partial \rho}{\partial y} + \frac{\partial f}{\partial \phi}\frac{\partial \phi}{\partial y} \\
			&= \frac{1}{2}\frac{2y}{(x^2 + y^2)^{1/2}}\frac{\partial f}{\partial \rho} + \frac{1}{1 + (\tfrac{y}{x})^2}\frac{1}{x}\frac{\partial f}{\partial \phi} \\
			&= \frac{y}{(x^2 + y^2)^{1/2}}\frac{\partial f}{\partial \rho} + \frac{x}{x^2 + y^2}\frac{\partial f}{\partial \phi} \\
			&= \frac{\rho\sin\phi}{\rho}\frac{\partial f}{\partial \rho} + \frac{\rho\cos\phi}{\rho^2}\frac{\partial f}{\partial \phi} \\
			&= \sin\phi\frac{\partial f}{\partial \rho} + \frac{\cos\phi}{\rho}\frac{\partial f}{\partial \phi} \\
		\end{align*}
		which implies
		\begin{align*}
			\frac{\partial}{\partial x} &\to \cos\phi\frac{\partial}{\partial \rho} - \frac{\sin\phi}{\rho}\frac{\partial}{\partial \phi} \\
			\frac{\partial}{\partial y} &\to \sin\phi\frac{\partial}{\partial \rho} + \frac{\cos\phi}{\rho}\frac{\partial}{\partial \phi} \\
		\end{align*}
		Substituting this into $L_z = XP_y - YP_x$ reveals
		\begin{align*}
			L_z &= -i\hbar x\frac{\partial}{\partial y} + i\hbar y\frac{\partial}{\partial x} \\
			&= -i\hbar\rho\cos\phi\Big(\sin\phi\frac{\partial}{\partial \rho} + \frac{\cos\phi}{\rho}\frac{\partial}{\partial \phi}\Big) + i\hbar\rho\sin\phi\Big(\cos\phi\frac{\partial}{\partial \rho} - \frac{\sin\phi}{\rho}\frac{\partial}{\partial \phi}\Big) \\
			&= -i\hbar\rho\Big(\cos\phi\sin\phi - \cos\phi\sin\phi\Big)\frac{\partial}{\partial \rho} - i\hbar(\cos^2\phi + \sin^2\phi)\frac{\partial}{\partial\phi} \\
			&= -i\hbar\frac{\partial}{\partial\phi}
		\end{align*}
		Alternatively, if we require that $L_z$ generate infinitesimal translations, i.e.
		$$\langle \rho, \phi|I - \frac{i}{\hbar}\varepsilon_z L_z|\psi\rangle = \psi(\rho, \phi) - \frac{i}{\hbar}\varepsilon_z\langle\rho,\phi|L_z|\psi\rangle = \psi(\rho, \phi - \varepsilon_z) = \psi(\rho, \phi) - \frac{\partial\psi}{\partial\phi}\varepsilon_z$$
		then we immediately have
		$$L_z = {-i\hbar\frac{\partial}{\partial\phi}}$$
	\end{solution}
	
	\question Rederive the equivalent of Eq. (12.2.23) keeping terms of order $\varepsilon_x\varepsilon_z^2$. (You may assume $\varepsilon_y=0$.) Use this information to rewrite Eq. (12.2.24) to order $\varepsilon_x\varepsilon_z^2$. By equating coefficients of this term deduce the constraint
	$$-2L_zP_xL_z + P_xL_z^2 + L_z^2P_x = \hbar^2P_x$$
	This seems to conflict with statement (1) made above, but not really, in view of the identity
	$$-2\Lambda\Omega\Lambda + \Omega\Lambda^2 + \Lambda^2\Omega \equiv [\Lambda, [\Lambda, \Omega]]$$
	Using the identity, verify that the new constraint coming from the $\varepsilon_x\varepsilon_z^2$ term is satisfied given the commutation relations between $P_x$, $P_y$, and $L_z$.
	
	\begin{solution}
		Tracing the steps outlined in the text, we consider the following sequence of operators
		$$U[R(-\varepsilon_z\mathbf{k})]T(-\varepsilon_x\mathbf{i})U[R(\varepsilon_z\mathbf{k})]T(\varepsilon_x\mathbf{i})$$
		but now use a rotation matrix of the form
		$$U[R(\varepsilon_z\mathbf{k})] = \begin{bmatrix}
			\cos\varepsilon_z & -\sin\varepsilon_z \\ 
			\sin\varepsilon_z & \cos\varepsilon_z
		\end{bmatrix} \approx \begin{bmatrix}
		1 - \frac{\varepsilon_z^2}{2} & -\varepsilon_z \\
		\varepsilon_z & 1 - \frac{\varepsilon_z^2}{2}
	\end{bmatrix}.$$
		Applied to a point $(x, y)$, this has the effect of
		\begin{align*}
			\begin{bmatrix}x \\ y\end{bmatrix} &\to \begin{bmatrix}x + \varepsilon_x \\ y\end{bmatrix} \\
			&\to \begin{bmatrix}(x + \varepsilon_x)(1 - \frac{\varepsilon_z^2}{2}) - y\varepsilon_z \\ (x + \varepsilon_x)\varepsilon_z + y(1 - \frac{\varepsilon_z^2}{2})\end{bmatrix} \\
			&\to \begin{bmatrix}
					x(1 - \frac{\varepsilon_z^2}{2}) - y\varepsilon_z - \frac{\varepsilon_x\varepsilon_z^2}{2} \\
					x\varepsilon_z + y(1 - \frac{\varepsilon_z^2}{2}) + \varepsilon_x\varepsilon_z
				\end{bmatrix} \\
			&\to \begin{bmatrix}
				(x(1 - \frac{\varepsilon_z^2}{2}) - y\varepsilon_z - \frac{\varepsilon_x\varepsilon_z^2}{2})(1 - \frac{\varepsilon_z^2}{2}) + (x\varepsilon_z + y(1 - \frac{\varepsilon_z^2}{2}) + \varepsilon_x\varepsilon_z)\varepsilon_z \\
				-(x(1 - \frac{\varepsilon_z^2}{2}) - y\varepsilon_z - \frac{\varepsilon_x\varepsilon_z^2}{2})\varepsilon_z + (x\varepsilon_z + y(1 - \frac{\varepsilon_z^2}{2}) + \varepsilon_x\varepsilon_z)(1 - \frac{\varepsilon_z^2}{2})
			\end{bmatrix} \\
			&= \begin{bmatrix}
				x + \frac{\varepsilon_x\varepsilon_z^2}{2} + \mathcal{O}(\varepsilon_z^4)\\
				y + \varepsilon_x\varepsilon_z + \mathcal{O}(\varepsilon_z^4)
			\end{bmatrix}
		\end{align*}
		Writing this out in terms of operators, we see that we must have
		$$\Big(I + \frac{i\varepsilon_z}{\hbar}L_z - \frac{\varepsilon_z^2}{2\hbar^2}L_z^2\Big)\Big(I + \frac{i\varepsilon_x}{\hbar}P_x\Big)\Big(I - \frac{i\varepsilon_z}{\hbar}L_z - \frac{\varepsilon_z^2}{2\hbar^2}L_z^2\Big)\Big(I - \frac{i\varepsilon_x}{\hbar}P_x\Big) = I - \frac{i\varepsilon_x\varepsilon_z^2}{2\hbar}P_x - \frac{i\varepsilon_x\varepsilon_z}{\hbar}P_y$$
		We can expand the lefthand side---keeping terms up to $\mathcal{O}(\varepsilon_z^2)$---to find
		$$I + \frac{i\varepsilon_x\varepsilon_z^2}{\hbar^3}(L_zP_xL_z - \frac{1}{2}L_z^2P_x - \frac{1}{2}P_xL_z^2) + \frac{\varepsilon_x\varepsilon_z}{\hbar^2}[P_x, L_z] = I - \frac{i\varepsilon_x\varepsilon_z^2}{2\hbar}P_x - \frac{i\varepsilon_x\varepsilon_z}{\hbar}P_y$$
		which can only hold if
		\begin{align*}
			[P_x, L_z] &= -i\hbar P_y \\
			-2L_zP_xL_z + L_z^2P_x + P_xL_z^2 &= \hbar^2P_x
		\end{align*}
		The first of these is true because
		$$[P_x, L_z] = [P_x, XP_y - YP_x] = [P_x, X]P_y = -i\hbar P_y$$
		while the second can be seen to hold because of the relation given in the problem statement,
		$$[L_z, [L_z, P_x]] = [L_z, i\hbar P_y] = i\hbar[XP_y - YP_x, P_y] = -i\hbar[Y, P_y]P_x = \hbar^2P_x$$
	\end{solution}
	
	\setcounter{subsection}{2}
	\setcounter{question}{0}
	\subsection{The Eigenvalue Problem of $L_z$}
	\question Provide the steps linking Eq. (12.3.5) to Eq. (12.3.6).
	\begin{solution}
		Imposing the Hermiticity of $L_z$ gives
		\begin{align*}
			-i\hbar\int_0^\infty\!\!\!\int_0^{2\pi}\psi_1^*\frac{\partial\psi_2}{\partial\phi}\rho\,\mathrm{d}\phi\,\mathrm{d}\rho &= \Big[{-i\hbar}\int_0^\infty\!\!\!\int_0^{2\pi}\psi_2^*\frac{\partial\psi_1}{\partial\phi}\rho\,\mathrm{d}\phi\,\mathrm{d}\rho\Big]^* \\
			&= i\hbar\int_0^{\infty}\!\!\!\int_0^{2\pi}\psi_2\frac{\partial\psi_1^*}{\partial\phi}\rho\,\mathrm{d}\phi\,\mathrm{d}\rho \\
			&= i\hbar\int_0^\infty\big(\psi_1^*\psi_2\big)\Big|_{\phi=0}^{\phi=2\pi}\rho\,\mathrm{d}\rho - i\hbar\int_0^{\infty}\!\!\!\int_0^{2\pi}\psi_1^*\frac{\partial\psi_2}{\partial\phi}\rho\,\mathrm{d}\phi\,\mathrm{d}\rho
		\end{align*}
		Clearly, this equality holds only if
		$$\psi_1^*(\rho, 0)\psi_2(\rho, 0) = \psi_1^*(\rho, 2\pi)\psi_2(\rho, 2\pi)$$
		which, given each $\psi_i$ is arbitrary, implies that
		$$\psi(\rho, 0) = \psi(\rho, 2\pi).$$
	\end{solution}
	
	\question Let us try to deduce the restriction on $l_z$ from another angle. Consider a superposition of two allowed $l_z$ eigenstates:
	$$\psi(\rho, \phi) = A(\rho)e^{i\phi l_z/\hbar} + B(\rho)e^{i\phi l_z'/\hbar}$$
	By demanding that upon a $2\pi$ rotation we get the same physical state (not necessarily the same state vector), show that $l_z - l_z' = m\hbar$, where $m$ is an integer. By arguing on the grounds of symmetry that the allowed values of $l_z$ must be symmetric about zero, show that these values are \textit{either} $\dots, 3\hbar/2, \hbar/2, -\hbar/2, -3\hbar/2, \dots$ or $\dots, 2\hbar, \hbar, 0 -\hbar, -2\hbar, \dots$. It is not possible to restrict $l_z$ any further this way.
	\begin{solution}
		The physical state of the system is described by its probability distribution, 
		$$|\psi(\rho, \phi)|^2 = A^2(\rho) + A(\rho)B^*(\rho)e^{i\phi(l_z - l_z')/\hbar} + A^*(\rho)B(\rho)e^{-i\phi(l_z - l_z')/\hbar} + B(\rho)^2$$
		In order for this to remain undisturbed under a rotation by $2\pi$, we must have
		$$e^{i2\pi(l_z - l_z')/\hbar} = 1$$
		and
		$$e^{-i2\pi(l_z - l_z')/\hbar}=1$$
		or, equivalently,
		$$l_z - l_z' = m\hbar, \quad m\in\mathbb{Z}.$$
		Given that there is no preference to positive or negative $l_z$ values in nature, the eigenvalues of $L_z$ should be spaced evenly about $0$. The only possibilities for $l_z$ that satisfy these two constraints are
		$$l_z = m\hbar$$
		or
		$$l_z = \frac{m\hbar}{2}$$
	\end{solution}
	
	\question A particle is described by a wave function
	$$\psi(\rho, \phi) = Ae^{-\rho^2/2\Delta^2}\cos^2\phi$$
	Show (by expressing $\cos^2\phi$ in terms of $\Phi_m$) that
	\begin{gather*}
		P(l_z = 0) = 2/3 \\
		P(l_z = 2\hbar) = 1/6 \\
		P(l_z = -2\hbar) = 1/6
	\end{gather*}
	(Hint: Argue that the radial part $e^{-\rho^2/2\Delta^2}$ is irrelevant here.)
	\begin{solution}
		First, note that we can use
		$$\Phi_m(\phi) = (2\pi)^{-1/2}e^{im\phi},$$
		to write
		\begin{align*}
			\cos^2\phi &= \Big(\frac{e^{i\phi} + e^{-i\phi}}{2}\Big)^2 \\
			&= \frac{1}{2} + \frac{e^{i2\phi}}{4} + \frac{e^{-i2\phi}}{4} \\
			&= \frac{(2\pi)^{1/2}}{2}\Phi_0(\phi) + \frac{(2\pi)^{1/2}}{4}\Phi_2(\phi) + \frac{(2\pi)^{1/2}}{4}\Phi_{-2}(\phi)
		\end{align*}
		Now, given the fact that the wave function is separable, we may write $\psi(\rho, \phi) = \Omega(\rho)\theta(\phi)$ and focus only on the angular component, $\theta(\phi)$. We can normalize this as
		\begin{align*}
			\int_0^{2\pi}|\theta(\phi)|^2\,\mathrm{d}\phi &= B^2\int_0^{2\pi}\cos^4(\phi)\,\mathrm{d}\phi \\
			&= B^2\int_0^{2\pi}\Big(\frac{1}{2} + \frac{1}{2}\cos(2\phi)\Big)^2\mathrm{d}\phi \\
			&= B^2\int_0^{2\pi}\frac{1}{4} + \frac{1}{2}\cos(2\phi) + \frac{1}{4}\cos^2(2\phi)\,\mathrm{d}\phi \\
			&= B^2\int_0^{2\pi}\frac{1}{4} + \frac{1}{2}\cos(2\phi) + \frac{1}{8} + \frac{1}{8}\cos(4\phi)\,\mathrm{d}\phi \\
			&= B^2\cdot\frac{6\pi}{8}
		\end{align*}
		or $B = (4/3\pi)^{1/2}$. Putting everything together, the angular wave function is
		$$\theta(\phi) = \Big(\frac{2}{3}\Big)^{1/2}\Big(\Phi_0(\phi) + \frac{1}{2}\Phi_2(\phi) + \frac{1}{2}\Phi_{-2}(\phi)\Big)$$
		The probabilities associated with finding the particle in various states of definite angular momentum are
		\begin{align*}
			P(l_z = 0) &= \frac{2}{3}\Big|\int_0^{2\pi}\Phi_0^*(\phi)\Big(\Phi_0(\phi) + \frac{1}{2}\Phi_2(\phi) + \frac{1}{2}\Phi_{-2}(\phi)\Big)\,\mathrm{d}\phi\Big|^2 \\
			&= \frac{2}{3}\cdot 1 \\
			&= \frac{2}{3} \\
			P(l_z = 2\hbar) &= \frac{2}{3}\Big|\int_0^{2\pi}\Phi_2^*(\phi)\Big(\Phi_0(\phi) + \frac{1}{2}\Phi_2(\phi) + \frac{1}{2}\Phi_{-2}(\phi)\Big)\,\mathrm{d}\phi\Big|^2 \\
			&= \frac{2}{3}\cdot\frac{1}{4} \\
			&= \frac{1}{6} \\
			P(l_z = -2\hbar) &= \frac{2}{3}\Big|\int_0^{2\pi}\Phi_{-2}^*(\phi)\Big(\Phi_0(\phi) + \frac{1}{2}\Phi_2(\phi) + \frac{1}{2}\Phi_{-2}(\phi)\Big)\,\mathrm{d}\phi\Big|^2 \\
			&= \frac{2}{3}\cdot\frac{1}{4} \\
			&= \frac{1}{6}
		\end{align*}
	\end{solution}
	
	\question A particle is described by a wave function
	$$\psi(\rho, \phi) = Ae^{-\rho^2/2\Delta^2}\Big(\frac{\rho}{\Delta}\cos\phi + \sin\phi\Big)$$
	Show that
	$$P(l_z = \hbar) = P(l_z = -\hbar) = \tfrac{1}{2}$$
	\begin{solution}
		Though it is not required to solve the problem, we first find the normalization factor $A$,
		\begin{align*}
			&A^2\int_0^\infty\!\!\!\int_0^{2\pi}e^{-\rho^2/\Delta^2}\Big(\frac{\rho}{\Delta}\cos\phi + \sin\phi\Big)^2\rho\,\mathrm{d}\phi\,\mathrm{d}\rho \\ =\,&A^2\int_0^\infty\!\!\!\int_0^{2\pi}e^{-\rho^2/\Delta^2}\Big(\frac{\rho^2}{\Delta^2}\cos^2\phi + \frac{2\rho}{\Delta}\cos\phi\sin\phi + \sin^2\phi\Big)\rho\,\mathrm{d}\phi\,\mathrm{d}\rho \\
			=\,&A^2\int_0^\infty\!\!\!\int_0^{2\pi}e^{-\rho^2/\Delta^2}\Big[\frac{\rho^2}{\Delta^2}\Big(\frac{1}{2} + \frac{1}{2}\cos(2\phi)\Big) + \frac{\rho}{\Delta}\sin(2\phi) + \Big(\frac{1}{2} - \frac{1}{2}\cos(2\phi)\Big)\Big]\rho\,\mathrm{d}\phi\,\mathrm{d}\rho \\
			=\,&\pi A^2\int_0^\infty e^{-\rho^2/\Delta^2}\Big[\frac{\rho^2}{\Delta^2} + 1\Big]\rho\,\mathrm{d}\rho \\
			&= \pi A^2\Big(\frac{1}{\Delta^2}\int_0^\infty\rho^3e^{-\rho^2/\Delta^2}\mathrm{d}\rho + \int_0^\infty \rho e^{-\rho^2/\Delta^2}\mathrm{d}\rho\Big) \\
			&= \pi A^2 \Big(\frac{1}{\Delta^2}\frac{\Delta^4}{2} + \frac{\Delta^2}{2}\Big) \\
			&= \pi A^2\Delta^2
		\end{align*}
		i.e. $A = (\pi\Delta^2)^{-1/2}$. Now we rewrite $\psi(\rho, \phi)$ in terms of $\Phi_m(\phi)$ as
		\begin{align*}
			\psi(\rho, \phi) &= \frac{e^{-\rho^2/2\Delta^2}}{(\pi\Delta^2)^{1/2}}\Big(\frac{\rho}{2\Delta}(e^{i\phi} + e^{-i\phi}) - \frac{i}{2}(e^{i\phi} - e^{-i\phi})\Big) \\
			&= \frac{e^{-\rho^2/2\Delta^2}}{(\pi\Delta^2)^{1/2}}\frac{(2\pi)^{1/2}}{2}\Big(\frac{\rho}{\Delta}\Phi_1(\phi) + \frac{\rho}{\Delta}\Phi_{-1}(\phi) - i\Phi_1(\phi) + i\Phi_{-1}(\phi)\Big) \\
			&= \frac{e^{-\rho^2/2\Delta^2}}{(2\Delta^2)^{1/2}}\Big(\Phi_1(\phi)\Big[\frac{\rho}{\Delta} - i\Big] + \Phi_{-1}(\phi)\Big[\frac{\rho}{\Delta} + i\Big]\Big)
		\end{align*}
		This is a superposition of two angular momentum eigenstates: one of $l_z = \hbar$ and the other of $l_z = -\hbar$. Since these states are orthogonal to one another, we know
		\begin{align*}
			\int_0^{2\pi}\Phi_1^*(\phi)\psi(\rho, \phi)\mathrm{d}\phi &= \frac{e^{-\rho^2/2\Delta^2}}{(2\Delta^2)^{1/2}}\Big(\frac{\rho}{\Delta} - i\Big) \\
			\int_0^{2\pi}\Phi_2^*(\phi)\psi(\rho,\phi)\mathrm{d}\phi &= \frac{e^{-\rho^2/2\Delta^2}}{(2\Delta^2)^{1/2}}\Big(\frac{\rho}{\Delta} + i\Big)
		\end{align*}
		Since the magnitude of the inner product of the wave function with the $l_z = \hbar$ and $l_z = -\hbar$ angular momentum eigenfunctions is the same no matter which one we choose, they are equally likely to occur. That is, $P(l_z = \hbar) = P(l_z = -\hbar) = \tfrac{1}{2}$.
	\end{solution}
	
	\question Note that the angular momentum seems to generate a repulsive potential in Eq. (12.3.13). Calculate its gradient and identify it as the centrifugal force.
	\begin{solution}
		Isolating $V(\rho)$ gives
		\begin{align*}
			V(\rho) &= \frac{1}{R(\rho)}\Big[ER(\rho) + \frac{\hbar^2}{2\mu}\Big(\frac{\mathrm{d}^2R(\rho)}{\mathrm{d}\rho^2} + \frac{1}{\rho}\frac{\mathrm{d}R(\rho)}{\mathrm{d}\rho} - \frac{m^2}{\rho^2}R(\rho)\Big)\Big] \\
			&= E + \frac{\hbar^2}{2\mu}\frac{1}{R(\rho)}\frac{\mathrm{d}^2R(\rho)}{\mathrm{d}\rho^2} + \frac{\hbar^2}{2\mu}\frac{1}{\rho R(\rho)}\frac{\mathrm{d}R(\rho)}{\mathrm{d}\rho} - \frac{\hbar^2}{2\mu}\frac{m^2}{\rho^2}
		\end{align*}
		The gradient of this is
		\begin{align*}
			\frac{\mathrm{d}V(\rho)}{\mathrm{d}\rho}\mathbf{e}_\rho &= \frac{\hbar^2}{2\mu}\Big({-\frac{1}{R^2}}\frac{\mathrm{d}^2R}{\mathrm{d}\rho^2}\frac{\mathrm{d}R}{\mathrm{d}\rho} + \frac{1}{R}\frac{\mathrm{d}^3R}{\mathrm{d}\rho^3} - \frac{1}{\rho^2R}\frac{\mathrm{d}R}{\mathrm{d}\rho} - \frac{1}{\rho R^2}\frac{\mathrm{d}R}{\mathrm{d}\rho} + \frac{1}{\rho R}\frac{\mathrm{d}^2R}{\mathrm{d}\rho^2} + 2\frac{m^2}{\rho^3}\Big)\mathbf{e}_\rho \\
			&= \frac{\hbar^2}{2\mu}\Big(\frac{\mathrm{d}^3R}{\mathrm{d}\rho^3}\frac{1}{R} + \frac{\mathrm{d}^2R}{\mathrm{d}\rho^2}\Big[\frac{1}{\rho R} - \frac{1}{R^2}\frac{\mathrm{d}R}{\mathrm{d}\rho}\Big] - \frac{\mathrm{d}R}{\mathrm{d}\rho}\Big[\frac{1}{\rho^2 R} + \frac{1}{\rho R^2}\Big] + 2\frac{m^2}{\rho^3}\Big)\mathbf{e}_\rho
		\end{align*}
	\end{solution}
	
	\question Consider a particle of mass $\mu$ constrained to move on a circle of radius $a$. Show that $H = L_z^2/2\mu a^2$. Solve the eigenvalue problem of $H$ and interpret the degeneracy.
	\begin{solution}
		For a (quasi) free particle constrained to move on a circle of radius $a$, the kinetic energy (and thus classical Hamiltonian) is given by
		\begin{align*}
			T &= \frac{1}{2}\mu v^2 \\
			&= \frac{1}{2}\mu a^2\dot{\phi}^2 \\
			&= \frac{1}{2}\mu a^2\Big(\frac{\mathrm{d}}{\mathrm{d}t}\tan^{-1}(\tfrac{y}{x})\Big)^2 \\
			&= \frac{1}{2}\mu a^2\Big[\frac{1}{1 + (\tfrac{y}{x})^2}\Big(\frac{\dot{y}}{x} - \frac{y}{x^2}\dot{x}\Big)\Big]^2 \\
			&= \frac{1}{2}\mu a^2\Big(\frac{x\dot{y} - y\dot{x}}{x^2 + y^2}\Big)^2 \\
			&= \frac{1}{2}\frac{a^2}{\mu}\Big(\frac{x\mu\dot{y} - y\mu\dot{x}}{a^2}\Big)^2 \\
			&= \frac{(xp_y - yp_x)^2}{2\mu a^2}
		\end{align*}
		Identifying the classical angular momentum operator with its quantum variant gives
		$$H = \frac{L_z^2}{2\mu a^2}$$
		We know that rotationally invariant Hamiltonians permit solutions of the form
		$$\psi_m(\rho, \phi) = R_{\textit{Em}}(\rho)\Phi_m(\phi)$$
		From the problem statement, $R_{\textit{Em}}(\rho)\propto\delta(\rho - a)$. Using the angular coordinate form of $L_z$ produces the time-independent equation
		$$-\frac{\hbar^2}{2\mu a^2}\frac{\partial^2}{\partial\phi^2}\psi_m(\rho, \phi) = \frac{m^2\hbar^2}{2\mu a^2}\psi_m(\rho, \phi) = E_m\psi(\rho, \phi)$$
		i.e. 
		$$E_m = \frac{m^2\hbar^2}{2\mu a^2}$$
		Clearly, states of $m = k$ have the same energy as states of $m = -k$. This makes sense, as the sign of $m$ corresponds only to the direction of rotation.
	\end{solution}
	
	\question \textit{(The Isotropic Oscillator).} Consider the Hamiltonian
	$$H = \frac{P_x^2 + P_y^2}{2\mu} + \frac{1}{2}\mu\omega^2(X^2 + Y^2)$$
	(1) Convince yourself $[H, L_z] = 0$ and reduce the eigenvalue problem of $H$ to the radial differential equation for $R_{\textit{Em}}(\rho)$.
	
	(2) Examine the equation as $\rho \to 0$ and show that
	$$R_{\textit{Em}}(\rho)\xrightarrow[\rho \to 0]{} \rho^{|m|}$$
	
	(3) Show likewise that up to powers of $\rho$
	$$R_{\textit{Em}}(\rho)\xrightarrow[\rho\to\infty]{} e^{-\mu\omega\rho^2/2\hbar}$$
	So assume that $R_{\textit{Em}}(\rho) = \rho^{|m|}e^{-\mu\omega\rho^2/2\hbar}U_{\textit{Em}}(\rho)$.
	
	(4) Switch to dimensionless variables $\varepsilon = E/\hbar\omega$, $y = (\mu\omega/\hbar)^{1/2}\rho$.
	
	(5) Convert the equation for $R$ into an equation for $U$. (I suggest proceeding in two stages: $R = y^{|m|}f$, $f = e^{-y^2/2}U$.) You should end up with
	$$U'' + \Big[\Big(\frac{2|m| + 1}{y}\Big) - 2y\Big]U' + (2\varepsilon - 2|m| - 2)U = 0$$
	
	(6) Argue that a power series for $U$ of the form
	$$U(y) = \sum_{r=0}^{\infty}C_ry^r$$
	will lead to a \textit{two-term} recursion relation.
	
	(7) Find the relation between $C_{r+2}$ and $C_r$. Argue that the series must terminate at some finite $r$ if the $y\to\infty$ behavior of the solution is to be acceptable. Show that $\varepsilon = r + |m| + 1$ leads to termination after $r$ terms. Now argue that $r$ is necessarily even---i.e., $r = 2k$. (Show that if $r$ is odd, the behavior of $R$ as $\rho \to 0$ is not $\rho^{|m|}$.) So finally you must end up with
	$$E = (2k + |m| + 1)\hbar\omega, \qquad k = 0, 1, 2, \dots$$
	Define $n = 2k + |m|$, so that
	$$E_n = (n + 1)\hbar\omega$$
	
	(8) For a given $n$, what are the allowed values of $|m|$? Given this information show that for a given $n$, the degeneracy is $n + 1$. Compare this to what you found in Cartesian coordinates (Exercise 10.2.2).
	
	(9) Write down all the normalized eigenfunctions corresponding to $n = 0, 1$.
	
	(10) Argue that the $n = 0$ function \textit{must} equal the corresponding one found in Cartesian coordinates. Show that the two $n = 1$ solutions are linear combinations of their counterparts in Cartesian coordinates. Verify that the parity of the states is $(-1)^n$ as you found in Cartesian coordinates.
	\begin{solution}
		First, we compute the commutators of squared state variables with $L_z$,
		\begin{align*}
			[X^2, L_z] &= X[X, L_z] + [X, L_z]X \\
			&= -i\hbar XY - i\hbar YX \\
			&= -2i\hbar XY \\
			[Y^2, L_z] &= Y[Y, L_z] + [Y, L_z]Y \\
			&= i\hbar YX + i\hbar XY \\
			&= 2i\hbar XY \\
			[P_x^2, L_z] &= P_x[P_x, L_z] + [P_x, L_z]P_x \\
			&= -i\hbar P_xP_y - i\hbar P_yP_x \\
			&= -2i\hbar P_xP_y \\
			[P_y^2, L_z] &= P_y[P_y, L_z] + [P_y, L_z]P_y \\
			&= i\hbar P_yP_x + i\hbar P_xP_y \\
			&= 2i\hbar P_xP_y
		\end{align*}
		From this, we note that $[P_x^2 + P_y^2, L_z] = [X^2 + Y^2, L_z] = 0$, and thus $[H, L_z] = 0$.
		
		The potential in angular coordinates is $V(\rho) = \tfrac{1}{2}\mu\omega^2\rho^2$, and thus the radial equation is
		$$\Big[{-\frac{\hbar^2}{2\mu}}\Big(\frac{\mathrm{d}^2}{\mathrm{d}\rho^2} + \frac{1}{\rho}\frac{\mathrm{d}}{\mathrm{d}\rho} - \frac{m^2}{\rho^2}\Big) + \frac{1}{2}\mu\omega^2\rho^2\Big]R_{\textit{Em}}(\rho) = ER_{\textit{Em}}(\rho)$$
		As $\rho$ goes to $0$, the $\rho^2$ and $E$ terms become negligible, outshined by terms proportional to $\rho^{-1}$ and $\rho^{-2}$. Since we do not know the second derivative of the radial part of the wave function we keep that term as well, giving a limiting equation of
		$$\frac{\mathrm{d}^2R}{\mathrm{d}\rho^2} + \frac{1}{\rho}\frac{\mathrm{d}R}{\mathrm{d}\rho} - \frac{m^2}{\rho^2}R = 0$$
		Using the ansatz $R = \rho^{k}$ gives
		$$k(k - 1)\rho^{k - 2} + k\rho^{k - 2} - m^2\rho^{k - 2} = 0$$
		or $k^2 - m^2 = 0$, i.e. $k = |m|$ (the exponent must be positive to ensure decent behavior at $\rho = 0$).
		If we now examine the behavior as $\rho$ goes to $\infty$, the $\rho^2$ term is much larger than all others (the term proportional only to $R_{Em}(\rho)$ must go to $0$ at infinity), with the possible exception of the term containing $\mathrm{d}^2/\mathrm{d}\rho^2$, so we have
		
		$$-\frac{\hbar^2}{2\mu}\frac{\mathrm{d}^2R}{\mathrm{d}\rho^2} + \frac{1}{2}\mu\omega^2\rho^2 R = 0$$
		Knowing that $f'(x) \propto xf(x)$ when $f(x) \propto e^{\pm \alpha x^2/2}$, we substitute this into the above to find
		\begin{align*}
			-\frac{\hbar^2}{2\mu}\frac{\mathrm{d}}{\mathrm{d}\rho}\Big({\pm\alpha\rho e^{\pm\alpha\rho^2/2}}\Big) + \frac{1}{2}\mu\omega^2\rho^2e^{\pm\alpha \rho^2/2} &= \mp\frac{\alpha\hbar^2}{2\mu}e^{\pm\alpha\rho^2/2} - \frac{\alpha^2\hbar^2}{2\mu}\rho^2e^{\pm\alpha\rho^2/2} + \frac{1}{2}\mu\omega^2\rho^2e^{\pm\alpha\rho^2/2} \\
			&\approx -\frac{\alpha^2\hbar^2}{2\mu}\rho^2e^{\pm\alpha\rho^2/2} + \frac{1}{2}\mu\omega^2\rho^2e^{\pm\alpha\rho^2/2}
		\end{align*}
		where we have used the approximate equality since $\rho^2 e^{\pm\alpha\rho^2/2} \gg e^{\pm\alpha\rho^2/2}$ as $\rho$ goes to infinity. Now, for a well-behaved $R(\rho)$, we must take the negative exponent in our ansatz. Choosing $\alpha = \mu\omega/\hbar$ satisfies the limiting differential equation.
		
		From these limiting answers, we assume
		$$R(\rho) = \rho^{|m|}e^{-\mu\omega\rho^2/2\hbar}U(\rho)$$
		
		To switch to the given dimensionless variables, we make the usual replacements, as well as
		$$\frac{\mathrm{d}}{\mathrm{d}\rho} \to \Big(\frac{\mu\omega}{\hbar}\Big)^{1/2}\frac{\mathrm{d}}{\mathrm{d}y}$$
		Altogether, we find
		$$\Big[{-\frac{\hbar^2}{2\mu}}\Big(\frac{\mu\omega}{\hbar}\frac{\mathrm{d}^2}{\mathrm{d}y^2} + \frac{\mu\omega}{\hbar}\frac{1}{y}\frac{\mathrm{d}}{\mathrm{d}y} - \frac{\mu\omega}{\hbar}\frac{m^2}{y^2}\Big) + \frac{1}{2}\mu\omega^2\frac{\hbar}{\mu\omega}y^2\Big]R_{\textit{Em}}(\rho) = \varepsilon \hbar\omega R_{\textit{Em}}(\rho)$$
		or
		$$\Big({-\frac{\mathrm{d}^2}{\mathrm{d}y^2}} - \frac{1}{y}\frac{\mathrm{d}}{\mathrm{d}y} + \frac{m^2}{y^2} + y^2 - 2\varepsilon\Big)R_{\textit{Em}}(\rho) = 0$$
		Taking Shankar's suggestion to first substitute $R = y^{|m|}f$ gives term-wise results of
		\begin{align*}
			 -\frac{\mathrm{d}^2}{\mathrm{d}y^2}\big(y^{|m|}f\big) &= -\frac{\mathrm{d}}{\mathrm{d}y}\big(|m|y^{|m|-1}f + y^{|m|}f'\big) \\
			 &= -|m|(|m| - 1)y^{|m| - 2}f - |m|y^{|m| - 1}f' - |m|y^{|m| - 1}f' - y^{|m|}f'' \\
			 &= (|m| - m^2)y^{|m|-2}f - 2|m|y^{|m| - 1}f' - y^{|m|}f'' \\
			 -\frac{1}{y}\frac{\mathrm{d}}{\mathrm{d}y}\big(y^{|m|}f\big) &= -|m|y^{|m| - 2}f - y^{|m|-1}f' \\
			 \frac{m^2}{y^2}\big(y^{|m|}f\big) &= m^2y^{|m| - 2}f \\
			 y^2\big(y^{|m|}f\big) &= y^{|m| + 2}f
		\end{align*}
		and a total result of
		$$-y^{|m|}f'' - (2|m| + 1)y^{|m| - 1}f' + (y^{2} - 2\varepsilon)y^{|m|}f = 0$$
		Given that
		\begin{align*}
			f &= Ue^{-y^2/2} \\
			f' &= (-yU + U')e^{-y^2/2} \\
			f'' &= \big((y^2 - 1)U - 2yU' + U''\big)e^{-y^2/2}
		\end{align*}
		we have
		$$y^{|m|}\Big({-U''} + \big(2y - \tfrac{2|m| - 1}{y}\big)U' + (2|m| + 2 - 2\varepsilon)U\Big)e^{-y^2/2} = 0$$
		Dividing by $-y^{|m|}e^{-y^2/2}$ gives the form in the problem statement,
		$$U'' + \Big[\Big(\frac{2|m| + 1}{y}\Big) - 2y\Big]U' + (2\varepsilon - 2|m| - 2)U = 0$$
		Assuming that $U$ can be expanded in a power series,
		$$U = \sum_{r=0}^{\infty}C_ry^{r}$$
		the above differential equation becomes
		\begin{align*}
			&\sum_{r=0}^{\infty}\Big[C_rr(r - 1)y^{r - 2} + C_rr(2|m| + 1)y^{r - 2} - 2C_rry^{r} + C_r(2\varepsilon - 2|m| - 2)y^{r}\Big] \\
			=\,&\sum_{k=0}^{\infty}C_{k+2}\big((k+2)(k+1) + (k+2)(2|m|+1)\big)y^k + \sum_{r=0}^{\infty}C_r(2\varepsilon - 2|m| - 2r - 2)y^r \\
			=\,&\sum_{r=0}^{\infty}\Big[C_{r+2}\Big((r+2)(r+1) + (k+2)(2|m| + 1)\Big) + C_r\Big(2\varepsilon - 2|m| - 2r - 2\Big)\Big]y^r \\
			=\,&0
		\end{align*}
		This will be true only if 
		$$C_{r+2} = C_r\frac{2(|m| + r + 1 - \varepsilon)}{r^2 + 2r|m| + 4r + 4|m| + 4}$$
		which is our two-term recurrence relation. Explicitly, given $C_0$ and $C_1$, we have
		\begin{align*}
			U(y) =\,&C_0\Big(1 + \frac{|m| + 1 - \varepsilon}{2|m| + 2}y^2 + \Big[\frac{|m| + 1 - \varepsilon}{2|m| + 2}\Big]\Big[\frac{|m| + 3 - \varepsilon}{4|m| + 8}\Big]y^4 + \cdots\Big) \\
			&+ C_1\Big(y + \frac{2|m| + 4 - 2\varepsilon}{6|m| + 9}y^3 + \Big[\frac{2|m| + 4 - 2\varepsilon}{6|m| + 9}\Big]\Big[\frac{2|m| + 8 - 2\varepsilon}{10|m| + 25}\Big]y^5 + \cdots\Big)
		\end{align*}
		Now, to ensure that $e^{-y^2/2}$ dies faster than $U(y)$ grows, the parenthesized expressions must contain a finite number of terms. This is because the $C_r$ coefficients die more slowly than $1/r!$---which have the approximate relation $C_{r+2} = C_r/r^2$---and so an untruncated $U(y)$ grows at a faster-than-exponential rate. From inspection, we can guarantee that $U(y)$ contains a finite number of terms by choosing $\varepsilon = |m| + r + 1$.

		Now, to first order, the odd $r$ terms in $U(y)$ must be made zero to avoid $R(\rho)$ behaving like $\rho^{|m|+1}$ as $\rho \to 0$. Keeping only the terms proportional to $C_0$, we rewrite $\varepsilon = |m| + 2k + 1$, where $k\in\mathbb{N}$. Defining $n = |m| + 2k$ and reinstating units gives
		$$E_n = (n + 1)\hbar\omega.$$
		Let us now consider the allowed values of $m$ for a given $n$. Since $n - 2k = |m|$, valid $m$ values will each be separated by $2$ and centered around the origin. That is,
		\begin{align*}
			n = 0 &\implies m = 0 \\
			n = 1 &\implies m = {-1}, 1 \\
			n = 2 &\implies m = {-2}, 0, 2 \\
			n = 3 &\implies m = {-3}, {-1}, {1}, {3} \\
			&\vdots
		\end{align*}
		Thus, $m$ has $n + 1$ degrees of freedom, which is exactly what we found in 10.2.2. The first few eigenfunctions are
		\begin{align*}
			\psi_{E0}(\rho, \phi) &= R_{E0}(\rho)\Phi_0(\phi) \\
			&= \frac{C_0}{(2\pi)^{1/2}}e^{-\mu\omega\rho^2/2\hbar} \\
			\psi_{E(-1)}(\rho, \phi) &= R_{E(-1)}(\rho)\Phi_{-1}(\phi) \\
			&= \frac{C_0}{(2\pi)^{1/2}}\rho e^{-\mu\omega\rho^2/2\hbar}e^{-i\phi} \\
			\psi_{E1}(\rho, \phi) &= R_{E1}(\rho)\Phi_{1}(\phi) \\
			&= \frac{C_0}{(2\pi)^{1/2}}\rho e^{-\mu\omega\rho^2/2\hbar}e^{i\phi}
		\end{align*}
		Given that we have used the same Hamiltonian as that of 10.2.2, all solutions we find here should be linear combinations of those we found before. In particular, the solution for $n = 0$ should be exactly the same, being perfectly isotropic. We find that this is indeed the case, with the $n = 1$ eigenfunctions differing only by $e^{\pm i \phi}$ weighted by $\cos\phi$ or $\sin\phi$.
	\end{solution}
	
	\question Consider a particle of mass $\mu$ and charge $q$ in a vector potential
	$$\mathbf{A} = \frac{B}{2}(-y\mathbf{i} + x\mathbf{j})$$
	
	(1) Show that the magnetic field is $\mathbf{B} = B\mathbf{k}$.
	
	(2) Show that a classical particle in this potential will move in circles at an angular frequency $\omega_0 = qB/\mu c$.
	
	(3) Consider the Hamiltonian for the corresponding quantum problem:
	$$H = \frac{[P_x + qYB/2c]^2}{2\mu} + \frac{[P_y - qXB/2c]^2}{2\mu}$$
	Show that $Q = (cP_x + qYB/2)/qB$ and $P = (P_y - qXB/2c)$ are canonical. Write $H$ in terms of $P$ and $Q$ and show that allowed levels are $E = (n + 1/2)\hbar\omega_0$.
	
	(4) Expand $H$ out in terms of the original variables and show
	$$H = H\Big(\frac{\omega_0}{2}, \mu\Big) - \frac{\omega_0}{2}L_z$$
	where $H(\omega_0/2, \mu)$ is the Hamiltonian for an isotropic two-dimensional harmonic oscillator of mass $\mu$ and frequency $\omega_0/2$. Argue that the same basis that diagonalized $H(\omega_0/2, \mu)$ will diagonalize $H$. By thinking in terms of this basis, show that the allowed levels for $H$ are $E = (k + \tfrac{1}{2}|m| - \tfrac{1}{2}m + \tfrac{1}{2})\hbar\omega_0$, where $k$ is any integer and $m$ is the angular momentum. Convince yourself that you get the same levels from this formula as from the earlier one [$E = (n + 1/2)\hbar\omega_0$]. We shall return to this problem in Chapter 21.

	\begin{solution}
		From basic electromagnetism, the magnetic field is given by
		\begin{align*}	
			\mathbf{B} &= \nabla \times \mathbf{A} \\
			&= (\partial_yA_z - \partial_zA_y)\mathbf{i} - (\partial_xA_z - \partial_zA_x)\mathbf{j} + (\partial_xA_y - \partial_yA_x)\mathbf{k} \\
			&= \Big(\frac{B}{2} + \frac{B}{2}\Big)\mathbf{k} \\
			&= B\mathbf{k}
		\end{align*}
		The force on a classical particle moving in this field is described, in CGS units, by 
		$$\mathbf{F} = \mu\dot{\mathbf{v}} = \frac{q}{c}\mathbf{v}\times\mathbf{B}$$
		which, componentwise, can be written as
		$$\mu\begin{bmatrix}\dot{v}_x \\ \dot{v}_y \\ \dot{v}_z\end{bmatrix} = \frac{q}{c}\begin{bmatrix}v_yB \\ -v_xB \\ 0\end{bmatrix}$$
		Taking the time derivative of both sides of this and using the definitions of $\dot{v}_i$ given above, we find
		\begin{gather*}
			\mu\ddot{v}_x = \frac{q}{c}\dot{v}_yB = -\frac{q}{c}\Big(\frac{q}{\mu c}v_xB\Big)B = -\frac{q^2B^2}{\mu c^2}v_x \\
			\mu\ddot{v}_y = -\frac{q}{c}\dot{v}_xB = -\frac{q}{c}\Big(\frac{q}{\mu c}v_yB\Big)B = -\frac{q^2B^2}{\mu c^2}v_y
		\end{gather*}
		where we have left out the $z$ component on account of its clear solution. Recognizing both the $x$ and $y$ components of the velocity to be undergoing simple harmonic motion with $\omega_0 = qB/\mu c$, we may write
		$$\mathbf{v} = v_0\cos(\omega_0 t + \phi_0)\mathbf{i} - v_0\sin(\omega_0 t + \phi_0)\mathbf{j} + v_1\mathbf{k}$$
		which immediately implies the position of the particle is given by
		$$\mathbf{x} = \frac{v_0}{\omega_0}\sin(\omega_0 t + \phi_0)\mathbf{i} + \frac{v_0}{\omega_0}\cos(\omega_0 t + \phi_0) + v_1t\mathbf{k}$$
		i.e. the particle gyrates around the magnetic field lines with frequency $\omega_0$.

		To confirm the canonical nature of the given $Q$ and $P$ coordinates, we need to verify that $[Q, P] = i\hbar$,
		\begin{align*}
			[Q, P] &= \frac{1}{qB}[cP_x + \frac{q}{2}YB, P_y - \frac{q}{2c}XB] \\
			&= \frac{1}{qB}\Big(c[P_x, P_y] - \frac{q}{2}[P_x, X]B + \frac{q}{2}[Y, P_y]B - \frac{q^2}{4c}[Y, X]B^2\Big) \\
			&= \frac{1}{qB}\Big(\frac{qB}{2}(i\hbar) + \frac{qB}{2}(i\hbar)\Big) \\
			&= i\hbar
		\end{align*}
		We can rewrite $H$ in terms of $Q$ and $P$ as
		$$H = \Big(\frac{qB}{c}\Big)^2\frac{Q^2}{2\mu} + \frac{P^2}{2\mu} = \frac{P^2}{2\mu} + \frac{1}{2}\mu\omega_0^2Q^2$$
		which is exactly the Hamiltonian of the harmonic oscillator, implying the allowable energy levels are $E = (n + \frac{1}{2})\hbar\omega_0$.

		Returning to the original Hamiltonian, we may expand it to find
		\begin{align*}
			H &= \frac{[P_x + qYB/2c]^2}{2\mu} + \frac{[P_y - qXB/2c]^2}{2\mu} \\
			&= \frac{1}{2\mu}\Big[P_x^2 + \frac{qB}{c}P_xY + \Big(\frac{qB}{2c}\Big)^2Y^2\Big] + \frac{1}{2\mu}\Big[P_y^2 - \frac{qB}{c}P_yX + \Big(\frac{qB}{2c}\Big)^2X^2\Big] \\
			&= \frac{P_x^2}{2\mu} + \frac{P_y^2}{2\mu} + \frac{1}{2}\mu\Big(\frac{qB}{2\mu c}\Big)^2X^2 + \frac{1}{2}\mu\Big(\frac{qB}{2\mu c}\Big)^2Y^2 - \frac{qB}{2\mu c}\Big(XP_y - YP_x\Big) \\
			&= \frac{P_x^2}{2\mu} + \frac{P_y^2}{2\mu} + \frac{1}{2}\mu\Big(\frac{\omega_0}{2}\Big)^2X^2 + \frac{1}{2}\mu\Big(\frac{\omega_0}{2}\Big)^2Y^2 - \frac{\omega_0}{2}L_z \\
			&= H\Big(\frac{\omega_0}{2}, \mu\Big) - \frac{\omega_0}{2}L_z
		\end{align*}
		where $H\big(\frac{\omega_0}{2}, \mu\big)$ is the isotropic two-dimensional oscillator from with mass $\mu$ and frequency $\omega_0/2$. These components of the overall Hamiltonian share a common, diagonal basis if they commute. Since
		\begin{align*}
			[X^2 + Y^2, L_z] =\,&[X^2 + Y^2, XP_y - YP_x] \\
			=\,&[X^2, XP_y] - [X^2, YP_x] + [Y^2, XP_y] - [Y^2, YP_x] \\
			=\,&-{\Big(X[X, X]P_y + [X, Y]XP_x + YX[X, P_x] + Y[X, P_x]X\Big)} \\
			&+\Big(Y[Y, X]P_y + [Y, X]YP_y + XY[Y, P_y] + X[Y, P_y]Y\Big) \\
			=\,&-i\hbar YX - i\hbar YX + i\hbar XY + i\hbar XY \\
			=\,&0
		\end{align*}
		and
		\begin{align*}
			[P_x^2 + P_y^2, L_z] =\,&[P_x^2 + P_y^2, XP_y - YP_x] \\
			=\,&[P_x^2, XP_y] - [P_x^2, YP_x] + [P_y^2, XP_y] - [P_y^2, YP_x] \\
			=\,&\Big(P_x[P_x, X]P_y + [P_x, X]P_xP_y + [X, P_x]P_xP_y + X[P_x, P_y]P_x\Big) \\
			&{-\Big(P_y[P_y, Y]P_x + [P_y, Y]P_yP_x + [Y, P_y]P_yP_x + Y[P_y, P_x]P_y\Big)} \\
			=\,&-i\hbar P_xP_y - i\hbar P_xP_y + i\hbar P_xP_y + i\hbar P_yP_x + i\hbar P_yP_x - i\hbar P_yP_x \\
			=\,&0
		\end{align*}
		the two parts of the Hamiltonian \textit{do} commute, and hence can be diagonalize by a common basis. Given that the eigenfunctions of $H\big(\frac{\omega_0}{2}, \mu\big)$ can be written as $R(\rho)\Phi(\phi)$ (where $\Phi(\phi)$ are the eigenfunctions of $L_z$) and $L_z$ commutes with any function of $\rho$, it is clear that the two parts of the Hamiltonian can, more specifically, be diagonalized by the same basis as that found in the previous problems on the isotropic two-dimensional oscillator. If we apply this Hamiltonian to a state built from that basis, we find
		\begin{align*} 
			H|\psi_{Em}\rangle &= H\Big(\frac{\omega_0}{2}, \mu\Big)|\psi_{Em}\rangle - \frac{\omega_0}{2}L_z|\psi_{Em}\rangle \\
			&= (2k + |m| + 1)\frac{\hbar\omega_0}{2}|\psi_{Em}\rangle - \frac{\omega_0}{2}m\hbar|\psi_{Em}\rangle \\
			&= \Big(k + \frac{1}{2}|m| - \frac{1}{2}m + \frac{1}{2}\Big)\hbar\omega_0|\psi_{Em}\rangle
		\end{align*}
		which implies the energy is quantized as
		$$E = \Big(k + \frac{1}{2}|m| - \frac{1}{2}m + \frac{1}{2}\Big)\hbar\omega_0$$
		Since $k \in \mathbb{N}$ and $|m| - m \geq 0$, this expression gives the exact same energy levels as $E = (n + 1/2)\hbar\omega_0$.
	\end{solution}

	\setcounter{subsection}{3}
	\setcounter{question}{0}
	\subsection{Angular Momentum in Three Dimensions}
	\question (1) Verify that Eqs. (12.4.9) and Eq. (12.4.8) are equivalent, given the definition of $\varepsilon_{ijk}$.
	
	(2) Let $U_1$, $U_2$, and $U_3$ be three energy eigenfunctions of a single particle in some potential. Construct the wave function $\psi_A(x_1, x_2, x_3)$ fo three fermions in this potential, one of which is in $U_1$, one in $U_2$, and one in $U_3$, using the $\varepsilon_{ijk}$ tensor.
	\begin{solution}
		Since 
		\begin{align*} 
			\mathbf{c} = c_x\mathbf{i} + c_y\mathbf{j} + c_z\mathbf{k} &= \mathbf{a}\times\mathbf{b} \\
			&= \begin{vmatrix}\mathbf{i} & \mathbf{j} & \mathbf{k} \\ a_x & a_y & a_z \\ b_x & b_y & b_z\end{vmatrix} \\
			&= (a_yb_z - a_zb_y)\mathbf{i} + (a_zb_x - a_xb_z)\mathbf{j} + (a_xb_y - a_yb_x)\mathbf{k}
		\end{align*}
		we may write
		\begin{align*}
			c_1 &= a_2b_3 - a_3b_2 \\
			c_2 &= a_3b_1 - a_1b_3 \\
			c_3 &= a_1b_2 - a_2b_1
		\end{align*}
		If we want to find each $c_i$ with (12.4.9), we evaluate
		\begin{align*}
			c_1 &= \sum_{j=1}^3\sum_{k=1}^3\varepsilon_{1jk}a_jb_k \\
			&= \varepsilon_{123}a_2b_3 + \varepsilon_{132}a_3b_2 \\
			&= a_2b_3 - a_3b_2 \\
			c_2 &= \sum_{j=1}^3\sum_{k=1}^3\varepsilon_{2jk}a_jb_k \\
			&= \varepsilon_{213}a_1b_3 + \varepsilon_{231}a_3b_1 \\
			&= {-a_1b_3} + a_3b_1 \\
			c_3 &= \sum_{j=1}^3\sum_{k=1}^3\varepsilon_{3jk}a_jb_k \\
			&= \varepsilon_{312}a_1b_2 + \varepsilon_{321}a_2b_1 \\
			&= a_1b_2 - a_2b_1 \\
		\end{align*}
		which is equivalent to the first set of $c_i$ values.
		
		For the second part of this problem, we are looking to write a wavefunction that is completely antisymmetric with respect to $U_1$, $U_2$, and $U_3$. This can be done with the antisymmetric tensor via
		$$\psi_A(x_1, x_2, x_3) = \frac{1}{6}\varepsilon_{ijk}U_i(x_1)U_j(x_2)U_k(x_3)$$
		where we have invoked the Einstein summation convention. The $1/6$ is present for normalization purposes.
	\end{solution}

	\question (1) Verify Eq. (12.4.2) by first constructing the $3\times3$ matrices corresponding to $R(\varepsilon_x\mathbf{i})$ and $R(\varepsilon_y\mathbf{j})$, to order $\varepsilon$.
	
	(2) Provide the steps connecting Eqs. (12.4.3) and (12.4.4a).

	(3) Verify that $L_x$ and $L_y$ defined in Eq. (12.4.1) satisfy Eq. (12.4.4a). The proof for other commutators follows by cyclic permutation.
	
	\begin{solution}
		If we define positive rotations as anticlockwise, the requested rotation matrices are
		\begin{align*}
			R(\varepsilon_x \mathbf{i}) &= \begin{bmatrix}1 & 0 & 0 \\ 0 & \cos\varepsilon_x & -\sin\varepsilon_x \\ 0 & \sin\varepsilon_x & \cos\varepsilon_x\end{bmatrix} \\
			R(\varepsilon_y \mathbf{j}) &= \begin{bmatrix}\cos\varepsilon_y & 0 & \sin\varepsilon_y \\ 0 & 1 & 0 \\ -\sin\varepsilon_y & 0 & \cos\varepsilon_y\end{bmatrix}
		\end{align*}
		As we take both $\varepsilon_x$ and $\varepsilon_y$ to $0$, these become
		\begin{align*}
			\lim_{\varepsilon_x\to 0}R(\varepsilon_x\mathbf{i}) &= \begin{bmatrix}1 & 0 & 0 \\ 0 & 1 & -\varepsilon_x \\ 0 & \varepsilon_x & 1\end{bmatrix} \\
			\lim_{\varepsilon_y\to0}R(\varepsilon_y\mathbf{j}) &= \begin{bmatrix}1 & 0 & \varepsilon_y \\ 0 & 1 & 0 \\ -\varepsilon_y & 0 & 1\end{bmatrix}
		\end{align*}
		We can now write out (12.4.2) in matrix form as
		\begin{align*}
			R(-\varepsilon_y\mathbf{j})R(-\varepsilon_x\mathbf{i})R(\varepsilon_y\mathbf{j})R(\varepsilon_x\mathbf{i}) &= \begin{bmatrix}1 & 0 & -\varepsilon_y \\ 0 & 1 & 0 \\ \varepsilon_y & 0 & 1\end{bmatrix}\begin{bmatrix}1 & 0 & 0 \\ 0 & 1 & \varepsilon_x \\ 0 & -\varepsilon_x & 1\end{bmatrix}\begin{bmatrix}1 & 0 & \varepsilon_y \\ 0 & 1 & 0 \\ -\varepsilon_y & 0 & 1\end{bmatrix}\begin{bmatrix}1 & 0 & 0 \\ 0 & 1 & -\varepsilon_x \\ 0 & \varepsilon_x & 1\end{bmatrix} \\
			&= \begin{bmatrix}1 & \varepsilon_x\varepsilon_y & -\varepsilon_y \\ 0 & 1 & \varepsilon_x \\ \varepsilon_y & -\varepsilon_x & 1\end{bmatrix}\begin{bmatrix}1 & \varepsilon_x\varepsilon_y & \varepsilon_y \\ 0 & 1 & -\varepsilon_x \\ -\varepsilon_y & \varepsilon_x & 1\end{bmatrix} \\
			&= \begin{bmatrix}
				1 + \varepsilon_y^2 & \varepsilon_x\varepsilon_y & -\varepsilon_x^2\varepsilon_y \\ -\varepsilon_x\varepsilon_y & 1 + \varepsilon_x^2 & 0 \\ 0 & \varepsilon_x\varepsilon_y^2 & \varepsilon_y^2 + \varepsilon_x^2 + 1
			\end{bmatrix} \\
			&\approx \begin{bmatrix}
				1 & \varepsilon_x\varepsilon_y & 0 \\ -\varepsilon_x\varepsilon_y & 1 & 0 \\ 0 & 0 & 1
			\end{bmatrix} \\
			&= R(-\varepsilon_x\varepsilon_y\mathbf{k})
		\end{align*}
		where the last line follows from the fact that 
		$$\lim_{\varepsilon_z\to0}R(\varepsilon_z\mathbf{k}) = \lim_{\varepsilon_z\to0}\begin{bmatrix}\cos\varepsilon_z & -\sin\varepsilon_z & 0 \\ \sin\varepsilon_z & \cos\varepsilon_z & 0 \\ 0 & 0 & 1 \end{bmatrix} = \begin{bmatrix}1 & -\varepsilon_z & 0 \\ \varepsilon_z & 1 & 0 \\ 0 & 0 & 1\end{bmatrix}$$
		Starting from (12.4.3), we have
		\begin{align*}
			U[R(-\varepsilon_y\mathbf{j})]\cdots U[R(\varepsilon_x\mathbf{i})] =\,&e^{i\varepsilon_yL_y/\hbar}e^{i\varepsilon_xL_x/\hbar}e^{-i\varepsilon_yL_y/\hbar}e^{-i\varepsilon_xL_x/\hbar} \\
			\approx\,&(1 + \tfrac{i\varepsilon_y}{\hbar}L_y)(1 + \tfrac{i\varepsilon_x}{\hbar}L_x)(1 - \tfrac{i\varepsilon_y}{\hbar}L_y)(1 - \tfrac{i\varepsilon_x}{\hbar}L_x) \\
			=\,&(1 + \tfrac{i\varepsilon_y}{\hbar}L_y + \tfrac{i\varepsilon_x}{\hbar}L_x - \tfrac{\varepsilon_x\varepsilon_y}{\hbar^2}L_yL_x)(1 - \tfrac{i\varepsilon_y}{\hbar}L_y - \tfrac{i\varepsilon_x}{\hbar}L_x - \tfrac{\varepsilon_x\varepsilon_y}{\hbar^2}L_yL_x) \\
			=\,&1 - \tfrac{i\varepsilon_y}{\hbar}L_y - \tfrac{i\varepsilon_x}{\hbar}L_x - \tfrac{\varepsilon_x\varepsilon_y}{\hbar^2}L_yL_x \\
			&+ \tfrac{i\varepsilon_y}{\hbar}L_y + \tfrac{\varepsilon_y^2}{\hbar^2}L_y^2 + \tfrac{\varepsilon_x\varepsilon_y}{\hbar^2}L_yL_x - \tfrac{i\varepsilon_x\varepsilon_y^2}{\hbar^3}L_y^2L_x \\
			&+ \tfrac{i\varepsilon_x}{\hbar}L_x + \tfrac{\varepsilon_x\varepsilon_y}{\hbar^2}L_xL_y + \tfrac{\varepsilon_x^2}{\hbar^2}L_x^2 - \tfrac{i\varepsilon_x^2\varepsilon_y}{\hbar^3}L_xL_yL_x \\
			&- \tfrac{\varepsilon_x\varepsilon_y}{\hbar^2}L_yL_x + \tfrac{i\varepsilon_x\varepsilon_y^2}{\hbar^3}L_yL_xL_y + \tfrac{i\varepsilon_x^2\varepsilon_y}{\hbar^3}L_yL_x^2 + \tfrac{\varepsilon_x^2\varepsilon_y^2}{\hbar^4}L_yL_xL_yL_x
		\end{align*}
	\end{solution}
	\begin{solution}
		After canceling terms and neglecting those of order $\varepsilon_x^2$ or $\varepsilon_y^2$, we find
		\begin{align*}
			U[R(-\varepsilon_y\mathbf{j})]U[R(-\varepsilon_x\mathbf{i})]U[R(\varepsilon_y\mathbf{j})] U[R(\varepsilon_x\mathbf{i})] &\approx 1 + \tfrac{\varepsilon_x\varepsilon_y}{\hbar^2}L_xL_y - \tfrac{\varepsilon_x\varepsilon_y}{\hbar^2}L_yL_x \\
			&= 1 + \tfrac{\varepsilon_x\varepsilon_y}{\hbar^2}[L_x, L_y] \\
			&= U[R(-\varepsilon_x\varepsilon_y\mathbf{k})]
		\end{align*}
		which only works if
		$$[L_x, L_y] = i\hbar L_z$$
		To verify that this is actually the case, we compute
		\begin{align*}
			[L_x, L_y] =\,&[YP_z - ZP_y, ZP_x - XP_z] \\
			=\,&[YP_z, ZP_x] - [YP_z, XP_z] - [ZP_y, ZP_x] + [ZP_y, XP_z] \\
			=\,&[YP_z, ZP_x] + [ZP_y, XP_z] \\ 
			=\,&Y[P_z, Z]P_x + [Y, Z]P_zP_x + ZY[P_z, P_x] + Z[Y, P_x]P_z \\
			&+ Z[P_y, X]P_z + [Z, X]P_yP_z + XZ[P_y, P_z] + X[Z, P_z]P_y \\
			=\,&{-i\hbar YP_x + i\hbar XP_y} \\
			=\,&i\hbar(XP_y - YP_x) \\
			=\,&i\hbar L_z
		\end{align*}
	\end{solution}

	\question We would like to show that $\hat{\theta}\cdot\mathbf{L}$ generates rotations about the axis parallel to $\hat{\theta}$. Let $\delta\boldsymbol{\theta}$ be an infinitesimal rotation parallel to $\boldsymbol{\theta}$.
	
	(1) Show that when a vector $\mathbf{r}$ is rotated by an angle $\delta\boldsymbol{\theta}$, it changes to $\mathbf{r}+ \delta\boldsymbol{\theta}\times\mathbf{r}$. (It might help to start with $\mathbf{r}\perp\delta\boldsymbol{\theta}$ and then generalize.)
	
	(2) We therefore demand that (to first order, as usual)
	$$\psi(\mathbf{r})\xrightarrow[{U[R(\delta\boldsymbol{\theta})]}]{} \psi(\mathbf{r} - \delta\boldsymbol{\theta}\times\mathbf{r}) = \psi(\mathbf{r}) - (\delta\boldsymbol{\theta}\times\mathbf{r})\cdot\nabla\psi$$
	
	Comparing to $U[R(\delta\boldsymbol{\theta})] = I - (i\delta\theta/\hbar)L_{\hat{\theta}}$, show that $L_{\hat{\theta}}=\hat{\theta}\cdot\mathbf{L}$.
	
	\begin{solution}
		We can encode a rotation of $\theta$ about an arbitrary axis via
		$$R(\theta\boldsymbol{\theta}) = R(-\phi_z\mathbf{k})R(-\phi_y\mathbf{j})R(\theta\mathbf{i})R(\phi_y\mathbf{j})R(\phi_z\mathbf{k})$$
		where $\phi_y$ and $\phi_z$ are suitably chosen to align the rotation axis first within the $x$-$z$ plane, and then along the $x$ axis. That is, if the rotation axis is given by $\mathbf{u} = \begin{bmatrix}u_x & u_y & u_z\end{bmatrix}^T$ (where $\|\mathbf{u}\| = 1$), then $R(\phi_z\mathbf{k})$ should zero out $y$ component and $R(\phi_y\mathbf{j})$ should zero out the (resulting) $z$ component. Since
		\begin{align*}
			R(\phi_z\mathbf{k})\mathbf{u} &= \begin{bmatrix}\cos\phi_z & -\sin\phi_z & 0 \\ \sin\phi_z & \cos\phi_z & 0 \\ 0 & 0 & 1\end{bmatrix}\begin{bmatrix}u_x \\ u_y \\ u_z\end{bmatrix} \\
			&= \begin{bmatrix}u_x\cos\phi_z - u_y\sin\phi_z \\ u_x\sin\phi_z + u_y\cos\phi_z \\ u_z\end{bmatrix}
		\end{align*}
		this first condition is satisfied when
		$$u_x\sin\phi_z + u_y\cos\phi_z = 0$$
		i.e. $\tan\phi_z = -u_y/u_x$. Since rotations preserve length, we know immediately the this rotation must take $\mathbf{u}$ to $\mathbf{u}= \begin{bmatrix}(u_x^2 + u_y^2)^{1/2} & 0 & u_z\end{bmatrix}^T$. Applying the next rotation moves this vector to
		\begin{align*}
			R(\phi_y\mathbf{j})\mathbf{u} &= \begin{bmatrix}\cos\phi_y & 0 & \sin\phi_y \\ 0 & 1 & 0 \\ -\sin\phi_y & 0 & \cos\phi_y\end{bmatrix}\begin{bmatrix}(u_x^2+u_y^2)^{1/2} \\ 0 \\ u_z\end{bmatrix} \\
			&= \begin{bmatrix}
				(u_x^2 + u_y^2)^{1/2}\cos\phi_y + u_z\sin\phi_y \\ 0 \\ -(u_x^2 + u_y^2)^{1/2}\sin\phi_y + u_z\cos\phi_y
			\end{bmatrix}
		\end{align*}
		The second condition is met when
		$$-(u_x^2 + u_y^2)^{1/2}\sin\phi_y + u_z\cos\phi_y = 0$$
		or $\tan\phi_y = u_z/(u_x^2 + u_y^2)^{1/2}$. Using basic trigonometry, we know
		\begin{align*}
			\cos\phi_z &= \cos\arctan\Big({-\frac{u_y}{u_x}}\Big) = \frac{u_x}{(u_x^2 + u_y^2)^{1/2}} \\
			\sin\phi_z &= \sin\arctan\Big({-\frac{u_y}{u_x}}\Big) = {-\frac{u_y}{(u_x^2 + u_y^2)^{1/2}}} \\
			\cos\phi_y &= \cos\arctan\Big(\frac{u_z}{(u_x^2 + u_y^2)^{1/2}}\Big) = \frac{(u_x^2 + u_y^2)^{1/2}}{(u_x^2 + u_y^2 + u_z^2)^{1/2}} \\
			\sin\phi_y &= \sin\arctan\Big(\frac{u_z}{(u_x^2 + u_y^2)^{1/2}}\Big) = \frac{u_z}{(u_x^2 + u_y^2 + u_z^2)^{1/2}}
		\end{align*}
		and so
		\begin{align*}
			R(\phi_y\mathbf{j})R(\phi_z\mathbf{k}) &= \frac{1}{(1 - u_z^2)^{1/2}}\begin{bmatrix}
				(1 - u_z^2)^{1/2} & 0 & u_z \\
				0 & 1 & 0 \\
				-u_z & 0 & (1 - u_z^2)^{1/2}
			\end{bmatrix}\begin{bmatrix}
				u_x & u_y & 0 \\
				-u_y & u_x & 0 \\
				0 & 0 & (1 - u_z^2)^{1/2}
			\end{bmatrix} \\
			&= \frac{1}{(1 - u_z^2)^{1/2}}\begin{bmatrix}
				u_x(1 - u_z^2)^{1/2} & u_y(1 - u_z^2)^{1/2} & u_z(1 - u_z^2)^{1/2} \\
				-u_y & u_x & 0 \\
				-u_xu_z & -u_yu_z & 1 - u_z^2
			\end{bmatrix}
		\end{align*}
		Since $R(-\phi_z\mathbf{k})R(-\phi_y\mathbf{j}) =  R(\phi_z\mathbf{k})^TR(\phi_y\mathbf{j})^T = [R(\phi_y\mathbf{j})R(\phi_z\mathbf{k})]^T$, we can immediately write
		$$R(-\phi_z\mathbf{k})R(-\phi_y\mathbf{j}) = \frac{1}{(1 - u_z^2)^{1/2}}\begin{bmatrix}
			u_x(1 - u_z^2)^{1/2} & -u_y & -u_xu_z \\
			u_y(1 - u_z^2)^{1/2} & u_x & -u_yu_z \\
			u_z(1 - u_z^2)^{1/2} & 0 & 1 - u_z^2
		\end{bmatrix}$$
		Now, when $\delta\theta$ is small, the central $R(\delta\theta\mathbf{i})$ matrix becomes
		$$R(\theta\mathbf{i}) = \begin{bmatrix}
			1 & 0 & 0 \\
			0 & 1 & -\delta\theta \\
			0 & \delta\theta & 1
		\end{bmatrix}$$
		and so a small rotation about an arbitrary axis can be written as
		\begin{align*}
			&\frac{1}{1 - u_z^2}\begin{bmatrix}
				u_x(1 - u_z^2)^{1/2} & -u_y & -u_xu_z \\
				u_y(1 - u_z^2)^{1/2} & u_x & -u_yu_z \\
				u_z(1 - u_z^2)^{1/2} & 0 & 1 - u_z^2
			\end{bmatrix}\begin{bmatrix}
				1 & 0 & 0 \\
				0 & 1 & -\delta\theta \\
				0 & \delta\theta & 1
			\end{bmatrix}\begin{bmatrix}
			u_x(1 - u_z^2)^{1/2} & u_y(1 - u_z^2)^{1/2} & u_z(1 - u_z^2)^{1/2} \\
			-u_y & u_x & 0 \\
			-u_xu_z & -u_yu_z & 1 - u_z^2
		\end{bmatrix} \\
		=\,&\frac{1}{1 - u_z^2}\begin{bmatrix}
			u_x(1 - u_z^2)^{1/2} & -u_y & -u_xu_z \\
			u_y(1 - u_z^2)^{1/2} & u_x & -u_yu_z \\
			u_z(1 - u_z^2)^{1/2} & 0 & 1 - u_z^2
		\end{bmatrix}\begin{bmatrix}
			u_x(1 - u_z^2)^{1/2} & u_y(1 - u_z^2)^{1/2} & u_z(1 - u_z^2)^{1/2} \\
			-u_y + \delta\theta u_xu_z & u_x + \delta\theta u_y u_z & -\delta\theta(1 - u_z^2) \\
			-u_xu_z - \delta\theta u_y & -u_yu_z + \delta\theta u_x & 1 - u_z^2
		\end{bmatrix} \\
		=\,&\frac{1}{1 - u_z^2}\begin{bmatrix}
			1 - u_z^2 & -\delta\theta u_z(1 - u_z^2) & \delta\theta u_y(1 - u_z^2) \\
			\delta\theta u_z(1 - u_z^2) & 1 - u_z^2 & -\delta\theta u_x(1 - u_z^2) \\
			-\delta\theta u_y(1 - u_z^2) & \delta\theta u_x (1 - u_z^2) & 1 - u_z^2
		\end{bmatrix} \\
		=\,&\begin{bmatrix}
			1 & 0 & 0 \\
			0 & 1 & 0 \\
			0 & 0 & 1
		\end{bmatrix} + \delta\theta\begin{bmatrix}
			0 & -u_z & u_y \\
			u_z & 0 & -u_x \\
			-u_y & u_x & 0
		\end{bmatrix}
		\end{align*}
		which is the operator $(I + \delta\boldsymbol{\theta}\times)$, i.e. a small rotation of $\mathbf{r}$ by an angle $\delta\theta$ about $\mathbf{u}$ can be written as
		$$\mathbf{r} + \delta\boldsymbol{\theta}\times\mathbf{r}$$
		This fact can also be seen geometrically, as the rotated vector moves perpendicularly to both $\delta\boldsymbol{\theta}$ and $\mathbf{r}$ with a displacement proportional to $\|\mathbf{r}\| \|\delta\boldsymbol{\theta}\|\sin\delta\theta$, the magnitude of $\delta\boldsymbol{\theta}\times\mathbf{r}$.
		
		To first order, we can Taylor expand $\psi(\mathbf{r})$ under a small rotation as
		\begin{align*}
			\psi(\mathbf{r} - \delta\boldsymbol{\theta}\times\mathbf{r}) &= \psi(\mathbf{r}) - (\delta\boldsymbol{\theta}\times\mathbf{r})\cdot\nabla\psi \\
			&= \psi(\mathbf{r}) - (\delta\theta_y z - \delta\theta_z y)\frac{\partial\psi}{\partial x} - (\delta\theta_zx - \delta\theta_xz)\frac{\partial\psi}{\partial y} - (\delta\theta_xy - \delta\theta_yx)\frac{\partial\psi}{\partial z} \\
			&= \psi(\mathbf{r}) - \delta\theta_x\Big(y\frac{\partial}{\partial z} - z\frac{\partial}{\partial y}\Big)\psi(\mathbf{r}) - \delta\theta_y\Big(z\frac{\partial}{\partial x} - x\frac{\partial}{\partial z}\Big)\psi(\mathbf{r}) - \delta\theta_z\Big(x\frac{\partial}{\partial y} - y\frac{\partial}{\partial x}\Big)\psi(\mathbf{r}) \\
			&= \Big[I - \frac{i}{\hbar}\Big(\delta\theta_x(YP_z - ZP_y) - \delta\theta_y(ZP_x + XP_z) + \delta\theta_z(XP_y - YP_x)\Big)\Big]\psi(\mathbf{r}) \\
			&= \Big(I - \frac{i}{\hbar}\delta\boldsymbol{\theta}\cdot\mathbf{L}\Big)\psi(\mathbf{r})
		\end{align*}
		Equating this operator to $I - i\delta\theta L_{\hat{\theta}}/\hbar$, we see
		$$L_{\hat{\theta}} = \hat{\theta}\cdot\mathbf{L}$$
		where $\hat{\theta} = \mathbf{u}$.
	\end{solution}
	
	\question Recall that $\mathbf{V}$ is a vector operator if its components $V_i$ transform as
	$$U^\dagger[R]V_iU[R] = \sum_j R_{ij} V_j$$
	(1) For an infinitesimal rotation $\delta\boldsymbol{\theta}$, show, on the basis of the previous exercise, that
	$$\sum_j R_{ij}V_j = V_i + (\delta\boldsymbol{\theta}\times\mathbf{V})_i = V_i + \sum_j\sum_k\varepsilon_{ijk}(\delta\theta)_jV_k$$
	(2) Feed in $U[R] = 1 - (i/\hbar)\delta\boldsymbol{\theta}\cdot\mathbf{L}$ into the left-hand side of Eq. (12.4.13) and deduce that
	$$[V_i, L_j] = i\hbar\sum_k \varepsilon_{ijk}V_k$$
	This is as good a definition of a vector operator as Eq. (12.4.13). By setting $\mathbf{V} = \mathbf{L}$, we can obtain the commutation rules among the $L$'s.
	
	\begin{solution}
		The first part of this exercise was accomplished in the previous question's solution. In indicial notation, a small rotation of $\delta\theta$ about an arbitrary axis $\hat{\theta}$ can be written (in index notation) as
		$$V_i + \sum_j\sum_k\varepsilon_{ijk}(\delta\theta)_jV_k$$
		Feeding the given $U[R]$ into $U^\dagger[R]V_iU[R]$ and expanding gives
		\begin{align*}
			U^\dagger[R]V_iU[R] &= (1 - (i/\hbar)\delta\boldsymbol{\theta}\cdot\mathbf{L})^\dagger V_i(1 - (i/\hbar)\delta\boldsymbol{\theta}\cdot\mathbf{L}) \\
			&= (1 + \tfrac{i}{\hbar}(\delta\theta)_jL_j)V_i(1 - \tfrac{i}{\hbar}(\delta\theta)_kL_k) \\
			&= V_i + \tfrac{i}{\hbar}(\delta\theta)_jL_jV_i - \tfrac{i}{\hbar}(\delta\theta)_kV_iL_k + \tfrac{1}{\hbar^2}(\delta\theta)_j(\delta\theta)_kL_jV_iL_k
		\end{align*}
		Equating this to the first equation, canceling $V_i$, and dropping the term proportional to $(\delta\theta)^2$ gives the requirement
		$$-\tfrac{i}{\hbar}(\delta\theta)_j(V_iL_j - L_jV_i) = -\tfrac{i}{\hbar}(\delta\theta)_j[V_i, L_j] = \varepsilon_{ijk}(\delta\theta)_jV_k$$
		or
		$$[V_i, L_j] = i\hbar\varepsilon_{ijk}V_k$$
	\end{solution}


	\setcounter{subsection}{4}
	\setcounter{question}{0}
	\subsection{The Eigenvalue Problem of $L^2$ and $L_z$}
	\question Consider a vector field $\boldsymbol{\Psi}(x, y)$ in two dimensions. From Fig. 12.1 it follows that under an infinitesimal rotation $\varepsilon_z\mathbf{k}$,
	\begin{align*}
		\psi_x \to \psi_x'(x, y) &= \psi_x(x + y\varepsilon_z, y - x\varepsilon_z) - \psi_y(x + y\varepsilon_z, y - x\varepsilon_z)\varepsilon_z \\
		\psi_y \to \psi_y'(x, y) &= \psi_x(x + y\varepsilon_z, y - x\varepsilon_z)\varepsilon_z + \psi_y(x + y\varepsilon_z, y - x\varepsilon_z)
	\end{align*}
	Show that (to order $\varepsilon_z$)
	$$\begin{bmatrix}
		\psi'_x \\ \psi'_y
		\end{bmatrix}= \Big(\begin{bmatrix}1 & 0 \\ 0 & 1\end{bmatrix} - \frac{i\varepsilon_z}{\hbar}\begin{bmatrix}L_z & 0 \\ 0 & L_z\end{bmatrix} - \frac{i\varepsilon_z}{\hbar}\begin{bmatrix}0 & -i\hbar \\ i\hbar & 0\end{bmatrix}\Big)\begin{bmatrix}\psi_x \\ \psi_y\end{bmatrix}$$
	so that
	\begin{align*}
		J_z &= L_z^{(1)}\otimes I^{(2)} + I^{(1)}\otimes S_z^{(2)} \\
		&= L_z + S_z
	\end{align*}
	where $I^{(2)}$ is a $2\times 2$ identity matrix with respect to the vector components, $I^{(1)}$ is the identity operator with respect to the argument $(x, y)$ of $\boldsymbol{\Psi}(x, y)$. This example only illustrates the fact that $J_z = L_z + S_z$ if the wave function is not a scalar. An example of half-integral eigenvalues will be provided when we consider spin in a later chapter. (In the present example, $S_z$ has eigenvalues $\pm\hbar$.)
	
	\begin{solution}
		Since $\varepsilon_k$ is an infinitesimal value, we may Taylor expand the transformed components of $\boldsymbol{\Psi}(x, y)$ to find
		\begin{align*}
			\psi_x' &= \psi_x + \frac{\partial\psi_x}{\partial x}y\varepsilon_z - \frac{\partial\psi_x}{\partial y}x\varepsilon_z - \Big(\psi_y + \frac{\partial \psi_y}{\partial x}y\varepsilon_z - \frac{\partial\psi_y}{\partial y}x\varepsilon_z\Big)\varepsilon_z \\
			&= \psi_x + \frac{i\varepsilon_z}{\hbar}YP_x\psi_x - \frac{i\varepsilon_z}{\hbar}XP_y\psi_x - \psi_y\varepsilon_z \\
			&= \psi_x - \frac{i\varepsilon_z}{\hbar}(XP_y - YP_x)\psi_x - \frac{i\varepsilon_z}{\hbar}(-i\hbar\psi_y) \\
			&= \psi_x - \frac{i\varepsilon_z}{\hbar}L_z\psi_x - \frac{i\varepsilon_z}{\hbar}(-i\hbar\psi_y) \\
			\psi_y' &= \Big(\psi_x + \frac{\partial \psi_x}{\partial x}y\varepsilon_z - \frac{\partial \psi_x}{\partial y}x\varepsilon_z\Big)\varepsilon_z + \psi_y + \frac{\partial\psi_y}{\partial x}y\varepsilon_z - \frac{\partial \psi_y}{\partial y}x\varepsilon_z \\
			&= \psi_x\varepsilon_z + \psi_y + \frac{i\varepsilon_z}{\hbar}YP_x\psi_y - \frac{i\varepsilon_z}{\hbar}XP_y\psi_y \\
			&= \psi_y - \frac{i\varepsilon_z}{\hbar}(XP_y - YP_x)\psi_y - \frac{i\varepsilon_z}{\hbar}(i\hbar\psi_x)
		\end{align*}
		where we have suppressed the arguments to $\psi_x(x, y)$ and $\psi_y(x, y)$. We can collect these equations in matrix form as
		$$\begin{bmatrix}
		\psi'_x \\ \psi'_y
		\end{bmatrix}= \Big(\begin{bmatrix}1 & 0 \\ 0 & 1\end{bmatrix} - \frac{i\varepsilon_z}{\hbar}\begin{bmatrix}L_z & 0 \\ 0 & L_z\end{bmatrix} - \frac{i\varepsilon_z}{\hbar}\begin{bmatrix}0 & -i\hbar \\ i\hbar & 0\end{bmatrix}\Big)\begin{bmatrix}\psi_x \\ \psi_y\end{bmatrix}$$
		Now, the matrix containing $L_z$ terms acts to rotate the argument of $\boldsymbol{\Psi}(x, y)$, while the matrix containing $i\hbar$ terms rotates the components of $\boldsymbol{\Psi}$ itself. Thinking of these components as two separate `spaces' (the domain and codomain), we can write the generator of vectorial rotations as
		$$J_z = L_z^{(1)}\otimes I^{(2)} + I^{(1)}\otimes S_z^{(2)} = L_z + S_z$$
		\end{solution}
	
		\question (1) Verify that the $2\times2$ matrices $J_x^{(1/2)}$, $J_y^{(1/2)}$, and $J_z^{(1/2)}$ obey the commutation rule $[J_x^{(1/2)}, J_y^{(1/2)}] = i\hbar J_z^{(1/2)}$.
		
		(2) Do the same for the $3\times3$ matrices $J_i^{(1)}$.
		
		(3) Construct the $4\times4$ matrices and verify that
		$$[J_x^{(3/2)}, J_y^{(3/2)}] = i\hbar J_z^{(3/2)}$$
		
		\begin{solution}
			From what we have learned of the raising and lowering operators $J_+$ and $J_-$, we know that, for a given $j$, 
			\begin{align*}
				\langle jm'|J_x|jm\rangle &= \langle jm'|\frac{J_+ + J_-}{2}|jm\rangle \\
				&= \frac{\hbar}{2}\{\delta_{m',m+1}[(j-m)(j+m+1)]^{1/2} + \delta_{m',m-1}[(j+m)(j-m+1)]^{1/2}\} \\
				\langle jm'|J_y|jm\rangle &= \langle jm'|\frac{J_+ - J_-}{2i}|jm\rangle \\
				&= \frac{\hbar}{2i}\{\delta_{m',m+1}[(j-m)(j+m+1)]^{1/2} - \delta_{m',m-1}[(j+m)(j-m+1)]^{1/2}\} \\
				\langle jm'|J_z|jm\rangle &= \delta_{m', m}m\hbar
			\end{align*}
			Recalling that $m$ takes on integer spacing between $-j$ and $j$, we choose the matrix indices $k$ and $l$ (i.e. $[J_x^{(j)}]_{kl}$, etc.) such that $m = j - (l - 1)$ and $m' = j - (k - 1)$. Both indices start at $1$. With this choice, the index forms of our matrices are
			\begin{align*}
				[J_x^{(j)}]_{kl} &= \frac{\hbar}{2}\{\delta_{k, l - 1}[(l - 1)(2j - l + 2)]^{1/2} + \delta_{k, l + 1}[l(2j - l + 1)]^{1/2}\} \\
				[J_y^{(j)}]_{kl} &= -\frac{i\hbar}{2}\{\delta_{k, l - 1}[(l - 1)(2j - l + 2)]^{1/2} - \delta_{k, l + 1}[l(2j - l + 1)]^{1/2}\} \\
				[J_z^{(j)}]_{kl} &= \delta_{kl}(j - l + 1)\hbar
			\end{align*}
			We can now produce the set of $2\times2$ matrices corresponding to $j = 1/2$.
			\begin{align*}
				J_x^{(1/2)} &= \begin{bmatrix}0 & \hbar/2 \\ \hbar/2 & 0 \end{bmatrix} \\
				J_y^{(1/2)} &= \begin{bmatrix} 0 & -i\hbar/2 \\ i\hbar/2 & 0\end{bmatrix} \\
				J_z^{(1/2)} &= \begin{bmatrix}\hbar / 2 & 0 \\ 0 & -\hbar/2\end{bmatrix}
			\end{align*}
			The commutator is
			\begin{align*}
				[J_x^{(1/2)}, J_y^{(1/2)}] &= \frac{i\hbar^2}{4}\begin{bmatrix}0 & 1 \\ 1 & 0\end{bmatrix}\begin{bmatrix}0 & -1 \\ 1 & 0\end{bmatrix} - \frac{i\hbar^2}{4}\begin{bmatrix}0 & -1 \\ 1 & 0\end{bmatrix}\begin{bmatrix}0 & 1\\1 & 0\end{bmatrix} \\
				&= \frac{i\hbar^2}{4}\begin{bmatrix}1 & 0 \\ 0 & -1\end{bmatrix} - \frac{i\hbar^2}{4}\begin{bmatrix}-1 & 0 \\ 0 & 1\end{bmatrix} \\
				&= \frac{i\hbar^2}{4}\begin{bmatrix}2 & 0 \\ 0 & -2\end{bmatrix} \\
				&= i\hbar\begin{bmatrix}\hbar/2 & 0 \\ 0 & -\hbar/2\end{bmatrix} \\
				&= i\hbar J_z^{(1/2)}
			\end{align*}
			The set of $3\times3$ matrices is
			\begin{align*}
				J_x^{(1)} &= \begin{bmatrix}0 & \hbar/2^{1/2} & 0 \\
				\hbar/2^{1/2} & 0 & \hbar/2^{1/2} \\
				0 & \hbar/2^{1/2} & 0\end{bmatrix} \\
				J_y^{(1)} &= \begin{bmatrix}0 & -i\hbar/2^{1/2} & 0 \\ i\hbar/2^{1/2} & 0 & -i\hbar/2^{1/2} \\
				0 & i\hbar/2^{1/2} & 0\end{bmatrix} \\
				J_z^{(1)} &= \begin{bmatrix}\hbar & 0 & 0 \\ 0 & 0 & 0 \\ 0 & 0 & -\hbar\end{bmatrix}
			\end{align*}
			Their commutator is
			\begin{align*}
				[J_x^{(1)}, J_y^{(1)}] &= \frac{i\hbar^2}{2}\begin{bmatrix}0 & 1 & 0 \\ 1 & 0 & 1 \\ 0 & 1 & 0 \end{bmatrix}\begin{bmatrix} 0 & -1 & 0 \\ 1 & 0 & -1 \\ 0 & 1 & 0 \end{bmatrix} - \frac{i\hbar^2}{2}\begin{bmatrix}0 & -1 & 0 \\ 1 & 0 & -1 \\ 0 & 1 & 0\end{bmatrix}\begin{bmatrix}0 & 1 & 0 \\ 1 & 0 & 1 \\ 0 & 1 & 0 \end{bmatrix} \\
				&= \frac{i\hbar^2}{2}\begin{bmatrix}1 & 0 & -1\\ 0 & 0 & 0\\ 1 & 0 & -1\end{bmatrix} - \frac{i\hbar^2}{2}\begin{bmatrix}-1 & 0 & -1 \\ 0 & 0 & 0\\ 1 & 0 & 1\end{bmatrix} \\
				&= \frac{i\hbar^2}{2}\begin{bmatrix}2 & 0 & 0 \\ 0 & 0 & 0 \\ 0 & 0 & -2\end{bmatrix} \\
				&= i\hbar\begin{bmatrix}\hbar & 0 & 0 \\ 0 & 0 & 0 \\ 0 & 0 & -\hbar\end{bmatrix} \\
				&= i\hbar J_z^{(1)}
			\end{align*}
			Finally, the set of $4\times4$ matrices is
			\begin{align*}
				J_x^{(3/2)} &= \frac{\hbar}{2}\begin{bmatrix}0 & 3^{1/2} & 0 & 0 \\ 3^{1/2} & 0 & 2 & 0 \\ 0 & 2 & 0 & 3^{1/2} \\ 0 & 0 & 3^{1/2} & 0\end{bmatrix}\\
				J_y^{(3/2)} &= \frac{i\hbar}{2}\begin{bmatrix}0 & -3^{1/2} & 0 & 0 \\ 3^{1/2} & 0 & -2 & 0 \\ 0 & 2 & 0 & -3^{1/2} \\ 0 & 0 & 3^{1/2} & 0  \end{bmatrix} \\
				J_z^{(3/2)} &= \frac{\hbar}{2}\begin{bmatrix}3 & 0 & 0 & 0 \\ 0 & 1 & 0 & 0 \\ 0 & 0 & -1 & 0 \\ 0 & 0 & 0 & -3\end{bmatrix}
			\end{align*}
			Their commutation relation is
			\begin{align*}
				[J_x^{(3/2)}, J_y^{(3/2)}] =\,&\frac{i\hbar^2}{4}\begin{bmatrix}0 & 3^{1/2} & 0 & 0 \\ 3^{1/2} & 0 & 2 & 0 \\ 0 & 2 & 0 & 3^{1/2} \\ 0 & 0 & 3^{1/2} & 0\end{bmatrix}\begin{bmatrix}0 & -3^{1/2} & 0 & 0 \\ 3^{1/2} & 0 & -2 & 0 \\ 0 & 2 & 0 & -3^{1/2} \\ 0 & 0 & 3^{1/2} & 0  \end{bmatrix} \\
				&- \frac{i\hbar^2}{4}\begin{bmatrix}0 & -3^{1/2} & 0 & 0 \\ 3^{1/2} & 0 & -2 & 0 \\ 0 & 2 & 0 & -3^{1/2} \\ 0 & 0 & 3^{1/2} & 0  \end{bmatrix}\begin{bmatrix}0 & 3^{1/2} & 0 & 0 \\ 3^{1/2} & 0 & 2 & 0 \\ 0 & 2 & 0 & 3^{1/2} \\ 0 & 0 & 3^{1/2} & 0\end{bmatrix} \\
				=\,&\frac{i\hbar^2}{4}\begin{bmatrix}3 & 0 & -2\cdot3^{1/2} & 0 \\ 0 & 1 & 0 & -2\cdot 3^{1/2} \\ 2\cdot3^{1/2} & 0 & -1 & 0 \\ 0 & 2\cdot3^{1/2} & 0 & -3\end{bmatrix} \\
				&-\frac{i\hbar^2}{4}\begin{bmatrix}-3 & 0 & -2\cdot3^{1/2} & 0 \\ 0 & -1 & 0 & -2\cdot 3^{1/2} \\ 2\cdot3^{1/2} & 0 & 1 & 0 \\ 0 & 2\cdot3^{1/2} & 0 & 3\end{bmatrix} \\
				=\,&\frac{i\hbar^2}{4}\begin{bmatrix}6 & 0 & 0 & 0 \\ 0 & 2 & 0 & 0 \\ 0 & 0 & -2 & 0 \\ 0 & 0 & 0 & -6\end{bmatrix} \\
				=\,&i\hbar\cdot\frac{\hbar}{2}\begin{bmatrix}3 & 0 & 0 & 0 \\ 0 & 1 & 0 & 0 \\ 0 & 0 & -1 & 0 \\ 0 & 0 & 0 & -3\end{bmatrix} \\
				=\,&i\hbar J_z^{(3/2)}
			\end{align*}
		\end{solution}
		
		\question (1) Show that $\langle J_x\rangle = \langle J_y\rangle = 0$ in a state $|jm\rangle$.
		
		(2) Show that in these states
		$$\langle J_x^2\rangle = \langle J_y^2\rangle = \tfrac{1}{2}\hbar^2[j(j+1)-m^2]$$
		(use symmetry arguments to relate $\langle J_x^2\rangle$ to $\langle J_y^2\rangle$).
		
		(3) Check that $\Delta J_x \cdot \Delta J_y$ from part (2) satisfies the inequality imposed by the uncertainty principle [Eq. (9.2.9)].
		
		(4) Show that the uncertainty bound is saturated in the state $|j, \pm j\rangle$.
		\begin{solution}
			The first problem is a simple exercise in algebra,
			\begin{align*}
				\langle J_x\rangle &= \langle jm|\frac{J_+ + J_-}{2}|jm\rangle \\
				&= \frac{C_+(j, m)}{2}\langle jm|j, m+1\rangle + \frac{C_-(j, m)}{2}\langle jm|j, m - 1\rangle \\
				&= 0 \\
				\langle J_y\rangle &= \langle jm|\frac{J_+ - J_-}{2i}|jm\rangle \\
				&= \frac{C_+(j, m)}{2i}\langle jm|j, m+1\rangle - \frac{C_-(j, m)}{2i}\langle jm|j, m-1\rangle \\
				&= 0
			\end{align*}
			For the second, note that
			\begin{align*}
				\Big(\frac{J_+ + J_-}{2}\Big)^2 &= \frac{1}{4}\Big(J_+^2 + J_+J_- + J_-J_+ + J_-^2\Big) \\
				&= \frac{1}{4}\Big(J_+^2 + (J^2 - J_z^2 + \hbar J_z) + (J^2 - J_z^2 - \hbar J_z) + J_-^2)\Big) \\
				&= \frac{1}{4}\Big(J_+^2 + J_-^2 + 2J^2 - 2J_z^2\Big) \\
				\Big(\frac{J_+ - J_-}{2i}\Big)^2 &= -\frac{1}{4}\Big(J_+^2 - J_+J_- - J_-J_+ + J_-^2\Big) \\
				&= -\frac{1}{4}\Big(J_+^2 - (J^2 - J_z^2 + \hbar J_z) - (J^2 - J_z^2 - \hbar J_z) + J_-^2)\Big) \\
				&= -\frac{1}{4}\Big(J_+^2 + J_-^2 - 2J^2 + 2J_z^2\Big)
			\end{align*}
			and so
			\begin{align*}
				\langle J_x^2\rangle &= \frac{1}{4}\langle jm|J_+^2 + J_-^2 + 2J^2 - 2J_z^2|jm\rangle \\
				&= \frac{1}{2}\langle jm|J^2 - J_z^2|jm\rangle \\
				&= \frac{1}{2}\langle jm|\hbar^2 j(j+1) - \hbar^2m^2|jm\rangle \\
				&= \frac{1}{2}\hbar^2[j(j + 1) - m^2] \\
				\langle J_y^2\rangle &= -\frac{1}{4}\langle jm|J_+^2 + J_-^2 - 2J^2 + 2J_z^2|jm\rangle \\
				&= \frac{1}{2}\langle jm|J^2 - J_z^2|jm\rangle \\
				&= \frac{1}{2}\hbar^2[j(j + 1) - m^2]
			\end{align*}
			Further noting that
			\begin{align*}
				J_xJ_y &= \frac{1}{4i}(J_+ + J_-)(J_+ - J_-) \\
				&= \frac{1}{4i}(J_+^2 + J_-J_+ - J_+J_- - J_-^2) \\
				J_yJ_x &= \frac{1}{4i}(J_+ - J_-)(J_+ + J_-) \\
				&= \frac{1}{4i}(J_+^2 + J_+J_- - J_-J_+ - J_-^2)
			\end{align*}
			and thus
			$$[J_x, J_y]_+ = J_xJ_y + J_yJ_x = \frac{1}{2i}(J_+^2 - J_-^2) \implies \langle jm|[J_x, J_y]_+|jm\rangle = 0,$$
			the uncertainty principle tells us that
			\begin{align*}
				\Delta J_x\cdot\Delta J_y = \frac{1}{2}\hbar^2|j(j + 1) - m^2| &\geq \frac{1}{2}|\langle jm|\hbar J_z|jm\rangle| = \frac{|m|\hbar^2}{2}
			\end{align*}
			which effectively says
			$$j^2 + j \geq |m^2 + m|,$$
			a statement we know to be true (as $j$---the total angular momentum---must always be greater than or equal in magnitude to $m$---the $z$ component of angular momentum). Clearly, this reaches an equality when $m = j$.
		\end{solution}
	
		\question (1) Argue that the eigenvalues of $J_x^{(i)}$ and $J_y^{(i)}$ are the same as those of $J_z^{(i)}$, namely, $j\hbar, (j - 1)\hbar, \dots, (-j\hbar)$. Generalize the result to $\hat{\theta}\cdot\mathbf{J}^{(j)}$.
		
		(2) Show that
		$$(J - j\hbar)[J - (j-1)\hbar][J - (j - 2)\hbar]\cdots(J + j\hbar) = 0$$
		where $J \equiv \hat{\theta}\cdot\mathbf{J}^{(j)}$. (Hint: In the case $J = J_z$ what happens when both sides are applied to an arbitrary eigenket $|jm\rangle$? What about an arbitrary superposition of such kets?)
		
		(3) It follows from (2) that $J^{2j + 1}$ is a linear combination of $J^0, J^1, \dots, J^{2j}$. Argue that the same goes for $J^{2j + k}$, $k = 1, 2, \dots$.
		
		\begin{solution}
			As there is no preferred direction and we may have just as easily chosen $\hat{\theta}\cdot\mathbf{J}$ as $J_z$ to be diagonal in the shared $J^2$, $J_i$ basis, all $\hat{\theta}\cdot\mathbf{J}^{(j)}$ must have the same eigenvalues as $J_z^{(j)}$. The fact that
			$$(J - j\hbar)[J - (j-1)\hbar][J - (j - 2)\hbar]\cdots(J + j\hbar) = 0$$
			is a consequence of angular momentum being quantized to some value of $m$ along \textit{any} axis. That is, it is a restatement of the observation made in the first sentence.
			
			The linear dependence of $J^{2j + k}$ can be seen by writing those factors of $J$ whose powers exceed $2j$ as a linear combination of lower powers of $J$.
		\end{solution}
		
		\question \textit{(Hard).} Using results from the previous exercise and Eq. (12.5.23), show that
		
		(1) $D^{(1/2)}[R] = \exp(-i\hat{\theta}\cdot\mathbf{J}^{(1/2)}/\hbar) = \cos(\theta/2)I^{1/2} - (2i/\hbar)\sin(\theta/2)\hat{\theta}\cdot\mathbf{J}^{(1/2)}$
		
		(2) $D^{(1)}[R] = \exp(-i\theta_xJ_x^{(1)}/\hbar) = (\cos\theta_x - 1)\Big(\frac{J_x^{(1)}}{\hbar}\Big)^2 - i\sin\theta_x\Big(\frac{J_x^{(1)}}{\hbar}\Big) + I^{(1)}$
		
		\begin{solution}
			From the previous problem, we know that
			$$(\hat{\theta}\cdot\mathbf{J}^{(1/2)} - \tfrac{1}{2}\hbar)(\hat{\theta}\cdot\mathbf{J}^{(1/2)} + \tfrac{1}{2}\hbar) = 0$$
			or
			$$(\hat{\theta}\cdot\mathbf{J}^{(1/2)})^2 = \tfrac{1}{4}\hbar^2$$
			Armed with this piece of information (and calling $\hat{\theta}\cdot\mathbf{J}^{(1/2)} \equiv J$), we can expand $D^{(1/2)}[R]$ as
			\begin{align*}
				D^{(1/2)}[R] &= \sum_{n=0}^{\infty}\Big({\frac{-i\theta}{\hbar}}\Big)^nJ^n\frac{1}{n!} \\
				&= I - i\Big(\frac{\theta}{\hbar}\Big)J - \frac{1}{2!}\Big(\frac{\theta}{\hbar}\Big)^2J^2 + i\frac{1}{3!}\Big(\frac{\theta}{\hbar}\Big)^3J^3 + \frac{1}{4!}\Big(\frac{\theta}{\hbar}\Big)^4J^4 - i\frac{1}{5!}\Big(\frac{\theta}{\hbar}\Big)^5J^5 - \cdots \\
				&= \Big[I - \frac{1}{2!}\Big(\frac{\theta}{\hbar}\Big)^2J^2 + \frac{1}{4!}\Big(\frac{\theta}{\hbar}\Big)^4J^4 - \cdots\Big] - i\Big[\Big(\frac{\theta}{\hbar}\Big)J - \frac{1}{3!}\Big(\frac{\theta}{\hbar}\Big)^3J^3 + \frac{1}{5!}\Big(\frac{\theta}{\hbar}\Big)^5J^5 - \cdots\Big] \\
				&= \Big[I - \frac{1}{2!}\Big(\frac{\theta}{\hbar}\Big)^2\Big(\frac{\hbar}{2}\Big)^2 + \frac{1}{4!}\Big(\frac{\theta}{\hbar}\Big)^4\Big(\frac{\hbar}{2}\Big)^4 - \cdots\Big] - i\Big[\Big(\frac{\theta}{\hbar}\Big)J - \frac{1}{3!}\Big(\frac{\theta}{\hbar}\Big)^3\Big(\frac{\hbar}{2}\Big)^2J + \frac{1}{5!}\Big(\frac{\theta}{\hbar}\Big)^5\Big(\frac{\hbar}{2}\Big)^4J - \cdots\Big] \\
				&= \Big[I - \frac{1}{2!}\Big(\frac{\theta}{2}\Big)^2 + \frac{1}{4!}\Big(\frac{\theta}{2}\Big)^4 - \cdots\Big] - \frac{2i}{\hbar}\Big[\Big(\frac{\theta}{2}\Big) - \frac{1}{3!}\Big(\frac{\theta}{2}\Big)^3 + \frac{1}{5!}\Big(\frac{\theta}{2}\Big)^5 - \cdots \Big]J \\
				&= \cos(\theta/2)I^{(1/2)} - (2i/\hbar)\sin(\theta/2)\hat{\theta}\cdot\mathbf{J}^{(1/2)}
			\end{align*}
			For the second part, we draw upon the previous problem to write
			$$(J_x^{(1)} - \hbar)J_x^{(1)}(J_x^{(1)} + \hbar) = 0$$
			or
			$$(J_x^{(1)})^3 = \hbar^2J_x^{(1)}$$
			Calling $J_x^{(1)}\equiv J$, we find
			\begin{align*}
				D^{(1)}[R] &= \sum_{n=0}^{\infty}\Big({\frac{-i\theta}{\hbar}}\Big)^nJ^n\frac{1}{n!} \\
				&= I - i\Big(\frac{\theta}{\hbar}\Big)J - \frac{1}{2!}\Big(\frac{\theta}{\hbar}\Big)^2J^2 + i\frac{1}{3!}\Big(\frac{\theta}{\hbar}\Big)^3J^3 + \frac{1}{4!}\Big(\frac{\theta}{\hbar}\Big)^4J^4 - i\frac{1}{5!}\Big(\frac{\theta}{\hbar}\Big)^5J^5 - \cdots \\
				&= I - i\Big(\frac{\theta}{\hbar}\Big)J - \frac{1}{2!}\Big(\frac{\theta}{\hbar}\Big)^2J^2 + i\frac{1}{3!}\Big(\frac{\theta}{\hbar}\Big)^3\hbar^2J + \frac{1}{4!}\Big(\frac{\theta}{\hbar}\Big)^4\hbar^2J^2 - i\frac{1}{5!}\Big(\frac{\theta}{\hbar}\Big)^5\hbar^4J - \cdots \\
				&= I + \Big[-\frac{1}{2!}\theta^2\frac{J^2}{\hbar^2} + \frac{1}{4!}\theta^4\frac{J^2}{\hbar^2} - \cdots\Big] - i\Big[\theta\frac{J}{\hbar} - \frac{1}{3!}\theta^3\frac{J}{\hbar} + \frac{1}{5!}\theta^5\frac{J}{\hbar} - \cdots\Big] \\
				&= I + (1 - \frac{1}{2!}\theta^2 + \frac{1}{4!}\theta^4 - \cdots - 1)\frac{J^2}{\hbar^2} - i\Big[\theta - \frac{1}{3!}\theta^3 + \frac{1}{5!}\theta^5 - \cdots\Big]\frac{J}{\hbar} \\
				&= I^{(1)} + (\cos\theta_x - 1)\Big(\frac{J_x^{(1)}}{\hbar}\Big)^2 - i\sin\theta_x\Big(\frac{J_x^{(1)}}{\hbar}\Big)
			\end{align*}
		\end{solution}
	
		\question Consider the family of states $|jj\rangle,\dots,|jm\rangle,\dots,|j,-j\rangle$. One refers to them as states of the same magnitude but different orientations of angular momentum. If one takes this remark literally, i.e., in the classical sense, one is led to believe that one may rotate these into each other, as is the case for classical states with these properties. Consider, for instance, the family $|1, 1\rangle$, $|1, 0\rangle$, $|1, -1\rangle$. It may seem, for example, that the state with zero angular momentum along the $z$ axis, $|1, 0\rangle$, may be obtained by rotating $|1, 1\rangle$ by some suitable ($\tfrac{1}{2}\pi$?) angle about the $x$ axis. Using $D^{(1)}[R(\theta_x\mathbf{i})]$ from part (2) in the last exercise show that
		$$|1, 0\rangle \neq D^{(1)}[R(\theta_x\mathbf{i})]|1, 1\rangle\quad\text{for any }\theta_x$$
		The error stems from the fact that classical reasoning should be applied to $\langle\mathbf{J}\rangle$, which responds to rotations like an ordinary vector, and not directly to $|jm\rangle$, which is a vector in Hilbert space. Verify that $\langle\mathbf{J}\rangle$ responds to rotations like its classical counterpart, by showing that $\langle\mathbf{J}\rangle$ in the state $D^{(1)}[R(\theta_x\mathbf{i})]|1, 1\rangle$ is $\hbar[-\sin\theta_x\mathbf{j} + \cos\theta_x\mathbf{k}]$.
		
		It is not too hard to see why we can't always satisfy
		$$|jm'\rangle = D^{(j)}[R]|jm\rangle$$
		or more generally, for two normalized kets $|\psi'_j\rangle$ and $|\psi_j\rangle$, satisfy
		$$|\psi_j'\rangle = D^{(j)}[R]|\psi_j\rangle$$
		by any choice of $R$. These abstract equations imply $(2j + 1)$ linear, complex relations between the components of $|\psi'_j\rangle$ and $|\psi_j\rangle$ that can't be satisfied by varying $R$, which depends on only three parameters, $\theta_x$, $\theta_y$, and $\theta_z$. (Of course one can find a unitary matrix in $\mathbb{V}_J$ that takes $|jm\rangle$ into $|jm'\rangle$ or $|\psi_j\rangle$ in $\psi_j'\rangle$, but it will not be a \textit{rotation} matrix corresponding to $U[R]$.
		
		\begin{solution}
			Using the results from the previous exercise and (12.5.23), we have
			\begin{align*}
				D^{(1)}[R(\theta_x\mathbf{i})] &= \frac{1}{\hbar^2}(\cos\theta_x - 1)\begin{bmatrix}0 & \hbar/2^{1/2} & 0 \\ \hbar/2^{1/2} & 0 & \hbar/2^{1/2} \\ 0 & \hbar/2^{1/2} & 0\end{bmatrix}^2 - \frac{i}{\hbar}\sin\theta_x\begin{bmatrix}0 & \hbar/2^{1/2} & 0 \\ \hbar/2^{1/2} & 0 & \hbar/2^{1/2} \\ 0 & \hbar/2^{1/2} & 0\end{bmatrix} + \begin{bmatrix}1 & 0 & 0 \\ 0 & 1 & 0 \\ 0 & 0 & 1\end{bmatrix} \\
				&= (\cos\theta_x - 1)\begin{bmatrix}1/2 & 0 & 1/2 \\ 0 & 1 & 0 \\ 1/2 & 0 & 1/2\end{bmatrix} - i\sin\theta_x\begin{bmatrix}0 & 2^{-1/2} & 0 \\ 2^{-1/2} & 0 & 2^{-1/2} \\ 0 & 2^{-1/2} & 0\end{bmatrix} + \begin{bmatrix}1 & 0 & 0 \\ 0 & 1 & 0 \\ 0 & 0 & 1\end{bmatrix} \\
				&= \begin{bmatrix}
					(\cos\theta_x + 1)/2 & -i\sin(\theta_x)/2^{1/2} & (\cos\theta_x - 1)/2 \\
					-i\sin(\theta_x)/2^{1/2} & \cos\theta_x & -i\sin(\theta_x)/2^{1/2} \\
					(\cos\theta_x - 1)/2 & -i\sin(\theta_x)/2^{1/2} & (\cos\theta_x+ 1)/2
				\end{bmatrix}
			\end{align*}
			where the basis is represented as
			$$\begin{bmatrix}1 \\ 0 \\ 0\end{bmatrix} = |1, 1\rangle, \qquad \begin{bmatrix}0 \\ 1 \\ 0\end{bmatrix} = |1, 0\rangle, \qquad \begin{bmatrix}0 \\ 0 \\ 1\end{bmatrix} = |1, {-1}\rangle$$
			Applying the rotation matrix to $|1, 1\rangle$ gives
			$$D^{(1)}[R(\theta_x\mathbf{i})]|1, 1\rangle = \begin{bmatrix}(\cos\theta_x + 1)/2 \\ -i\sin(\theta_x)/2^{1/2} \\ (\cos\theta_x - 1)/2\end{bmatrix}$$
			As the first and third entries cannot be simultaneously nonzero, this operation will never change $|1, 1\rangle$ into $|1, 0\rangle$.
			
			What of the expectation values of $J_x^{(1)}$, $J_y^{(1)}$, and $J_z^{(1)}$ in this rotated state? For these we have
			\begin{align*}
				\langle J_x\rangle &= \begin{bmatrix}(\cos\theta_x + 1)/2 & i\sin(\theta_x)/2^{1/2} & (\cos\theta_x - 1)/2\end{bmatrix}\begin{bmatrix}0 & \hbar/2^{1/2} & 0 \\ \hbar/2^{1/2} & 0 & \hbar/2^{1/2} \\ 0 & \hbar/2^{1/2} & 0\end{bmatrix}\begin{bmatrix}(\cos\theta_x + 1)/2 \\ -i\sin(\theta_x)/2^{1/2} \\ (\cos\theta_x - 1)/2\end{bmatrix} \\
				&= \begin{bmatrix}(\cos\theta_x + 1)/2 & i\sin(\theta_x)/2^{1/2} & (\cos\theta_x - 1)/2\end{bmatrix}\begin{bmatrix}-i\hbar\sin(\theta_x)/2 \\ \hbar\cos\theta_x/2^{1/2} \\ -i\hbar\sin(\theta_x)/2\end{bmatrix} \\
				&= -i\hbar(\cos\theta_x - 1)\sin(\theta_x)/4 - i\hbar(\cos\theta_x + 1)\sin(\theta_x)/4 + i\hbar\sin(\theta_x)\cos(\theta_x) / 2 \\
				&= 0 \\
				\langle J_y\rangle &= \begin{bmatrix}(\cos\theta_x + 1)/2 & i\sin(\theta_x)/2^{1/2} & (\cos\theta_x - 1)/2\end{bmatrix}\begin{bmatrix}0 & -i\hbar/2^{1/2} & 0 \\ i\hbar/2^{1/2} & 0 & -i\hbar/2^{1/2} \\ 0 & i\hbar/2^{1/2} & 0\end{bmatrix}\begin{bmatrix}(\cos\theta_x + 1)/2 \\ -i\sin(\theta_x)/2^{1/2} \\ (\cos\theta_x - 1)/2\end{bmatrix} \\
				&= \begin{bmatrix}(\cos\theta_x + 1)/2 & i\sin(\theta_x)/2^{1/2} & (\cos\theta_x - 1)/2\end{bmatrix}\begin{bmatrix}-\hbar\sin(\theta_x)/2 \\ i\hbar/2^{1/2} \\ \hbar\sin(\theta_x)/2\end{bmatrix} \\
				&= \hbar(\cos\theta_x - 1)\sin(\theta_x)/4 - \hbar(\cos\theta_x + 1)\sin(\theta_x)/4 - \hbar\sin(\theta_x)/2 \\
				&= -\hbar\sin\theta_x \\
				\langle J_z\rangle &= \begin{bmatrix}(\cos\theta_x + 1)/2 & i\sin(\theta_x)/2^{1/2} & (\cos\theta_x - 1)/2\end{bmatrix}\begin{bmatrix}\hbar & 0 & 0 \\ 0 & 0 & 0 \\ 0 & 0 & -\hbar\end{bmatrix}\begin{bmatrix}(\cos\theta_x + 1)/2 \\ -i\sin(\theta_x)/2^{1/2} \\ (\cos\theta_x - 1)/2\end{bmatrix} \\
				&= \begin{bmatrix}(\cos\theta_x + 1)/2 & i\sin(\theta_x)/2^{1/2} & (\cos\theta_x - 1)/2\end{bmatrix}\begin{bmatrix}\hbar(\cos\theta_x + 1)/2 \\ 0 \\ -\hbar(\cos\theta_x - 1)/2 \end{bmatrix} \\
				&= \hbar[(\cos\theta_x + 1)/2]^2 - \hbar[(\cos\theta_x - 1)/2]^2 \\
				&= \hbar\cos^2(\theta_x/2) - \hbar\sin^2(\theta_x/2) \\
				&= \hbar\cos\theta_x
			\end{align*}
			or, compactly,
			$$\langle \mathbf{J}\rangle = \hbar\begin{bmatrix}0 \\ -\sin\theta_x \\ \cos\theta_x\end{bmatrix}$$
			where the basis is now in physical space, not Hilbert space. In words, while we are not able to na\"ively rotate the state into $|1, 0\rangle$, we are able to rotate the expectation value until.
		\end{solution}
		
		\question \textit{Euler Angles}. Rather than parameterize an arbitrary rotation by the angle $\boldsymbol{\theta}$, which describes a \textit{single} rotation by $\theta$ about an axis parallel to $\boldsymbol{\theta}$, we may parameterize it by three angles, $\gamma$, $\beta$, and $\alpha$ called \textit{Euler angles}, which define three successive rotations:
		$$U[R(\alpha, \beta, \gamma)] = e^{-i\alpha J_z/\hbar}e^{-i\beta J_y/\hbar}e^{-i\gamma J_z/\hbar}$$
		(1) Construct $D^{(1)}[R(\alpha, \beta, \gamma)]$ explicitly as a product of three $3\times3$ matrices. (Use the result from Exercise 12.5.5 with $J_x\to J_y$.)
		
		(2)Let it act on $|1, 1\rangle$ and show that $\langle\mathbf{J}\rangle$ in the resulting state is
		$$\langle\mathbf{J}\rangle = \hbar(\sin\beta\cos\alpha\mathbf{i} + \sin\beta\sin\alpha\mathbf{j} + \cos\beta\mathbf{k})$$
		
		(3) Show that for no value of $\alpha$, $\beta$, and $\gamma$ can one rotate $|1, 1\rangle$ into just $|1, 0\rangle$.
		
		(4) Show that one can always rotate any $|1, m\rangle$ into a linear combination that involves $|1, m'\rangle$, i.e.
		$$\langle 1, m'|D^{(1)}[R(\alpha, \beta, \gamma)]|1, m\rangle \neq 0$$
		for some $\alpha$, $\beta$, $\gamma$ and any $m$, $m'$.
		
		(5) To see that one can occasionally rotate $|jm\rangle$ into $|jm'\rangle$, verify that a $180^\circ$ rotation about the $y$ axis applied to $|1, 1\rangle$ turns it into $|1, -1\rangle$.
		
		\begin{solution}
			The $J_z^{(1)}$ matrix is easy to exponentiate,
			$$e^{-i\alpha J_z^{(1)}/\hbar} = \begin{bmatrix}e^{-i\alpha} & 0 & 0 \\ 0 & 0 & 0 \\ 0 & 0 & e^{i\alpha}\end{bmatrix}, \qquad e^{-i\gamma J_z^{(1)}/\hbar} = \begin{bmatrix}e^{-i\gamma} & 0 & 0 \\ 0 & 0 & 0 \\ 0 & 0 & e^{i\gamma}\end{bmatrix}$$
			The compact form of $e^{-i\beta J_y/\hbar}$ is
			\begin{align*}
				D^{(1)}[R(\beta\mathbf{j})] =\,&\frac{1}{\hbar^2}(\cos\beta - 1)\begin{bmatrix}0 & -i\hbar/2^{1/2} & 0 \\ i\hbar/2^{1/2} & 0 & -i\hbar/2^{1/2} \\ 0 & i\hbar/2^{1/2} & 0\end{bmatrix}^2 - \frac{i}{\hbar}\sin\beta\begin{bmatrix}0 & -i\hbar/2^{1/2} & 0 \\ i\hbar/2^{1/2} & 0 & -i\hbar/2^{1/2} \\ 0 & i\hbar/2^{1/2} & 0\end{bmatrix} \\
				&+ \begin{bmatrix}1 & 0 & 0 \\ 0 & 1 & 0 \\ 0 & 0 & 1\end{bmatrix} \\
				=\,&(\cos\beta - 1)\begin{bmatrix}1/2 & 0 & -1/2 \\ 0 & 1 & 0 \\ -1/2 & 0 & 1/2\end{bmatrix} + \frac{1}{2^{1/2}}\sin\beta\begin{bmatrix}0 & -1 & 0 \\ 1 & 0 & -1 \\ 0 & 1 & 0\end{bmatrix} + \begin{bmatrix}1 & 0 & 0 \\ 0 & 1 & 0 \\ 0 & 0 & 1\end{bmatrix} \\
				=\,&\begin{bmatrix}
				(\cos\beta + 1)/2 & -\sin(\beta)/2^{1/2} & -(\cos\beta - 1)/2 \\
				\sin(\beta)/2^{1/2} & \cos\beta & -\sin(\beta)/2^{1/2} \\
				-(\cos\beta - 1)/2 & \sin(\beta)/2^{1/2} & (\cos\beta + 1)/2
				\end{bmatrix}
			\end{align*}
			and so
			\begin{align*}
				D^{(1)}[R(\alpha, \beta, \gamma)] &= \begin{bmatrix}e^{-i\alpha} & 0 & 0 \\ 0 & 0 & 0 \\ 0 & 0 & e^{i\alpha}\end{bmatrix}\begin{bmatrix}
					(\cos\beta + 1)/2 & -\sin(\beta)/2^{1/2} & -(\cos\beta - 1)/2 \\
					\sin(\beta)/2^{1/2} & \cos\beta & -\sin(\beta)/2^{1/2} \\
					-(\cos\beta - 1)/2 & \sin(\beta)/2^{1/2} & (\cos\beta + 1)/2
				\end{bmatrix}\begin{bmatrix}e^{-i\gamma} & 0 & 0 \\ 0 & 0 & 0 \\ 0 & 0 & e^{i\gamma}\end{bmatrix} \\
				&= \begin{bmatrix}e^{-i\alpha} & 0 & 0 \\ 0 & 0 & 0 \\ 0 & 0 & e^{i\alpha}\end{bmatrix}\begin{bmatrix}e^{-i\gamma}(\cos\beta+ 1)/2 & -\sin(\beta)/2^{1/2} & -e^{i\gamma}(\cos\beta - 1)/2 \\
				e^{-i\gamma}\sin(\beta)/2^{1/2} & \cos\beta & -e^{i\gamma}\sin(\beta)/2^{1/2} \\
			-e^{-i\gamma}(\cos\beta - 1)/2 & \sin(\beta)/2^{1/2} & e^{i\gamma}(\cos\beta + 1)/2\end{bmatrix} \\
				&= \begin{bmatrix}
					e^{-i(\alpha + \gamma)}(\cos\beta + 1)/2 & -e^{-i\alpha}\sin(\beta)/2^{1/2} & -e^{-i(\alpha - \gamma)}(\cos\beta - 1)/2 \\
					e^{-i\gamma}\sin(\beta)/2^{1/2} & \cos\beta & -e^{i\gamma}\sin(\beta)/2^{1/2} \\
					-e^{i(\alpha - \gamma)}(\cos\beta - 1)/2 & e^{i\alpha}\sin(\beta)/2^{1/2} & e^{i(\alpha + \gamma)}(\cos\beta + 1)/2
					\end{bmatrix} \\
				&= \begin{bmatrix}
					e^{-i(\alpha + \gamma)}\cos^2(\beta/2) & -e^{-i\alpha}\sin(\beta)/2^{1/2} & e^{-i(\alpha - \gamma)}\sin^2(\beta/2) \\
					e^{-i\gamma}\sin(\beta)/2^{1/2} & \cos\beta & -e^{i\gamma}\sin(\beta)/2^{1/2} \\
					e^{i(\alpha - \gamma)}\sin^2(\beta/2) & e^{i\alpha}\sin(\beta)/2^{1/2} & e^{i(\alpha + \gamma)}\cos^2(\beta/2)
				\end{bmatrix}
			\end{align*}
			To find $\langle \mathbf{J}\rangle$, we calculate $\langle 1, 1|D^{(1)}[R(\alpha, \beta, \gamma)]^\dagger J_i^{(1)}D^{(1)}[R(\alpha, \beta, \gamma)]|1, 1\rangle$ for $i = x, y, z$. The algebra is tedious and not particularly enlightening, so we use a CAS to find
			\begin{align*}
				\langle J_x^{(1)}\rangle &= \hbar\sin\beta\cos\alpha \\
				\langle J_y^{(1)}\rangle &= \hbar\sin\alpha\sin\beta \\
				\langle J_z^{(1)}\rangle &= \hbar\cos\beta
			\end{align*}
			i.e.
			$$\langle\mathbf{J}\rangle = \hbar(\sin\beta\cos\alpha\mathbf{i} + \sin\beta\sin\alpha\mathbf{j} + \cos\beta\mathbf{k})$$
			By examining
			$$D^{(1)}[R(\alpha, \beta, \gamma)]|1, 1\rangle = \begin{bmatrix}
			e^{-i(\alpha + \gamma)}\cos^2(\beta/2) \\
			e^{-i\gamma}\sin(\beta)/2^{1/2} \\
			e^{i(\alpha - \gamma)}\sin^2(\beta/2)
			\end{bmatrix}$$
			we find that the middle entry can never be $1$ for any choice of $\alpha$, $\beta$, $\gamma$, i.e. this state can never be $|1, 0\rangle$. In fact, when $\beta$ is chosen to be $\pi/2$ (so that the middle entry takes on its largest value), both the first and third entries are nonzero, meaning the closest we can get to $|1, 0\rangle$ is a linear combination of states containing a bit of $|1, 0\rangle$.
			
			Part 4 follows from the fact that any entry of $D^{(1)}[R(\alpha, \beta, \gamma)]|1, 1\rangle$ can be made nonzero through an appropriate choice of $\alpha, \beta, \gamma$, and thus we can always perform a rotation from $|1, m\rangle$ into a state that contains $|1, m'\rangle$. (Of course, the rotated state may contain more than just $|1, m'\rangle$, as we saw above.)
			
			From $D^{(1)}[R(\alpha, \beta, \gamma)]|1, 1\rangle$ (see above), it is immediately clear that $\beta = \pi$ takes us from $|1, 1\rangle$ to $|1, {-1}\rangle$, so we may occasionally have the ability to rotate one pure state to another.
		\end{solution}
		
		\question \textit{(Optional).} Verify that
		\begin{gather*}
			L_x \xrightarrow[\text{coordinate basis}]{} i\hbar\Big(\sin\phi\frac{\partial}{\partial\theta} + \cos\phi\cot\theta\frac{\partial}{\partial\phi}\Big) \\
			L_y \xrightarrow[\text{coordinate basis}]{} i\hbar\Big({-\cos\phi}\frac{\partial}{\partial\theta} + \sin\phi\cot\theta\frac{\partial}{\partial\phi}\Big) \\
		\end{gather*}
		\begin{solution}
			From basic geometry, we know that
			\begin{align*}
				x &= r\cos\phi\sin\theta \\
				y &= r\sin\phi\sin\theta \\
				z &= r\cos\theta
			\end{align*}
			from which we can get
			\begin{align*}
				r &= (x^2 + y^2 + z^2)^{1/2} \\
				\phi &= \tan^{-1}\big(\tfrac{y}{x}\big) \\
				\theta &= \tan^{-1}\Big(\tfrac{(x^2 + y^2)^{1/2}}{z}\Big)
			\end{align*}
			To make the coming process easier, we calculate the derivatives of the spherical coordinates with respect to each rectilinear coordinate, then substitute in our first three relations to obtain results entirely in spherical coordinates:
			\begin{align*}
				\frac{\partial r}{\partial x} &= \frac{1}{2}(x^2 + y^2 + z^2)^{-1/2}\cdot 2x \\
				&= x/r \\
				&= \cos\phi\sin\theta \\
				\frac{\partial r}{\partial y} &= \frac{1}{2}(x^2 + y^2 + z^2)^{-1/2}\cdot 2y \\
				&= y/r \\
				&= \sin\phi\sin\theta\\
				\frac{\partial r}{\partial z} &= \frac{1}{2}(x^2 + y^2 + z^2)^{-1/2}\cdot 2z\\
				&= z/r \\
				&= \cos\theta \\
				\frac{\partial \phi}{\partial x} &= \frac{1}{1 + \tfrac{y^2}{x^2}}\cdot{-\frac{y}{x^2}} \\
				&= {-\frac{y}{x^2 + y^2}} \\
				&= -\frac{r\sin\phi\sin\theta}{r^2\sin^2\theta(\cos^2\phi + \sin^2\phi)} \\
				&= {-\frac{\sin\phi}{r\sin\theta}} \\
				\frac{\partial \phi}{\partial y} &= \frac{1}{1 + \tfrac{y^2}{x^2}}\cdot\frac{1}{x} \\
				&= \frac{x}{x^2 + y^2} \\
				&= \frac{r\cos\phi\sin\theta}{r^2\sin^2\theta(\cos^2\phi + \sin^2\phi)} \\
				&= \frac{\cos\phi}{r\sin\theta} \\
				\frac{\partial\phi}{\partial z} &= 0 \\
				\frac{\partial\theta}{\partial x} &= \frac{1}{1 + \tfrac{x^2 + y^2}{z^2}}\cdot\frac{1}{2z}(x^2 + y^2)^{-1/2}\cdot 2x \\
				&= \frac{z}{r^2}\frac{x}{r\sin\theta} \\
				&= \frac{r\cos\theta}{r^2}\frac{r\cos\phi\sin\theta}{r\sin\theta} \\
				&= \frac{\cos\phi\cos\theta}{r} \\
				\frac{\partial \theta}{\partial y} &= \frac{1}{1 + \tfrac{x^2 + y^2}{z^2}}\cdot\frac{1}{2z}(x^2 + y^2)^{-1/2}\cdot 2y \\
				&= \frac{z}{r^2}\frac{y}{r\sin\theta} \\
				&= \frac{r\cos\theta}{r^2}\frac{r\sin\phi\sin\theta}{r\sin\theta} \\
				&= \frac{\sin\phi\cos\theta}{r} \\
				\frac{\partial \theta}{\partial z} &= \frac{1}{1 + \tfrac{x^2 + y^2}{z^2}}\cdot{-\frac{(x^2 + y^2)^{1/2}}{z^2}} \\
				&= {-\frac{r\sin\theta}{r^2}} \\
				&= {-\frac{\sin\theta}{r}}
			\end{align*}
			From here, it is a simple exercise to express the angular momentum operators in terms of the usual rectilinear coordinate basis, then change to spherical coordinates with the help of the above relations,
			\begin{align*}
			L_x &= YP_z - ZP_y \\
			&= -i\hbar y\frac{\partial}{\partial z} + i\hbar z\frac{\partial}{\partial y} \\
			&= i\hbar\Big[{-r\sin\phi}\sin\theta\Big(\frac{\partial r}{\partial z}\frac{\partial}{\partial r} + \frac{\partial \phi}{\partial z}\frac{\partial}{\partial \phi} + \frac{\partial \theta}{\partial z}\frac{\partial}{\partial\theta}\Big) + r\cos\theta\Big(\frac{\partial r}{\partial y}\frac{\partial}{\partial r} + \frac{\partial \phi}{\partial y}\frac{\partial}{\partial \phi} + \frac{\partial \theta}{\partial y}\frac{\partial}{\partial\theta}\Big)\Big] \\
			&= i\hbar\Big[{-r\sin\phi}\sin\theta\Big(\cos\theta\frac{\partial}{\partial r} - \frac{\sin\theta}{r}\frac{\partial}{\partial\theta}\Big) + r\cos\theta\Big(\sin\phi\sin\theta\frac{\partial}{\partial r} + \frac{\cos\phi}{r\sin\theta}\frac{\partial}{\partial \phi} + \frac{\sin\phi\cos\theta}{r}\frac{\partial}{\partial\theta}\Big)\Big] \\
			&= i\hbar\Big[\Big(r\cos\theta\frac{\cos\phi}{r\sin\theta}\Big)\frac{\partial}{\partial \phi} + \Big(r\sin\phi\sin\theta\frac{\sin\theta}{r} + r\cos\theta\frac{\sin\phi\cos\theta}{r}\Big)\frac{\partial}{\partial\theta}\Big] \\
			&= i\hbar\Big(\cos\phi\cot\theta\frac{\partial}{\partial\phi} + \sin\phi\frac{\partial}{\partial\theta}\Big) \\
			L_y &= ZP_x - XP_z \\
			&= -i\hbar z\frac{\partial}{\partial x} + i\hbar x\frac{\partial}{\partial z} \\
			&= i\hbar\Big[{-r\cos\theta}\Big(\frac{\partial r}{\partial x}\frac{\partial}{\partial r} + \frac{\partial \phi}{\partial x}\frac{\partial}{\partial \phi} + \frac{\partial \theta}{\partial x}\frac{\partial}{\partial\theta}\Big) + r\cos\phi\sin\theta\Big(\frac{\partial r}{\partial z}\frac{\partial}{\partial r} + \frac{\partial \phi}{\partial z}\frac{\partial}{\partial \phi} + \frac{\partial \theta}{\partial z}\frac{\partial}{\partial\theta}\Big)\Big] \\
			&= i\hbar\Big[{-r\cos\theta}\Big(\cos\phi\sin\theta\frac{\partial}{\partial r} - \frac{\sin\phi}{r\sin\theta}\frac{\partial}{\partial \phi} + \frac{\cos\phi\cos\theta}{r}\frac{\partial}{\partial \theta}\Big) + r\cos\phi\sin\theta\Big(\cos\theta\frac{\partial}{\partial r} - \frac{\sin\theta}{r}\frac{\partial}{\partial \theta}\Big)\Big] \\
			&= i\hbar\Big[\Big(r\cos\theta\frac{\sin\phi}{r\sin\theta}\Big)\frac{\partial}{\partial \phi} + \Big({-r\cos\theta}\frac{\cos\phi\cos\theta}{r} - r\cos\phi\sin\theta\frac{\sin\theta}{r}\Big)\frac{\partial}{\partial\theta}\Big] \\
			&= i\hbar\Big(\sin\phi\cot\theta\frac{\partial}{\partial\phi} - \cos\phi\frac{\partial}{\partial\theta}\Big)
			\end{align*}
		\end{solution}
	
		\question Show that $L^2$ is Hermitian in the sense
		$$\int \psi_1^*(L^2\psi_2)\mathrm{d}\Omega = \Big[\int\psi_2^*(L^2\psi_1)\mathrm{d}\Omega\Big]^*$$
		The same goes for $L_z$, which is insensitive to $\theta$ and is Hermitian with respect to the $\phi$ integration.
		
		\begin{solution}
			This is a simple math exercise. Starting with the left side, we have
			\begin{align*}
				&{-\hbar^2}\int\psi_1^*\Big(\frac{\partial^2}{\partial\theta^2} + \cot\theta\frac{\partial}{\partial \theta} + \frac{1}{\sin^2\theta}\frac{\partial^2}{\partial\phi^2}\Big)\psi_2\,\mathrm{d}\Omega \\
				=\,&{-\hbar^2}\int_0^\pi\int_0^{2\pi}\Big(\psi_1^*\frac{\partial^2\psi_2}{\partial\theta^2} + \psi_1^*\cot\theta\frac{\partial\psi_2}{\partial\theta} + \psi_1^*\frac{1}{\sin^2\theta}\frac{\partial^2\psi_2}{\partial\phi^2}\Big)\sin\theta\,\mathrm{d}\phi\,\mathrm{d}\theta \\
				=\,&{-\hbar^2}\int_0^\pi\int_0^{2\pi}\Big(\psi_1^*\sin\theta\frac{\partial^2\psi_2}{\partial\theta^2} + \psi_1^*\cos\theta\frac{\partial\psi_2}{\partial\theta} + \psi_1^*\frac{1}{\sin\theta}\frac{\partial^2\psi_2}{\partial\phi^2}\Big)\mathrm{d}\phi\,\mathrm{d}\theta \\
				=\,&{-\hbar^2}(\psi_1^*\psi_2\cos\theta)\big|_0^{\pi} - \hbar^2\int_0^{\pi}\int_0^{2\pi}\Big({-\frac{\partial}{\partial\theta}}(\psi_1^*\sin\theta)\frac{\partial\psi_2}{\partial\theta} - \frac{\partial}{\partial\theta}(\psi_1^*\cos\theta)\psi_2 + \frac{1}{\sin\theta}\psi_2\frac{\partial^2\psi_1^*}{\partial\phi^2}\Big)\mathrm{d}\phi\,\mathrm{d}\theta \\
				=\,&{-\hbar^2}(\psi_1^*\psi_2\cos\theta)\big|_0^{\pi} - \hbar^2\int_0^{\pi}\int_0^{2\pi}\Big({-\sin\theta}\frac{\partial\psi_1^*}{\partial\theta}\frac{\partial\psi_2}{\partial \theta} - \psi_1^*\cos\theta\frac{\partial \psi_2}{\partial\theta} - \psi_2\cos\theta\frac{\partial\psi_1^*}{\partial\theta} + \psi_1^*\psi_2\sin\theta + \frac{1}{\sin\theta}\psi_2\frac{\partial^2\psi_1^*}{\partial\phi^2}\Big)\mathrm{d}\phi\,\mathrm{d}\theta \\
				=\,&{-\hbar^2}\int_0^{\pi}\int_0^{2\pi}\Big(\frac{\partial}{\partial\theta}\Big[\sin\theta\frac{\partial\psi_1^*}{\partial\theta}\Big]\psi_2 + \frac{\partial}{\partial\theta}(\psi_1^*\cos\theta)\psi_2 + \psi_2\Big[{-\cos\theta}\frac{\partial\psi_1^*}{\partial\theta} + \psi_1^*\sin\theta + \frac{1}{\sin\theta}\frac{\partial^2\psi_1^*}{\partial\phi^2}\Big]\Big)\mathrm{d}\phi\mathrm{d}\,\theta \\
				=\,&{-\hbar^2}\int_0^{\pi}\int_0^{2\pi}\Big(\psi_2\Big[\cos\theta\frac{\partial\psi_1^*}{\partial\theta} + \sin\theta\frac{\partial^2\psi_1^*}{\partial\theta^2} + \cos\theta\frac{\partial\psi_1^*}{\partial\theta} - \psi_1^*\sin\theta - \cos\theta\frac{\partial\psi_1^*}{\partial\theta} + \psi_1^*\sin\theta + \frac{1}{\sin\theta}\frac{\partial^2\psi_1^*}{\partial\phi^2}\Big]\Big)\mathrm{d}\phi\,\mathrm{d}\theta \\
				=\,&{-\hbar^2}\int_0^{\pi}\int_0^{2\pi}\psi_2\Big(\frac{\partial^2\psi_1^*}{\partial\theta^2} + \cot\theta\frac{\partial\psi_1^*}{\partial\theta} + \frac{1}{\sin^2\theta}\frac{\partial^2\psi_1^*}{\partial\phi^2}\Big)\sin\theta\,\mathrm{d}\phi\,\mathrm{d}\theta \\
				=\,&{-\hbar^2}\int\psi_2\Big(\frac{\partial^2}{\partial\theta^2} + \cot\theta\frac{\partial}{\partial\theta} + \frac{1}{\sin^2\theta}\frac{\partial^2}{\partial\phi^2}\Big)\psi_1^*\,\mathrm{d}\Omega \\
				=\,&\Big[\int\psi_2^*(L^2\psi_1)\mathrm{d}\Omega\Big]^*
			\end{align*}
			where we have made generous use of integration-by-parts, taking advantage of the fact that $\sin\theta$ vanishes at $\theta = 0$ and $\theta = \pi$.
		\end{solution}
	
		\question Write the differential equation corresponding to
		$$L^2|\alpha\beta\rangle = \alpha|\alpha\beta\rangle$$
		in the coordinate basis, using the $L^2$ operator given in Eq. (12.5.36). We already know $\beta=m\hbar$ from the analysis of $-i\hbar(\partial/\partial\phi)$. So assume that the simultaneous eigenfunctions have the form
		$$\psi_{\alpha m}(\theta, \phi) = P_{\alpha}^m(\theta)e^{im\phi}$$
		and show that $P_{\alpha}^m$ satisfies the equation
		$$\Big(\frac{1}{\sin\theta}\frac{\partial}{\partial\theta}\sin\theta\frac{\partial}{\partial\theta} + \frac{\alpha}{\hbar^2} - \frac{m^2}{\sin^2\theta}\Big)P_{\alpha}^m(\theta) = 0$$
		We need to show that
		\begin{align*}
			&(1)\quad\frac{\alpha}{\hbar^2} = l(l + 1), \qquad l = 0, 1, 2, \dots \\
			&(2)\quad|m| \leq l
		\end{align*}
		We will consider only part (1) and that too for the case $m = 0$. By rewriting the equation in terms of $u=\cos\theta$, show that $P_{\alpha}^0$ satisfies
		$$(1 - u^2)\frac{\mathrm{d}^2P_{\alpha}^0}{\mathrm{d}u^2} - 2u\frac{\mathrm{d}P_{\alpha}^0}{\mathrm{d}u} + \Big(\frac{\alpha}{\hbar^2}\Big)P_{\alpha}^0 = 0$$
		Convince yourself that a power series solution
		$$P_{\alpha}^0 = \sum_{n=0}^{\infty}C_nu^n$$
		will lead to a two-term recursion relation. Show that $(C_{n+2}/C_n)\to 1$ as $n \to \infty$. Thus the series diverges when $|u|\to 1$ ($\theta\to 0$ or $\pi$). Show that if $\alpha/\hbar^2 = (l)(l + 1)$; $l = 0, 1, 2, \dots$, the series will terminate and be either an even or odd function of $u$. The functions $P_{\alpha}^0(u) = P_{l(l+1)\hbar^2}^0(u) \equiv P_{l}^0(u) = P_l(u)$ are just the Legendre polynomials up to a scale factor. Determine $P_0$, $P_1$, and $P_2$ and compare (ignoring overall scales) with the $Y_l^0$ functions.
		
		\begin{solution}
			Transforming to the coordinate basis and using the ansatz $\psi_{\alpha m}(\theta, \phi) = P_{\alpha}^m(\theta)e^{im\phi}$, we have
			$$\langle x|L^2|x\rangle\langle x|\alpha\beta\rangle = \alpha\langle x|\alpha \beta\rangle \to -\hbar^2\Big(\frac{1}{\sin\theta}\frac{\partial}{\partial\theta}\sin\theta\frac{\partial}{\partial\theta} + \frac{1}{\sin^2\theta}\frac{\partial^2}{\partial\phi^2}\Big)P_\alpha^m(\theta)e^{im\phi} = \alpha P_\alpha^m(\theta)e^{im\phi}$$
			Taking $\partial/\partial\phi^2$ of $e^{im\phi}$ and then dividing by the latter gives
			$$-\hbar^2\Big(\frac{1}{\sin\theta}\frac{\partial}{\partial\theta}\sin\theta\frac{\partial}{\partial\theta} - \frac{m^2}{\sin^2\theta}\Big)P_\alpha^m(\theta) = \alpha P_\alpha^m(\theta)$$
			Collecting terms, dividing by $\hbar^2$, and multiplying by ${-1}$, we have
			$$\Big(\frac{1}{\sin\theta}\frac{\partial}{\partial\theta}\sin\theta\frac{\partial}{\partial\theta} + \frac{\alpha}{\hbar^2} - \frac{m^2}{\sin^2\theta}\Big)P_\alpha^m(\theta) = 0$$
			Setting $m = 0$ and using $u = \cos\theta$ (and so $\partial/\partial\theta = -\sin\theta\cdot\partial/\partial u$), we find
			\begin{align*}
				\Big(\frac{1}{\sin\theta}(-\sin\theta)\frac{\partial}{\partial u} (-\sin^2\theta)\frac{\partial}{\partial u} + \frac{\alpha}{\hbar^2}\Big)P_\alpha^0 &= \Big({-\frac{\partial}{\partial u}}(u^2 - 1)\frac{\partial}{\partial u} + \frac{\alpha}{\hbar^2}\Big)P_{\alpha}^0 \\
				&= -\frac{\partial}{\partial u}\Big((u^2 - 1)\frac{\mathrm{d}P_\alpha^0}{\mathrm{d}u}\Big) + \Big(\frac{\alpha}{\hbar^2}\Big)P_\alpha^0 \\
				&= -(u^2 - 1)\frac{\mathrm{d}^2P_\alpha^0}{\mathrm{d}u^2} - 2u\frac{\mathrm{d}P_\alpha^0}{\mathrm{d}u} + \Big(\frac{\alpha}{\hbar^2}\Big)P_\alpha^0 \\
				&= (1 - u^2)\frac{\mathrm{d}^2P_\alpha^0}{\mathrm{d}u^2} - 2u\frac{\mathrm{d}P_\alpha^0}{\mathrm{d}u} + \Big(\frac{\alpha}{\hbar^2}\Big)P_\alpha^0 \\
				&= 0
			\end{align*}
			where we have used $\cos^2\theta + \sin^2\theta = 1$. Assuming a power series solution exists, we find
			\begin{align*}
				&(1 - u^2)\frac{\mathrm{d}^2}{\mathrm{d}u^2}\sum_{n=0}^\infty C_n u^n - 2u\frac{\mathrm{d}}{\mathrm{d}u}\sum_{n=0}^\infty C_n u^n + \Big(\frac{\alpha}{\hbar^2}\Big)\sum_{n=0}^\infty C_n u^n \\
				=\,&\sum_{n=0}^{\infty}(1 - u^2)\cdot C_n n(n-1) u^{n-2} - \sum_{n=0}^{\infty}2u\cdot C_nnu^{n-1} + \sum_{n=0}^{\infty}\Big(\frac{\alpha}{\hbar^2}\Big)C_n u^n \\
				=\,&\sum_{n=0}^{\infty}C_nn(n-1)u^{n-2} + \sum_{n=0}^{\infty}C_n\Big[\frac{\alpha}{\hbar^2} - 2n - n(n-1)\Big]u^n \\
				=\,&\sum_{m={-2}}^{\infty}C_{m + 2}(m + 2)(m + 1)u^m + \sum_{n=0}^{\infty}C_n\Big[\frac{\alpha}{\hbar^2} - 2n - n^2 + n\Big]u^n \\
				=\,&\sum_{n=0}^{\infty}\Big(C_{n+2}(n + 2)(n + 1) + C_n\big(\frac{\alpha}{\hbar^2} - n(n + 1)\big)\Big)u^n \\
				=\,&0
			\end{align*}
			which implies we must have
			$$\frac{C_{n+2}}{C_n} = \frac{n(n + 1) - \alpha/\hbar^2}{(n + 2)(n + 1)}$$
			Since the leading term in both the numerator and denominator is $n^2$, $\lim_{n\to\infty} C_{n+2}/C_n = 1$, implying the solution does not exist for $\theta = \pi k$, $k \in \mathbb{Z}$.
			
			If we choose $\alpha = \hbar^2l(l + 1)$, the power series is guaranteed to terminate after a finite set of terms (specifically, whenever $n = l$), even for $\theta = \pi k$. We can return to the differential equation in terms of $u$ to get $C_0$ and $C_1$. When $l = 0$, $\alpha = 0$ and we get
			$$(1 - u^2)\frac{\mathrm{d}^2}{\mathrm{d}u^2}C_0 - 2u\frac{\mathrm{d}}{\mathrm{d}u}C_0 = 0$$
			which is trivially true. Disregarding normalization, we choose $C_0 = 1$, or $P_0(u) = 1$. For $l = 1$, we have
			\begin{align*}
				(1 - u^2)\frac{\mathrm{d}^2}{\mathrm{d}u^2}(C_0 + C_1u) - 2u\frac{\mathrm{d}}{\mathrm{d}u}(C_0 + C_1u) + 2(C_0 + C_1u) &=-2C_1u + 2C_0 + 2C_1u \\
				&=2C_0 \\
				&=0
			\end{align*}
			Clearly, $C_0 = 0$. For simplicity we choose $C_1 = 1$, setting $P_1(u) = u$. For $l = 2$, we will need to set $C_0$, $C_1$, and $C_2$. We can relate $C_2$ to $C_0$ with our recurrence relation, taking the latter to be $-1$ arbitrarily. We must choose $C_1 = 0$, since our recurrence relation would not terminate otherwise. (In fact, all of our solutions must be either purely even or odd for this reason.) Putting these observations together, we have
			$$C_2(0 + 2)(0 + 1) = [0(0 + 1) - 2(2 + 1)]C_0$$
			or $C_2 = -3C_0 = 3$. Altogether, the first three Legendre polynomials for $m = 0$ are
			\begin{align*}
				P_0(u) &= 1 \\
				P_1(u) &= u \\
				P_2(u) &= 3u^2 - 1
			\end{align*}
			If we scale these and make the replacement $u = \cos\theta$, we obtain the first three spherical harmonics with $m = 0$.
		\end{solution}
	
		\question Derive $Y_1^1$ starting from Eq. (12.5.28) and normalize it yourself. [Remember the $(-1)^l$ factor from Eq. (12.5.32).] Lower it to get $Y_1^0$ and $Y_1^{-1}$ and compare it with Eq. (12.5.39).
		
		\begin{solution}
			Starting at (12.5.28) would result in a mere repeat of the next few equations, so we jump to the result of
			$$\psi_1^1(r, \theta, \phi) \propto R(r)\sin\theta \,e^{i\phi}$$
			where we can immediately identify $Y_1^1 = A(\sin\theta)e^{i\phi}$. We can find the normalization factor $A$ via
			\begin{align*}
				\int |Y_1^1|^2\mathrm{d}\Omega &= |A|^2\int_0^\pi\int_0^{2\pi}\sin^2\theta\cdot\sin\theta\,\mathrm{d}\phi\,\mathrm{d}\theta \\
				&= 2\pi|A|^2\int_0^{\pi}\sin^3\theta\,\mathrm{d}\theta \\
				&= 2\pi|A|^2\int_{1}^{-1}(u^2 - 1)\mathrm{d}u \\
				&= 2\pi|A|^2\Big(\frac{u^3}{3} - u\Big)\Big|_1^{-1} \\
				&= 2\pi|A|^2\Big({-\frac{1}{3}} + 1 - \frac{1}{3} + 1\Big) \\
				&= \frac{8\pi}{3}|A|^2 \\
				&= 1
			\end{align*}
			so
			$$A = e^{i\alpha}\Big(\frac{3}{8\pi}\Big)^{1/2}$$
			(In the above, we made the substitution $u = \cos\theta$ and used $\cos^2\theta + \sin^2\theta = 1$.) To be consistent with Shankar's convention, we take $\alpha = \pi$, resulting in
			$$Y_1^1(\theta, \phi) = {-\Big(\frac{3}{8\pi}\Big)}^{1/2}\sin\theta\,e^{i\phi}$$
			The other two spherical harmonics for $l = 1$ are
			\begin{align*}
				Y_1^0 &= \frac{L_-Y_1^1}{\hbar[(1 + 1)(1 - 1 + 1)]^{1/2}} \\
				&= {-\frac{\hbar e^{-i\phi}}{2^{1/2}\hbar }} \Big(\frac{\partial}{\partial\theta} - i\cot\theta\frac{\partial}{\partial\phi}\Big)\Big[{-\Big(\frac{3}{8\pi}\Big)}^{1/2}\sin\theta\, e^{i\phi}\Big] \\
				&=  e^{-i\phi}\Big(\frac{3}{16\pi}\Big)^{1/2}\Big(\cos\theta\,e^{i\phi} + \cos\theta\,e^{i\phi}\Big) \\
				&= \Big(\frac{3}{4\pi}\Big)^{1/2}\cos\theta \\
				Y_1^{-1} &= \frac{L_-Y_1^0}{\hbar[(1 + 0)(1 - 0 + 1)]^{1/2}} \\
				&= {-\frac{\hbar e^{-i\phi}}{2^{1/2}\hbar }} \Big(\frac{\partial}{\partial\theta} - i\cot\theta\frac{\partial}{\partial\phi}\Big)\Big[\Big(\frac{3}{4\pi}\Big)^{1/2}\cos\theta\Big] \\
				&= {-e^{-i\phi}}\Big(\frac{3}{8\pi}\Big)^{1/2}\Big({-\sin\theta}\Big) \\
				&= \Big(\frac{3}{8\pi}\Big)^{1/2}\sin\theta\,e^{-i\phi}
			\end{align*}
			In both of these calculations, we have removed the factor of $C_-(l, m)$ introduced by the application of $L_-$. Both results agree with (12.5.39).
		\end{solution}
		
		\question Since $L^2$ and $L_z$ commute with $\Pi$, they should share a basis with it. Verify that under parity $Y_l^m \to (-1)^l Y_l^m$. (First show that $\theta \to \pi - \theta$, $\phi \to \phi + \pi$ under parity. Prove the result for $Y_l^l$. Verify that $L_-$ does not alter the parity, thereby proving the result for all $Y_l^m$.)
		
		\begin{solution}
			Since
			\begin{align*}
				r &= (x^2 + y^2 + z^2)^{1/2} \\
				\phi &= \tan^{-1}\Big(\frac{y}{x}\Big) \\
				\theta &= \tan^{-1}\Big(\frac{(x^2 + y^2)^{1/2}}{z}\Big)
			\end{align*}
			and $\Pi$ takes $x\to{-x}$, $y\to{-y}$, and $z\to{-z}$, $\Pi$ makes the following changes in spherical coordinates
			\begin{align*}
				r &\to r \\
				\phi &\to \phi + \pi \\
				\theta &\to \pi - \theta
			\end{align*}
			The transformation of $\phi$ can be understood as a shifting of quadrants ($y \to {-y}$, $x \to {-x}$), while the transformation of $\theta$ can be understood as a reflection about the $x$-$y$ plane. Given this, 
			\begin{align*}
				\Pi\,Y_l^l(\theta, \phi) &= \Pi\,(-1)^l\Big[\frac{(2l + 1)!}{4\pi}\Big]^{1/2}\frac{1}{2^ll!}(\sin\theta)^l e^{il\phi} \\
				&= (-1)^l\Big[\frac{(2l + 1)!}{4\pi}\Big]^{1/2}\frac{1}{2^ll!}\big(\sin(\pi - \theta)\big)^l e^{il(\phi + \pi)} \\
				&= e^{il\pi}(-1)^l\Big[\frac{(2l + 1)!}{4\pi}\Big]^{1/2}\frac{1}{2^ll!}(\sin\theta)^l e^{il\phi} \\
				&= (-1)^lY_l^l(\theta, \phi)
			\end{align*} 
			Now, in the coordinate basis,
			\begin{align*}
				[\Pi, L_-]|l, l\rangle \to\,&\Pi\Big[{-\hbar e^{-i\phi}}\Big(\frac{\partial}{\partial\theta} - i\cot\theta\frac{\partial}{\partial\phi}\Big)\Big]Y_l^l(\theta, \phi) - \Big[{-\hbar e^{-i\phi}}\Big(\frac{\partial}{\partial\theta} - i\cot\theta\frac{\partial}{\partial\phi}\Big)\Big]\Pi Y_l^l(\theta, \phi) \\
				=\,&\Pi\Big[{-\hbar e^{-i\phi}}\Big(\frac{\partial Y_l^l(\theta, \phi)}{\partial \theta} - i\cot\theta\frac{\partial Y_l^l(\theta, \phi)}{\partial \phi}\Big)\Big] + \Big[{\hbar e^{-i\phi}}\Big(\frac{\partial}{\partial\theta} - i\cot\theta\frac{\partial}{\partial\phi}\Big)\Big]Y_l^l(\pi - \theta, \phi + \pi) \\
				=\,&\Big[{-\hbar e^{-i(\phi  + \pi)}}\Big({\frac{\partial Y_l^l(\pi - \theta, \phi + \pi)}{\partial \theta}} - i\cot(\pi -\theta)\frac{\partial Y_l^l(\pi - \theta, \phi + \pi)}{\partial \phi}\Big)\Big] \\
				&+ \Big[{\hbar e^{-i\phi}}\Big({-\frac{\partial Y_l^l(\pi - \theta, \phi + \pi)}{\partial\theta}} - i\cot\theta\frac{\partial Y_l^l(\pi - \theta, \phi + \pi)}{\partial\phi}\Big)\Big] \\
				=\,&\Big[{\hbar e^{-i\phi}}\Big({\frac{\partial Y_l^l(\pi - \theta, \phi + \pi)}{\partial \theta}} + i\cot(\theta)\frac{\partial Y_l^l(\pi - \theta, \phi + \pi)}{\partial \phi}\Big)\Big] \\
				&+ \Big[{\hbar e^{-i\phi}}\Big({-\frac{\partial Y_l^l(\pi - \theta, \phi + \pi)}{\partial\theta}} - i\cot\theta\frac{\partial Y_l^l(\pi - \theta, \phi + \pi)}{\partial\phi}\Big)\Big] \\
				=\,&0
			\end{align*}
			which implies $L_-$ does not change the parity of a state. Therefore, $\Pi\,Y_l^m \to (-1)^l Y_l^m$.
		\end{solution}
		
		\question Consider a particle in a state described by
		$$\psi = N(x + y + 2z)e^{-\alpha r}$$
		where $N$ is a normalization factor.
		
		(1) Show, by rewriting $Y_1^{\pm 1, 0}$ functions in terms of $x$, $y$, $z$, and $r$, that
		\begin{gather*}
			Y_1^{\pm 1} = \mp\Big(\frac{3}{4\pi}\Big)^{1/2}\frac{x\pm iy}{2^{1/2}r} \\
			Y_1^0 = \Big(\frac{3}{4\pi}\Big)^{1/2}\frac{z}{r}
		\end{gather*}
	
		(2) Using this result, show that for a particle described by $\psi$ above, $P(l_z = 0) = 2/3$; $P(l_z = +\hbar) = 1/6 = P(l_z=-\hbar)$.
		
		\begin{solution}
			Consulting 
			\begin{align*}
				r &= (x^2 + y^2 + z^2)^{1/2} \\
				\phi &= \tan^{-1}\Big(\frac{y}{x}\Big) \\
				\theta &= \tan^{-1}\Big(\frac{(x^2 + y^2)^{1/2}}{z}\Big)
			\end{align*}
			we find that
			\begin{align*}
				\sin\theta e^{\pm i\phi} &= \sin\theta\cos\phi \pm i\sin\theta\sin\phi \\
				&= \frac{(x^2 + y^2)^{1/2}}{r}\frac{x}{(x^2 + y^2)^{1/2}} \pm i\frac{(x^2 + y^2)^{1/2}}{r}\frac{y}{(x^2 + y^2)^{1/2}} \\
				&= \frac{x \pm iy}{r} \\
				\cos\theta &= \frac{z}{r}
			\end{align*}
			which allows us to immediately write
			\begin{align*}
				Y_1^{1} &= {-\Big(\frac{3}{8\pi}\Big)^{1/2}}\frac{x+ iy}{r} \\
				Y_1^0 &= \Big(\frac{3}{4\pi}\Big)^{1/2}\frac{z}{r} \\
				Y_1^{-1} &= \Big(\frac{3}{8\pi}\Big)^{1/2}\frac{x- iy}{r} \\
			\end{align*}
			Now, by inspection, 
			\begin{align*}
				x &= r\Big(\frac{2\pi}{3}\Big)^{1/2}(Y_1^{-1} - Y_1^1) \\
				y &= ir\Big(\frac{2\pi}{3}\Big)^{1/2}(Y_1^{-1} + Y_1^1) \\
				z &= r\Big(\frac{4\pi}{3}\Big)^{1/2}Y_1^0
			\end{align*}
			and so our state can be rewritten as
			\begin{align*}
				\psi &= N(x + y + 2z)e^{-\alpha r} \\
				&= N\Big(\frac{2\pi}{3}\Big)^{1/2}\Big[(Y_1^{-1} - Y_1^1) + i(Y_1^{-1} + Y_1^1) + 2\cdot2^{1/2}Y_1^0\Big]re^{-\alpha r} \\
				&= N\Big(\frac{2\pi}{3}\Big)^{1/2}\Big[(i - 1)Y_1^1 + 8^{1/2}Y_1^0 + (i + 1)Y_1^{-1}\Big]
			\end{align*}
			The squared magnitude of the $C_1^1$ and $C_1^{-1}$ coefficients is $2$, while the squared magnitude of $C_1^0$ is $8$. As the state is completely described by the weighted sum of these three spherical harmonics, we can determine the probability of measuring the different values of $l_z$ by looking at their ratios. That is,
			\begin{align*}
				P(l_z = 1) &= \frac{2}{2 + 8 + 2} = \frac{1}{6} \\
				P(l_z = 0) &= \frac{8}{2 + 8 + 2} = \frac{2}{3} \\
				P(l_z = {-1}) &= \frac{2}{2 + 8 + 2} = \frac{1}{6}
			\end{align*}
		\end{solution}
		
		\question Consider a rotation $\theta_x\mathbf{i}$. Under this
		\begin{align*}
			x &\to x \\
			y &\to y\cos\theta_x - z\sin\theta_x \\
			z &\to z\cos\theta_x + y\sin\theta_x
		\end{align*}
		Therefore we must have
		$$\psi(x, y, z) \xrightarrow[U(R(\theta_x\mathbf{i}))]{}\psi_R = \psi(x, y\cos\theta_x + z\sin\theta_x, z\cos\theta_x - y\sin\theta_x)$$
		Let us verify this prediction for a special case
		$$\psi = Aze^{-r^2/a^2}$$
		which must go into
		$$\psi_R = A(z \cos\theta_x - y\sin\theta_x)e^{-r^2/a^2}$$
		
		(1) Expand $\psi$ in terms of $Y_1^1$, $Y_1^0$, and $Y_1^{-1}$.
		
		(2) Use the matrix $e^{-i\theta_x L_x/\hbar}$ to find the fate of $\psi$ under this rotation. (See Exercise 12.5.5) Check your result against that anticipated above. [Hint: (1) $\psi \sim Y_1^0$, which corresponds to
		$$\begin{bmatrix}0 \\ 1 \\ 0\end{bmatrix}$$
		
		(3) Use Eq. (12.5.42).]
		
		\begin{solution}
			From the previous exercise, 
			$$\psi = rA\Big(\frac{4\pi}{3}\Big)^{1/2}Y_1^0e^{-r^2/a^2}$$
			which, in the basis chosen in Exercise 12.5.5, can be written as
			$$\psi = A\Big(\frac{4\pi}{3}\Big)^{1/2}re^{-r^2/a^2}\begin{bmatrix}0 \\ 1 \\ 0\end{bmatrix}$$
			Collecting the factors constant or dependent on $r$ under $B$ (as these should not change under rotation), we can use our results from Exercise 12.5.6 to write
			\begin{align*}
				D^{(1)}[R(\theta_x\mathbf{i})]\psi &= B\begin{bmatrix}
					(\cos\theta_x + 1)/2 & -i\sin(\theta_x)/2^{1/2} & (\cos\theta_x - 1)/2 \\
					-i\sin(\theta_x)/2^{1/2} & \cos\theta_x & -i\sin(\theta_x)/2^{1/2} \\
					(\cos\theta_x - 1)/2 & -i\sin(\theta_x)/2^{1/2} & (\cos\theta_x+ 1)/2
				\end{bmatrix}\begin{bmatrix}0 \\ 1 \\0\end{bmatrix} \\
				&= B\begin{bmatrix}
					-i\sin(\theta_x)/2^{1/2} \\ \cos\theta_x \\ -i\sin(\theta_x)/2^{1/2}
				\end{bmatrix} \\
				&\to -iB\frac{\sin\theta_x}{2^{1/2}}Y_1^1 + B\cos\theta_x Y_1^0 - iB\frac{\sin\theta_x}{2^{1/2}}Y_1^{-1} \\
				&= -iB\frac{\sin\theta_x}{2^{1/2}}\Big[{-\Big(\frac{3}{8\pi}\Big)^{1/2}}\frac{x+ iy}{r}\Big] + B\cos\theta_x\Big[\Big(\frac{3}{4\pi}\Big)^{1/2}\frac{z}{r}\Big] - iB\frac{\sin\theta_x}{2^{1/2}}\Big[\Big(\frac{3}{8\pi}\Big)^{1/2}\frac{x- iy}{r}\Big] \\
				&= \frac{B}{r}\Big(\frac{3}{4\pi}\Big)^{1/2}\Big[x\Big(\frac{i\sin\theta_x}{2} - \frac{i\sin\theta_x}{2}\Big) + y\Big({-\frac{\sin\theta_x}{2}} - \frac{\sin\theta_x}{2}\Big) + z\cos\theta_x\Big] \\
				&= A\Big(\frac{4\pi}{3}\Big)^{1/2}\frac{re^{-r^2/a^2}}{r}\Big(\frac{3}{4\pi}\Big)^{1/2}(z\cos\theta_x - y\sin\theta_x) \\
				&= A(z\cos\theta_x - y\sin\theta_x)e^{-r^2/a^2}
			\end{align*}
			which is exactly what we expected.
		\end{solution}
	
		\setcounter{subsection}{5}
		\setcounter{question}{0}
		\subsection{Solution of Rotationally Invariant Problems}
		
		\question A particle is described by the wave function
		$$\psi_E(r, \theta, \phi) = Ae^{-r/a_0}\qquad (a_0=\mathrm{const})$$
		(1) What is the angular momentum content of the state?
		
		(2) Assuming $\psi_E$ is an eigenstate in a potential that vanishes as $r\to\infty$, find $E$. (Match leading terms in Schr\"odinger's equation.)
		
		(3) Having found $E$, consider finite $r$ and find $V(r)$.
		
		\begin{solution}
			This state has zero angular momentum, being entirely described by a function of $r$. Feeding this into (12.6.3), we find
			\begin{align*}
				&\Big\{{-\frac{\hbar^2}{2\mu}}\Big[\frac{1}{r^2}\frac{\partial}{\partial r}r^2\frac{\partial}{\partial r} - \frac{l(l + 1)}{r^2}\Big] + V(r)\Big\}Ae^{-r/a_0} \\
				=\,&{-\frac{\hbar^2}{2\mu}}\frac{1}{r^2}\frac{\partial}{\partial r}\Big({-\frac{A}{a_0}r^2e^{-r/a_0}}\Big) + \frac{\hbar^2}{2\mu}\frac{l(l + 1)}{r^2}Ae^{-r/a_0} + AV(r)e^{-r/a_0} \\
				=\,&\frac{A}{a_0}\frac{\hbar^2}{2\mu}\frac{1}{r^2}\Big(2re^{-r/a_0} - \frac{r^2}{a_0}e^{-r/a_0}\Big) + A\frac{\hbar^2}{2\mu}\frac{l(l + 1)}{r^2}e^{-r/a_0} + AV(r)e^{-r/a_0} \\
				=\,&\frac{A}{a_0}\frac{\hbar^2}{\mu}\frac{1}{r}e^{-r/a_0} - \frac{A}{a_0^2}\frac{\hbar^2}{2\mu}e^{-r/a_0} + A\frac{\hbar^2}{2\mu}\frac{l(l + 1)}{r^2}e^{-r/a_0} + AV(r)e^{-r/a_0} \\
				=\,&\Big(\frac{\hbar^2}{2\mu}\frac{l(l + 1)}{r^2} + \frac{1}{a_0}\frac{\hbar^2}{\mu}\frac{1}{r} - \frac{1}{a_0^2}\frac{\hbar^2}{2\mu} + V(r)\Big)Ae^{-r/a_0} = EAe^{-r/a_0}
			\end{align*}
			As the state has zero angular momentum, $l = 0$. We can combine this with our assumption that $\lim_{r\to\infty}V(r) = 0$ and remove common factors to write
			\begin{align*}
				\lim_{r\to\infty}\Big(\frac{\hbar^2}{a_0 \mu r} - \frac{\hbar^2}{2 a_0^2\mu}\Big) = E
			\end{align*}
			or
			$$E = -\frac{\hbar^2}{2a_0^2\mu}$$
			Returning to the non-limiting form of the equation, we can isolate $V(r)$ to get
			$$V(r) = E + \frac{\hbar^2}{2a_0^2\mu} - \frac{\hbar^2}{a_0\mu r} = {-\frac{\hbar^2}{a_0\mu r}}$$
			i.e. we are dealing with a Coulomb-like potential.
		\end{solution}
	
		\question Provide the steps connecting Eq. (12.6.3) and Eq. (12.6.5).
		\begin{solution}
			Making the suggested substitution $R_{El} = U_{El}/r$, we find
			\begin{align*}
				&\Big\{{-\frac{\hbar^2}{2\mu}}\Big[\frac{\mathrm{d}^2}{\mathrm{d} r^2} + \frac{2}{r}\frac{\mathrm{d}}{\mathrm{d} r} - \frac{l(l + 1)}{r^2}\Big] + V(r)\Big\}\frac{U_{El}}{r} \\
				=\,&{-\frac{\hbar^2}{2\mu}}\frac{\mathrm{d}}{\mathrm{d}r}\Big(\frac{U_{El}'}{r} - \frac{U_{El}}{r^2}\Big) - \frac{\hbar^2}{2\mu}\frac{2}{r}\Big(\frac{U_{El}'}{r} - \frac{U_{El}}{r^2}\Big) + \frac{\hbar^2}{2\mu}\frac{l(l + 1)}{r^3}U_{El} + V(r)\frac{U_{El}}{r} \\
				=\,&{-\frac{\hbar^2}{2\mu}}\Big(\frac{U_{El}''}{r} - \frac{U_{El}'}{r^2}\Big) + \frac{\hbar^2}{2\mu}\Big(\frac{U_{El}'}{r^2} - \frac{2U_{El}}{r^3}\Big) - \frac{\hbar^2}{2\mu}\frac{2}{r}\Big(\frac{U_{El}'}{r} - \frac{U_{El}}{r^2}\Big) + \frac{\hbar^2}{2\mu}\frac{l(l + 1)}{r^3}U_{El} + V(r)\frac{U_{El}}{r} \\
				=\,&\Big({-\frac{\hbar^2}{2\mu r}}\Big)U''_{El} + \Big(\frac{\hbar^2}{2\mu r^2} + \frac{\hbar^2}{2\mu r^2} - \frac{\hbar^2}{\mu r^2}\Big)U_{El}' + \Big({-\frac{\hbar^2}{\mu r^3}} + \frac{\hbar^2}{\mu r^3} + \frac{\hbar^2}{2\mu}\frac{l(l + 1)}{r^3} + \frac{V(r)}{r}\Big)U_{El} \\
				=\,&{-\frac{\hbar^2}{2\mu r}}U''_{El} + \Big(\frac{V(r)}{r} + \frac{\hbar^2}{2\mu}\frac{l(l + 1)}{r^3}\Big)U_{El} = E\frac{U_{El}}{r}
			\end{align*}
			Collecting all terms on one side and multiplying by $-2\mu r/\hbar^2$ gives
			$$\Big\{\frac{\mathrm{d}^2}{\mathrm{d}r^2} + \frac{2\mu}{\hbar^2}\Big[E - V(r) - \frac{l(l + 1)\hbar^2}{2\mu r^2}\Big]\Big\}U_{El} = 0$$
		\end{solution}
	
		\question Show that Eq. (12.6.7b) follows from Eq. (12.6.7a).
		\begin{solution}
			Starting with the left side,
			\begin{align*}
				&\int_0^\infty U_1^*(D_lU_2)\,\mathrm{d}r \\
				=\,&\int_0^\infty U_1^*\Big({-\frac{\hbar^2}{2\mu}}\frac{\mathrm{d}^2}{\mathrm{d}r^2} + V(r) + \frac{l(l + 1)\hbar^2}{2\mu r^2}\Big)U_2\,\mathrm{d}r \\
				=\,&{-\frac{\hbar^2}{2\mu}}\int_0^\infty U_1^*\frac{\mathrm{d}^2U_2}{\mathrm{d}r^2}\,\mathrm{d}r + \int_0^\infty\Big(V(r) + \frac{l(l + 1)\hbar^2}{2\mu r^2}\Big)U_1^* U_2\,\mathrm{d}r \\
				=\,&\Big({-U_1^*}\frac{\mathrm{d}U_2}{\mathrm{d}r}\Big)\Big|_0^{\infty} + \frac{\hbar^2}{2\mu}\int_0^\infty\frac{\mathrm{d}U_1^*}{\mathrm{d}r}\frac{\mathrm{d}U_2}{\mathrm{d}r}\,\mathrm{d}r + \int_0^\infty\Big(V(r) + \frac{l(l + 1)\hbar^2}{2\mu r^2}\Big)U_1^* U_2\,\mathrm{d}r \\
				=\,&\Big({-U_1^*}\frac{\mathrm{d}U_2}{\mathrm{d}r} + U_2\frac{\mathrm{d}U_1^*}{\mathrm{d}r}\Big)\Big|_0^{\infty} - \frac{\hbar^2}{2\mu}\int_0^\infty\frac{\mathrm{d}^2U_1^*}{\mathrm{d}r^2}U_2\,\mathrm{d}r  + \int_0^\infty\Big(V(r) + \frac{l(l + 1)\hbar^2}{2\mu r^2}\Big)U_1^* U_2\,\mathrm{d}r \\
				=\,&\Big({-U_1^*}\frac{\mathrm{d}U_2}{\mathrm{d}r} + U_2\frac{\mathrm{d}U_1^*}{\mathrm{d}r}\Big)\Big|_0^{\infty} + \int_0^\infty\Big[\Big({-\frac{\hbar^2}{2\mu}}\frac{\mathrm{d}^2}{\mathrm{d}r^2} + V(r) + \frac{l(l + 1)\hbar^2}{2\mu r^2}\Big)U_1\Big]^*U_2\,\mathrm{d}r \\
				=\,&\Big({-U_1^*}\frac{\mathrm{d}U_2}{\mathrm{d}r} + U_2\frac{\mathrm{d}U_1^*}{\mathrm{d}r}\Big)\Big|_0^{\infty} + \int_0^\infty (D_lU_1)^*U_2\,\mathrm{d}r
			\end{align*}
			This equals the right side only when
			$$\Big(U_1^*\frac{\mathrm{d}U_2}{\mathrm{d}r} - U_2\frac{\mathrm{d}U_1^*}{\mathrm{d}r}\Big)\Big|_0^{\infty} = 0$$
		\end{solution}
	
		\question (1) Show that
		$$\delta^3(\mathbf{r}- \mathbf{r}') \equiv \delta(x - x')\delta(y - y')\delta(z - z') = \frac{1}{r^2\sin\theta}\delta(r - r')\delta(\theta - \theta')\delta(\phi - \phi')$$
		(consider a test function).
		
		(2) Show that
		$$\nabla^2(1/r) = -4\pi\delta^3(\mathbf{r})$$
		(Hint: First show that $\nabla^2(1/r) = 0$ if $r \neq 0$. To see what happens at $r =0$, consider a small sphere centered at the origin and use Gauss's law and the identity $\nabla^2\phi = \nabla\cdot\nabla\phi$).)
		
		\begin{solution}
			A defining property of the Dirac delta function is that
			$$\int f(\mathbf{r}')\delta^3(\mathbf{r} - \mathbf{r}')\,\mathrm{d}\mathbf{r}' = f(\mathbf{r})$$
			Applying this to a function expressed in spherical coordinates gives
			\begin{align*}
				&\int_0^\infty\int_0^\pi\int_0^{2\pi}f(r', \theta', \phi')A(r, \theta, \phi)\delta(r - r')\delta(\theta - \theta')\delta(\phi - \phi')r'^2\sin\theta'\,\mathrm{d}r'\,\mathrm{d}\theta'\,\mathrm{d}\phi' \\
				=\,&A(r, \theta, \phi)r^2\sin\theta f(r, \theta, \phi)
			\end{align*}
			where $A(r, \theta, \phi)$ is a presumed correction factor to the delta function. Clearly, for the defining property of the Dirac delta function to hold, 
			$$A(r, \theta, \phi) = (r^2\sin\theta)^{-1}$$
			Moving on to part (2), the gradient of $1/r$ is (in rectilinear coordinates)
			\begin{align*}
				\nabla\Big(\frac{1}{r}\Big) &= \nabla(x^2 + y^2 + z^2)^{-1/2} \\
				&= -\frac{1}{2}(x^2 + y^2 + z^2)^{-3/2}\begin{bmatrix}
					2x \\ 2y \\ 2z
				\end{bmatrix} \\
				&= -\frac{1}{r^3}\begin{bmatrix}
					r\cos\phi\sin\theta \\
					r\sin\phi\sin\theta \\
					r\cos\theta
				\end{bmatrix} \\
				&= -\frac{1}{r^2}\begin{bmatrix}\cos\phi\sin\theta \\ \sin\phi\sin\theta \\ \cos\theta\end{bmatrix}
			\end{align*}
			Now, the derivative with respect to $x$ of the $x$-component of this vector is
			\begin{align*}
				\frac{\partial^2}{\partial x^2}\Big(\frac{1}{r}\Big) &= \frac{3}{4}(x^2 + y^2 + z^2)^{-5/2}2x\cdot 2x - \frac{1}{2}(x^2 + y^2 + z^2)^{-3/2}2 \\
				&= \frac{3x^2}{r^5} - \frac{1}{r^3} \\
				&= \frac{1}{r^5}(3x^2 - r^2)
			\end{align*}
			so, by symmetry,
			\begin{align*}
				\nabla^2\Big(\frac{1}{r}\Big) &= \frac{1}{r^5}(3x^2 - r^2 + 3y^2 - r^2 + 3z^2 - r^2) \\
				&= \frac{1}{r^5}(3r^2 - 3r^2)
			\end{align*}
			Clearly, at nonzero $r$, $\nabla^2(1/r) = 0$. However, at $r = 0$, the limiting behavior of the above function suggests a different result. To investigate this, we integrate $\nabla^2(1/r)$ on a ball of radius $r_0$ centered at the origin
			\begin{align*}
				&\int_0^r \int_0^{2\pi}\int_0^{\pi}\nabla^2\Big(\frac{1}{r}\Big)\,r^2\sin\theta\,\mathrm{d}r\,\mathrm{d}p\,\mathrm{d}\phi \\
				=\,&\int_0^{2\pi}\int_0^{\pi}\begin{bmatrix}\cos\phi\sin\theta \\ \sin\phi\sin\theta \\ \cos\theta\end{bmatrix}\cdot\nabla\Big(\frac{1}{r}\Big)\Big|_{r=r_0}r_0^2\sin\theta\,\mathrm{d}\theta\,\mathrm{d}\phi \\
				=\,&-\int_0^{2\pi}\int_0^{\pi}\begin{bmatrix}\cos\phi\sin\theta \\ \sin\phi\sin\theta \\ \cos\theta\end{bmatrix}\cdot\begin{bmatrix}\cos\phi\sin\theta \\ \sin\phi\sin\theta \\ \cos\theta\end{bmatrix}\sin\theta\,\mathrm{d}\theta\,\mathrm{d}\phi \\
				=\,&-\int_0^{2\pi}\int_0^{\pi}\sin\theta\,\mathrm{d}\theta\,\mathrm{d}\phi \\
				=\,&{-2\pi}\int_0^\pi\sin\theta\,\mathrm{d}\theta \\
				=\,&{-4\pi}
			\end{align*}
			A function that integrates to a finite value at the origin yet returns $0$ everywhere else is proportional to $\delta^3(\mathbf{r})$, so we must have
			$$\nabla^2(1/r) = -4\pi\delta^3(\mathbf{r})$$
		\end{solution}
	
		\question Show that $D_l$ is nondegenerate in the space of functions $U$ that vanish as $r\to 0$. (Recall the proof of Theorem 15, Section 5.6.) Note that $U_{El}$ is nondegenerate even for $E>0$. This means that $E$, $l$, and $m$, label a state fully in three dimensions.
		
		\begin{solution}
			Consider two functions $U_1$, $U_2$ that give the energy $E$ when acted upon by $D_l$, i.e.
			\begin{align*}
				D_lU_1 &= \Big({-\frac{\hbar^2}{2\mu}}\frac{\mathrm{d}^2}{\mathrm{d}r^2} + V(r) + \frac{l(l + 1)\hbar^2}{2\mu r^2}\Big)U_1 = EU_1 \\
				D_lU_2 &= \Big({-\frac{\hbar^2}{2\mu}}\frac{\mathrm{d}^2}{\mathrm{d}r^2} + V(r) + \frac{l(l + 1)\hbar^2}{2\mu r^2}\Big)U_2 = EU_2
			\end{align*}
			Left-multiplying the first equation by $U_2$ and the second by $U_1$, then subtracting, gives
			$$U_2D_lU_1 - U_1D_lU_2 = {-\frac{\hbar^2}{2\mu}}\Big(U_2\frac{\mathrm{d}^2U_1}{\mathrm{d}r^2} - U_1\frac{\mathrm{d}^2U_2}{\mathrm{d}r^2}\Big) = 0$$
			which can be rewritten as
			$$\frac{\mathrm{d}}{\mathrm{d}r}\Big(U_2\frac{\mathrm{d}U_1}{\mathrm{d}r} - U_1\frac{\mathrm{d}U_2}{\mathrm{d}r}\Big) = 0$$
			This is satisfied when
			$$U_2\frac{\mathrm{d}U_1}{\mathrm{d}r} - U_1\frac{\mathrm{d}U_2}{\mathrm{d}r} = c$$
			for $c$ constant. Since this must hold for all $r$ and $\lim_{r\to\infty}U_i(r) = 0$, we immediately have $c = 0$. This implies
			$$\frac{1}{U_1}\frac{\mathrm{d}U_1}{\mathrm{d}r} = \frac{1}{U_2}\frac{\mathrm{d}U_2}{\mathrm{d}r}$$
			or, after integrating,
			$$\ln U_1 = \ln U_2 + c$$
			i.e.
			$$U_1 = e^c U_2$$
			But if $U_1$ is simply a scaled version of $U_2$, they describe the same physical state. Therefore $D_l$ is nondegenerate.
		\end{solution}
	
		\question (1) Verify that Eqs. (12.6.21) and (12.6.22) are equivalent to Eq. (12.6.20).
		
		(2) Verify Eq. (12.6.24).
		
		\begin{solution}
			Starting with Eq. (12.6.22), 
			\begin{align*}
				(d_ld_l^\dagger)U_l &= \Big(\frac{\mathrm{d}}{\mathrm{d}\rho} + \frac{l + 1}{\rho}\Big)\Big({-\frac{\mathrm{d}}{\mathrm{d}\rho}} + \frac{l + 1}{\rho}\Big)U_l \\
				&= \Big(\frac{\mathrm{d}}{\mathrm{d}\rho} + \frac{l + 1}{\rho}\Big)\Big({-\frac{\mathrm{d}U_l}{\mathrm{d}\rho}} + \frac{l + 1}{\rho}U_l\Big) \\
				&= {-\frac{\mathrm{d}^2U_l}{\mathrm{d}\rho^2}} - \frac{l + 1}{\rho^2}U_l + \frac{l + 1}{\rho}\frac{\mathrm{d}U_l}{\mathrm{d}\rho} - \frac{l + 1}{\rho}\frac{\mathrm{d}U_l}{\mathrm{d}\rho} + \frac{(l + 1)^2}{\rho^2}U_l \\
				&= {-\frac{\mathrm{d}^2U_l}{\mathrm{d}\rho^2}} + \frac{l^2 + 2l + 1 - l - 1}{\rho^2}U_l \\
				&= {-\frac{\mathrm{d}^2U_l}{\mathrm{d}\rho^2}} + \frac{l(l + 1)}{\rho^2}U_l \\
				&= \Big({-\frac{\mathrm{d}^2}{\mathrm{d}\rho^2}} + \frac{l(l + 1)}{\rho^2}\Big)U_l
			\end{align*}
			which is exactly Eq. (12.6.20). For Eq. (12.6.24), we start with the left side and act on a test function $U_l$,
			\begin{align*}
				d_l^\dagger d_lU_l &= \Big({-\frac{\mathrm{d}}{\mathrm{d}\rho}} + \frac{l + 1}{\rho}\Big)\Big(\frac{\mathrm{d}}{\mathrm{d}\rho} + \frac{l + 1}{\rho}\Big)U_l \\
				&= \Big({-\frac{\mathrm{d}}{\mathrm{d}\rho}} + \frac{l + 1}{\rho}\Big)\Big(\frac{\mathrm{d}U_l}{\mathrm{d}\rho} + \frac{l + 1}{\rho}U_l\Big) \\
				&= {-\frac{\mathrm{d}^2U_l}{\mathrm{d}\rho^2}} + \frac{l + 1}{\rho^2}U_l - \frac{l + 1}{\rho}\frac{\mathrm{d}U_l}{\mathrm{d}\rho} + \frac{l + 1}{\rho}\frac{\mathrm{d}U_l}{\mathrm{d}\rho} + \frac{(l + 1)^2}{\rho^2}U_l \\
				&= {-\frac{\mathrm{d}^2U_l}{\mathrm{d}\rho^2}} + \frac{l^2 + 2l + 1 + l + 1}{\rho^2}U_l \\
				&= {-\frac{\mathrm{d}^2U_l}{\mathrm{d}\rho^2}} + \frac{(l + 1)(l + 2)}{\rho^2}U_l \\
				&= \Big({-\frac{\mathrm{d}^2}{\mathrm{d}\rho^2}} + \frac{(l + 1)(l + 2)}{\rho^2}\Big)U_l
			\end{align*}
			This is exactly $d_{l+1}d_{l+1}^\dagger$, as can be seen by returning to our first result and making the replacement $l\to l + 1$.
		\end{solution}
		
		\question Verify that $j_0$ and $j_1$ have the limits given in Eq (12.6.33).
		
		\begin{solution}
			Rewriting the trigonometric functions in $j_0$ and $j_1$ as Taylor series shows
			\begin{align*}
				j_0 &= \frac{\sin\rho}{\rho} \\
				&= 1 - \frac{\rho^2}{3!} + \frac{\rho^4}{5!} - \cdots \\
				j_1 &= \frac{\sin\rho}{\rho^2} - \frac{\cos\rho}{\rho} \\
				&= 	\Big(\frac{1}{\rho} - \frac{\rho}{3!} + \frac{\rho^3}{5!} - \cdots\Big) - \Big(\frac{1}{\rho} - \frac{\rho}{2!} + \frac{\rho^3}{4!} - \cdots\Big) \\
				&= \Big(\frac{1}{2!} - \frac{1}{3!}\Big)\rho - \Big(\frac{1}{4!} - \frac{1}{5!}\Big)\rho^3 + \cdots \\
				&= \frac{\rho}{3} - \frac{\rho^2}{30} + \cdots
			\end{align*}
			which implies
			\begin{align*}
				\lim_{\rho\to 0}j_0 &= 1 \\
				\lim_{\rho\to 0}j_1 &= \frac{\rho}{3}
			\end{align*}
			which is consistent with (12.6.33). More generally, since
			\begin{align*}
				j_l &= (-\rho)^l\Big(\frac{1}{\rho}\frac{\mathrm{d}}{\mathrm{d}\rho}\Big)^l\Big(\frac{\sin\rho}{\rho}\Big) \\
				&= (-\rho)^l\Big(\frac{1}{\rho}\frac{\mathrm{d}}{\mathrm{d}\rho}\Big)^l\sum_{n=0}^\infty\frac{(-1)^{n}\rho^{2n}}{(2n + 1)!}
			\end{align*}
			each application of $(1/\rho)\mathrm{d}/\mathrm{d}\rho$ to the series expansion of $\sin(\rho)/\rho$ reduces the exponent by $2$, introduces the original exponent as a multiplicative factor, and increments the start of $n$ by $1$, i.e.
			$$\Big(\frac{1}{\rho}\frac{\mathrm{d}}{\mathrm{d}\rho}\Big)\sum_{n=0}^\infty\frac{(-1)^{n}\rho^{2n}}{(2n + 1)!} = \sum_{n=1}^{\infty}\frac{(-1)^n\rho^{2(n - 1)}}{(2n+1)(2n-1)!}$$
			The dominating term as $\rho\to0$ is the first term in the above summation, which is
			$$\frac{(-1)^l}{(2l + 1)!!}$$
			Multiplying this by $(-\rho)^l$ gives the limiting expression presented in the book,
			$$\lim_{\rho\to0}j_l(\rho) = \frac{\rho^l}{(2l + 1)!!}$$
		\end{solution}
		
		\question Find the energy levels of a particle in a spherical box of radius $r_0$ in the $l=0$ sector.
		
		\begin{solution}
			Since the interior of the box has no potential, the $l = 0$ solution will by the free-particle solution of rotationally invariant Hamiltonians,
			$$\psi_{E00}(r, \theta, \phi) = j_0(kr)Y_0^0(\theta, \phi) = \frac{1}{(4\pi)^{1/2}}\frac{\sin kr}{kr}$$
			Since the particle is in a box of radius $r_0$, we must have that
			$$\psi_{E00}(r) = \frac{1}{(r\pi)^{1/2}}\frac{\sin kr_0}{kr_0} = 0$$
			Excepting the zero at $r\to\infty$, this is satisfied when
			$$kr_0 = n\pi, \quad n\in \mathbb{Z}$$ 
			Expressing $k$ in terms of $E$ gives us the quantization condition,
			$$\Big(\frac{2\mu E}{\hbar^2}\Big)^{1/2}r_0 = n\pi$$
			or 
			$$E = \frac{\hbar^2n^2\pi^2}{2\mu r_0^2}$$
		\end{solution}
		
		\question Show that the quantization condition for $l=0$ bound states in a spherical well of depth ${-V_0}$ and radius $r_0$ is
		$$k'/\kappa = {-\tan(k'r_0)}$$
		where $k'$ is the wave number inside the well and $i\kappa$ is the complex wave number for the exponential tail outside. Show that there are no bound states for $V_0 < \pi^2\hbar^2/8\mu r_0^2$. (Recall Exercise 5.2.6.)
		
		\begin{solution}
			With $l = 0$, the wave function will be completely classified by its radial component $R_{E0}(r) = U_{E0}/r$, which can be found by solving
			$$\Big({-\frac{\hbar^2}{2\mu}}\frac{\mathrm{d}^2}{\mathrm{d}r^2} + V(r) - E\Big)U_{E0} = 0$$
			Within the spherical well this equation is
			$$\frac{\mathrm{d}^2U_{E0}}{\mathrm{d}r^2} = -\frac{2\mu}{\hbar^2}(V_0 - |E|)U_{E0}$$
			while outside it is
			$$\frac{\mathrm{d}^2U_{E0}}{\mathrm{d}r^2} = \frac{2\mu}{\hbar^2}|E|U_{E0}$$
			In both regions, we have made the replacement $E \to -|E|$ to make clear that we are investigating a bound state, i.e. one for which $E<0$. Clearly, $-V_0 < E < 0$ for a bound state to exist in this scenario, and so $V_0 - |E|> 0$. The general solutions to each region are then
			$$U_{E0}(r) = \begin{cases}
				A\cos(k'r) + B\sin(k'r), \quad &r< r_0 \\
				Ce^{\kappa r} + De^{-\kappa r}, &r > r_0
			\end{cases}$$
			where $k' = (2\mu[V_0-|E|]/\hbar^2)^{1/2}$ and $\kappa = (2\mu|E|/\hbar^2)^{1/2}$. For a physical solution that is well behaved as $r \to 0$ and $r\to\infty$, $A = C = 0$. Matching both solutions at the boundary (and their first derivatives) gives
			\begin{align*}
				B\sin(k'r_0) &= De^{-\kappa r_0} \\
				k'B\cos(k'r_0) &= -\kappa De^{-\kappa r_0}
			\end{align*}
			Dividing the first equation by the second gives the sought after normalization condition,
			$$\frac{k'}{\kappa} = {-\tan(k'r_0)}$$
			We can show the second part by noting that
			$$k^2 + \kappa^2 = \frac{2\mu V_0}{\hbar^2},$$
			treating $\kappa$ as our dependent variable, and looking for the intersection of the two plots given by
			\begin{align*}
				\kappa &= -k'\cot(k'r_0) \\
				\kappa &= \Big(\frac{2\mu V_0}{\hbar^2} - k^2\Big)^{1/2}
			\end{align*}
			The first possible intersection occurs at the first zero of $-k'\cot(k'r_0)$, i.e.
			$$k'r_0 = \Big(\frac{2\mu(V_0 - |E|)}{\hbar^2}\Big)^{1/2}r_0 = \frac{\pi}{2}$$
			or
			$$V_0 = \frac{\pi^2\hbar^2}{8\mu r_0^2} + |E|$$
			Since all non-trivial solutions require $|E| > 0$, we see that no interesting bound-state solutions can exist for
			$$V_0 < \frac{\pi^2\hbar^2}{8\mu r_0^2}$$
		\end{solution}
	
		\question \textit{(Optional).} Verify Eq. (12.6.41) given that
		\begin{align*}
			&\text{(1)}\quad\int_{-1}^1P_l(\cos\theta)P_{l'}(\cos\theta)\mathrm{d}(\cos\theta) = [2/(2l + 1)]\delta_{ll'} \\
			&\text{(2)}\quad P_l(x) = \frac{1}{2^ll!}\frac{\mathrm{d}^l(x^2 - 1)^l}{\mathrm{d}x^l} \\
			&\text{(3)}\quad\int_0^1(1 - x^2)^m\mathrm{d}x = \frac{(2m)!!}{(2m + 1)!!}
		\end{align*}
		Hint: Consider the limit $kr \to 0$ after projecting out $C_l$.
		
		\begin{solution}
			We follow the hint, but instead consider the limit as $kr \to \infty$. Beginning on the right side, we have
			\begin{align*}
				\lim_{kr\to\infty}\int_{-1}^1\sum_{l'=0}^{\infty}C_{l'}j_{l'}(kr)P_{l'}(\cos\theta)P_l(\cos\theta)\,\mathrm{d}(\cos\theta) &= \lim_{kr\to\infty}\sum_{l'=0}^{\infty}C_{l'}j_{l'}(kr)\frac{2}{2l' + 1}\delta_{ll'} \\
				&= \lim_{kr\to\infty}\frac{2}{2l + 1}C_lj_l(kr) \\
				&= C_l\frac{2}{2l + 1}\frac{1}{kr}\sin\Big(kr - \frac{l\pi}{2}\Big)
			\end{align*}
			On the left, we have
			\begin{align*}
				&\lim_{kr\to\infty}\int_{-1}^1P_l(u)e^{ikru}\,\mathrm{d}u \\
				=\,&\lim_{kr\to\infty}\frac{1}{ikr}(P_l(u)e^{ikru})\Big|_{-1}^1 - \frac{1}{ikr}\int_{-1}^1P'_l(u)e^{ikru}\,\mathrm{d}u \\
				=\,&\lim_{kr\to\infty}\frac{1}{ikr}(e^{ikr} - (-1)^le^{-ikr}) - \frac{1}{ikr}\Big[\frac{1}{ikr}(P_l'(u)e^{ikru})\Big|_{-1}^1 - \frac{1}{ikr}\int_{-1}^1P''_l(u)e^{ikru}\mathrm{d}u\Big] \\
				=\,&\lim_{kr\to\infty}\frac{1}{ikr}(e^{ikr} - e^{il\pi}e^{-ikr}) + \mathcal{O}\big((kr)^{-2}\big) \\
				=\,&\lim_{kr\to\infty}\frac{e^{il\pi/2}}{kr}\frac{e^{i(kr - l\pi/2)} - e^{-i(kr - l\pi/2)}}{i} + \mathcal{O}\big((kr)^{-2}\big) \\
				=\,&\frac{2i^l}{kr}\sin\Big(kr - \frac{l\pi}{2}\Big)
			\end{align*}
			where we have made the replacement $u = \cos\theta$ and used the fact that $|P_l(1)| = |P_l(-1)| = 1$. Equating our findings gives
			$$C_l\frac{2}{2l + 1}\frac{1}{kr}\sin\Big(kr - \frac{l\pi}{2}\Big) = \frac{2i^l}{kr}\sin\Big(kr - \frac{l\pi}{2}\Big)$$
			or
			$$C_l = i^l(2l + 1)$$
		\end{solution}
	
		\question (1) By combining Eqs. (12.6.48) and (12.6.49) derive the two-term recursion relation. Argue that $C_0 \neq 0$ if $U$ is to have the right properties near $y=0$. Derive the quantization condition, Eq. (12.6.50).
		
		(2) Calculate the degeneracy and parity at each $n$ and compare with Exercise 10.2.3, where the problem was solved in Cartesian coordinates.
		
		(3) Construct the normalized eigenfunction $\psi_{nlm}$ for $n = 0$ and $1$. Write them as linear combinations of the $n=0$ and $n = 1$ eigenfunctions obtained in Cartesian coordinates.
		
		\begin{solution}
			Using
			$$v(y) = \sum_{m=0}^{\infty}C_my^{m + l + 1}$$
			we find
			\begin{align*}
				&v'' - 2yv' + \Big[2\lambda - 1 - \frac{l(l + 1)}{y^2}\Big]v \\
				=\,&\sum_{m=0}^\infty C_m(m + l + 1)(m + l)y^{m + l - 1} - 2y\sum_{m=0}^{\infty}C_m(m + l + 1)y^{m + l} + \Big[2\lambda - 1 - \frac{l(l + 1)}{y^2}\Big]\sum_{m=0}^\infty C_my^{m + l + 1} \\
				=\,&\sum_{m=0}^{\infty}C_m(m + l + 1)(m + l)y^{m + l - 1} - 2C_m(m + l + 1)y^{m + l + 1} + C_m(2\lambda - 1)y^{m + l + 1} - C_m l(l+1)y^{m + l - 1} \\
				=\,&\sum_{m=0}^\infty C_m\big((m + l + 1)(m + l) - (l + 1)\big)y^{m + l - 1} + \sum_{m=0}^\infty C_m\big((2\lambda - 1) - 2(m + l + 1)\big)y^{m + l + 1}
			\end{align*}
			which implies the recurrence relation
			$$\frac{C_{m + 2}}{C_m} = \frac{2(m + l + 1) - (2\lambda - 1)}{(m + l + 1)(m + l) - (l + 1)} = \frac{2(m + l + \tfrac{3}{2} - \lambda)}{m^2 + (2l + 1)m + l^2 - 1}$$
			Since $C_0 \neq 0$ for $U \propto y^{l}$ near $0$, the recurrence relation must terminate on an even value of $m$. Writing $m = 2k$ and $n = 2k + l$ gives the quantization condition
			$$\lambda = 2k + l + \frac{3}{2} = n + \frac{3}{2}$$
			To find the degeneracy, let us concentrate on even values of $n$. For such an energy, there are $n/2 + 1$ states with $m = 0$. For $m \neq 0$, there are
			$$\sum_{l' = 0}^{n/2}\sum_{m=0}^{2l'}2$$
			such states, where each $l'$ corresponds to a value of $l = 2l'$. Altogether, the number of states for a given even $n$ is
			\begin{align*}
				\frac{n}{2} + 1 + \sum_{l'=0}^{n/2}\sum_{m=0}^{2l'}2 &= \frac{n + 2}{2} + \sum_{l'=0}^{n/2}4l' \\
				&= \frac{n + 2}{2} + 4\frac{(n/2)(n/2 + 1)}{2} \\
				&= \frac{n + 2}{2} + \frac{n(n + 2)}{2} \\
				&= \frac{n + 2 + n^2 + 2n}{2} \\
				&= \frac{n^2 + 3n + 2}{2} \\
				&= \frac{(n + 1)(n + 2)}{2}
			\end{align*}
			which is exactly what we found in Exercise 10.2.3. For parity, it is obvious that $U(r)/r$ is even when $n$ is even, with $U(r)/r$ odd otherwise.
			
			The normalized eigenfunctions for $n = 0$ and $1$ are
			\begin{align*}
				\psi_{0, 0, 0} &= \frac{C_0}{2}\Big(\frac{\mu\omega}{\pi\hbar}\Big)^{1/2}e^{-\mu\omega r^2/2\hbar} \\
				\psi_{1, 1, -1} &= \frac{C_0}{2}\Big(\frac{3}{2\pi}\Big)^{1/2}\Big(\frac{\mu\omega}{\hbar}\Big)r e^{-\mu\omega r^2/2\hbar}e^{-i\phi}\sin\theta \\
				\psi_{1, 1, 0} &= \frac{C_0}{2}\Big(\frac{3}{\pi}\Big)^{1/2}\Big(\frac{\mu\omega}{\hbar}\Big)r e^{-\mu\omega r^2/2\hbar}\cos\theta \\
				\psi_{1, 1, 1} &= {-\frac{C_0}{2}}\Big(\frac{3}{2\pi}\Big)^{1/2}\Big(\frac{\mu\omega}{\hbar}\Big)r e^{-\mu\omega r^2/2\hbar}e^{i\phi}\sin\theta
			\end{align*}
			Writing the above functions as linear combinations of the $n = 0$ and $1$ eigenfunctions in Cartesian coordinates is trivial and amounts to rewriting the above in terms of Cartesian coordinates. Because of this, we skip this part of the problem.
		\end{solution}
	\end{questions}
\end{document}