\documentclass[../principles-of-quantum-mechanics.tex]{subfiles}

\begin{document}
	\printanswers
	
	\section{Spin}
	
	\begin{questions}
		\setcounter{subsection}{2}
		\setcounter{question}{0}
		\subsection{Kinematics of Spin}
		\question Let us verify the above corollary explicitly. Take some spinor with components $\alpha = \rho_1e^{i\phi_1}$ and $\beta = \rho_2e^{i\phi_2}$. From $\langle\chi|\chi\rangle = 1$, deduce that we can write $\rho_1=\cos(\theta/2)$ and $\rho_2=\sin(\theta/2)$ for some $\theta$. Next pull out a common phase factor so that the spinor takes the form in Eq. (14.3.28a). This verifies the corollary and also fixes $\hat{n}$.
		
		\begin{solution}
			From the normalization condition, we have
			$$(\rho_1e^{i\phi_1})(\rho_1^*e^{-i\phi_1}) + (\rho_2e^{i\phi_2})(\rho_2^*e^{-i\phi_2}) = \rho_1^2 + \rho_2^2 = 1$$
			which will always be true if we choose
			\begin{align*}
				\rho_1 &= \cos(\phi) \\
				\rho_2 &= \sin(\phi)
			\end{align*}
			Of course, we can just as easily rename $\phi \to \theta/2$ to obtain the specific relation sought after. If we then factor out the common phase factor $e^{i(\phi_1 + \phi_2)/2}$ we get
			$$|\chi\rangle = e^{i(\phi_1 + \phi_2)/2}\begin{bmatrix}
				\cos(\theta/2)e^{-i(\phi_2 - \phi_1)/2} \\
				\sin(\theta/2)e^{i(\phi_2 - \phi_1)/2}
			\end{bmatrix}$$
			Throwing out the common phase factor and labeling $\phi = \phi_2 - \phi_1$ gives
			$$|\chi\rangle = \begin{bmatrix}
				\cos(\theta/2)e^{-i\phi/2} \\
				\sin(\theta/2)e^{i\phi/2}
			\end{bmatrix}$$
		\end{solution}
	
		\question (1) Show that the eigenvectors of $\boldsymbol{\sigma}\cdot\hat{n}$ are given by Eq. (14.3.28). (2) Verify Eq. (14.3.29).
		
		\begin{solution}
			Following (14.3.26) and (14.3.27), it is obvious that we can write
			$$\boldsymbol{\sigma}\cdot\hat{n} = \begin{bmatrix}
				\cos\theta & \sin\theta e^{-i\phi} \\
				\sin\theta e^{i\phi} & -\cos\theta
			\end{bmatrix}$$
			The eigenvector equations for the two eigenvalues are given by
			$$\begin{bmatrix}
				\cos\theta \mp 1 & \sin\theta e^{-i\phi} \\
				\sin\theta e^{i\phi} & -\cos\theta \mp 1
			\end{bmatrix}|\hat{n}_\pm\rangle = 0$$
			By subtracting $\sin\theta e^{i\phi}/(\cos\theta \mp 1)$ times the first row from the second row, this system becomes
			\begin{align*}
				\begin{bmatrix}
					\cos\theta \mp 1 & \sin\theta e^{-i\phi} \\
					 0 & -\cos\theta \mp 1 - \sin\theta e^{-i\phi}\frac{\sin\theta e^{i\phi}}{\cos\theta \mp 1}
				\end{bmatrix}|\hat{n}_\pm\rangle &= \begin{bmatrix}
				\cos\theta \mp 1 & \sin\theta e^{-i\phi} \\
				0 & \frac{(-\cos\theta \mp 1)(\cos\theta \mp 1) - \sin^2\theta}{\cos\theta \mp 1}
			\end{bmatrix}|\hat{n}_\pm\rangle \\
			&= \begin{bmatrix}
				\cos\theta \mp 1 & \sin\theta e^{-i\phi} \\
				0 & \frac{-\cos^2\theta \pm \cos\theta \mp \cos\theta + 1 - \sin^2\theta}{\cos\theta \mp 1}
			\end{bmatrix}|\hat{n}_\pm\rangle \\
			&= \begin{bmatrix}
				\cos\theta \mp 1 & \sin\theta e^{-i\phi} \\
				0 & 0
			\end{bmatrix}|\hat{n}_\pm\rangle = 0
			\end{align*}
			which has the (non-normalized) solution
			$$|\hat{n}_\pm\rangle = e^{i\alpha}\begin{bmatrix}
				\sin\theta e^{-i\phi} \\
				-(\cos\theta \mp 1)
			\end{bmatrix}$$
			where $\alpha$ is an arbitrary phase factor. Given that
			$$\langle \hat{n}_\pm|\hat{n}_\pm\rangle = \sin^2\theta + \cos^2\theta \mp 2\cos\theta + 1 = 2 \mp 2\cos\theta$$
			we can write the normalized eigenkets as
			$$|\hat{n}_\pm\rangle = \frac{e^{i\alpha}}{(2 \mp 2\cos\theta)^{1/2}}\begin{bmatrix}
				\sin\theta e^{-i\phi} \\
				-(\cos\theta \mp 1)
			\end{bmatrix}$$
			Now, we know that
			\begin{align*}
				\cos\theta + 1 &= 2\cos^2(\theta/2) \\
				\cos\theta - 1 &= -2\sin^2(\theta/2) \\
				\sin\theta &= 2\sin(\theta/2)\cos(\theta/2)
			\end{align*}
			and so we may further simplify our eigenkets to
			\begin{align*}
				|\hat{n}_+\rangle &= \frac{e^{i\alpha}}{2\sin(\theta/2)}\begin{bmatrix}
					2\sin(\theta/2)\cos(\theta/2)e^{-i\phi} \\
					2\sin^2(\theta/2)
				\end{bmatrix} = e^{i\alpha}\begin{bmatrix}
				\cos(\theta/2)e^{-i\phi} \\
				\sin(\theta/2)
			\end{bmatrix} \\
			|\hat{n}_-\rangle &= \frac{e^{i\alpha}}{2\cos(\theta/2)}\begin{bmatrix}
				2\sin(\theta/2)\cos(\theta/2)e^{-i\phi} \\
				-2\cos^2(\theta/2)
			\end{bmatrix} = e^{i\alpha}\begin{bmatrix}
				\sin(\theta/2)e^{-i\phi} \\
				-\cos(\theta/2)
			\end{bmatrix}
			\end{align*}
			Choosing $\alpha = \phi/2$ and multiplying $|\hat{n}_-\rangle$ by ${-1}$ gives us our final answer,
			\begin{align*}
				|\hat{n}_+\rangle &= \begin{bmatrix}
					\cos(\theta/2)e^{-i\phi/2} \\
					\sin(\theta/2)e^{i\phi/2}
					\end{bmatrix} \\
				|\hat{n}_-\rangle &= \begin{bmatrix}
					-\sin(\theta/2)e^{-i\phi/2} \\
					\cos(\theta/2)e^{i\phi/2}
				\end{bmatrix}
			\end{align*}
			Now, let us examine the expectation values of the spin operators on a generic $|\hat{n}_+\rangle$ state,
			\begin{align*}
				\langle \hat{n}_+|S_x|\hat{n}_+\rangle &= \frac{\hbar}{2}\begin{bmatrix}
					\cos(\theta/2)e^{i\phi/2} &
					\sin(\theta/2)e^{-i\phi/2}
				\end{bmatrix}\begin{bmatrix}
					0 & 1 \\ 1 & 0
				\end{bmatrix}\begin{bmatrix}
					\cos(\theta/2)e^{-i\phi/2} \\
					\sin(\theta/2)e^{i\phi/2}
				\end{bmatrix} \\
				&= \frac{\hbar}{2}\begin{bmatrix}
					\cos(\theta/2)e^{i\phi/2} &
					\sin(\theta/2)e^{-i\phi/2}
				\end{bmatrix}\begin{bmatrix}
					\sin(\theta/2)e^{i\phi/2} \\
					\cos(\theta/2)e^{-i\phi/2}
				\end{bmatrix} \\
				&= \frac{\hbar}{2}\Big(\cos(\theta/2)\sin(\theta/2)e^{i\phi} + \sin(\theta/2)\cos(\theta/2)e^{-i\phi}\Big) \\
				&= \frac{\hbar}{2}\Big(\frac{\sin\theta e^{i\phi}}{2} + \frac{\sin\theta e^{-i\phi}}{2}\Big) \\
				&= \frac{\hbar}{2}\sin\theta\cos\phi \\
				\langle \hat{n}_+ | S_y | \hat{n}_+\rangle &= \frac{\hbar}{2}\begin{bmatrix}
					\cos(\theta/2)e^{i\phi/2} &
					\sin(\theta/2)e^{-i\phi/2}
				\end{bmatrix}\begin{bmatrix}
					0 & -i \\ i & 0
				\end{bmatrix}\begin{bmatrix}
					\cos(\theta/2)e^{-i\phi/2} \\
					\sin(\theta/2)e^{i\phi/2}
				\end{bmatrix} \\
				&= \frac{\hbar}{2}\begin{bmatrix}
					\cos(\theta/2)e^{i\phi/2} &
					\sin(\theta/2)e^{-i\phi/2}
				\end{bmatrix}\begin{bmatrix}
					-i\sin(\theta/2)e^{i\phi/2} \\
					i\cos(\theta/2)e^{-i\phi/2}
				\end{bmatrix} \\
				&= \frac{\hbar}{2}\Big({-i\cos}(\theta/2)\sin(\theta/2)e^{i\phi} + i\sin(\theta/2)\cos(\theta/2)e^{-i\phi}\Big) \\
				&= \frac{\hbar}{2}\Big(\frac{\sin\theta e^{i\phi}}{2i} - \frac{\sin\theta e^{-i\phi}}{2i}\Big) \\
				&= \frac{\hbar}{2}\sin\theta\sin\phi \\
				\langle \hat{n}_+|S_z|\hat{n}_+\rangle &= \frac{\hbar}{2}\begin{bmatrix}
					\cos(\theta/2)e^{i\phi/2} &
					\sin(\theta/2)e^{-i\phi/2}
				\end{bmatrix}\begin{bmatrix}
					1 & 0 \\ 0 & -1
				\end{bmatrix}\begin{bmatrix}
					\cos(\theta/2)e^{-i\phi/2} \\
					\sin(\theta/2)e^{i\phi/2}
				\end{bmatrix} \\
				&= \frac{\hbar}{2}\begin{bmatrix}
					\cos(\theta/2)e^{i\phi/2} &
					\sin(\theta/2)e^{-i\phi/2}
				\end{bmatrix}\begin{bmatrix}
					\cos(\theta/2)e^{-i\phi/2} \\
					-\sin(\theta/2)e^{i\phi/2}
				\end{bmatrix} \\
				&= \frac{\hbar}{2}\Big(\cos^2(\theta/2) - \sin^2(\theta/2)\Big) \\
				&= \frac{\hbar}{2}\cos\theta
			\end{align*}
			or, written another way,
			$$\langle \hat{n}_+|\mathbf{S}|\hat{n}_+\rangle = (\hbar/2)(\mathbf{i}\sin\theta\cos\phi + \mathbf{j}\sin\theta\sin\phi + \mathbf{k}\cos\theta)$$
			Since the procedure to find $\langle \hat{n}_-|\mathbf{S}|\hat{n}_-\rangle$ is nearly the same, we do not show it here.
		\end{solution}
		
		\question Using Eqs. (14.3.32) and (14.3.33) show that the Pauli matrices are traceless.
		
		\begin{solution}
			Noting that the $\mathrm{Tr}(-A) = -\mathrm{Tr}(A)$ and using (14.3.32) and (14.3.33), we find
			$$\mathrm{Tr}(\sigma_k) = -i\mathrm{Tr}(\sigma_i\sigma_j) = i\mathrm{Tr}(\sigma_j\sigma_i)$$
			However, the trace of a product of operators is unchanged under cyclic permutation of those operators, and so we also know
			$$\mathrm{Tr}(\sigma_k) = -i\mathrm{Tr}(\sigma_i\sigma_j) = -i\mathrm{Tr}(\sigma_j\sigma_i)$$
			Since the only way a quantity can be equal to the negative of itself is if it is 0, we must have $\mathrm{Tr}(\sigma_k) = 0$ for $k = x, y, z$.
		\end{solution}
		
		\question Derive Eq. (14.3.39) in two different ways.
		
		(1) Write $\sigma_i\sigma_j$ in terms of $[\sigma_i, \sigma_j]_+$ and $[\sigma_i, \sigma_j]$.
		
		(2) Use Eqs. (14.3.42) and (14.3.43).
		
		\begin{solution}
			We can write $\sigma_i\sigma_j$ as
			$$\sigma_i\sigma_j = \frac{1}{2}\sigma_i\sigma_j - \frac{1}{2}\sigma_j\sigma_i + \frac{1}{2}\sigma_i\sigma_j + \frac{1}{2}\sigma_j\sigma_i = \frac{1}{2}[\sigma_i, \sigma_j] + \frac{1}{2}[\sigma_i, \sigma_j]_+ = \delta_{ij}I + i\varepsilon_{ijk}\sigma_k$$
			Permuting the Levi-Civita indices from $ijk$ to $kij$, we can use this to write
			$$(\mathbf{A}\cdot\boldsymbol{\sigma})(\mathbf{B}\cdot\boldsymbol{\sigma}) \to A_i\sigma_iB_j\sigma_j = A_iB_i(\delta_{ij}I + i\varepsilon_{kij}\sigma_k) \to \mathbf{A}\cdot\mathbf{B}I + i(\mathbf{A}\times\mathbf{B})\cdot\boldsymbol{\sigma}$$
			If we instead approach this by looking at the inner product of each of the Pauli matrices with $A_iB_j\sigma_i\sigma_j$, we find
			\begin{align*}
				m_0 &= \frac{1}{2}\mathrm{Tr}(A_iB_j\sigma_i\sigma_j\sigma_0) \\
				&= \frac{1}{2}A_iB_j\mathrm{Tr}(\sigma_i\sigma_j) \\
				&= \frac{1}{2}A_iB_j\cdot2\delta_{ij} \\
				&= \mathbf{A}\cdot\mathbf{B} \\
				m_k &= \frac{1}{2}\mathrm{Tr}(A_iB_j\sigma_i\sigma_j\sigma_k) \\
				&= \frac{1}{2}A_iB_j\mathrm{Tr}(i\varepsilon_{ijl}\sigma_l\sigma_k) \\
				&= \frac{1}{2}A_iB_j\cdot 2i\varepsilon_{ijl}\delta_{lk} \\
				&= i\varepsilon_{kij}A_iB_j \\
				&= i[\mathbf{A}\times\mathbf{B}]_k
			\end{align*}
			where the $m_\alpha$ are defined from
			$$(\mathbf{A}\cdot\boldsymbol{\sigma})(\mathbf{B}\cdot\boldsymbol{\sigma}) = \sum_\alpha m_\alpha \sigma_\alpha$$
			Comparing our found $m_\alpha$ values to the first part of this problem, we see both methods produce identical results.
		\end{solution}
		
		\question Express the following matrix $M$ in terms of the Pauli matrices
		$$M = \begin{bmatrix}
			\alpha & \beta \\
			\gamma & \delta
		\end{bmatrix}$$
	
		\begin{solution}
			Writing
			$$M = \sum_\alpha m_\alpha\sigma_\alpha$$
			and noting that
			$$m_\beta = \frac{1}{2}\mathrm{Tr}(M\sigma_\beta)$$
			we find
			\begin{align*}
				m_0 &= \frac{1}{2}\mathrm{Tr}\begin{bmatrix}
					\alpha & \beta \\
					\gamma & \delta
				\end{bmatrix} \\
				&= \frac{1}{2}(\alpha + \delta) \\
				m_x &= \frac{1}{2}\mathrm{Tr}\begin{bmatrix}
					\alpha & \beta \\
					\gamma & \delta
				\end{bmatrix}\begin{bmatrix}0 & 1 \\ 1 & 0\end{bmatrix} \\
				&= \frac{1}{2}\mathrm{Tr}\begin{bmatrix}\beta & \alpha \\ \delta & \gamma\end{bmatrix} \\
				&= \frac{1}{2}(\beta + \gamma) \\
				m_y &= \frac{1}{2}\mathrm{Tr}\begin{bmatrix}
					\alpha & \beta \\
					\gamma & \delta
				\end{bmatrix}\begin{bmatrix}0 & -i \\ i & 0\end{bmatrix} \\
				&= \frac{1}{2}\mathrm{Tr}\begin{bmatrix}i\beta & -i\alpha \\ i\delta & -i\gamma\end{bmatrix} \\
				&= \frac{i}{2}(\beta - \gamma) \\
				m_z &= \frac{1}{2}\mathrm{Tr}\begin{bmatrix}
					\alpha & \beta \\
					\gamma & \delta
				\end{bmatrix}\begin{bmatrix}1 & 0 \\ 0 & -1\end{bmatrix} \\
				&= \frac{1}{2}\mathrm{Tr}\begin{bmatrix}\alpha & -\beta \\ \gamma & -\delta\end{bmatrix} \\
				&= \frac{1}{2}(\alpha - \delta)
			\end{align*}
			and so
			$$M = \frac{1}{2}(\alpha + \delta)\sigma_0 + \frac{1}{2}(\beta + \gamma)\sigma_x + \frac{i}{2}(\beta - \gamma)\sigma_y + \frac{1}{2}(\alpha - \delta)\sigma_z$$
		\end{solution}
		
		\question (1) Argue that $|\hat{n}, +\rangle = U[R(\phi\mathbf{k})]U[R(\theta\mathbf{j})]|s_z = \hbar/2\rangle$. (2) Verify by explicit calculation.
		
		\begin{solution}
			In the order in which the operations are performed, the equation describes rotating a pure, positive $z$ spin in the $x$-$z$ plane by an altitudinal angle $\theta$ before moving it azimuthally by $\phi$. The final orientation is exactly that of $|\hat{n}\rangle$, and so we expect the two sides of the proposed relationship to be equal.
			
			By analogy with angular momentum, the unitary representation of the spinorial rotation operator should be
			$$U[R(\boldsymbol{\theta})] = e^{-i\boldsymbol{\theta}\cdot\mathbf{S}/\hbar}$$
			Noting that $\hat{n} = \mathbf{j}$ implies $n_\theta = n_\phi = \pi/2$, $\hat{n} = \mathbf{k}$
			implies $n_\theta = n_\phi = 0$, and recalling we found $\mathbf{S}\cdot\boldsymbol{\theta} = \theta\mathbf{S}\cdot\hat{n} = \theta(\hbar/2)\boldsymbol{\sigma}\cdot\hat{n}$ in problem 14.3.2, we can write
			\begin{align*}
				U[R(\theta\mathbf{j})] &= \exp\Big({-\frac{i}{\hbar}}\frac{\hbar}{2}\theta\begin{bmatrix}0 & -i \\ i & 0\end{bmatrix}\Big) \\
				&= \exp\Big(\frac{\theta}{2}\begin{bmatrix}0 & -1 \\ 1 & 0\end{bmatrix}\Big) \\
				&= \begin{bmatrix}1 & 0 \\ 0 & 1\end{bmatrix} + \frac{\theta}{2}\begin{bmatrix}0 & -1 \\ 1 & 0\end{bmatrix} + \frac{1}{2!}\frac{\theta^2}{2^2}\begin{bmatrix}-1 & 0 \\ 0 & -1\end{bmatrix} + \frac{1}{3!}\frac{\theta^3}{2^3}\begin{bmatrix}0 & 1 \\ -1 & 0\end{bmatrix} + \cdots \\
				&= \begin{bmatrix}\cos(\theta/2) & -\sin(\theta/2) \\ \sin(\theta/2) & \cos(\theta/2) \end{bmatrix} \\
				U[R(\phi\mathbf{k})] &= \exp\Big({-\frac{i}{\hbar}}\frac{\hbar}{2}\phi\begin{bmatrix}1 & 0 \\ 0 & -1\end{bmatrix}\Big) \\
				&= \begin{bmatrix}e^{-i\phi/2} & 0 \\ 0 & e^{i\phi/2}\end{bmatrix}
			\end{align*}
			and so
			\begin{align*}
				U[R(\phi\mathbf{k})]U[R(\theta\mathbf{j})]|s_z = \hbar/2\rangle &= \begin{bmatrix}e^{-i\phi/2} & 0 \\ 0 & e^{i\phi/2}\end{bmatrix}\begin{bmatrix}\cos(\theta/2) & -\sin(\theta/2) \\ \sin(\theta/2) & \cos(\theta/2) \end{bmatrix}\begin{bmatrix}1 \\ 0\end{bmatrix} \\
				&= \begin{bmatrix}e^{-i\phi/2} & 0 \\ 0 & e^{i\phi/2}\end{bmatrix}\begin{bmatrix}\cos(\theta/2) \\ \sin(\theta/2) \end{bmatrix} \\
				&= \begin{bmatrix}
					\cos(\theta/2)e^{-i\phi/2} \\
					\sin(\theta/2)e^{i\phi/2}
				\end{bmatrix}
			\end{align*}
			which equals $|\hat{n}, +\rangle$.
		\end{solution}
		
		\question Express the following as linear combinations of the Pauli matrices and $I$:
		
		(1) $(I + i\sigma_x)^{1/2}$. (Relate it to half a certain rotation.)
		
		(2) $(2I + \sigma_x)^{-1}$.
		
		(3) $\sigma_x^{-1}$.
		
		\begin{solution}
			If we find diagonalize $I + i\sigma_x$ by rewriting it as $U^{-1}\Lambda U$, we can find $(I + i\sigma_x)^{1/2}$ as $U^{-1}\Lambda^{1/2}U$. To find the eigenvalues, we examine the characteristic polynomial,
			$$\det(I + i\sigma_x - \lambda I) = \begin{vmatrix}1 - \lambda & i \\ i & 1 - \lambda\end{vmatrix} = (1 - \lambda)^2 + 1 = \lambda^2 - 2\lambda + 2 = 0$$
			This has the roots
			$$\lambda_{\pm} = 1 \pm i$$
			which, from inspection, correspond to the eigenvectors
			$$\eta_+ = \begin{bmatrix}1 \\ 1\end{bmatrix}, \qquad \eta_- = \begin{bmatrix}1 \\ {-1}\end{bmatrix}$$
			From here, we can write
			$$U = \begin{bmatrix}\eta_+ & \eta_-\end{bmatrix} = \begin{bmatrix}1 & 1 \\ 1 & -1\end{bmatrix}, \qquad U^{-1} = \begin{bmatrix}1/2 & 1/2 \\ 1/2 & -1/2\end{bmatrix}$$
			and so
			$$I + i\sigma_x = \begin{bmatrix}1/2 & 1/2 \\ 1/2 & -1/2\end{bmatrix}\begin{bmatrix}1 + i & 0 \\ 0 & 1 - i\end{bmatrix}\begin{bmatrix}1 & 1 \\ 1 & -1\end{bmatrix}$$
			Writing $1 \pm i = 2^{1/2}e^{\pm i\pi/4}$, we see
			\begin{align*}
				(I + i\sigma_x)^{1/2} &= \begin{bmatrix}1/2 & 1/2 \\ 1/2 & -1/2\end{bmatrix}\begin{bmatrix}2^{1/4}e^{i\pi/8} & 0 \\ 0 & 2^{1/4}e^{-i\pi/8}\end{bmatrix}\begin{bmatrix}1 & 1 \\ 1 & -1\end{bmatrix} \\
				&= 2^{1/4}\begin{bmatrix}1/2 & 1/2 \\ 1/2 & -1/2\end{bmatrix}\begin{bmatrix}e^{i\pi/8}  & e^{i\pi/8} \\ e^{-i\pi/8} & -e^{-i\pi/8}\end{bmatrix} \\
				&= 2^{1/4}\begin{bmatrix}(e^{i\pi/8} + e^{-i\pi/8})/2 & (e^{i\pi/8} - e^{-i\pi/8})/2 \\ (e^{i\pi/8} - e^{-i\pi/8})/2 & (e^{i\pi/8} + e^{-i\pi/8})/2\end{bmatrix} \\
				&= 2^{1/4}\begin{bmatrix}\cos\pi/8 & i\sin\pi/8 \\ i\sin\pi/8 & \cos\pi/8\end{bmatrix} \\
				&= (2^{1/4}\cos\pi/8)I + (2^{1/4}i\sin\pi/8)\sigma_x
			\end{align*}
			For (2), we may easily write
			$$(2I + \sigma_x)^{-1} = \begin{bmatrix}2 & 1 \\ 1 & 2\end{bmatrix}^{-1} = \frac{1}{4-1}\begin{bmatrix}2 & -1 \\ -1 & 2\end{bmatrix} = \begin{bmatrix}2/3 & -1/3 \\ -1/3 & 2/3\end{bmatrix} = (2/3)I + (-1/3)\sigma_x$$
			Since $\sigma_x$ is its own inverse, (3) can be easily answered: $\sigma_x^{-1} = \sigma_x$.
		\end{solution}
		
		\question (1) Show that any matrix that commutes with $\boldsymbol{\sigma}$ is a multiple of the unit matrix.
		
		(2) Show that we cannot find a matrix that anticommutes with all three Pauli matrices. (If such a matrix exists, it must equal zero.)
		\begin{solution}
			Since we can write any operator $M$ as $M = m_0I + \mathbf{m}\cdot\boldsymbol{\sigma}$, a matrix $A$ that commutes with $\boldsymbol{\sigma}$ also commutes with every operator. The only class of matrices with this property is the set of multiples of the identity.
			
			Since we can write any matrix $A$ as a linear combination of Pauli matrices $A = a_0I + \sum_{i=1}^{3}a_i\sigma_i$, any $A$ that anticommutes with all three Pauli matrices satisfies
			\begin{align*}
				[A, \sigma_k]_+ &= a_0[I, \sigma_k]_+ + \sum_{i=1}^3a_i[\sigma_i, \sigma_k]_+ \\
				&= 2a_0\sigma_k + \sum_{i=1}^3a_i2\delta_{ik}I \\
				&= 2a_0\sigma_k + 2a_kI \\
				&= 0
			\end{align*}
			The only way for this to hold is for $a_0 = a_1 = a_2 = a_3 = 0$, i.e. $A$ must be the zero matrix.
		\end{solution}
	
		\setcounter{subsection}{3}
		\setcounter{question}{0}
		\subsection{Dynamics of Spin}
		
		\question Show that if $H = -\gamma\mathbf{L}\cdot\mathbf{B}$, and $\mathbf{B}$ is position independent,
		$$\frac{\mathrm{d}\langle\mathbf{L}\rangle}{\mathrm{d}t} = \langle\boldsymbol{\mu}\times\mathbf{B}\rangle = \langle\boldsymbol{\mu}\rangle\times\mathbf{B}$$
		Comparing this to Eq. (14.4.8), we see that $\langle\boldsymbol{\mu}\rangle$ evolves exactly like $\boldsymbol{\mu}$. Notice that his conclusion is valid even if $\mathbf{B}$ depends on time and also if we are talking about spin instead of orbital angular momentum. A more explicit verification follows in Exercise 14.4.3.
		\begin{solution}
			Since $\mathbf{L}$ does not depend on time, Ehrenfest's theorem tells us
			\begin{align*}
				i\hbar\frac{\mathrm{d}\langle L_i\rangle}{\mathrm{d}t} &= \langle[L_i, H]\rangle \\
				&= \langle [L_i, -\gamma\mathbf{L}\cdot\mathbf{B}]\rangle \\
				&= -\gamma\langle [L_i, L_j]B_j\rangle \\
				&= -\gamma i\hbar\langle \varepsilon_{ijk}L_kB_j\rangle \\
				&=  i\hbar\langle \varepsilon_{ikj}(\gamma L_k)B_j\rangle \\
				&=  i\hbar\langle \varepsilon_{ikj}\mu_kB_j\rangle
			\end{align*}
			or, in vector notation,
			$$i\hbar\frac{\mathrm{d}\langle\mathbf{L}\rangle}{\mathrm{d}t} = i\hbar\langle\boldsymbol{\mu}\times\mathbf{B}\rangle$$
			We can drop the common factor of $i\hbar$ and, since $\mathbf{B}$ is independent of position, write
			$$\frac{\mathrm{d}\langle\mathbf{L}\rangle}{\mathrm{d}t} = \langle\boldsymbol{\mu}\rangle\times\mathbf{B}$$
			to arrive at the final result. This holds even if $\mathbf{B}$ is dependent on time, since $\boldsymbol{\mu}$ depends only on position and derivatives of position.
		\end{solution}
		
		\question Derive (14.4.31) by studying Fig 14.3.
		\begin{solution}
			Visually, we can tell that $\mu_z$ will oscillate about a constant value corresponding to its magnitude doubly projected: first onto the axis of oscillation and then onto the axis of rotation (the $z$-axis). By similar triangles, this value is $\mu\cos\alpha\cdot\cos\alpha = \mu\cos^2\alpha$, where $\alpha$ is both the angle of the oscillation axis and the angle of the magnetic field vector with respect to the $z$-axis. (The diagram shows these as two separate angles, but they should coincide.)
			
			The amount by which $\mu_z$ will change with $\cos\omega_rt$ is the magnitude projected first onto the plane of precession and then second onto the axis of rotation. Using similar triangles yet again, we find this is equal to $\mu\sin\alpha\cdot\sin\alpha = \mu^2\sin\alpha$. Altogether, we now have
			$$\mu_z(t) = \mu\cos^2\alpha + \mu\sin^2\alpha\cos\omega_rt$$
			
			Since $\alpha$ is the angle $\boldsymbol{\omega}_r$ makes with the $z$-axis and
			$$\omega_r = [\gamma^2B^2 + (\omega - \omega)^2]^{1/2}$$
			we can rewrite this as
			$$\mu_z(t) = \mu(0)\Big[\frac{(\omega_0 - \omega)^2}{(\omega_0 - \omega)^2 + \gamma^2B^2} + \frac{\gamma^2B^2\cos\omega_rt}{(\omega_0 - \omega)^2 + \gamma^2B^2}\Big]$$
		\end{solution}
		
		\question We would like to study here the evolution of a state that starts out as $\begin{pmatrix}1 \\ 0 \end{pmatrix}$ and is subject to the $\mathbf{B}$ field given in Eq. (14.4.27). This state obeys
		$$i\hbar\frac{\mathrm{d}}{\mathrm{d}t}|\psi(t)\rangle = H|\psi\rangle$$
		where $H = -\gamma\mathbf{S}\cdot\mathbf{B}$, and $\mathbf{B}$ is time dependent. Since classical reasoning suggests that in a frame rotating at frequency $(-\omega\mathbf{k})$ the Hamiltonian should be time independent and governed by $\mathbf{B}_r$, [Eq. (14.4.29)], consider the ket in the rotating frame, $|\psi_r(t)\rangle$, related to $|\psi(t)\rangle$ by a rotation angle $\omega t$:
		$$|\psi_r(t)\rangle = e^{-i\omega tS_z/\hbar}|\psi(t)\rangle$$
		Combine Eqs. (14.4.34) and (14.4.35) to derive Schr\"odinger's equation for $|\psi_r(t)\rangle$ in the $S_z$ basis and verify that the classical expectation is borne out. Solve for $|\psi_r(t)\rangle = U_r(t)|\psi_r(0)\rangle$ by computing $U_r(t)$, the propagator in the rotating frame. Rotate back to the lab and show that
		$$|\psi(t)\rangle \xrightarrow[S_z\text{ basis}]{}\begin{bmatrix}\Big[\cos\Big(\frac{\omega_rt}{2}\Big) + i\frac{\omega_0 - \omega}{\omega_r}\sin\Big(\frac{\omega_rt}{2}\Big)\Big]e^{+i\omega t/2} \\ \frac{i\gamma B}{\omega_r}\sin\Big(\frac{\omega_rt}{2}\Big)e^{-i\omega t/2}\end{bmatrix}$$
		Compare this to the state $|\hat{\mathbf{n}}, +\rangle$ and see what is happening to the spin for the case $\omega_0 = \omega$.
		
		Calculate $\langle \mu_z(t)\rangle$ and verify that it agrees with Eq. (14.4.31).
		
		\begin{solution}
			Substituting $|\psi(t)\rangle = e^{i\omega tS_z/\hbar}|\psi_r(t)\rangle$ into the left side of the Schr\"odinger equation gives
			$$i\hbar\frac{\mathrm{d}}{\mathrm{d}t}\Big(e^{i\omega tS_z/\hbar}|\psi_r(t)\rangle\Big) = -\omega S_ze^{i\omega tS_z/\hbar}|\psi_r(t)\rangle + i\hbar e^{i\omega tS_z/\hbar}\frac{\mathrm{d}}{\mathrm{d}t}|\psi_r(t)\rangle$$
			while making the substitution on the right gives
			$$-\gamma\mathbf{S}\cdot\mathbf{B}|\psi(t)\rangle = -\gamma(S_xB\cos\omega t - S_yB\sin\omega t + S_zB_0)e^{i\omega tS_z/\hbar}|\psi_r(t)\rangle$$
			Combining these into one equation yields
			\begin{align*}
				i\hbar\frac{\mathrm{d}}{\mathrm{d}t}|\psi_r(t)\rangle &= -\gamma e^{-i\omega tS_z/\hbar}\Big[S_xB\cos\omega t - S_yB\sin\omega t + S_z\Big(B_0 - \frac{\omega}{\gamma}\Big)\Big]e^{i\omega tS_z/\hbar}|\psi_r(t)\rangle \\
				&= -\gamma e^{-i\omega tS_z/\hbar}\Big[S_xB\cos\omega t - S_yB\sin\omega t\Big]e^{i\omega tS_z/\hbar}|\psi_r(t)\rangle - \gamma S_z\Big(B_0 - \frac{\omega}{\gamma}\Big)|\psi_r(t)\rangle
			\end{align*}
			To simplify this further, we must determine both
			$$e^{-i\omega tS_z/\hbar}S_xe^{i\omega tS_z/\hbar}\qquad\text{and}\qquad e^{-i\omega tS_z/\hbar}S_ye^{i\omega tS_z/\hbar}$$
			The first of these is
			\begin{align*}
				e^{-i\omega tS_z/\hbar}S_xe^{i\omega tS_z/\hbar} &= (I\cos\tfrac{\omega t}{2} - \tfrac{2i}{\hbar}\sin\tfrac{\omega t}{2}S_z)S_x(I\cos\tfrac{\omega t}{2} + \tfrac{2i}{\hbar}\sin\tfrac{\omega t}{2}S_z) \\
				&= S_x\cos^2\tfrac{\omega t}{2} + \tfrac{2i}{\hbar}\cos\tfrac{\omega t}{2}\sin\tfrac{\omega t}{2}[S_x, S_z] +  S_zS_xS_z\tfrac{4}{\hbar^2}\sin^2\tfrac{\omega t}{2} \\
				&= S_x\cos^2\tfrac{\omega t}{2} + \frac{i}{\hbar}\sin\omega t(-i\hbar S_y) + (S_z^2S_x - i\hbar S_zS_y)\tfrac{4}{\hbar^2}\sin^2\tfrac{\omega t}{2} \\
				&= S_x\cos^2\tfrac{\omega t}{2} + S_y\sin\omega t + (\tfrac{\hbar^2}{4}S_x - \tfrac{i\hbar}{2}S_zS_y + \tfrac{i\hbar}{2}S_yS_z)\tfrac{4}{\hbar^2}\sin^2\tfrac{\omega t}{2} \\
				&= S_x\cos^2\tfrac{\omega t}{2} + S_y\sin\omega t + (\tfrac{\hbar^2}{4}S_x + \tfrac{i\hbar}{2}[S_y, S_z])\tfrac{4}{\hbar^2}\sin^2\tfrac{\omega t}{2} \\
				&= S_x\cos^2\tfrac{\omega t}{2} + S_y\sin\omega t + (\tfrac{\hbar^2}{4}S_x - \tfrac{\hbar^2}{2}S_x)\tfrac{4}{\hbar^2}\sin^2\tfrac{\omega t}{2} \\
				&= S_x(\cos^2\tfrac{\omega t}{2} - \sin^2\tfrac{\omega t}{2}) + S_y\sin\omega t \\
				&= S_x\cos\omega t + S_y\sin\omega t
			\end{align*}
			while the second is
			\begin{align*}
				e^{-i\omega tS_z/\hbar}S_ye^{i\omega tS_z/\hbar} &= (I\cos\tfrac{\omega t}{2} - \tfrac{2i}{\hbar}\sin\tfrac{\omega t}{2}S_z)S_y(I\cos\tfrac{\omega t}{2} + \tfrac{2i}{\hbar}\sin\tfrac{\omega t}{2}S_z) \\
				&= S_y\cos^2\tfrac{\omega t}{2} + \tfrac{2i}{\hbar}\cos\tfrac{\omega t}{2}\sin\tfrac{\omega t}{2}[S_y, S_z] +  S_zS_yS_z\tfrac{4}{\hbar^2}\sin^2\tfrac{\omega t}{2} \\
				&= S_y\cos^2\tfrac{\omega t}{2} + \frac{i}{\hbar}\sin\omega t(i\hbar S_x) + (S_z^2S_y + i\hbar S_zS_x)\tfrac{4}{\hbar^2}\sin^2\tfrac{\omega t}{2} \\
				&= S_y\cos^2\tfrac{\omega t}{2} - S_x\sin\omega t + (\tfrac{\hbar^2}{4}S_y + \tfrac{i\hbar}{2}S_zS_x - \tfrac{i\hbar}{2}S_xS_z)\tfrac{4}{\hbar^2}\sin^2\tfrac{\omega t}{2} \\
				&= S_y\cos^2\tfrac{\omega t}{2} - S_x\sin\omega t + (\tfrac{\hbar^2}{4}S_y + \tfrac{i\hbar}{2}[S_z, S_x])\tfrac{4}{\hbar^2}\sin^2\tfrac{\omega t}{2} \\
				&= S_y\cos^2\tfrac{\omega t}{2} - S_x\sin\omega t + (\tfrac{\hbar^2}{4}S_y - \tfrac{\hbar^2}{2}S_y)\tfrac{4}{\hbar^2}\sin^2\tfrac{\omega t}{2} \\
				&= S_y(\cos^2\tfrac{\omega t}{2} - \sin^2\tfrac{\omega t}{2}) - S_x\sin\omega t \\
				&= S_y\cos\omega t - S_x\sin\omega t
			\end{align*}
			and so the modified Schr\"odinger equation becomes
			\begin{align*}
				i\hbar\frac{\mathrm{d}}{\mathrm{d}t}|\psi_r(t)\rangle &= -\gamma \Big[(S_x\cos\omega t + S_y\sin\omega t)B\cos\omega t - (S_y\cos\omega t - S_x\sin\omega t)B\sin\omega t + S_z\Big(B_0 - \frac{\omega}{\gamma}\Big)\Big]|\psi_r(t)\rangle \\
				&= -\gamma \Big[S_xB + S_z\Big(B_0 - \frac{\omega}{\gamma}\Big)\Big]|\psi_r(t)\rangle
			\end{align*}
			This coincides with our classical expectation, i.e. the magnetic field appears stationary with a modified $z$ component. Identifying $H_r = -\gamma(S_xB + S_z(B_0 - \omega/\gamma)) = -\gamma\mathbf{S}\cdot\mathbf{B}_r$, where
			\begin{gather*}
				\gamma\mathbf{B}_r = \begin{bmatrix}\gamma B & \omega_0 - \omega\end{bmatrix}^T, \\
				\omega_r = {-\gamma}B_r, \\
				\omega_0 = \gamma B_0,
			\end{gather*}
			the propagator becomes
			\begin{align*}
				U_r(t) = e^{-iH_rt/\hbar} &= \exp\big({-i}(-\gamma\mathbf{S}\cdot\mathbf{B}_r)t/\hbar\big) \\
				&= \exp\big({-i}(-\gamma B_r t\,\hat{\mathbf{B}}_r)\cdot\boldsymbol{\sigma}/2\big) \\
				&= \exp\big({-i}(\omega_rt\,\hat{\mathbf{B}}_r)\cdot\boldsymbol{\sigma}/2\big) \\
				&= \cos\Big(\frac{\omega_rt}{2}\Big)I + i\sin\Big(\frac{\omega_rt}{2}\Big)\hat{\mathbf{B}}_r\cdot\boldsymbol{\sigma} \\
				&= \cos\Big(\frac{\omega_rt}{2}\Big)I + i\sin\Big(\frac{\omega_rt}{2}\Big)\big(\gamma B S_x + (\omega_0-\omega)S_z\big)\omega_r^{-1} \\
				&= \begin{bmatrix}
					\cos(\tfrac{\omega_rt}{2}) + i\tfrac{\omega_0-\omega}{\omega_r}\sin(\tfrac{\omega_rt}{2}) & i\tfrac{\gamma B}{\omega_r}\sin(\tfrac{\omega_rt}{2}) \\
					i\tfrac{\gamma B}{\omega_r}\sin(\tfrac{\omega_rt}{2}) & \cos(\tfrac{\omega_rt}{2}) - i\tfrac{\omega_0-\omega}{\omega_r}\sin(\tfrac{\omega_rt}{2})
				\end{bmatrix}
			\end{align*}
				and so
				$$|\psi_r(t)\rangle = \begin{bmatrix}
					\cos(\tfrac{\omega_rt}{2}) + i\tfrac{\omega_0-\omega}{\omega_r}\sin(\tfrac{\omega_rt}{2}) \\
					i\tfrac{\gamma B}{\omega_r}\sin(\tfrac{\omega_rt}{2})
				\end{bmatrix}$$
			Returning to the original frame amounts to left multiplication by $e^{i\omega tS_z/\hbar} = \mathrm{diag}\begin{pmatrix}e^{i\omega t/2} & e^{-i\omega t / 2}\end{pmatrix}$, i.e.
			$$|\psi(t)\rangle = \begin{bmatrix}\Big(\cos(\frac{\omega_rt}{2}) + i\frac{\omega_0 - \omega}{\omega_r}\sin(\frac{\omega_rt}{2})\Big)e^{i\omega t/2} \\ \frac{i\gamma B}{\omega_r}\sin(\frac{\omega_rt}{2})e^{-i\omega t/2}\end{bmatrix}$$
			Setting $\omega = \omega_0$ (implying $\gamma B = \omega_r$) and comparing with
			$$|\hat{n}_+\rangle = \begin{bmatrix}
				\cos(\theta/2)e^{-i\phi/2} \\
				\sin(\theta/2)e^{i\phi/2}
			\end{bmatrix}$$
			we see that our state is oscillating in the altitudinal and azimuthal directions at different rates. Finally, since $\mu_z = \gamma S_z$,
			\begin{align*}
				\langle\mu_z(t)\rangle &= \langle\psi|\begin{bmatrix}\gamma\hbar/2 & 0 \\ 0 & -\gamma\hbar/2\end{bmatrix}|\psi\rangle \\
				&= \frac{\gamma\hbar}{2}\begin{bmatrix}\big(\cos(\frac{\omega_rt}{2}) - i\frac{\omega_0 - \omega}{\omega_r}\sin(\frac{\omega_rt}{2})\big)e^{-i\omega t/2} \\ -\frac{i\gamma B}{\omega_r}\sin(\frac{\omega_rt}{2})e^{i\omega t/2}\end{bmatrix}^T\begin{bmatrix}\big(\cos(\frac{\omega_rt}{2}) + i\frac{\omega_0 - \omega}{\omega_r}\sin(\frac{\omega_rt}{2})\big)e^{i\omega t/2} \\ -\frac{i\gamma B}{\omega_r}\sin(\frac{\omega_rt}{2})e^{-i\omega t/2}\end{bmatrix} \\
				&= \frac{\gamma\hbar}{2}\Big[\cos^2(\tfrac{\omega_rt}{2}) + (\tfrac{\omega_0-\omega}{\omega_r})^2\sin^2(\tfrac{\omega_rt}{2}) - (\tfrac{\gamma B}{\omega_r})^2\sin^2(\tfrac{\omega_rt}{2})\Big] \\
				&= \frac{\gamma\hbar}{2\omega_r}\Big[(\omega_0 - \omega)^2\cos^2(\tfrac{\omega_rt}{2}) + (\gamma B)^2\cos^2(\tfrac{\omega_rt}{2}) + (\omega_0-\omega)^2\sin^2\tfrac{\omega_rt}{2} - (\gamma B)^2\sin^2(\tfrac{\omega_rt}{2})\Big] \\
				&= \frac{\gamma\hbar}{2\omega_r}\Big[(\omega_0-\omega)^2 + (\gamma B)^2\cos(\omega_rt)\Big] \\
				&= \frac{\gamma\hbar}{2}\Big[\frac{(\omega_0-\omega)^2}{(\omega_0-\omega)^2 + \gamma^2B^2} + \frac{\gamma^2B^2\cos(\omega_rt)}{(\omega_0-\omega)^2 + \gamma^2B^2}\Big]
			\end{align*}
			which, identifying $\mu_z(0) = \gamma\hbar/2$, is exactly Eq. (14.4.31).
		\end{solution}
		
		\question At $t = 0$, an electron is in the state with $s_z = \hbar/2$. A steady field $\mathbf{B} = B\mathbf{i}$, $B = 100\text{ G}$, is turned on. How many seconds will it take for the spin to flip?
		
		\begin{solution}
			The Hamiltonian for this system is $H = -\gamma \mathbf{S}\cdot\mathbf{B} = -\gamma S_xB$, giving a propagator of
			$$U(t) = e^{-iHt/\hbar} = \exp\Big({\frac{i\gamma Bt}{2}}\begin{bmatrix}0 & 1 \\ 1 & 0\end{bmatrix}\Big) = \begin{bmatrix}\cos\gamma Bt/2 & i\sin\gamma Bt/2 \\ i\sin \gamma Bt/2 & \cos\gamma Bt/2\end{bmatrix}$$
			We are looking for the value of $t = t_1$ that results in
			$$|\psi(t = t_1)\rangle = \alpha\begin{bmatrix}0 \\ 1 \end{bmatrix} = \begin{bmatrix}\cos\gamma Bt_1/2 & i\sin\gamma Bt_1/2 \\ i\sin \gamma Bt_1/2 & \cos\gamma Bt_1/2\end{bmatrix}\begin{bmatrix}1 \\ 0\end{bmatrix} = \begin{bmatrix}\cos\gamma Bt_1/2 \\ i\sin \gamma Bt_1/2\end{bmatrix}$$
			where $\alpha$ is a simple phase factor. From inspection, this occurs when $\gamma Bt_1/2 = \pi/2$, or
			$$ t_1 = \frac{\pi}{\gamma B} = \frac{\pi mc}{eB} = \frac{\pi(9.1\cdot10^{-28}\text{ g})(3\cdot10^{10}\text{ cm}\text{ s}^{-1})}{(4.8\cdot 10^{-10}\text{ cm}^{3/2}\text{ g}^{1/2}\text{ s}^{-1})(100\text{ cm}^{-1/2}\text{ g}^{1/2}\text{ s}^{-1})} = 1.8\text{ ns}$$
		\end{solution}
		
		\question We would like to establish the validity of Eq. (14.4.26) when $\boldsymbol{\omega}$ and $\mathbf{B}_0$ are not parallel.
		
		(1) Consider a vector $\mathbf{V}$ in the inertial (nonrotating) frame which changes by $\Delta\mathbf{V}$ in a time $\Delta t$. Argue, using the results from Exercise 12.4.3, that the change as seen in a frame rotating at an angular velocity $\boldsymbol{\omega}$, is $\Delta \mathbf{V} - \boldsymbol{\omega}\times\mathbf{V}\Delta t$. Obtain a relation between the time derivatives of $\mathbf{V}$ in the two frames.
		
		(2) Apply this result to the case of $\mathbf{l}$ [Eq. (14.4.8)], and deduce the formula for the effective field in the rotating frame.
		
		\begin{solution}
			We will take the $z$-axis to coincide with $\boldsymbol{\omega}$, in which case our non-inertial frame coordinates are continuously transformed as
			$$\begin{bmatrix}
				\cos\omega t & {-\sin\omega} t & 0 \\ {\sin\omega t} & \cos\omega t & 0 \\ 0 & 0 & 1
			\end{bmatrix}$$
			Or, equivalently, all other vectors are transformed oppositely. During the small time period $\Delta t$, this other transformation becomes
			$$\begin{bmatrix}
				1 & {\omega\Delta t} & 0 \\ {-\omega\Delta t} & 1 & 0 \\ 0 & 0 & 1
			\end{bmatrix}$$
			and so
			\begin{align*}
				(\mathbf{V} + \Delta\mathbf{V}) - \mathbf{V} \to R(\omega\Delta t\mathbf{k})(\mathbf{V} + \Delta\mathbf{V}) - \mathbf{V} &= \begin{bmatrix}
					1 & {\omega\Delta t} & 0 \\ {-\omega\Delta t} & 1 & 0 \\ 0 & 0 & 1
				\end{bmatrix}\begin{bmatrix}v_x + \Delta v_x \\ v_y + \Delta v_y \\ v_z + \Delta v_z\end{bmatrix} - \begin{bmatrix}v_x \\ v_y \\ v_z\end{bmatrix} \\
				&= \begin{bmatrix}v_x + \Delta v_x + \omega v_y\Delta t \\ {-\omega v_x\Delta t} + v_y + \Delta v_y \\ v_z + \Delta v_z\end{bmatrix} - \begin{bmatrix}v_x \\ v_y \\ v_z\end{bmatrix} \\
				&= \begin{bmatrix}\Delta v_x \\ \Delta v_y \\ \Delta v_z\end{bmatrix} - \begin{bmatrix}-\omega v_y\Delta t \\ \omega v_x\Delta t \\ 0\end{bmatrix} \\
				&= \Delta\mathbf{V} - \boldsymbol{\omega} \times \mathbf{V}\Delta t
			\end{align*}
			where we have dropped terms of order $\mathcal{O}(\Delta^2)$ and considered $v_x, v_y, v_z$ to describe the initial position of $\mathbf{V}$ in the rotating frame. Since this is the change $\Delta\mathbf{V}'$ as seen in the rotating frame, we can divide by $\Delta t$ and take the limit to find
			$$\frac{\mathrm{d}\mathbf{V}'}{\mathrm{d}t} = \frac{\mathrm{d}\mathbf{V}}{\mathrm{d}t} - \boldsymbol{\omega}\times\mathbf{V}$$
			
			If we substitute $\mathbf{V} = \mathbf{l}$, Eq. (14.4.8) becomes
			$$\frac{\mathrm{d}\mathbf{l}'}{\mathrm{d}t} + \boldsymbol{\omega}\times\mathbf{l} = \gamma(\mathbf{l}\times\mathbf{B})$$
			or
			$$\frac{\mathrm{d}\mathbf{l}'}{\mathrm{d}t} = \gamma\mathbf{l}\times(\mathbf{B} + \boldsymbol{\omega}/\gamma)$$
			which predicts the effective magnetic field in the rotating frame changes as
			$$\mathbf{B} \to \mathbf{B} + \boldsymbol{\omega}/\gamma$$
			exactly as given in Eq. (14.4.26).
		\end{solution}
		
		\question \textit{(A Density Matrix Problem).} (1) Show that the density matrix for an ensemble of spin-1/2 articles may be written as
		$$\rho = \tfrac{1}{2}(I + \mathbf{a}\cdot\boldsymbol{\sigma})$$
		where $\mathbf{a}$ is a $c$-number vector.
		
		(2) Show that $\mathbf{a}$ is the mean polarization, $\langle \bar{\boldsymbol{\sigma}}\rangle$.
		
		(3) An ensemble of electrons in a magnetic field $\mathbf{B} = B\mathbf{k}$, is in thermal equilibrium at temperature $T$. Construct the density matrix for this ensemble. Calculate $\langle \bar{\boldsymbol{\mu}}\rangle$.
		
		\begin{solution}
			The density matrix is given by
			$$\rho = \sum_i p_i|\psi_i\rangle\langle\psi_i|$$
			where the group of $|\psi_i\rangle$ make a complete set of states and $\sum_i p_i = 1$. Using the generic $|n_\pm\rangle$ states, we find
			\begin{align*}
				\rho &= p_1|n_+\rangle\langle n_+| + p_2|n_-\rangle\langle n_-| \\
				&= p_1\begin{bmatrix}\cos(\tfrac{\theta}{2})e^{-i\phi/2} \\ \sin(\tfrac{\theta}{2})e^{i\phi/2}\end{bmatrix}\begin{bmatrix}\cos(\tfrac{\theta}{2})e^{i\phi/2} \\ \sin(\tfrac{\theta}{2})e^{-i\phi/2}\end{bmatrix}^T + p_2\begin{bmatrix}-\sin(\tfrac{\theta}{2})e^{-i\phi/2} \\ \cos(\tfrac{\theta}{2})e^{i\phi/2}\end{bmatrix}\begin{bmatrix}-\sin(\tfrac{\theta}{2})e^{i\phi/2} \\ \cos(\tfrac{\theta}{2})e^{-i\phi/2}\end{bmatrix}^T \\
				&= p_1\begin{bmatrix}\cos^2(\tfrac{\theta}{2}) & \cos(\tfrac{\theta}{2})\sin(\tfrac{\theta}{2})e^{-i\phi} \\
				\cos(\tfrac{\theta}{2})\sin(\tfrac{\theta}{2})e^{i\phi} & \sin^2(\tfrac{\theta}{2})\end{bmatrix} + p_2\begin{bmatrix}
				\sin^2(\tfrac{\theta}{2}) & -\cos(\tfrac{\theta}{2})\sin(\tfrac{\theta}{2})e^{-i\phi} \\
				-\cos(\tfrac{\theta}{2})\sin(\tfrac{\theta}{2})e^{i\phi} & \cos^2(\tfrac{\theta}{2})
			\end{bmatrix} \\
			&= \begin{bmatrix}
				p_1\cos^2(\tfrac{\theta}{2}) + p_2\sin^2(\tfrac{\theta}{2}) & \tfrac{1}{2}(p_1 - p_2)\sin(\theta)e^{-i\phi} \\
				\tfrac{1}{2}(p_1-p_2)\sin(\theta)e^{i\phi} & p_1\sin^2(\tfrac{\theta}{2}) + p_2\cos^2(\tfrac{\theta}{2})
			\end{bmatrix}
			\end{align*}
			To decompose this into the Pauli matrices, we compute
			\begin{align*}
				2m_0 = \mathrm{Tr}(\rho\sigma_0) &= p_1\big(\cos^2(\tfrac{\theta}{2}) + \sin^2(\tfrac{\theta}{2})\big) + p_2\big(\sin^2(\tfrac{\theta}{2}) + \cos^2(\tfrac{\theta}{2})\big) \\
				&= p_1 + p_2 \\
				&= 1 \\
				2m_1 = \mathrm{Tr}(\rho\sigma_x) &= \tfrac{1}{2}(p_1 - p_2)\sin(\theta)(e^{-i\phi} + e^{i\phi}) \\
				&= (p_1 - p_2)\sin(\theta)\cos(\phi) \\
				2m_2 = \mathrm{Tr}(\rho\sigma_y) &= \tfrac{i}{2}(p_1 - p_2)\sin(\theta)(e^{-i\phi} - e^{i\phi}) \\
				&= (p_1 - p_2)\sin(\theta)\sin(\phi) \\
				2m_3 = \mathrm{Tr}(\rho\sigma_z) &= p_1\big(\cos^2(\tfrac{\theta}{2}) - \sin^2(\tfrac{\theta}{2})\big) + p_2\big(\sin^2(\tfrac{\theta}{2}) - \cos^2(\tfrac{\theta}{2})\big) \\
				&= (p_1 - p_2)\cos(\theta)
			\end{align*}
			and so
			$$\rho = \sum_\alpha m_\alpha\sigma_\alpha = \tfrac{1}{2}(I + \mathbf{a}\cdot\boldsymbol{\sigma})$$
			where we recognize
			$$\mathbf{a} = (p_1 - p_2)(\sin\theta\cos\phi\,\hat{\mathbf{i}} + \sin\theta\sin\phi\,\hat{\mathbf{j}} + \cos\theta\,\hat{\mathbf{k}}) = (p_1 - p_2)\hat{n}$$
			This is, by definition, the mean polarization of the ensemble, $\langle\bar{\boldsymbol{\sigma}}\rangle$.
			
			In a magnetic field $\mathbf{B} = B\mathbf{k}$, the spin Hamiltonian is $H = -\gamma\mathbf{B}\cdot\mathbf{S} = -\gamma BS_z$, having the effect of rotating a state over time via
			$$U(t) = \begin{bmatrix}e^{i\omega_0t/2} & 0 \\ 0 & e^{-i\omega_0t/2}\end{bmatrix}$$
			where $\omega_0 = \gamma B$. Put another way, it causes a state to rotate around $\mathbf{B}$ with a frequency $\omega_0$, i.e. $\phi \to \phi - \omega_0t$. Therefore, the density matrix becomes
			$$\rho = \begin{bmatrix}
				p_1\cos^2(\tfrac{\theta}{2}) + p_2\sin^2(\tfrac{\theta}{2}) & \tfrac{1}{2}(p_1 - p_2)\sin(\theta)e^{-i(\phi - \omega_0t)} \\
				\tfrac{1}{2}(p_1-p_2)\sin(\theta)e^{i(\phi - \omega_0t)} & p_1\sin^2(\tfrac{\theta}{2}) + p_2\cos^2(\tfrac{\theta}{2})
			\end{bmatrix}$$
			or, written in terms of the Pauli matrices, 
			$$\rho = \sum_\alpha m_\alpha\sigma_\alpha = \tfrac{1}{2}(I + \mathbf{a}\cdot\boldsymbol{\sigma})$$
			with
			$$\mathbf{a} = (p_1 - p_2)(\sin(\theta)\cos(\phi - \omega_0t)\,\hat{\mathbf{i}} + \sin(\theta)\sin(\phi - \omega_0t)\,\hat{\mathbf{j}} + \cos(\theta)\,\hat{\mathbf{k}})$$
			Since $\bar{\boldsymbol{\mu}} = \gamma\mathbf{S} = \gamma\hbar\bar{\boldsymbol{\sigma}}/2$ and $\langle \bar{\boldsymbol{\sigma}}\rangle = \mathbf{a}$, we have
			$$\langle\bar{\boldsymbol{\mu}}\rangle = \frac{\gamma\hbar}{2}(p_1 - p_2)(\sin(\theta)\cos(\phi - \omega_0t)\,\hat{\mathbf{i}} + \sin(\theta)\sin(\phi - \omega_0t)\,\hat{\mathbf{j}} + \cos(\theta)\,\hat{\mathbf{k}})$$
		\end{solution}
	
	\setcounter{subsection}{4}
	\setcounter{question}{0}
	\subsection{Return of Orbital Degrees of Freedom}
	\question (1) Why is the coupling of the proton's intrinsic moment to $\mathbf{B}$ an order $m/M$ correction to Eq. (14.5.4)?
	
	(2) Why is the coupling of its orbital motion an order $(m/M)^2$ correction? (You may reason classically in both parts.)
	\begin{solution}
		When we include proton terms in the Hamiltonian, it becomes
		$$H = H_{\text{Coulomb}} - \Big(\frac{-eB}{2mc}\Big)L_z^e - \Big(\frac{-g_eeB}{2mc}\Big)S_z^e - \Big(\frac{eB}{2Mc}\Big)L_z^p - \Big(\frac{g_peB}{2Mc}\Big)S_z^p$$
		Clearly, both $L_z^p$ and $S_z^p$ are a factor of $m/M$ smaller than $L_z^e$ and $S_z^e$.
		
		The orbital angular momentum is a further factor of $m/M$ smaller on account of the proton's orbital radius being a factor $m/M$ smaller than the electron's.
	\end{solution}
	
	\question (1) Estimate the relative size of the level splitting in the $n=1$ state to the unperturbed energy of the $n=1$ state, when a field $\mathbf{B}=1000\text{ kG}$ is applied.
	
	(2) Recall that we have been neglecting the order $B^2$ term in $H$. Estimate its contribution in the $n=1$ state relative to the linear $(-\boldsymbol{\mu}\cdot\mathbf{B})$ term we have kept, by assuming the electron moves on a classical orbit of radius $a_0$. Above what $|\mathbf{B}|$ does it begin to be a poor approximation?
	
	\begin{solution}
		For $n = 1, m = 0$, the split energy levels are given by
		$$E = -\mathrm{Ry} \pm \frac{eB\hbar}{2mc}$$
		With $B = 1000\text{ kG}$, $eB\hbar/2mc \approx 5.8\cdot10^{-3}\text{ eV}$. With $\mathrm{Ry} \approx 13.6\text{ eV}$, we can say that the relative size of the level splitting is approximately $0.0043\%$.
		
		Returning to the Hamiltonian given in Eq. (14.4.11), with $\mathbf{A} = B({-y}\mathbf{i} + x\mathbf{j})/2$, we see that the quadratic term becomes
		$$H_\text{quad} = \frac{e^2|\mathbf{A}|^2}{2mc^2} = \frac{e^2B^2r^2}{8mc^2}$$
		If $r = a_0$, its relative energy compared to the linear term can be estimated as
		$$\frac{H_\text{quad}}{H_\text{int}} = \frac{e^2B^2a_0^2}{8mc^2}\frac{2mc}{eB\hbar} = \frac{eBa_0^2}{4\hbar c}$$
		which, when $B = 1000\text{ kG}$, is approximately $10^{-4}$. In order for our linear approximation to be reasonably accurate, this must be much less than unity, giving the condition
		$$B \ll \frac{4\hbar c}{ea_0^2}$$
	\end{solution}
	
	\question A beam of spin-1/2 particles moving along the $y$ axis goes through two collinear SG apparatuses, both with lower beams blocked. The first has its $\mathbf{B}$ field along the $z$ axis and the second has its $\mathbf{B}$ field along the $x$ axis (i.e., is obtained by rotating the first by an angle $\pi/2$ about the $y$ axis). What fraction of particles leaving the first will exit the second? If a third filter that transmits only spin up along the $z$ axis is introduced, what fraction of particles leaving the first will exit the third? If the middle filter transmits both spins up and down (no blocking) the $x$ axis, but the last one transmits only spin down the $z$ axis, what fraction of particles leaving the first will leave the last?
	
	\begin{solution}
		Before answering this question, it is useful to find the explicit eigenvectors of each Pauli matrix. These are
		\begin{align*}
			|z_+\rangle &= \begin{bmatrix}1 \\ 0\end{bmatrix} \\
			|z_-\rangle &= \begin{bmatrix}0 \\ 1\end{bmatrix} \\
			|x_+\rangle &= \frac{1}{2^{1/2}}\begin{bmatrix}1 \\ 1\end{bmatrix} \\
			|x_-\rangle &= \frac{1}{2^{1/2}}\begin{bmatrix}1 \\ -1\end{bmatrix} \\
			|y_+\rangle &= \frac{1}{2^{1/2}}\begin{bmatrix}i \\ 1\end{bmatrix} \\
			|y_-\rangle &= \frac{1}{2^{1/2}}\begin{bmatrix}1 \\ i\end{bmatrix}
		\end{align*}
		All particles exiting the first apparatus will be in the $|z_+\rangle$ state. Since this can be written as $|z_+\rangle = 2^{-1/2}(|x_+\rangle + |x_-\rangle)$ and the second apparatus only lets through particles in the state $|x_+\rangle$, only $1/2$ of the particles exiting the first SG machine will exit the second.
		
		At the tail end of the second machine, we only have particles in the $|x_+\rangle$ state. This can be written as $|x_+\rangle = 2^{-1/2}(|z_+\rangle + |z_-\rangle)$, and so half of all particles exiting the second machine will exit the third, i.e. $1/4$ of the particles exiting the first apparatus come out the tail end of the third SG machine.
		
		For the last scenario, half of the particles exiting the first machine come out of the second in the $|x_+\rangle$ state, while the other half exit the second SG machine in the $|x_-\rangle$ state. Since both of these states are an equal (in magnitude) combination of $|z_+\rangle$ and $|z_-\rangle$ states, half of each o these populations exit the third machine in the $|z_-\rangle$ state, i.e. $1/2$ of all particles exiting the first machine (in the $|z_+\rangle$ state) exit the third apparatus in the $|z_-\rangle$ state.
		
	\end{solution}
	
	\question A beam of spin-1 particles, moving along the $y$ axis, is incident on two collinear SG apparatuses, the first with $\mathbf{B}$ along the $z$ axis and the second with $\mathbf{B}$ along the $z'$ axis, which lies in the $x$-$z$ plane at an angle $\theta$ relative to the $z$ axis. Both apparatuses transmit only the uppermost beams. What fraction leaving the first will pass the second?
	
	\begin{solution}
		For spin-1 particles, the $z$ and $x$ spin operators are (from Chapter 12)
		\begin{gather*}
			S_z = \hbar\begin{bmatrix}1 & 0 & 0 \\ 0 & 0 & 0 \\ 0 & 0 & -1\end{bmatrix} \\
			S_x = \frac{\hbar}{2^{1/2}}\begin{bmatrix}0 & 1 & 0 \\ 1 & 0 & 1 \\ 0 & 1 & 0\end{bmatrix}
		\end{gather*}
		and so, with $\hat{n} = \hat{\mathbf{i}}\sin\theta + \hat{\mathbf{k}}\cos\theta$, a general spin operator in the $x$-$z$ plane is described by
		\begin{align*}
			\hat{n}\cdot\mathbf{S} &= \frac{\hbar}{2^{1/2}}\begin{bmatrix}2^{1/2}\cos\theta & \sin\theta & 0 \\
			\sin\theta & 0 & \sin\theta  \\
			0 & \sin\theta & -2^{1/2}\cos\theta\end{bmatrix}
		\end{align*}
		Using a CAS system and performing some simplifications by hand, the eigenvectors of this matrix are
		\begin{align*}
			|n_1\rangle &= \begin{bmatrix}\cos^2\tfrac{\theta}{2} \\
			2^{-1/2}\sin\theta \\
			\sin^2\tfrac{\theta}{2}\end{bmatrix} \\
			|n_0\rangle &= \frac{1}{2^{1/2}}\begin{bmatrix}
				-\sin\theta \\ 2^{1/2}\cos\theta \\ \sin\theta
			\end{bmatrix} \\
			|n_{-1}\rangle &= \begin{bmatrix}
				\sin^2\tfrac{\theta}{2} \\
				-2^{-1/2}\sin\theta \\
				\cos^2\tfrac{\theta}{2}
			\end{bmatrix}
		\end{align*}
		When the particles exit the first apparatus, they are all in the state $$|z_1\rangle = \begin{bmatrix}1 \\ 0 \\ 0 \end{bmatrix}$$
		We can rewrite this state in terms of the eigenvectors of the spin operator corresponding to the second apparatus via $|\psi'\rangle = \sum_i\langle z'_i|z_1\rangle|z'_i\rangle$, upon which we see that the probability of finding a particle in the state $|z'_1\rangle$ is given by $|\langle z'_1|z_1\rangle|^2$, or
		$$|\langle z'_1|z_1\rangle|^2 = \cos^4\theta/2$$
	\end{solution}
	\end{questions}
\end{document}