\documentclass[../principles-of-quantum-mechanics.tex]{subfiles}

\begin{document}
	\printanswers
	
	\section{Addition of Angular Momentum}
	
	\begin{questions}
		\setcounter{subsection}{0}
		\setcounter{question}{0}
		\subsection{A Simple Example}
		\question Derive Eqs. (15.1.10) and (15.1.11). It might help to use
		$$\mathbf{S}_1\cdot\mathbf{S}_2 = S_{1z}S_{2z} + \tfrac{1}{2}(S_{1+}S_{2-} + S_{1-}S_{2+})$$
		
		\question In addition to the Coulomb interaction, there exists another, called the hyperfine interaction, between the electron and proton in the hydrogen atom. The Hamiltonian describing this interaction, which is due to the magnetic moments of the two particles, is
		$$H_{hf} = A\mathbf{S}_1\cdot\mathbf{S}_2\qquad(A > 0)$$
		(This formula assumes the orbital state of the electron is $|1, 0, 0\rangle$.) The total Hamiltonian is thus the Coulomb Hamiltonian plus $H_{hf}$.
		
		(1) Show that $H_{hf}$ splits the ground state into two levels:
		\begin{align*}
			E_+ &= -\mathrm{Ry} + \frac{\hbar^2A}{4} \\
			E_- &= -\mathrm{Ry} - \frac{3\hbar^2A}{4}
		\end{align*}
		and that corresponding states are triplets and singlet, respectively.
		
		(2) Try to estimate the frequency of the emitted radiation as the atom jumps from the triplet to the singlet. To do so, you may assume that the electron and proton are two dipoles $\mu_e$ and $\mu_p$ separated by a distance $a_0$, with an interaction energy of the order
		$$\mathcal{H}_hf \simeq \frac{\mu_e\cdot\mu_p}{a_0^3}$$
		Show that this implies that the constant in Eq. (15.1.22) is
		$$A \sim \frac{2e}{2mc}\frac{(5.6)e}{2Mc}\frac{1}{a_0^3}$$
		(where $5.6$ is the $g$ factor for the proton), and that
		$$\Delta E = E_+ - E_- = A\hbar^2$$
		is a correction of order $(m/M)\alpha^2$ relative to the ground-state energy. Estimate that the frequency of emitted radiation is a few tens of centimeters, using the mnemonics from Chapter 13. The measured value is $21.4\,\text{cm}$. This radiation, called the \textit{21-cm line}, is a way to detect hydrogen in other parts of the universe.
		
		(3) Estimate the probability ratio $P(\text{triplet})/P(\text{singlet})$ of hydrogen atoms in thermal equilibrium at room temperature.

		\setcounter{subsection}{1}
		\setcounter{question}{0}
		\subsection{The General Problem}
		
		\question (1) Verify that $|j_1j_1, j_2j_2\rangle$ is indeed a state of $j = j_1 + j_2$ by letting $J^2 = J_1^2 + J_2^2 + 2J_{1z}J_{2z} + J_{1+}J_{2-} + J_{1-}J_{2+}$ act on it.
		
		(2) (optional) Verify that the right-hand side of Eq. (15.2.8) indeed has angular momentum $j = j_1 + j_2 - 1$.
		
		\begin{solution}
			For (1), we compute
			\begin{align*}
				J^2|j_1j_1, j_2j_2\rangle &= (J_1^2 + J_2^2 + 2J_{1z}J_{2z} + J_{1+}J_{2-} + J_{1-}J_{2+})|j_1j_1, j_2j_2\rangle \\
				&= (\hbar^2j_1(j_1 + 1) + \hbar^2j_2(j_2 + 1) + 2\hbar^2j_1j_2)|j_1j_1, j_2j_2\rangle \\
				&= \hbar^2(j_1^2 + j_2^2 + 2j_1j_2 + j_1 + j_2)|j_1j_1, j_2j_2\rangle \\
				&= \hbar^2(j_1 + j_2)(j_1 + j_2 + 1)|j_1j_1, j_2j_2\rangle
			\end{align*}
			where $J_{1+}J_{2-}|j_1j_1, j_2j_2\rangle$ and $J_{1-}J_{2+}|j_1j_1, j_2j_2\rangle$ both evaluate to $0$ on account of both particles being in a state of maximum $m_j$. The end result, being $J^2|\psi\rangle = \hbar^2 j(j + 1)|\psi\rangle$, is consistent with a system in the state $j = j_1 + j_2$.
			
			For (2), we first look at $J^2$ applied to $|j_1j_1, j_2(j_2 - 1)\rangle$
			\begin{align*}
				J^2|j_1j_1, j_2(j_2 - 1)\rangle =\,&(J_1^2 + J_2^2 + 2J_{1z}J_{2z} + J_{1+}J_{2-} + J_{1-}J_{2+})|j_1j_1, j_2(j_2 - 1)\rangle \\
				=\,&(\hbar^2j_1(j_1 + 1) + \hbar^2j_2(j_2 + 1) + 2\hbar^2j_1(j_2 - 1))|j_1j_1, j_2(j_2 - 1)\rangle \\
				&+ \hbar(2j_2)^{1/2}J_{1-}|j_1j_1, j_2j_2\rangle \\
				=\,&\hbar^2(j_1^2 + j_2^2 + 2j_1(j_2 - 1) + j_1 + j_2)|j_1j_1, j_2(j_2 - 1)\rangle + 2\hbar^2(j_1j_2)^{1/2}|j_1(j_1 - 1), j_2j_2\rangle
			\end{align*}
			and then $J^2$ applied to $|j_1(j_1 - 1), j_2j_2\rangle$,
			\begin{align*}
				J^2|j_1(j_1 - 1), j_2j_2\rangle =\,&(J_1^2 + J_2^2 + 2J_{1z}J_{2z} + J_{1+}J_{2-} + J_{1-}J_{2+})|j_1(j_1 - 1), j_2j_2\rangle \\
				=\,&(\hbar^2j_1(j_1 + 1) + \hbar^2j_2(j_2 + 1) + 2\hbar^2(j_1 - 1)j_2)|j_1(j_1 - 1), j_2j_2\rangle \\
				&+ \hbar(2j_2)^{1/2}J_{1+}|j_1(j_1 - 1), j_2(j_2 - 1)\rangle \\
				=\,&\hbar^2(j_1^2 + j_2^2 + 2(j_1 - 1)j_2 + j_1 + j_2)|j_1(j_1 - 1), j_2j_2\rangle \\
				&+ 2\hbar^2(j_1j_2)^{1/2}|j_1j_1, j_2(j_2 - 1)\rangle
			\end{align*}
			Multiplying the first of these by $[j_1/(j_1 + j_2)]^{1/2}$ and the second by $[j_2/(j_1 + j_2)]^{1/2}$, we find for the $|j_1j_1, j_2(j_2 - 1)\rangle$ term in $J^2|j_1 + j_2 - 1, j_1 + j_2 - 1\rangle$
			\begin{align*}
				&\hbar^2\Big(j_1^2 + j_2^2 + 2j_1(j_2 - 1) + j_1 + j_2 - 2j_2\Big)\Big(\frac{j_1}{j_1 + j_2}\Big)^{1/2}|j_1j_1, j_2(j_2 - 1)\rangle \\
				=\,&\hbar^2\Big(j_1^2 + j_2^2 + 2j_1j_2 - j_1 - j_2\Big)\Big(\frac{j_1}{j_1 + j_2}\Big)^{1/2}|j_1j_1, j_2(j_2 - 1)\rangle \\
				=\,&\hbar^2(j_1 + j_2 - 1)(j_1 + j_2)\Big(\frac{j_1}{j_1 + j_2}\Big)^{1/2}|j_1j_1, j_2(j_2 - 1)\rangle
			\end{align*}
			while, for the $|j_1(j_1 - 1), j_2j_2\rangle$ term, we have
			\begin{align*}
				&{-\hbar^2}\Big(j_1^2 + j_2^2 + 2(j_1 - 1)j_2 + j_1 + j_2 - 2j_1\Big)\Big(\frac{j_2}{j_1 + j_2}\Big)^{1/2}|j_1(j_1 - 1), j_2j_2\rangle \\
				=\,&{-\hbar^2}\Big(j_1^2 + j_2^2 + 2j_1j_2 - j_1 - j_2\Big)\Big(\frac{j_2}{j_1 + j_2}\Big)^{1/2}|j_1(j_1 - 1), j_2j_2\rangle \\
				=\,&{-\hbar^2}(j_1 + j_2 - 1)(j_1 + j_2)\Big(\frac{j_2}{j_1 + j_2}\Big)^{1/2}|j_1(j_1 - 1), j_2j_2\rangle
			\end{align*}
			Putting these results together, we see
			$$J^2|j_1 + j_2 - 1, j_1 + j_2 - 1\rangle = \hbar^2(j_1 + j_2 - 1)(j_1 + j_2)|j_1 + j_2 - 1, j_1 + j_2 - 1\rangle$$
			i.e. the system is in a state with $j = j_1 + j_2 - 1$.
		\end{solution}
		
		\question Find the CG coefficients of
		
		(1) $\tfrac{1}{2}\otimes 1 = \tfrac{3}{2}\oplus\tfrac{1}{2}$
		
		(2) $1\otimes 1 = 2\oplus 1 \oplus 0$
		
		\begin{solution}
			For convenience, we remind ourselves that
			$$S_\pm|j, m\rangle = \hbar[(j\mp m)(j\pm m + 1)]^{1/2}|j, m\pm 1\rangle$$
			Now, the product space of $\tfrac{1}{2}\otimes 1$ permits two total-$j$ values, $\tfrac{3}{2}$ and $\tfrac{1}{2}$. Let us start with $j = \tfrac{3}{2}$. The only way to write the $j = m = \tfrac{3}{2}$ state is
			$$|\tfrac{3}{2}, \tfrac{3}{2}\rangle = |\tfrac{1}{2}\tfrac{1}{2}, 11\rangle$$
			To find the state of next highest $m$, we apply
			\begin{align*}
				S_-|\tfrac{3}{2},\tfrac{3}{2}\rangle = 3^{1/2}\hbar|\tfrac{3}{2},\tfrac{1}{2}\rangle &= (S_{1-} + S_{2-})|\tfrac{1}{2}\tfrac{1}{2},11\rangle \\
				&= \hbar|\tfrac{1}{2}(-\tfrac{1}{2}), 11\rangle + 2^{1/2}\hbar|\tfrac{1}{2}\tfrac{1}{2},10\rangle
			\end{align*}
			and so
			$$|\tfrac{3}{2},\tfrac{1}{2}\rangle = (\tfrac{1}{3})^{1/2}|\tfrac{1}{2}(-\tfrac{1}{2}),11\rangle + (\tfrac{2}{3})^{1/2}|\tfrac{1}{2}\tfrac{1}{2},10\rangle$$
			Continuing in this manner,
			\begin{align*}
				S_-|\tfrac{3}{2},\tfrac{1}{2}\rangle = 2\hbar|\tfrac{3}{2},-\tfrac{1}{2}\rangle &= (S_{1-} + S_{2-})\Big((\tfrac{1}{3})^{1/2}|\tfrac{1}{2}(-\tfrac{1}{2}),11\rangle + (\tfrac{2}{3})^{1/2}|\tfrac{1}{2}\tfrac{1}{2},10\rangle\Big) \\
				&= (\tfrac{2}{3})^{1/2}\hbar|\tfrac{1}{2}(-\tfrac{1}{2}),10\rangle + (\tfrac{2}{3})^{1/2}|\tfrac{1}{2}(-\tfrac{1}{2}),10\rangle + (\tfrac{4}{3})^{1/2}\hbar|\tfrac{1}{2}\tfrac{1}{2},1({-1})\rangle \\
				&= (\tfrac{8}{3})^{1/2}\hbar|\tfrac{1}{2}(-\tfrac{1}{2}),10\rangle + (\tfrac{4}{3})^{1/2}\hbar|\tfrac{1}{2}\tfrac{1}{2},1(-1)\rangle
			\end{align*}
			or
			$$|\tfrac{3}{2},{-\tfrac{1}{2}}\rangle = (\tfrac{2}{3})^{1/2}|\tfrac{1}{2}(-\tfrac{1}{2}),10\rangle + (\tfrac{1}{3})^{1/2}|\tfrac{1}{2}\tfrac{1}{2},1(-1)\rangle$$
			and clearly
			$$|\tfrac{3}{2},{-\tfrac{3}{2}}\rangle = |\tfrac{1}{2}(-\tfrac{1}{2}),1(-1)\rangle$$
			For the $|\tfrac{1}{2},\tfrac{1}{2}\rangle$ state, we know it must be a combination of $|\tfrac{1}{2}(-\tfrac{1}{2}), 11\rangle$ and $|\tfrac{1}{2}\tfrac{1}{2}, 10\rangle$. If the coefficients weighting these are $\alpha$ and $\beta$ respectively, then the orthogonality of $|\tfrac{1}{2},\tfrac{1}{2}\rangle$ to $|\tfrac{3}{2},\tfrac{1}{2}\rangle$ and the normalization of $|\tfrac{1}{2},\tfrac{1}{2}\rangle$ imposes
			\begin{align*}
				\alpha + 2^{1/2}\beta &= 0 \\
				\alpha^2 + \beta^2 &= 1
			\end{align*}
			or $\alpha = -(2/3)^{1/2}$ and $\beta = (1/3)^{1/2}$. That is,
			$$|\tfrac{1}{2},\tfrac{1}{2}\rangle = -(\tfrac{2}{3})^{1/2}|\tfrac{1}{2}(-\tfrac{1}{2}), 11\rangle + (\tfrac{1}{3})^{1/2}|\tfrac{1}{2}\tfrac{1}{2}, 10\rangle$$
			and so
			\begin{align*}
				S_-|\tfrac{1}{2},\tfrac{1}{2}\rangle = \hbar|\tfrac{1}{2}(-\tfrac{1}{2})\rangle &= (S_{1-} + S_{2-})\Big({-(\tfrac{2}{3})^{1/2}}|\tfrac{1}{2}(-\tfrac{1}{2}), 11\rangle + (\tfrac{1}{3})^{1/2}|\tfrac{1}{2}\tfrac{1}{2}, 10\rangle\Big) \\
				&= -(\tfrac{4}{3})^{1/2}\hbar|\tfrac{1}{2}(-\tfrac{1}{2}),10\rangle + (\tfrac{1}{3})^{1/2}\hbar|\tfrac{1}{2}(-\tfrac{1}{2}),10\rangle + (\tfrac{2}{3})^{1/2}\hbar|\tfrac{1}{2}\tfrac{1}{2},1(-1)\rangle \\
				&= -(\tfrac{1}{3})^{1/2}\hbar |\tfrac{1}{2}(-\tfrac{1}{2}),10\rangle + (\tfrac{2}{3})^{1/2}\hbar|\tfrac{1}{2}\tfrac{1}{2},1(-1)\rangle
			\end{align*}
			or
			$$|\tfrac{1}{2}(-\tfrac{1}{2})\rangle = -(\tfrac{1}{3})^{1/2} |\tfrac{1}{2}(-\tfrac{1}{2}),10\rangle + (\tfrac{2}{3})^{1/2}|\tfrac{1}{2}\tfrac{1}{2},1(-1)\rangle$$
			The second part of this problem follows the same process. Since it is rather tedious, we simply display the results,
			\begin{align*}
				|2, 2\rangle &= |11,11\rangle \\
				|2, 1\rangle &= (\tfrac{1}{2})^{1/2}|10,11\rangle + (\tfrac{1}{2})^{1/2}|11, 10\rangle \\
				|2, 0\rangle &= (\tfrac{1}{6})^{1/2}|1(-1),11\rangle + (\tfrac{2}{3})^{1/2}|10,10\rangle + (\tfrac{1}{6})^{1/2}|11, 1(-1)\rangle \\
				|2, {-1}\rangle &= (\tfrac{1}{2})^{1/2}|10,1(-1)\rangle + (\tfrac{1}{2})^{1/2}|1(-1),10\rangle \\
				|2, {-2}\rangle &= |1(-1),1(-1)\rangle \\
				|1, 1\rangle &= (\tfrac{1}{2})^{1/2}|11, 10\rangle - (\tfrac{1}{2})^{1/2}|10, 11\rangle \\
				|1, 0\rangle &= (\tfrac{1}{2})^{1/2}|11, 1(-1)\rangle - (\tfrac{1}{2})^{1/2}|1(-1), 11\rangle \\
				|1, {-1}\rangle &= (\tfrac{1}{2})^{1/2}|10, 1(-1)\rangle - (\tfrac{1}{2})^{1/2}|1(-1), 10\rangle \\
				|0, 0\rangle &= (\tfrac{1}{3})^{1/2}|1(-1), 11\rangle - (\tfrac{1}{3})^{1/2}|10, 10\rangle + (\tfrac{1}{3})^{1/2}|11, 1(-1)\rangle
			\end{align*}
		\end{solution}
		
		\question Argue that $\tfrac{1}{2} \otimes \tfrac{1}{2} \otimes \tfrac{1}{2} = \tfrac{3}{2}\oplus \tfrac{1}{2} \oplus \tfrac{1}{2}$.
		
		\begin{solution}
			Given that the tensor product and direct sum are associative and distributive,
			\begin{align*}
				\tfrac{1}{2}\otimes\tfrac{1}{2}\otimes\tfrac{1}{2} &= (\tfrac{1}{2}\otimes\tfrac{1}{2})\otimes\tfrac{1}{2} \\
				&= (1 \oplus 0)\otimes\tfrac{1}{2} \\
				&= (1 \otimes \tfrac{1}{2}) \oplus (0 \otimes\tfrac{1}{2}) \\
				&= (\tfrac{3}{2}\oplus \tfrac{1}{2}) \oplus \tfrac{1}{2} \\
				&= \tfrac{3}{2} \oplus \tfrac{1}{2} \oplus \tfrac{1}{2}
			\end{align*}
		\end{solution}
	
		\question Derive Eqs. (15.2.19) and (15.2.20).
		
		\begin{solution}
			Applying $J^2$ to $|j = l \pm 1/2, m\rangle$ gives
			\begin{align*}
				&(L^2 + S^2 + 2L_zS_z + L_-S_+ + L_+S_-)\big(\alpha|l, m-1/2;1/2,1/2\rangle + \beta|l,m+1/2;1/2,{-1/2}\rangle\big) \\
				=\,&\alpha\big(\hbar^2l(l + 1) + (3/4)\hbar^2 + 2\hbar^2(m - 1/2)(1/2)\big)|l, m-1/2;1/2,1/2\rangle \\
				&+ \beta\big(\hbar^2l(l + 1) + (3/4)\hbar^2 + 2\hbar^2(m + 1/2)(-1/2)\big)|l, m+1/2;1/2,{-1/2}\rangle \\
				&+ \alpha\hbar^2[(l + m + 1/2)(l - m + 1/2)]^{1/2}|l, m + 1/2; 1/2, {-1/2}\rangle \\
				&+ \beta\hbar^2[(l + m + 1/2)(l - m + 1/2)]^{1/2}|l,m - 1/2; 1/2, 1/2\rangle \\
				=\,&\alpha\hbar^2\Big(l(l + 1) + \tfrac{3}{4} + 2(m - \tfrac{1}{2})(\tfrac{1}{2}) + \frac{\beta}{\alpha}(l + m + \tfrac{1}{2})^{1/2}(l - m + \tfrac{1}{2})^{1/2}\Big)|l, m - 1/2; 1/2, 1/2\rangle \\
				&+ \beta\hbar^2\Big(l(l + 1) + \tfrac{3}{4} + 2(m + \tfrac{1}{2})({-\tfrac{1}{2}}) + \frac{\alpha}{\beta}(l + m + \tfrac{1}{2})^{1/2}(l - m + \tfrac{1}{2})^{1/2}\Big)|l, m + 1/2; 1/2, {-1/2}\rangle \\
				=\,&\alpha\hbar^2\Big(l^2 + l + m + \tfrac{1}{4} + \frac{\beta}{\alpha}(l + m + \tfrac{1}{2})^{1/2}(l - m + \tfrac{1}{2})^{1/2}\Big)|l, m - 1/2; 1/2, 1/2\rangle \\
				&+ \beta\hbar^2\Big(l^2 + l - m + \tfrac{1}{4} + \frac{\alpha}{\beta}(l + m + \tfrac{1}{2})^{1/2}(l - m + \tfrac{1}{2})^{1/2}\Big)|l, m + 1/2; 1/2, {-1/2}\rangle
			\end{align*}
			In order for this to equal $\hbar^2(l + 1/2)(l + 3/2)|j = l + 1/2, m\rangle$, both parenthesized terms must be equal to $l^2 + 2l + 3/4$, or
			$$l + m + \tfrac{1}{2} = \frac{\alpha}{\beta}(l + m + \tfrac{1}{2})^{1/2}(l - m + \tfrac{1}{2})^{1/2}$$
			i.e.
			$$\frac{\alpha}{\beta} = \Big(\frac{l + m + \tfrac{1}{2}}{l - m + \tfrac{1}{2}}\Big)^{1/2}$$
			Putting this into the normalization condition yields
			\begin{align*}
				\alpha^2 + \beta^2 &= \Big(\frac{l + m + \tfrac{1}{2}}{l - m + \tfrac{1}{2}} + 1\Big)\beta^2 \\
				&= \Big(\frac{2l + 1}{l - m + \tfrac{1}{2}}\Big)\beta^2 \\
				&= 1
			\end{align*}
			or
			$$|\beta| = \Big(\frac{l - m + \tfrac{1}{2}}{2l + 1}\Big)^{1/2}$$
			and
			$$|\alpha| = \Big(\frac{l + m + \tfrac{1}{2}}{2l + 1}\Big)$$
			Imposing our sign convention gives us back (15.5.20).
		\end{solution}
	
		\question (1) Show that $\mathbb{P}_1 = \tfrac{3}{4}I + (\mathbf{S}_1\cdot\mathbf{S}_2)/\hbar^2$ and $\mathbb{P}_0 = \tfrac{1}{4}I - (\mathbf{S}_1\cdot\mathbf{S}_2)/\hbar^2$ are projection operators, i.e., obey $\mathbb{P}_i\mathbb{P}_j = \delta_{ij}\mathbb{P}_j$ [use Eq. (14.3.39)].
		
		(2) Show that these project into the spin-1 and spin-0 spaces in $\tfrac{1}{2}\otimes\tfrac{1}{2}=1\oplus0$.
		
		\begin{solution}
			Squaring each operator and expanding the $(\mathbf{S}_1\cdot\mathbf{S}_2)^2$ term with the help of Eq. (14.3.39) gives
			\begin{align*}
				\mathbb{P}_1^2 &= \frac{9}{16}I + \frac{3}{2\hbar^2}(\mathbf{S}_1\cdot\mathbf{S}_2) + \frac{1}{\hbar^4}(\mathbf{S}_1\cdot\mathbf{S}_2)^2 \\
				&= \frac{9}{16}I + \frac{3}{2\hbar^2}(\mathbf{S}_1\cdot\mathbf{S}_2) + \frac{1}{16}(\boldsymbol{\sigma}_1\cdot\boldsymbol{\sigma}_1) I + \frac{i}{16}(\boldsymbol{\sigma}_1\times\boldsymbol{\sigma}_1)\cdot\boldsymbol{\sigma}_2 \\
				&= \frac{9}{16}I + \frac{3}{2\hbar^2}(\mathbf{S}_1\cdot\mathbf{S}_2) + \frac{3}{16}I - \frac{1}{8}(\boldsymbol{\sigma}_1\cdot\boldsymbol{\sigma_2}) \\
				&= \frac{3}{4}I + \frac{3}{2\hbar^2}(\mathbf{S}_1\cdot\mathbf{S}_2) - \frac{1}{2\hbar^2}(\mathbf{S}_1\cdot\mathbf{S}_2) \\
				&= \frac{3}{4}I + \frac{1}{\hbar^2}(\mathbf{S}_1\cdot\mathbf{S}_2) \\
				&= \mathbb{P}_1 \\
				\mathbb{P}_0^2 &= \frac{1}{16}I - \frac{1}{2\hbar^2}(\mathbf{S}_1\cdot\mathbf{S}_2) + \frac{1}{\hbar^4}(\mathbf{S}_1\cdot\mathbf{S}_2)^2 \\
				&= \frac{1}{16}I - \frac{1}{2\hbar^2}(\mathbf{S}_1\cdot\mathbf{S}_2) + \frac{3}{16}I - \frac{1}{2\hbar^2}(\mathbf{S}_1\cdot\mathbf{S}_2) \\
				&= \frac{1}{4}I - \frac{1}{\hbar^2}(\mathbf{S}_1\cdot\mathbf{S}_2) \\
				&= \mathbb{P}_0^2
			\end{align*}
			while the product of the two operators is
			\begin{align*}
				\mathbb{P}_0\mathbb{P}_1 = \mathbb{P}_1\mathbb{P}_0 &= \frac{3}{16}I + \frac{1}{4\hbar^2}(\mathbf{S}_1\cdot\mathbf{S}_2) - \frac{3}{4\hbar^2}(\mathbf{S}_1\cdot\mathbf{S}_2) - \frac{1}{\hbar^4}(\mathbf{S}_1\cdot\mathbf{S}_2)^2 \\
				&= \frac{3}{16}I - \frac{1}{2\hbar^2}(\mathbf{S}_1\cdot\mathbf{S}_2) - \frac{3}{16}I + \frac{1}{2\hbar^2}(\mathbf{S}_1\cdot\mathbf{S}_2) \\
				&= 0
			\end{align*}
			i.e. $\mathbb{P}_i\mathbb{P}_j = \delta_{ij}\mathbb{P}_j$. To get a sense of how these operators act on the sum space, we write $\mathbf{S} = \mathbf{S}_1 + \mathbf{S}_2$. Since $S^2 = S_1^2 + 2(\mathbf{S}_1\cdot\mathbf{S}_2) + S_2^2 = \tfrac{3}{2}\hbar^2I + 2(\mathbf{S}_1\cdot\mathbf{S}_2)$, we can express the projection operators as
			\begin{align*}
				\mathbb{P}_1 &= \frac{3}{4}I + \frac{1}{2\hbar^2}(S^2 - \tfrac{3}{2}\hbar^2I) =\frac{1}{2\hbar^2}S^2 \\
				\mathbb{P}_0 &= \frac{1}{4}I - \frac{1}{2\hbar^2}(S^2 - \tfrac{3}{2}\hbar^2I) = I - \frac{1}{2\hbar^2}S^2
			\end{align*}
			Since $S^2|j = 1, m\rangle = 2\hbar^2|j = 1, m\rangle$ and $S^2|j = 0, m\rangle = 0$, we see that $\mathbb{P}_1$ projects a given state onto the $j = 1$ subspace, while $\mathbb{P}_0$ subtracts off the $j = 1$ subspace portion of a state, i.e. it projects a state onto the $j = 0$ subspace.
		\end{solution}
	
		\question Construct the projection operators $\mathbb{P}_\pm$ for the $j = l \pm 1/2$ subspaces in the addition $\mathbf{L}+\mathbf{S}=\mathbf{J}$.
		
		\begin{solution}
			Taking a cue from the previous problem, we compute $\mathbf{L}\cdot\mathbf{S}$ from
			\begin{align*}
				J^2 = \hbar^2j(j + 1)I = \hbar^2(l \pm \tfrac{1}{2})(l \pm \tfrac{1}{2} + 1)I &= L^2 + S^2 + 2(\mathbf{L}\cdot\mathbf{S}) \\
				&= \hbar^2l(l + 1)I + \tfrac{3}{4}\hbar^2I + 2(\mathbf{L}\cdot\mathbf{S})
			\end{align*}
			or
			\begin{align*}
				(\mathbf{L}\cdot\mathbf{S}) &= \frac{\hbar^2}{2}\Big((l \pm \tfrac{1}{2})(l \pm \tfrac{1}{2} + 1) - l(l + 1) - \tfrac{3}{4}\Big)I \\
				&= \frac{\hbar^2}{2}\Big(l^2 \pm \tfrac{1}{2}l + l \pm \tfrac{1}{2}l + \tfrac{1}{4} \pm \tfrac{1}{2} - l^2 - l - \tfrac{3}{4}\Big)I \\
				&= \pm\frac{\hbar^2}{2}(l + \tfrac{1}{2} \mp \tfrac{1}{2})I
			\end{align*}
			so
			\begin{align*}
				(\mathbf{L}\cdot\mathbf{S})|j=l+\tfrac{1}{2}\rangle &= l\frac{\hbar^2}{2}|j = l + \tfrac{1}{2}\rangle \\
				(\mathbf{L}\cdot\mathbf{S})|j=l-\tfrac{1}{2}\rangle &= -(l + 1)\frac{\hbar^2}{2}|j = l - \tfrac{1}{2}\rangle
			\end{align*}
			from which we can immediately write
			\begin{align*}
				\mathbb{P}_- &= \frac{Il\hbar^2 - 2(\mathbf{L}\cdot\mathbf{S})}{l\hbar^2 + \hbar^2(l + 1)} = \frac{l}{2l + 1}I - \frac{2}{\hbar^2}\frac{(\mathbf{L}\cdot\mathbf{S})}{2l + 1} \\
				\mathbb{P}_+ &= \frac{I(l + 1)\hbar^2 + 2(\mathbf{L}\cdot\mathbf{S})}{(l + 1)\hbar^2 + \hbar^2l} = \frac{l + 1}{2l + 1}I + \frac{2}{\hbar^2}\frac{(\mathbf{L}\cdot\mathbf{S})}{2l + 1}
			\end{align*}
		\end{solution}
		
		\question Show that when we add $j_1$ to $j_1$, the states with $j = 2j_1$ are symmetric. Show that the states with $j = 2j_1 - 1$ are antisymmetric. (Argue for the symmetry of the top states and show that lowering does not change symmetry.) This pattern of alternating symmetry continues as $j$ decreases, but is harder to prove.
		
		\begin{solution}
			We know that the highest $m$ states of $j = 2j_1$ and $j = 2j_1 - 1$ are given by
			\begin{gather*}
				|2j_1, 2j_1\rangle = |j_1j_1, j_1j_1\rangle \\
				|2j_1 - 1, 2j_1 - 1\rangle = 2^{-1/2}|j_1j_1, j_1(j_1-1)\rangle - 2^{-1/2}|j_1(j_1 - 1), j_1j_1\rangle
			\end{gather*}
			The first of these is symmetric with respect to the two product states; the second is antisymmetric. To find the lower $m$ states, we simply apply $J_- = J_{1-}^1 + J_{1-}^2$ to each in turn, where the superscripts here denote the first and second state.Since this is symmetric, its application to the topmost state will result in a symmetric, lower $m$ state, and so on.
			
			Similarly, since the $j = 2j_1 - 1$ state is antisymmetric, the application of the symmetric $J_-$ operator will result in another antisymmetric state, this time with lower $m$, and so on. 
		\end{solution}
	
		\setcounter{subsection}{2}
		\setcounter{question}{0}
		\subsection{Irreducible Tensor Operators}
		\question (1) Show that Eq. (15.3.11) follows from Eq. (15.3.10) when one considers infinitesimal rotations. (Hint: $D^{(k)}_{q'q}=\langle kq'|I - (i\delta\boldsymbol{\theta}\cdot\mathbf{J})/\hbar|kq\rangle$. Pick $\delta\boldsymbol{\theta}$ along, say, the $x$ direction and then generalize the result to the other directions.)
		
		(2) Verify that the spherical tensor $V_1^q$ constructed out of $\mathbf{V}$ as in Eq. (15.3.15) obeys Eq. (15.3.11).
		
		\question It is claimed that $\sum_q(-1)^qS^q_kT_k^{(-1)}$ is a scalar operator.
		
		(1) For $k = 1$ verify that this is just $\mathbf{S}\cdot\mathbf{T}$.
		
		(2) Prove it in general by considering its response to a rotation. [Hint: $D^{(j)}_{-m,-m'}=(-1)^{m-m'}(D^{(j)}_{m,m'})^*$.]
		
		\question (1) Using $\langle jj|jj, 10\rangle = [j/(j + 1)]^{1/2}$ show that
		$$\langle \alpha j||J_1||\alpha'j'\rangle = \delta_{\alpha\alpha'}\delta_{jj'}\hbar[j(j + 1)]^{1/2}$$
		
		(2) Using $\mathbf{J}\cdot\mathbf{A} = J_zA_z + \tfrac{1}{2}(J_-A_+ + J_+A_-)$ (where $A_{\pm} = A_x \pm iA_y$) argue that
		$$\langle\alpha'jm'|\mathbf{J}\cdot\mathbf{A}|\alpha jm\rangle = c\langle \alpha'j||A||\alpha j\rangle$$
		where $c$ is a constant independent of $\alpha$, $\alpha'$ and $\mathbf{A}$. Show that $c = \hbar[j(j + 1)]^{1/2}\delta_{m,m'}$.
		
		(3) Using the above, show that
		$$\langle \alpha'jm'|A^q|\alpha jm\rangle = \frac{\langle \alpha' jm|\mathbf{J}\cdot\mathbf{A}|\alpha jm\rangle}{\hbar^2j(j + 1)}\langle jm'|J^q|jm\rangle$$
		
		\question (1) Consider a system whose angular momentum consists of two parts $\mathbf{J}_1$ and $\mathbf{J}_2$ and whose magnetic moment is
		$$\mu = \gamma_1\mathbf{J}_1 + \gamma_2\mathbf{J}_2$$
		In a state $|jm, j_1j_2\rangle$ show, using Eq. (15.3.19), that
		\begin{align*}
			\langle \mu_x\rangle &= \langle \mu_y\rangle = 0 \\
			\langle \mu_z\rangle &= m\hbar\Big[\frac{\gamma_1 + \gamma_2}{2} + \frac{(\gamma_1 - \gamma_2)}{2}\frac{j_1(j_1 + 1) - j_2(j_2 + 1)}{j(j + 1)}\Big]
		\end{align*}
	
		(2) Apply this to the problem of a proton ($g = 5.6$) in a $^2P_{1/2}$ state and show that $\langle\mu_z\rangle = \pm 0.26$ nuclear magnetons.
		
		(3) For an electron in a $^2P_{1/2}$ state show that $\langle \mu_z\rangle = \pm\tfrac{1}{3}$ Bohr magnetons.
		
		\question Show that $\langle jm|T_k^q|jm\rangle = 0$ if $k > 2j$.
	\end{questions}
\end{document}