\documentclass[../principles-of-quantum-mechanics.tex]{subfiles}

\begin{document}
	\printanswers
	
	\section{Review of Classical Mechanics}
	
	\begin{questions}
		
		\question Consider the following system, called a \textit{harmonic oscillator}. The block has a mass $m$ and lies on a frictionless surface. The spring has a force constant $k$. Write the Lagrangian and get the equation of motion.
		
		\begin{solution}
			From the diagram, we can immediately write
			\[
				\mathcal{L} = T - V = \frac{1}{2}m\dot{x}^2 - \frac{1}{2}kx^2.
			\]
			This gives us a conjugate momentum and generalized force of
			\[
				\frac{\partial\mathcal{L}}{\partial{\dot{x}}} = m\dot{x} \qquad \frac{\partial\mathcal{L}}{\partial{x}} = -kx
			\]
			which can be combined to give the equation of motion,
			\[
				\frac{\mathrm{d}}{\mathrm{d}t}\frac{\partial\mathcal{L}}{\partial{\dot{x}}} = m\ddot{x} = -kx = \frac{\partial\mathcal{L}}{\partial{x}}
			\]
		\end{solution}
	
		\question Do the same for the coupled-mass problem discussed at the end of Section 1.8. Compare the equations of motion with Eqs. (1.8.24) and (1.8.25).
		
		\begin{solution}
			Returning to Figure 1.5, we can write
			\begin{align*}
				T &= \frac{1}{2}m\dot{x}_1^2 + \frac{1}{2}m\dot{x}_2^2 \\
				&= \frac{1}{2}m(\dot{x}_1^2 + \dot{x}_2^2) \\
				V &= \frac{1}{2}kx_1^2 + \frac{1}{2}kx_2^2 + \frac{1}{2}k(x_2-x_1)^2 \\
				&= k(x_1^2 - x_1x_2 + x_2^2) \\
			\end{align*}
			and hence
			\[
				\mathcal{L} = T - V = \frac{1}{2}m(\dot{x}_1^2 + \dot{x}_2^2) - k(x_1^2-x_1x_2+x_2^2)
			\]
			Feeding the above into the Euler-Lagrange equations gives us
			\begin{align*}
				\frac{\mathrm{d}}{\mathrm{d}t}\frac{\partial\mathcal{L}}{\partial\dot{x}_1} &= \frac{\mathrm{d}}{\mathrm{d}t}\Big(m\dot{x}_1\Big) = m\ddot{x}_1 = -2kx_1 + kx_2 = \frac{\partial\mathcal{L}}{\partial{x}_1} \\
				\frac{\mathrm{d}}{\mathrm{d}t}\frac{\partial\mathcal{L}}{\partial\dot{x}_2} &= \frac{\mathrm{d}}{\mathrm{d}t}\Big(m\dot{x}_2\Big) = m\ddot{x}_2 = x_1 - 2kx_2 = \frac{\partial\mathcal{L}}{\partial{x}_2}
			\end{align*}
			which is exactly Eqs. (1.8.24) and (1.8.25).
		\end{solution}
	
		\question A particle of mass $m$ moves in three dimensions under a potential $V(r, \theta, \phi) = V(r)$. Write its $\mathcal{L}$ and find the equations of motion.
		
		\begin{solution}
			Using a similar geometric argument to Shankar, we see that the distance covered by a particle in time $\Delta{t}$ is
			\[
				\mathrm{d}S = [(\mathrm{d}r)^2 + (r\sin(\theta)\mathrm{d}\phi)^2 + (r\mathrm{d}\theta)^2]
			\]
			where $\phi$ is the azimuthal angle and $\theta$ is the inclination. This gives us a squared velocity of
			\[
				v^2 = \dot{r}^2 + r^2\sin^2(\theta)\dot{\phi}^2 + r^2\dot{\theta}^2
			\]
			and thus a Lagrangian of
			\[
				\mathcal{L} = T - V = \frac{1}{2}m(\dot{r}^2 + r^2\sin^2(\theta)\dot{\phi}^2 + r^2\dot{\theta}^2) - V(r)
			\]
			The equations of motion for this particle are given by
			\begin{align*}
			\frac{\mathrm{d}}{\mathrm{d}t}\frac{\partial\mathcal{L}}{\partial\dot{r}} &= \frac{\mathrm{d}}{\mathrm{d}t}\Big(m\dot{r}\Big) = m\ddot{r} = mr\sin^2(\theta)\dot{\phi}^2 + mr\dot{\theta}^2 - \frac{\partial{V(r)}}{\partial{r}} = \frac{\partial\mathcal{L}}{\partial{r}} \\
			\frac{\mathrm{d}}{\mathrm{d}t}\frac{\partial\mathcal{L}}{\partial\dot{\phi}} &= \frac{\mathrm{d}}{\mathrm{d}t}\Big(mr^2\sin^2(\theta)\dot{\phi}\Big) = 2mr\dot{r}\sin^2(\theta)\dot{\phi} + 2mr^2\sin(\theta)\cos(\theta)\dot{\theta}\dot{\phi} + mr^2\sin^2(\theta)\ddot{\phi} = 0 = \frac{\partial\mathcal{L}}{\partial{\phi}} \\
			\frac{\mathrm{d}}{\mathrm{d}t}\frac{\partial\mathcal{L}}{\partial\dot{\theta}} &= \frac{\mathrm{d}}{\mathrm{d}t}\Big(mr^2\dot{\theta}\Big) = 2mr\dot{r}\dot{\theta} + mr^2\ddot{\theta} = mr^2\sin(\theta)\cos(\theta)\dot{\phi}^2 =  \frac{\partial\mathcal{L}}{\partial{\theta}}
			\end{align*}
			Simplifying, these become
			\begin{align*}
				m\ddot{r} &= mr(\dot{\phi}^2\sin^2\theta + \dot{\theta}^2) - \frac{\partial{V}(r)}{\partial{r}} \\
				m\ddot{\phi} &= -2m\dot{\phi}\Big(\frac{\dot{r}}{r} - \dot{\theta}\cot\theta\Big) \\
				m\ddot{\theta} &= m\Big(\dot{\phi}^2\sin\theta\cos\theta  - 2\frac{\dot{r}}{r}\dot{\theta}\Big)
			\end{align*}
		\end{solution}
	
		\question Derive Eq. (2.3.6) from (2.3.5) by changing variables.
		
		\begin{solution}
			This is a straightforward exercise of algebra,
			\begin{align*}
			\mathcal{L} &= \frac{1}{2}m_1|\dot{\mathbf{r}}_1|^2 + \frac{1}{2}|\dot{\mathbf{r}}_2|^2 - V(\mathbf{r}_1 - \mathbf{r}_2) \\
			&= \frac{1}{2}m_1\Big|\dot{\mathbf{r}}_{\mathrm{CM}} + \frac{m_2\dot{\mathbf{r}}}{m_1+m_2}\Big|^2 + \frac{1}{2}m_2\Big|\dot{\mathbf{r}}_{\mathrm{CM}} - \frac{m_1\dot{\mathbf{r}}}{m_1+m_2}\Big|^2 - V(\mathbf{r}) \\
			&= \frac{1}{2}m_1\Big(|\dot{\mathbf{r}}_{\mathrm{CM}}|^2 + \frac{2m_2\dot{\mathbf{r}}\cdot\dot{\mathbf{r}}_{\mathrm{CM}}}{m_1+m_2} + \frac{m_2^2|\dot{\mathbf{r}}|^2}{(m_1+m_2)^2}\Big) +  \frac{1}{2}m_2\Big(|\dot{\mathbf{r}}_{\mathrm{CM}}|^2 - \frac{2m_1\dot{\mathbf{r}}\cdot\dot{\mathbf{r}}_{\mathrm{CM}}}{m_1+m_2} + \frac{m_1^2|\dot{\mathbf{r}}|^2}{(m_1+m_2)^2}\Big) - V(\mathbf{r}) \\
			&= \frac{1}{2}(m_1+m_2)|\dot{\mathbf{r}}_{\mathrm{CM}}|^2 + \frac{m_1m_2\dot{\mathbf{r}}\cdot\dot{\mathbf{r}}_{\mathrm{CM}}}{m_1+m_2} - \frac{m_1m_2\dot{\mathbf{r}}\cdot\dot{\mathbf{r}}_{\mathrm{CM}}}{m_1+m_2} + \frac{1}{2}\frac{m_1m_2^2 + m_1^2m_2}{(m_1+m_2)^2}|\dot{\mathbf{r}}|^2 - V(r) \\
			&= \frac{1}{2}(m_1+m_2)|\dot{\mathbf{r}}_{\mathrm{CM}}|^2 + \frac{1}{2}\frac{m_1m_2(m_1 + m_2)}{(m_1+m_2)^2}|\dot{\mathbf{r}}|^2 - V(r) \\
			&= \frac{1}{2}(m_1+m_2)|\dot{\mathbf{r}}_{\mathrm{CM}}|^2 + \frac{1}{2}\frac{m_1m_2}{m_1+m_2}|\dot{\mathbf{r}}|^2 - V(r)
			\end{align*}
		\end{solution}
	
		\question Show that if $T = \sum_i\sum_j{T_{ij}}(q)\dot{q}^i\dot{q}^j$, where $\dot{q}$'s are generalized velocities, $\sum_ip_i\dot{q}^i=2T$.
		
		\begin{solution}
			Assuming the Lagrangian built from $T$ contains a potential term independent of velocity, the conjugate momentum to $q$ is
			\begin{align*}
				p_i &= \frac{\partial{L}}{\partial{\dot{q}^i}} \\
				&= \frac{\partial}{\partial{\dot{q}^i}}\Big(\sum_j\sum_kT_{kj}(q)\dot{q}^j\dot{q}^k\Big) \\
				&= \sum_j\sum_kT_{kj}(q)\frac{\partial{\dot{q}^j}}{\partial{\dot{q}^i}}\dot{q}^k + \sum_j\sum_kT_{kj}(q)\dot{q}^j\frac{\partial{\dot{q}^k}}{\partial{\dot{q}^i}} \\
				&= \sum_j\sum_kT_{kj}(q)\delta^j_i\dot{q}^k + \sum_j\sum_kT_{kj}(q)\dot{q}^j\delta^k_i \\
				&= \sum_kT_{ki}(q)\dot{q}^k + \sum_jT_{ij}(q)\dot{q}^j \\
				&= 2\sum_jT_{ij}(q)\dot{q}^j
			\end{align*}
			where we have assumed that $T_{ij}$ is symmetric in the last equality. From the above, we see
			\[
				\sum_ip_i\dot{q}^i = 2\sum_i\sum_jT_{ij}(q)\dot{q}^i\dot{q}^j = 2T.
			\]
		\end{solution}
	
		\question Using the conservation of energy, show that the trajectories in phase space for the oscillator are ellipses of the form $(x/a)^2 + (p/b)^2 = 1$, where $a^2 = 2E/k$ and $b^2 = 2mE$.
		
		\begin{solution}
			The Hamiltonian (and thus the energy) for the classical harmonic oscillator is given by
			\[
				\mathcal{H} = \frac{p^2}{2m} + \frac{k}{2}x^2 \equiv E.
			\]
			Since energy is conserved, $\partial\mathcal{H}/\partial{t} = 0$ and we can divide by $E$ (a constant) to get
			\[
				\Big(\frac{p}{\sqrt{2mE}}\Big)^2 + \Big(\frac{x}{\sqrt{2E/k}}\Big)^2 = 1,
			\]
			or, defining $a^2 = 2E/k$ and $b^2 = 2mE$,
			\[
				\Big(\frac{x}{a}\Big)^2 + \Big(\frac{p}{b}\Big)^2 = 1.
			\]
		\end{solution}
		
		\question Solve Exercise 2.1.2 using the Hamiltonian formalism.
		
		\begin{solution}
			In this simple case, we can make the replacement $\dot{x}_i^2 \to p_i^2/m^2$ and flip the sign of $V$ in the found $\mathcal{L}$ to arrive at
			\[
				\mathcal{H} = T + V = \frac{p_1^2 + p_2^2}{2m} + k(x_1^2 - x_1x_2 + x_2^2)
			\]
			To obtain the dynamical equations for the system, we compute
			\begin{align*}
				\frac{\partial\mathcal{H}}{\partial p_i} = \frac{p_i}{m} = \dot{x}_i \quad \text{and} \quad -\frac{\partial\mathcal{H}}{\partial x_i} = kx_j - 2kx_i = \dot{p}_i
			\end{align*}
			where $j \neq i$. Taking the time derivative of $\partial\mathcal{H}/\partial p_i$ allows us to substitute the resulting equations into $-\partial\mathcal{H}/\partial{x}_i$ to obtain
			\[
				\ddot{x}_i = \frac{k}{m}(x_j - 2x_i)
			\]
			which is exactly what we found in Exercise 2.1.2.
		\end{solution}
		
		\question Show that $\mathcal{H}$ corresponding to $\mathcal{L}$ in Eq. (2.3.6) is $\mathcal{H}=|\mathbf{p}_{\text{CM}}|^2/2M + |\mathbf{p}|^2/2\mu + V(\mathbf{r})$, where $M$ is the total mass, $\mu$ is the reduced mass, $\mathbf{p}_{\text{CM}}$ and $\mathbf{p}$ are the momenta conjugate to $\mathbf{r}_{\text{CM}}$ and $\mathbf{r}$, respectively.
		
		\begin{solution}
			Starting from the Lagrangian,
			\begin{align*}
				\mathcal{L} &= \frac{1}{2}(m_1+m_2)|\dot{\mathbf{r}}_{\mathrm{CM}}|^2 + \frac{1}{2}\frac{m_1m_2}{m_1+m_2}|\dot{\mathbf{r}}|^2 - V(r), \\
				&= \frac{M}{2}|\dot{\mathbf{r}}_{\mathrm{CM}}|^2 + \frac{\mu}{2}|\dot{\mathbf{r}}|^2 - V(r),
			\end{align*}
			we can find the conjugate momenta to $\mathbf{r}$ and $\mathbf{r}_{\mathrm{CM}}$ via
			\begin{align*}
				\mathbf{p}_{\mathrm{CM}} &= \frac{\partial\mathcal{L}}{\partial\dot{\mathbf{r}}_{\mathrm{CM}}} = M\dot{\mathbf{r}}_{\mathrm{CM}}, \\
				\mathbf{p} &= \frac{\partial\mathcal{L}}{\partial\dot{\mathbf{r}}} = \mu\dot{\mathbf{r}}.
			\end{align*}
			Performing the necessary Legendre transform reveals
			\begin{align*}
				\mathcal{H} &= \mathbf{p}\cdot\dot{\mathbf{r}} + \mathbf{p}_{\mathrm{CM}}\cdot\dot{\mathbf{r}}_{\mathrm{CM}} - \mathcal{L} \\
				&= \frac{|\mathbf{p}|^2}{\mu} + \frac{|\mathbf{p}_{\mathrm{CM}}|^2}{M} - \Big(\frac{M}{2}\frac{|\mathbf{p}_{\mathrm{CM}}|^2}{M^2} + \frac{\mu}{2}\frac{|\mathbf{p}|^2}{\mu^2} - V(r)\Big) \\
				&= \frac{|\mathbf{p}|^2}{2\mu} + \frac{|\mathbf{p}_{\mathrm{CM}}|^2}{2M} + V(r)
			\end{align*}
		\end{solution}
	\end{questions}
\end{document}