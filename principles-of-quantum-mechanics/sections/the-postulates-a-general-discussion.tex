\documentclass[../principles-of-quantum-mechanics.tex]{subfiles}

\begin{document}
	\printanswers
	
	\setcounter{section}{3}
	\section{The Postulates---a General Discussion}

	\begin{questions}
		\setcounter{subsection}{1}
		\setcounter{question}{0}
		\subsection{Discussion of Postulates I-III}
		\question (\textit{Very Important}). Consider the following operators on a Hilbert space $\mathbb{V}^3(C)$:
		$$L_x = \frac{1}{2^{1/2}}\begin{bmatrix}
			0 & 1 & 0 \\
			1 & 0 & 1 \\
			0 & 1 & 0
		\end{bmatrix}, \quad L_y = \frac{1}{2^{1/2}}\begin{bmatrix}
			0 & -i & 0 \\
			i & 0 & -i \\
			0 & i & 0
		\end{bmatrix}, \quad L_z = \begin{bmatrix}
			1 & 0 & 0 \\
			0 & 0 & 0 \\
			0 & 0 & -1
		\end{bmatrix}$$
		\begin{parts}
			\part What are the possible values one can obtain if $L_z$ is measured?
			\part Take the state in which $L_z = 1$. In this state what are $\langle L_x\rangle$, $\langle L_x^2\rangle$, and $\Delta L_x$?
			\part Find the normalized eigenstates and the eigenvalues of $L_x$ in the $L_z$ basis.
			\part If the particle is in the state with $L_z = {-1}$, and $L_x$ is measured, what are the possible outcomes and their probabilities?
			\part Consider the state
			$$|\psi\rangle = \begin{bmatrix}1/2 \\ 1/2 \\ 1/2^{1/2}\end{bmatrix}$$
			in the $L_z$ basis. If $L_z^2$ is measured in this state and a result $+1$ is obtained, what is the state after measurement? How probable was this result? If $L_z$ is measured immediately afterwards, what are the outcomes and respective probabilities?
			\part A particle is in a state for which the probabilites are $P(L_z=1) = 1/4$, $P(L_z=0)=1/2$, and $P(L_z={-1})=1/4$. Convince yourself that the most general, normalized state with this property is
			$$|\psi\rangle = \frac{e^{i\delta_1}}{2}|L_z=1\rangle + \frac{e^{i\delta_2}}{2^{1/2}}|L_z=0\rangle + \frac{e^{i\delta_3}}{2}|L_z = {-1}\rangle$$
			It was stated earlier on that if $|\psi\rangle$ is a normalized state then the state $e^{i\theta}|\psi\rangle$ is a physically equivalent normalized state. Does this mean that the factors $e^{i\delta_i}$ multiplying the $L_z$ eigenstates are irrelevant? [Calculate for example $P(L_x = 0)$.]
		\end{parts}
	\begin{solution}
		\begin{parts}
			\part These are the eigenvalues of $L_z$,
			$$L_z = 1, 0, \,\text{or}\,\,{-1}$$
			\part $L_z = 1$ when the state has the form
			$$|\psi\rangle = \begin{bmatrix}1 \\ 0 \\ 0\end{bmatrix}$$
			Using this, we calculate
			\begin{align*}
				\langle L_x \rangle &= \langle \psi | L_x |\psi \rangle \\
				&= \frac{1}{2^{1/2}}\begin{bmatrix}1 & 0 & 0\end{bmatrix}\begin{bmatrix}0 & 1 & 0 \\ 1 & 0 & 1 \\ 0 & 1 & 0\end{bmatrix}\begin{bmatrix}1 \\ 0 \\ 0\end{bmatrix} \\
				&= \frac{1}{2^{1/2}}\begin{bmatrix}1 & 0 & 0\end{bmatrix}\begin{bmatrix}0 \\ 1 \\ 0\end{bmatrix} \\
				&= 0 \\
				\langle L_x^2\rangle &= \langle\psi| L_x^2|\psi\rangle \\
				&= \frac{1}{2}\begin{bmatrix}1 & 0 & 0\end{bmatrix}\begin{bmatrix}0 & 1 & 0 \\ 1 & 0 & 1 \\ 0 & 1 & 0\end{bmatrix}\begin{bmatrix}0 & 1 & 0 \\ 1 & 0 & 1 \\ 0 & 1 & 0\end{bmatrix}\begin{bmatrix}1 \\ 0 \\ 0\end{bmatrix} \\
				&= \frac{1}{2}\begin{bmatrix}0 & 1 & 0\end{bmatrix}\begin{bmatrix}0 \\ 1 \\ 0\end{bmatrix} \\
				&= \frac{1}{2} \\
				\Delta L_x &= \langle \psi |(L_x - \langle L_x\rangle)^2|\psi\rangle \\
				&= \langle \psi | L_x^2 - L_x\langle L_x\rangle - \langle L_x\rangle L_x + \langle L_x\rangle^2|\psi\rangle \\
				&= \langle \psi | L_x^2 | \psi \rangle - 2\langle L_x\rangle\langle \psi | L_x |\psi\rangle + \langle L_x\rangle^2\langle \psi | \psi \rangle \\
				&= \langle L_x^2\rangle - \langle L_x\rangle^2 \\
				&= \langle L_x^2\rangle \\
				&= \frac{1}{2}
			\end{align*}
			\part Noting that we are already in the $L_z$ basis ($L_z$ is diagonal), this amounts to simply finding the eigenvalues and eigenvectors of $L_x$. We do so by examining the characteristic equation,
			\begin{align*}
				\det(\lambda I - L_x) &= \begin{vmatrix}\lambda & -\frac{1}{2^{1/2}} & 0 \\ -\frac{1}{2^{1/2}} & \lambda & -\frac{1}{2^{1/2}} \\ 0 & -\frac{1}{2^{1/2}} & \lambda\end{vmatrix} \\
				&= \lambda\Big(\lambda^2 - \frac{1}{2}\Big) + \frac{1}{2^{1/2}}\Big({-\frac{\lambda}{2^{1/2}}}\Big) \\
				&= \lambda^3 - \frac{\lambda}{2} - \frac{\lambda}{2} \\
				&= \lambda(\lambda^2 - 1)
			\end{align*}
			i.e. the eigenvalues of $L_x$ are
			$$L_x = 1, 0, \,\text{and}\,\,{-1}$$
			The eigenstate $|\lambda_i\rangle$ is found by solving $\ker(\lambda_i I - L_x)$,
			\begin{align*}
				(I - L_x)|L_x = 1\rangle &= \begin{bmatrix}
					1 & -\frac{1}{2^{1/2}} & 0 \\ -\frac{1}{2^{1/2}} & 1 & -\frac{1}{2^{1/2}} \\ 0 & -\frac{1}{2^{1/2}} & 1
					\end{bmatrix}|L_x=1\rangle = 0 \\
				{-L_x}|L_x=0\rangle &= \begin{bmatrix}
					0 & -\frac{1}{2^{1/2}} & 0 \\ -\frac{1}{2^{1/2}} & 0 & -\frac{1}{2^{1/2}} \\ 0 & -\frac{1}{2^{1/2}} & 0
				\end{bmatrix}|L_x = 0\rangle = 0 \\
				({-I} - L_x)|L_x = {-1}\rangle &= \begin{bmatrix}
					{-1} & -\frac{1}{2^{1/2}} & 0 \\ -\frac{1}{2^{1/2}} & {-1} & -\frac{1}{2^{1/2}} \\ 0 & -\frac{1}{2^{1/2}} & {-1}
				\end{bmatrix}|L_x = {-1}\rangle = 0
			\end{align*}
			By inspection, we can write
			\begin{align*}
				|L_x=1\rangle &\propto \begin{bmatrix}\frac{1}{2^{1/2}} \\ 1 \\ \frac{1}{2^{1/2}}\end{bmatrix} \\
				|L_x=0\rangle &\propto \begin{bmatrix}1 \\ 0 \\ {-1}\end{bmatrix} \\
				|L_x={-1}\rangle &\propto \begin{bmatrix}{-\frac{1}{2^{1/2}}} \\ 1 \\ -\frac{1}{2^{1/2}}\end{bmatrix}
			\end{align*}
			Normalizing these yields
			\begin{align*}
				|L_x=1\rangle &= \frac{1}{2}\!\!\begin{bmatrix}1 \\ \sqrt{2} \\ 1\end{bmatrix} \\
				|L_x=0\rangle &= \frac{1}{2^{1/2}}\!\!\begin{bmatrix}1 \\ 0 \\ {-1}\end{bmatrix} \\
				|L_x={-1}\rangle &= \frac{1}{2}\begin{bmatrix}{-1} \\ \sqrt{2} \\ {-1}\end{bmatrix}
			\end{align*}
		\part $L_z = {-1}$ when the state has the form
		$$|\psi\rangle = \begin{bmatrix}0 \\ 0 \\ 1 \end{bmatrix}$$
		The only possible measurements for $L_x$ are its eigenvalues, $1$, $0$, and ${-1}$. We project the state onto the $L_x$ eigenbasis via the projection operator $P_{L_x} = \sum_k|\lambda_k\rangle\langle\lambda_k|$
		\begin{align*}
			|\psi\rangle &= \sum_k|\lambda_k\rangle\langle\lambda_k|\psi\rangle \\
			&= \langle L_x=1|\psi\rangle |L_x=1\rangle + \langle L_x=0|\psi\rangle |L_x=0\rangle + \langle L_x={-1}|\psi\rangle|L_x = {-1}\rangle \\
			&= \frac{1}{2}\begin{bmatrix}1 & \sqrt{2} & 1\end{bmatrix}\begin{bmatrix}0 \\ 0 \\ 1\end{bmatrix}|L_x = 1\rangle + \frac{1}{2^{1/2}}\begin{bmatrix}1 & 0 & {-1}\end{bmatrix}\begin{bmatrix}0 \\ 0 \\ 1\end{bmatrix}|L_x = 0\rangle + \frac{1}{2}\begin{bmatrix}{-1} & \sqrt{2} & {-1}\end{bmatrix}\begin{bmatrix}0 \\ 0 \\ 1\end{bmatrix}|L_x = {-1}\rangle \\
			&= \frac{1}{2}|L_x = 1\rangle - \frac{1}{2^{1/2}}|L_x = 0\rangle - \frac{1}{2}|L_x = {-1}\rangle
		\end{align*}
		From this, we can read off the probabilities of the various measurements,
		\begin{align*}
			P(L_x = 1) &= \Big(\frac{1}{2}\Big)^2 = \frac{1}{4} \\
			P(L_x = 0) &= \Big({-\frac{1}{2^{1/2}}}\Big)^2 = \frac{1}{2} \\
			P(L_x = {-1}) &= \Big({-\frac{1}{2}}\Big)^2 = \frac{1}{4}
		\end{align*}
		\part $L_z^2$ will have eigenvalues that are the square of the eigenvalues of $L_z$, i.e.
		$$L_z^2 = 1\,\,\text{or}\,\,0.$$
		Since
		$$L_z^2 = \begin{bmatrix}1 & 0 & 0 \\ 0 & 0 & 0 \\ 0 & 0 & 1\end{bmatrix},$$
		a measurement of $1$ projects the state $|\psi\rangle$ onto the eigenspace associated with this eigenvalue,
		$$|1\rangle = \begin{bmatrix}1 \\ 0 \\ 0\end{bmatrix}, \quad |1'\rangle = \begin{bmatrix}0 \\ 0 \\ 1\end{bmatrix},$$
		and so the new (normalized) state is
		$$|\psi'\rangle = \frac{2}{3^{1/2}}\begin{bmatrix}1/2 \\ 0 \\ 1/2^{1/2}\end{bmatrix}$$
		The probability of this result is
		$$|\langle 1|\psi\rangle|^2 + |\langle 1'|\psi\rangle|^2 = \frac{1}{4} + \frac{1}{2} = \frac{3}{4}.$$
		If we measure $L_z$ in this new state, we find two possibilities: $L_z = -1, 1$ (the $L_z = 0$ state is excluded). The first has the associated state and probability
		$$|\psi_{+1}\rangle = \begin{bmatrix}1 \\ 0 \\ 0\end{bmatrix}, \quad P = |\langle {+1}|\psi'\rangle|^2 = \Big(\frac{2}{3^{1/2}}\cdot\frac{1}{2}\Big)^2 = \frac{1}{3}$$
		while the second has the associated state and probability
		$$|\psi_{-1}\rangle = \begin{bmatrix}0 \\ 0 \\ 1\end{bmatrix}, \quad P = |\langle{-1}|\psi'\rangle|^2 = \Big(\frac{2}{3^{1/2}}\cdot\frac{1}{2^{1/2}}\Big)^2 = \frac{2}{3}$$
		
		\part This is clear from how we measure the probability of a state. If the probability of finding $|\psi\rangle$ in the eigenstate $|\alpha\rangle$ is $p$ and $\langle\alpha|\psi\rangle = \psi_\alpha \in \mathbb{C}$, then 
		\begin{align*}
			p &= |\langle \alpha|\psi\rangle|^2 \\
			  &= \langle\alpha|\psi\rangle\langle\psi|\alpha\rangle \\
			  &= \psi_\alpha \cdot \bar{\psi}_\alpha \\
			  &= re^{i\delta}\cdot re^{-i\delta} \\
			  &= r^2
		\end{align*}
		where we have written $\psi_\alpha$ in polar form as $re^{i\delta}$. From this, it is clear that the most general form for $\psi_\alpha$ that guarantees $p = |\langle\alpha|\psi\rangle|^2$ is
		$$\psi_\alpha = \sqrt{p}\,e^{i\delta}$$
		Despite the seeming arbitrariness of $\delta$, it plays a role when performing a measurement with an operator whose eigenstates are different than the currently used eigenstates. To take the given example, 
		\begin{align*}
			P(L_x = 0) &= |\langle L_x = 0|\psi\rangle|^2 \\
			&= \Big|\frac{1}{2^{1/2}}\Big(\frac{e^{i\delta_1}}{2} - \frac{e^{i\delta_3}}{2}\Big)\Big|^2 \\
			&= \frac{1}{8}(e^{i\delta_1} - e^{i\delta_3})(e^{-i\delta_1} - e^{-i\delta_3}) \\
			&= \frac{1}{8}(2 - e^{i(\delta_1 - \delta_3)} - e^{-i(\delta_1 - \delta_3)}) \\
			&= \frac{1}{4}(1 - \cos(\delta_1 - \delta_3))
		\end{align*}
		From this, we see that it is the \textit{relative} phase values that significantly alter our measurements. This is similar to what we find in electromagnetism: the relative phase of two waves creates measurable effects, while the phase of a single, monochromatic plane wave is undetectable.
		\end{parts}
	\end{solution}

	\question Show that for a real wave function $\psi(x)$, the expectation value of momentum $\langle P\rangle = 0$. (Hint: Show that the probabilities for the momenta $\pm p$ are equal.) Generalize this result to the case $\psi = c\psi_r$, where $\psi_r$ is real and $c$ an arbitrary (real or complex) constant. (Recall that $|\psi\rangle$ and $\alpha|\psi\rangle$ are physically equivalent.)
	
	\begin{solution}
		Following the hint, we compute
		\begin{align*}
			P(P = {+p}) &= |\langle p |\psi\rangle|^2 \\
			&= \langle \psi |p \rangle \langle p |\psi\rangle \\
			&= \Big(\int_{-\infty}^{\infty}\langle \psi|x\rangle\langle x|p\rangle\,\mathrm{d}x\Big)\Big(\int_{-\infty}^{\infty}\langle p|x'\rangle\langle x'|\psi\rangle\,\mathrm{d}x'\Big) \\
			&= \Big(\int_{-\infty}^{\infty}\bar{c}\psi_r(x)\frac{e^{ipx/\hbar}}{(2\pi\hbar)^{1/2}}\,\mathrm{d}x\Big)\Big(\int_{-\infty}^{\infty}\frac{e^{-ipx'/\hbar}}{(2\pi\hbar)^{1/2}}c\psi_r(x')\,\mathrm{d}x'\Big) \\
			&= \frac{|c|^2}{2\pi\hbar}\Big(\int_{-\infty}^{\infty}\psi_r(x)e^{ipx/\hbar}\,\mathrm{d}x\Big)\Big(\int_{-\infty}^{\infty}\psi_r(x')e^{-ipx'/\hbar}\,\mathrm{d}x'\Big)
		\end{align*}
		Since this is invariant under the exchange $p \to -p$, the probability of finding a particle at momentum $p$ is exactly the same as finding that same particle at momentum $-p$. But any probability distribution $f_P(p)$ with this property is even about $p = 0$, and so $pf_P(p)$ is odd and $\langle P\rangle = 0$.
	\end{solution}
	
	\question Show that if $\psi(x)$ has mean momentum $\langle P\rangle$, $e^{ip_0 x/\hbar}\psi(x)$ has mean momentum $\langle P\rangle + p_0$.
	
	\begin{solution}
		Assuming the state $|\psi'\rangle$ is normalized, we have
		\begin{align*}
			\langle \psi'|P|\psi'\rangle &= \int_{-\infty}^{\infty}\int_{-\infty}^{\infty}\langle\psi'|x\rangle\langle x|P|x'\rangle\langle x'|\psi'\rangle\,\mathrm{d}x\,\mathrm{d}x' \\
			&= \int_{-\infty}^{\infty}\int_{-\infty}^{\infty}e^{-ip_0 x/\hbar}\bar{\psi}(x)\Big({-i\hbar\delta(x - x')}\frac{\mathrm{d}}{\mathrm{d}x'}\Big)e^{ip_0x'/\hbar}\psi(x')\,\mathrm{d}x\,\mathrm{d}x' \\
			&= \int_{-\infty}^{\infty}e^{-ip_0 x/\hbar}\bar{\psi}(x)\Big({-i\hbar}\frac{\mathrm{d}}{\mathrm{d}x}\Big)e^{ip_0x/\hbar}\psi(x)\,\mathrm{d}x \\
			&= \int_{-\infty}^{\infty} e^{-ip_0x/\hbar}\bar{\psi}(x)\Big({-i\hbar}\cdot\frac{i p_0}{\hbar}\cdot e^{ip_0 x/\hbar}\psi(x) - i\hbar\cdot e^{ip_0x/\hbar}\frac{\mathrm{d}}{\mathrm{d}x}\psi(x)\Big)\,\mathrm{d}x \\
			&= \int_{-\infty}^{\infty}\bar{\psi}(x)\Big(p_0 - i\hbar\frac{\mathrm{d}}{\mathrm{d}x}\Big)\psi(x)\,\mathrm{d}x \\
			&= p_0\int_{-\infty}^{\infty}|\psi(x)|^2\,\mathrm{d}x + \int_{-\infty}^{\infty}\bar{\psi}(x)\Big({-i\hbar}\frac{\mathrm{d}}{\mathrm{d}x}\Big)\psi(x)\,\mathrm{d}x \\
			&= p_0 + \langle P\rangle
		\end{align*}
		where in the last step we used the fact that $\langle P\rangle$ in the position basis is
		$$\langle P\rangle = \int_{-\infty}^{\infty}\bar{\psi}(x)\Big({-i\hbar}\frac{\mathrm{d}}{\mathrm{d}x}\Big)\psi(x)\,\mathrm{d}x.$$
		Another way to show this relation is to recall that position space and momentum space are dual in the Fourier sense, and so a multiplication by $e^{ip_0x}$ (or $e^{ipx_0}$) in one space becomes addition by $p_0$ (or $x_0$) in the other.
	\end{solution}
	\end{questions}
\end{document}