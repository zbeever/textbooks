\documentclass{exam}

\usepackage{scrextend}
\usepackage{tikz, tkz-euclide}
\usepackage{amsmath}
\usepackage{amssymb}
\usepackage{commath}
\usepackage{sectsty}
\usepackage{array}
\usepackage{etoolbox}
\usepackage{color}
\usepackage{subfiles}
\usepackage{wasysym}

\renewcommand{\solutiontitle}{}
\AtBeginEnvironment{solution}{}
\renewcommand{\thequestion}{\thesection.\arabic{question}}

\begin{document}
	\printanswers
	\begin{questions}
		\question Following the same line of argument that led to Eq. (5.28), show that the error on the integral evaluated using Simpson's rule is given, to leading order $n$, by Eq. (5.29).
		
		\begin{solution}
			Let us define two step sizes, $h_1 = (b-a)/N$ and $h_2 = (b-a)/2N = h_1/2$. Integrating using Simpson's rule gives us an approximation error of order $h^4$, so we may write
			\[
				I = I_1 + ch_1^4
			\]
			where $I$ is the true value of the integral, $I_1$ is the numerical value obtained using Simpson's rule, and $ch_1^4$ is the error ($c$ is an unknown constant).
			
			We may do the same with the smaller step size, writing $I = I_2 + ch_2^4$. Equating these two gives us
			\[
				I_1 + ch_1^4 = I_2 + ch_2^4.
			\]
			Using $h_1 = 2h_2$, we may rearrange this to find
			\[
				I_2 - I_1 = 15ch_2^4.
			\]
			Identifying the error on our second evaluation as $\epsilon_2 = ch_2^4$, we see
			\[
				\epsilon_2 = \frac{1}{15}(I_2-I_1).
			\]
		\end{solution}
	\end{questions}
\end{document}