\documentclass[../road-to-reality.tex]{subfiles}

\begin{document}
\printanswers

\section{The quantum particle}

\begin{questions}

\question Show that $(1 + D^2)\cos{x}=0$ and $(1 + D^2)\sin{x}=0$ (referring
  to formulae in $\S6.5$, if you need them).

  \begin{solution}
    Both cosine and sine are eigenfunctions of $D^2$ with eigenvaues of $-1$,
    and so
    \begin{align*}
      (1 + D^2)\cos{x} &= (1 - 1)\cos{x} = 0, \\
      (1 + D^2)\sin{x} &= (1 - 1)\sin{x} = 0.
    \end{align*}
  \end{solution}

\question Taking note of Exercise [21.1], find the \textit{general} solution
  of $(1 + D^2)y=x^5$, providing a proof that your solution is, in fact, the
  most general.

  \begin{solution}
    To find the most general solution of this equation, we must add to the
    particular solution the homogenous one, i.e. the set of solutions satisfying
    \[
      (1 + D^2)y = 0.
    \]
    This set is exactly that given in the solution to the above problem,
    yielding a general solution of
    \[
      y = x^5 - 20x^3+120x + A\cos{x} + B\sin{x}.
    \]
  \end{solution}

\question See if you can explain why the procedure given in the text isses
  most of the solutions given in Exercise [21.2]. Can you suggest a modified
  general procedure which finds them all? \textit{Hint}: To what extent does `$1
  - D^2 + D^4 - D^6 + \dots$' really satisfy the requirements for an inverse to
  $1+D^2$? Try acting on $(1 + D^2)\cos{x}$ with this infinite expression.

\question Why?

  \begin{solution}
    When $a\neq{b}$, we have $D_bx^a=x^aD_b$, i.e. $x$ and $D$ commute. This is
    patently obvious, though, as the partial derivative of an expression absent
    of the coordinate to differentiated is necessarily zero.
  \end{solution}

\question Solve this Schrodinger equation explicitly in the case of a particle
  of mass $m$ in a constant Newtonian gravitational field: $V = mgz$. (Here
  $z$ is the height above the Earth's surface and $g$ is the downward
  gravitational acceleration.)

  \begin{solution}
    There is no reason for the solution to such an equation to vary in either
    the $x$ or $y$ coordinate, and so the Schrodinger equation with the relevant
    Hamiltonian collapses to
    \[
      i\hbar\frac{\partial\psi}{\partial{t}} =
      -\frac{\hbar^2}{2m}\frac{\partial^2\psi}{\partial{z}^2} + mgz\psi.
    \]
    Let us suppose that our wavefunction is a product of two independent
    functions, $\psi(z; t) = \phi(z)\nu(t)$. Then our equation becomes
    \[
      i\hbar\frac{1}{\nu}\frac{\mathrm{d}\nu}{\mathrm{d}t} =
      -\frac{\hbar^2}{2m}\frac{1}{\phi}\frac{\mathrm{d}^2\phi}{\mathrm{d}z^2}
      + mgz.
    \]
    Because $\phi$ and $\nu$ are independent, this equation can only be
    satisified if both sides equal the same, constant value $E$.
    
    The time function $\nu$ is easily solved as $\nu = e^{-i\frac{E}{\hbar}t}$. Unfortunately, the solution of $\phi$ is not as nice. We are looking for a function $\phi$ that satisfies
    \[
    	{-\frac{\hbar^2}{2m}}\frac{\mathrm{d}^2\phi}{\mathrm{d}z^2} + (mgz - E)\phi = 0,
    \]
    or, equivalently,
    \[
    	\frac{\mathrm{d}^2\phi}{\mathrm{d}z^2} - \frac{mgz - E}{\hbar^2/2m}\phi = 0.
    \]
    Functions that satisfy such a relation are called Airy functions. If we assume $\phi$ goes to zero at infinity, $\phi$ is given by $\mathrm{Ai}(x)$, where such a function equals
    \[
    	\mathrm{Ai}(x) = \frac{1}{\pi}\int_0^{\infty}\cos\Big(\frac{t^3}{3}+xt\Big)\mathrm{d}t
    \]
    and $x$ is the variable multiplying $\phi$ in the rightmost term of the differential equation.
  \end{solution}

\question By transforming to the freely-falling frame with coordinates
  $X=x$, $Y=y$, $Z=z-\frac{1}{2}t^2g$, $T=t$, show that the Schrodinger
  equation of Exericse [21.5] transforms to one without a gravitational
  field, with wavefunction $\Psi = e^{i(\frac{1}{6}mt^3g^2+mtzg)}\psi$. What
  does this tell us about Einstein's principle of equivalence (see
  $\S17.4$), as applied to quantum systems? (Take note of $\S21.9$.)

\question Show that, if the quantum Hamiltonian $\mathcal{H}$ has a
  translation invariance, say being independent of the position variable
  $x^3$, then the corresponding momentum $p_3$ is conserved in the sense that
  the operator $p_3$ commutes with the time evolution $\partial/\partial{t}$.
  Explain, in the light fo the interpretations given later, why this
  commutation implies conservation.

  \begin{solution}
    From Schrodinger's equation, we know that
    \[
      \hat{\mathcal{H}} = i\hbar\frac{\partial}{\partial{t}},
    \]
    i.e. the Hamiltonian \textit{is} the time-evolution operator.
    Simultaneously, we also know that
    \[
      \hat{p}_a = -i\hbar\frac{\partial}{\partial{q}^a},
    \]
    i.e. the operator for the $a$th momentum tells us how the quantity acted
    upon changes in the $a$th coordinate. But if $\mathcal{H}$ is independent of
    the $a$th coordinate, then
    $\hat{p}_a(\hat{\mathcal{H}}\psi)=\hat{\mathcal{H}}(\hat{p}_a\psi)$, i.e. $p_a$ commutes
    with the Hamiltonian, or time-evolution, operator.

    The commutativity of $p_a$ and $\mathcal{H}$ allows us to equate both
    ordered products, and since $\hat{p}_a\mathcal{H} = 0$, so also does
    $\hat{\mathcal{H}}p_a=0$. But this last equation says that the $a$th
    momentum does not change in time, i.e. it is conserved.
  \end{solution}

\question See if you can see why the requirements of special relativity enable
  Planck's $E=h\nu$ to be deduced from de Broglie's $p=h\lambda^{-1}$.
  (\textit{Hint}: You may assume that the hyperplanes in $\mathbb{M}$ along which
  the wave takes a constant value are Lorentz-orthogonal to the particle's $4$-velocity.)
  
  \begin{solution}
  	From SR, we know that energy and momentum may be combined to form a four-vector 
  	\[
  		(E, \mathbf{p}) = (E, \hbar\mathbf{k}),
  	\]
  	where we have used de Broglie's $p = h\lambda^{-1} = \hbar{k}$ in the second equality. We can then \textit{define} the frequency as $E = \hbar\omega = h\nu$. 
 	\end{solution}
  
\question Why? Here linear dependence can involve continuous sums, namely integrals.

\question Why can I split it this way?

\begin{solution}
	This is because
	\begin{align*}
		e^{-iP_ax^a/\hbar} &= e^{-i(Et - p_xx - p_yy - p_zz)/\hbar} \\
		&= e^{-iEt/\hbar + i(p_xx + p_yy + p_zz)/\hbar} \\
		&= e^{-iEt/\hbar + i\mathbf{P}\cdot\mathbf{x}/\hbar} \\
		&= e^{-iEt/\hbar}e^{i\mathbf{P}\cdot\mathbf{x}/\hbar}
	\end{align*}
\end{solution}

\question Replacing the real number $C$ in the above displayed expression by the complex number $C + iD$ (where $C$ and $D$ are real), find the frequency of the wave packet and the location of its peak.

\begin{solution}
	Making the substitution and expanding yields
	\begin{align*}
		Ae^{-B^2(x-C)^2} &= Ae^{-B^2[x-(C+iD)]^2} \\
		&= Ae^{-B^2[x^2 - 2x(C + iD) + (C + iD)^2]} \\
		&= Ae^{-B^2[x^2 - 2xC - i2xD + C^2 + i2CD - D^2]} \\
		&= Ae^{-B^2(x^2-2xC+C^2-D^2)}e^{-i2B^2D(C-x)} \\
		&= \big[Ae^{B^2D^2}e^{-B^2(x-C)^2}\big]e^{-i2B^2D(C-x)}
	\end{align*}
	We can immediately identify the frequency as $\omega = 2B^2D$. The peak occurs when the argument of the exponential dictating the amplitude of the wave is largest. This occurs at $x = C$.
\end{solution}

\question Show that the probability of such double-spot appearances, according to such a picture, must be quite appreciable, whatever the law of probability of spot appearance in terms of wavefunction intensity might be. \textit{Hint}: Divide the screen into two parts, with equal probability of spot appearance in each.

\question Check this from the hyperfunctional definition given in \S9.7.

\begin{solution}
	The definition of the Dirac delta function is
	\[
		\delta(x) = \Big(\!\!\big|\frac{1}{2\pi{i}z}, \frac{1}{2\pi{i}z}\big|\!\!\Big).
	\]
	By the properties of hyperfunctions,
	\[
		x\delta(x) = \Big(\!\!\big|\frac{x}{2\pi{i}z}, \frac{x}{2\pi{i}z}\big|\!\!\Big),
	\]
	which, when evaluated at $0$, becomes
	\[
		0\cdot\delta(0) = \Big(\!\!\big|0, 0\big|\!\!\Big) = 0
	\]
\end{solution}

\question Show that replacing $\psi$ by $x^1\psi$ or by $i\hbar\partial\psi/\partial{x^1}$ corresponds, respectively, to replacing $\tilde{\psi}$ by $-i\hbar\partial\tilde{\psi}/\partial{p_1}$ or by $p_1\tilde{\psi}$. Show that replacing $\psi(x^a)$ by $\psi(x^a + C^a)$ corresponds to replacing $\psi$ by $e^{-iC^ap_a/\hbar}\tilde{\psi}$ (where $a$ ranges over $1$, $2$, $3$).

\begin{solution}
	These correspondences are easily seen when either representation is expressed as the Fourier transform of the other. For example,
	\begin{align*}
		i\hbar\frac{\partial\tilde{\psi}}{\partial{p_1}} &= i\hbar\frac{\partial}{\partial{p_1}}\Big((2\pi)^{-3/2}\int_{\mathbb{E}^3}\psi(\mathbf{X})e^{-i\mathbf{p}\cdot\mathbf{X}/\hbar}\mathrm{d}^3\mathbf{X}\Big), \\
		&= i\hbar(2\pi)^{-3/2}\int_{\mathbb{E}^3}\psi(\mathbf{X})\frac{\partial}{\partial{p_1}}\Big(e^{-i(p_1x^1+p_2x^2+p_3x^3)/\hbar}\Big)\mathrm{d}^3\mathbf{X}, \\
		&= i\hbar(2\pi)^{-3/2}\int_{\mathbb{E}^3}\psi(\mathbf{X})\Big(\frac{-ix^1}{\hbar}\Big)e^{-i(p_1x^1+p_2x^2+p_3x^3)/\hbar}\mathrm{d}^3\mathbf{X}, \\
		&= (2\pi)^{-3/2}\int_{\mathbb{E}^3}\Big(x^1\psi(\mathbf{X})\Big)e^{-i\mathbf{p}\cdot\mathbf{X}/\hbar}\mathrm{d}^3\mathbf{X},
	\end{align*}
	which shows the identification
	\[
		i\hbar\frac{\partial\tilde{\psi}}{\partial{p_1}} \leftrightarrow x^1\psi.
	\]
	Meanwhile, we have
	\begin{align*}
		-i\hbar\frac{\partial\psi}{\partial{x^1}} &= -i\hbar\frac{\partial}{\partial{x^1}}\Big((2\pi)^{-3/2}\int_{\mathbb{E}^3}\tilde{\psi}(\mathbf{p})e^{i\mathbf{p}\cdot\mathbf{X}/\hbar}\mathrm{d}^3\mathbf{p}\Big), \\
		&= -i\hbar(2\pi)^{-3/2}\int_{\mathbb{E}^3}\tilde{\psi}(\mathbf{p})\frac{\partial}{\partial{x^1}}\Big(e^{i(p_1x^1 + p_2x^2 + p_3x^3)/\hbar}\Big)\mathrm{d}^3\mathbf{p}, \\
		&= -i\hbar(2\pi)^{-3/2}\int_{\mathbb{E}^3}\tilde{\psi}(\mathbf{p})\Big(\frac{ip_1}{\hbar}\Big)e^{i(p_1x^1 + p_2x^2 + p_3x^3)/\hbar}\mathrm{d}^3\mathbf{p}, \\
		&= (2\pi)^{-3/2}\int_{\mathbb{E}^3}\Big(p_1\tilde{\psi}(\mathbf{p})\Big)e^{i\mathbf{p}\cdot\mathbf{X}/\hbar}\mathrm{d}^3\mathbf{p}, \\
	\end{align*}
	which shows 
	\[
		-i\hbar\frac{\partial{\psi}}{\partial{x^1}} \leftrightarrow p_1\tilde{\psi}.
	\]
	In both cases, I believe Penrose has mistakenly used the wrong sign.
	
	For the last case, observe that
	\begin{align*}
		e^{iC^ap_a/\hbar}\tilde{\psi} &= e^{iC^ap_a/\hbar}\Big((2\pi)^{-3/2}\int_{\mathbb{E}^3}\psi(x^a)e^{-ip_ax^a/\hbar}\mathrm{d}^3\mathbf{X}\Big), \\
		&= (2\pi)^{-3/2}\int_{\mathbb{E}^3}\psi(x^a)e^{-ip_ax^a/\hbar}e^{ip_aC^a/\hbar}\mathrm{d}^3\mathbf{X}, \\
		&= (2\pi)^{-3/2}\int_{\mathbb{E}^3}\psi(x^a)e^{-ip_a(x^a-C^a)/\hbar}\mathrm{d}^3\mathbf{X}, \\
		&= (2\pi)^{-3/2}\int_{\mathbb{E}^3}\psi(\bar{x}^a+C^a)e^{-ip_a\bar{x}^a/\hbar}\mathrm{d}^3\bar{\mathbf{X}},
	\end{align*}
	where we made the substitution $\bar{x}^a = x^a - C^a$ in the last line. This shows
	\[
		e^{-iC^ap_a/\hbar}\tilde{\psi} \leftrightarrow \psi(x^a+C^a).
	\]
	
	In all of the above cases, I believe Penrose has mistakenly used the wrong sign in each of the transforms' exponential arguments.
\end{solution}

\question Use the results of Exercises [21.11], [21.13], and [21.14] to show that the Fourier transform of the wave packet $\psi = Ae^{-B^2(x-C)^2}e^{i\omega{x}}$ is $\tilde{\psi}=(Ae^{i\omega{C}}/B\sqrt{2})e^{-(p-\omega)^2/4B^2}e^{-iCp}$ (putting $\hbar=1$ for convenience.)

\question Can you see how to justify the factor $-\frac{1}{2}\hbar\log{2}$? (The half-life is the time at which the probability of decay has reached one half.)

\end{questions}

\end{document}