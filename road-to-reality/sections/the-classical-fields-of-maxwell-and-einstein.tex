\documentclass[../road-to-reality.tex]{subfiles}

\begin{document}

\section{The classical fields of Maxwell and Einstein}

\begin{questions}

\question Check both these statement.

  \begin{solution}
    The raised version of the Levi-Civita symbol is
    \[
      \varepsilon^{0123} = g^{0a}g^{1b}g^{2c}g^{3d}\varepsilon_{abcd}
    \]
    which, given that $g^{ab}=\mathrm{diag}(1,-1,-1,-1)$, reduces to $\varepsilon^{0123}=-\varepsilon_{0123}$. Given that $\varepsilon_{0123}=1$, then, we have $\varepsilon^{0123}=-1$.

    Since $\epsilon$ is defined by the normalization $\epsilon\cdot\varepsilon = n!$, the choice of $\epsilon^{abcd}=-\varepsilon^{abcd}$ makes sense, as then we have $\epsilon\cdot\varepsilon = \varepsilon^{abcd}\varepsilon_{abcd}=n!$.
  \end{solution}

\question Write these out fully, in terms of the electric and magnetic field components, showing how these equations provide a time-evoluton of the electric and magnetic fields, in terms of the operator $\partial/\partial{t}$

  \begin{solution}
    The exterior derivative of $\mathbf{F}$ is given by
    \begin{align*}
      \mathrm{d}\mathbf{F} &= \frac{\partial}{\partial{x^{[a}}}F_{bc]} \\
                           &= \frac{1}{6}\Big(\frac{\partial{F_{bc}}}{\partial{x^a}} - \frac{\partial{F_{cb}}}{\partial{x^a}} + \frac{\partial{F_{ca}}}{\partial{x^b}} - \frac{\partial{F_{ac}}}{\partial{x^b}} + \frac{\partial{F_{ab}}}{\partial{x^c}} - \frac{\partial{F_{ba}}}{\partial{x^c}}\Big).
    \end{align*}
    When we choose $a = 0$, $b = 1$ and $c = 2$, this becomes
    \[
      \frac{1}{3}\Big(-\frac{\partial{B_3}}{\partial{t}} -
      \frac{\partial{E_2}}{\partial{x}} + \frac{\partial{E_1}}{\partial{y}}\Big)
      = 0,
    \]
    or
    \[
      \frac{\partial{B_3}}{\partial{t}} = \frac{\partial{E_1}}{\partial{y}} - \frac{\partial{E_2}}{\partial{x}}.
    \]
    Likewise, $a=0$, $b=1$, and $c=3$ gives
    \[
      \frac{\partial{B_2}}{\partial{t}} =
      \frac{\partial{E_3}}{\partial{x}} - \frac{\partial{E_1}}{\partial{z}},
    \]
    and $a=0$, $b=2$, and $c=3$ results in
    \[
      \frac{\partial{B_1}}{\partial{t}} = \frac{\partial{E_2}}{\partial{z}}
      - \frac{\partial{E_3}}{\partial{y}}.
    \]
    Finally, $a=1$, $b=2$ and $c=3$ gives
    \[
      \frac{\partial{B_1}}{\partial{x}} + \frac{\partial{B_2}}{\partial{y}} +
      \frac{\partial{B_3}}{\partial{z}} = 0.
    \]
    We can interpret the first three of these to be expressing Faraday's law,
    which states that a changing magnetic field is equal to the negative curl of the
    electric field. The final equation encodes the lack of magnetic monopoles
    (the divergence of the magnetic field is zero).

    The exterior derivative of ${}^*\mathbf{F}$ is given by a similar
    expression,
    \[
      \mathrm{d}^*\mathbf{F} = \frac{1}{6}\Big(\frac{\partial{^*F_{bc}}}{\partial{x^a}} -
      \frac{\partial{^*F_{cb}}}{\partial{x^a}} + \frac{\partial{^*F_{ca}}}{\partial{x^b}}
      - \frac{\partial{^*F_{ac}}}{\partial{x^b}} +
      \frac{\partial{^*F_{ab}}}{\partial{x^c}} -
      \frac{\partial{^*F_{ba}}}{\partial{x^c}}\Big) = \frac{4}{3}\pi^*J_{abc}.
    \]
    Choosing $a=0$, $b=1$, and $c=2$ yields
    \[
      \frac{1}{3}\Big(-\frac{\partial{E_3}}{\partial{t}} + \frac{\partial{B_2}}{\partial{x}} - \frac{\partial{B_1}}{\partial{y}}\Big) = \frac{4}{3}\pi{j_3},
    \]
    or
    \[
      \frac{\partial{E_3}}{\partial{t}} = \frac{\partial{B_2}}{\partial{x}} -
      \frac{\partial{B_1}}{\partial{y}} - 4\pi{j_3}.
    \]
    Moving on, choosing $a=0$, $b=1$, and $c=3$ gives us
    \[
      \frac{\partial{E_2}}{\partial{t}} = \frac{\partial{B_1}}{\partial{z}} -
      \frac{\partial{B_3}}{\partial{x}} -4\pi{j_2},
    \]
    while $a=0$, $b=2$, and $c=3$ yields
    \[
      \frac{\partial{E_1}}{\partial{t}} = \frac{\partial{B_3}}{\partial{y}} -
      \frac{\partial{B_2}}{\partial{z}} - 4\pi{j_1}.
    \]
    Lastly, $a=1$, $b=2$, and $c=3$ results in
    \[
      \frac{\partial{E_1}}{\partial{x}} + \frac{\partial{E_2}}{\partial{y}} +
      \frac{\partial{E_3}}{\partial{z}} = 4\pi\rho.
    \]
    We may interpret the first three of \textit{these} equations as Ampere's law
    expressing the change in the electric field as the curl of the magnetic field
    minus the current density. The final equation is Gauss's law, relating the
    divergence of the electric field to the enclosed charge density.
  \end{solution}

\question Show the equivalence to the previous pair of equations.

  \begin{solution}
    To show the equivalence of the two sets of equations, we may make the
    replacement $\partial_a\to\nabla_a$, as this is the generalization of the
    above equations to the curvilinear case. The first equations equivalence is
    immediately apparent, as it becomes
    \[
      \mathrm{d}\mathbf{F} = \nabla_{[a}F_{bc]} = 0.
    \]
    For the second equation, we may make use of the index expression of the
    Hodge dual to obtain
    \begin{align*}
      \mathrm{d}^*\mathbf{F} &= \nabla_{[a}^*F_{bc]} \\
                             &= \frac{1}{2}\nabla_{[a}\varepsilon_{bc]de}F^{de} \\
                             &= \frac{1}{6}\Big(\nabla_a\varepsilon_{bcde}F^{de} + \nabla_{b}\varepsilon_{cade}F^{de} + \nabla_{c}\varepsilon_{abde}F^{de}\Big).
    \end{align*}
    Meanwhile, the righthand side is given by
    \begin{align*}
      \frac{4}{3}\pi^*\mathbf{J} &= \frac{4}{3}\pi^*J_{abc} \\
                                 &= \frac{4}{3}\pi\varepsilon_{abcd}J^d.
    \end{align*}
    The combined expression has four possible values, set by the antisymmetry of $a$, $b$, and
    $c$. When $a=0$, $b=1$, and $c=2$, we get
    \[
      \frac{1}{6}\Big(\nabla_0(F^{03} - F^{30}) + \nabla_1(F^{13} - F^{31}) + \nabla_2(F^{23} -
      F^{32})\Big) = \frac{4}{3}\pi{J^3},
    \]
    or, by the antisymmetry of $F^{ab}$,
    \[
      \nabla_{a}F^{a3} = 4\pi{J^3}.
    \]
    Doing this for the remaining three valid permutations gives
    \begin{align*}
      \nabla_aF^{a2} &= 4\pi{J^2} \\
      \nabla_aF^{a1} &= 4\pi{J^1} \\
      \nabla_aF^{a0} &= 4\pi{J^0}
    \end{align*}
    which is succinctly encapsulated by $\nabla_aF^{ab}=4\pi{J^b}$.
  \end{solution}

\question Show that the two versions of this vanishing divergence are equivalent.

  \begin{solution}
    Following the same steps as above, we may expand the expression for the
    Hodge dual of the charge-current vector as
    \begin{align*}
      \mathrm{d}^*\mathbf{J} &= \nabla_{[a}{}^*J_{bcd]} \\
                             &= \frac{1}{4}\Big(\nabla_{a}\varepsilon_{bcde}J^e +
                               \nabla_{b}\varepsilon_{dcae}J^e + \nabla_{c}\varepsilon_{bdae}J^e + \nabla_d\varepsilon_{bace}J^e\Big).
    \end{align*}
    Because both $\{a,b,c,d\}$ and $\{b,c,d,e\}$ are completely antisymmetric,
    a moment's thought reveals that the coordinate shared by $\nabla$ will
    also by shared by $\mathbf{J}$, and therefore this expression becomes
    \[
      \frac{1}{4}\nabla_aJ^a = 0
    \]
    which, of course, is equivalent to $\nabla_aJ^a = 0$.
  \end{solution}

\question Show this, first demonstrating that dualizing twice yields minus the original quantity. Does this sign relate to the Lorentzian signature of spacetime? Explain.

  \begin{solution}
    The relationship between the raised and lowered volume element is, as given
    by Penrose, $\varepsilon^{abcd}=-\varepsilon_{abcd}$. Utilizing this, and from the
    definition of the Hodge dual, it is easy to see that dualizing twice (which
    introduces both the raised and lowered volume element) returns the negative
    of the original quantity. Explicitly, we have 
    \begin{align*}
      {}^*({}^*\mathbf{F}) &= {}^*\Big(\frac{1}{2}\varepsilon_{abcd}F^{cd}\Big) \\
                           &= -\frac{1}{4}\varepsilon^{efab}\varepsilon_{abcd}F^{cd} \\
                           &= -\frac{1}{2}(\delta^e_c\delta^f_d - \delta^e_d\delta^f_c)F^{cd} \\
                           &= -\frac{1}{2}(F^{ef} - F^{fe}) \\
                           &= -F^{ef} \\
                           &= -\mathbf{F},
    \end{align*}
    where the fourth to last line uses the expression for the partial
    contraction of two volume elements given in Fig. 12.18.

    Knowing that dualizing a quantity twice returns negative the original, we see
    \begin{align*}
      {}^*({}^+\mathbf{F}) &= \frac{1}{2}({}^*\mathbf{F} + i\mathbf{F}) \\
                           &= i\frac{1}{2}(\mathbf{F} - i^*\mathbf{F}) \\
                           &= i^+\mathbf{F}
    \end{align*}
    and
    \begin{align*}
      {}^*({}^-\mathbf{F}) &= \frac{1}{2}({}^*\mathbf{F} - i\mathbf{F}) \\
                           &= -i\frac{1}{2}(\mathbf{F} + i^*\mathbf{F}) \\
                           &= -i^-\mathbf{F}.
    \end{align*}
    This sign does indeed relate to the Lorentzian signature of our metric, as
    the sign reversal of the antisymmetric symbol upon index juggling occurs
    for this reason.
  \end{solution}

\question Can you spell this out? What happens to the components of $\mathbf{F}$ and ${}^*\mathbf{F}$ in a general curvilinear coordinate system? Why are the Maxwell equations unaffected if expressed correctly?

\question Although correct, this argument has been given somewhat glibly.
  Spell out hte details more fully, in teh case when $\mathcal{R}$ is a
  spacetime `cylinder' consisting of some bounded spatial region that is
  constant in time, for a fixed finite interval of the time coordinate $t$.
  Explain the different notions of `flux of charge' involved, contrasting this
  for spacelike `base' and `top' of the cylinder with that for the timelike
  `sides'.

\question Spell out why this is just the electric flux.

  \begin{solution}
    The spacelike components of ${}^*\mathbf{F}$ are those of the electric
    field, so any integral for fixed time of this quantity will necessarily
    refer to the electric field. Hence, Gauss's law applied to this quantity
    relates the flux of the electric field to the charge enclosed by the
    spatial cross-section of spacetime under consideration.
  \end{solution}

\question Why can we add such a quantity?

  \begin{solution}
    Because $\mathrm{d}^2=0$, or, explicitly,
    \begin{align*}
      \mathrm{d}(\mathbf{A} + \mathrm{d}\Theta) &= \mathrm{d}\mathbf{A} + \mathrm{d}^2\Theta \\
                                                &= \mathrm{d}\mathbf{A}.
    \end{align*}
    $\Theta$ must be a scalar to have a well-defined quantity: $\mathbf{A}$ is a
    one-form, and, if $\Theta$ is a scalar, $\mathrm{d}\Theta$ is also a one-form. 
  \end{solution}

\question Show this. \textit{Hint}: Have a look at $\S15.8$.

  \begin{solution}
    This is easily seen by referring to the coordinate expression of the
    curvature tensor given a connection,
    \[
      (\nabla_a\nabla_b-\nabla_b\nabla_a)\xi^d = {R_{abc}}^d\xi^c.
    \]
    We will modify the above in accordance with $\S15.8$, where the last
    two indices refer to the directions in the fibre. Since our fibre is
    $U(1)$, these coordinates refer to just one direction, and so we will omit them.
    Given a connection of $\nabla_a=\partial/\partial{x}^a - ieA_a$, the above
    expression then becomes
    \begin{align*}
      (\nabla_a\nabla_b-\nabla_b\nabla_a)\psi &= \Big(\frac{\partial}{\partial{x}^a} - ieA_a\Big)\Big(\frac{\partial\psi}{\partial{x}^b} - ieA_b\psi\Big) - \Big(\frac{\partial}{\partial{x}^b} - ieA_b\Big)\Big(\frac{\partial\psi}{\partial{x}^a} - ieA_a\psi\Big) \\
                                              &= \frac{\partial^2\psi}{\partial{x}^a\partial{x}^b} - ie\frac{\partial{A_b}}{\partial{x}^a}\psi - ieA_b\frac{\partial{\psi}}{\partial{x}^a} - ieA_a\frac{\partial\psi}{\partial{x}^b} - e^2A_aA_b\psi \\
                                              &\quad - \frac{\partial^2\psi}{\partial{x}^b\partial{x}^a} + ie\frac{\partial{A_a}}{\partial{x}^b}\psi + ieA_a\frac{\partial{\psi}}{\partial{x}^b} + ieA_b\frac{\partial\psi}{\partial{x}^a} + e^2A_bA_a\psi \\
                                              &= -ie\Big(\frac{\partial{A_b}}{\partial{x}^a} - \frac{\partial{A_a}}{\partial{x}^b}\Big)\psi \\
                                              &= -ieF_{ab}\psi
    \end{align*}
    where the last line is the indicial expression for
    $\mathbf{F}=2\mathrm{d}\mathbf{A}$.
  \end{solution}

\question Explain this.

  \begin{solution}
    Because $\mathrm{d}\mathbf{A}$ vanishes within $\mathcal{R}$, expanding
    or otherwise distorting the contour of integration does not change the
    resulting value. If it did, the region the loop would be
    entering would have some nonzero value. This is somewhat analagous to
    contour integration around poles in the complex plane, wherein the analyticity of the
    function outside the singularity would give zero if that pole were removed---it is the global nature of 
    the function that introduces a non-zero value.
  \end{solution}

\question How do the individual components ${T^a}_b$ relate to $T_{ab}$, in a
  local Minkowskian frame, where the components $g_{ab}$ have the diagonal form
  $(1,-1,-1,-1)$?

  \begin{solution}
    Upon raising the $a$ index, ${T^a}_b = g^{ac}T_{cb}$, we see that the only
    surviving terms in the summation are those where $c = a$. When $a=0$, these
    terms are exactly $T_{ab}$, while $a\neq{0}$ resuts in $-T_{ab}$.
    Altogether, we have
    \begin{align*}
      {T^0}_b &= T_{0b}, \\
      {T^1}_b &= -T_{1b}, \\
      {T^2}_b &= -T_{2b}, \\
      {T^3}_b &= -T_{3b}.
    \end{align*}
  \end{solution}

\question Show that this satisfies the conservation equation $\nabla^aT_{ab}=0$
  if $\mathbf{J}=0$. Obtain the $00$ component of this tensor, and recover
  Maxwell's original expression $(E^2+B^2)/{8\pi}$ for the energy density of an
  electromagnetic field in terms of $(E_1,E_2,E_3)$ and $(B_1,B_2,B_3)$.

  \begin{solution}
    When $\mathbf{J}=0$, ${}^*\mathbf{J}=0$, and we have tight symmetry between both of Maxwell's
    equation in the language of differential forms, namely
    \begin{align*}
      \mathrm{d}\mathbf{F} &= 0, \\
      \mathrm{d}{}^*\mathbf{F} &= 0.
    \end{align*}
    By this symmetry we can express the indicial versions of these equations in two ways,
    \begin{align*}
      \nabla_aF^{ab} &= 0, \\
      \nabla_{[a}F_{bc]} &= 0, \\
      \nabla_a{}^*F^{ab} &= 0, \\
      \nabla_{[a}{}^*F_{bc]} &= 0.
    \end{align*}
    Using these identities, and contracting the claimed energy-momentum tensor with the raised covariant
    derivative, gives
    \begin{align*}
      \nabla^a\Big(\frac{1}{8\pi}\big(F_{ac}{F^c}_b + {}^*F_{ac}{}^*{F^c}_b\big)\Big) &= \frac{1}{8\pi}({F^c}_b\nabla^aF_{ac} + F_{ac}\nabla^a{F^c}_b + {}^*{F^c}_b\nabla^a{}^*F_{ac} + {}^*F_{ac}\nabla^a{}^*{F^c}_b) \\
                                                                                      &= \frac{1}{8\pi}(F_{cb}\nabla_aF^{ac} + F^{ac}\nabla_aF_{cb} + {}^*F_{cb}\nabla_a{}^*F^{ac} + {}^*F^{ac}\nabla_a{}^*F_{cb}) \\
                                                                                      &= \frac{1}{8\pi}(F^{ac}\nabla_aF_{cb} + {}^*F^{ac}\nabla_a{}^*F_{cb})
    \end{align*}
    where the third line has used the fact that both $\nabla_aF^{ab}$ and
    $\nabla_a{}^*F^{ab}$ are zero. The remaining terms are also zero. To see
    this, notice that both are antisymmetry in $ac$ as well as $cb$, thereby
    being antisymmetric in $abc$. But the antisymmetric parts of
    $\nabla_aF_{bc}$ and $\nabla_a{}^*F_{bc}$ vanish, and so $T_{ab}$ satisfies
    the conservation equation $\nabla^aT_{ab}=0$.
  \end{solution}

\question Why? Why does this procedure specialize to the above
  $\nabla_a{T^a}_0=0$, etc.? Can you find an analogue of the continuous-field
  conservation law $\nabla^a(T_{ab}\kappa^b)=0$, for a discrete system of
  particles where $4$-momentum is conserved in collisons? \textit{Hint}: Find
  a quantity, given the Killing vector $\kappa^a$, that is constant for each
  particle between collisions.

  \begin{solution}
    Applying the covariant derivative to $L_a$ yields
    \begin{align*}
      \nabla^aL_a &= \nabla^a(T_{ab}\kappa^b) \\
                  &= \kappa^b\nabla^aT_{ab} + T_{ab}\nabla^a\kappa^b.
    \end{align*}
    When the energy-momentum tensor satisifes the conservation law
    $\nabla^aT_{ab}=0$, the first term vanishes. Meanwhile, the second term
    vanishes because, by virtue of $\nabla^a\kappa^a$ being contracted with a
    symmetric tensor, it is totally symmetric, and the symmetric part of
    $\nabla^a\kappa^a$ is zero (see the indicial definition of a Killing vector).

    I am unsure how to answer the second part of this question.
  \end{solution}

\question Why is $R_{ab}$ symmetric?

  \begin{solution}
    This is simply because
    \begin{align*}
      R_{ab} &= {R_{acb}}^c \\
             &= R_{acbd}g^{dc} \\
             &= R_{bdac}g^{dc} \\
             &= {R_{bda}}^d \\
             &= R_{ba}.
    \end{align*}
  \end{solution}

\question See if you can prove this using the Ricci identity and the
  properties of Lie derivative.

\question Show fully why we can `lop off' all the $t^a$s, explaining the role
  of the symmetry of the tensors.

  \begin{solution}
    Clearly, $t^at^b$ is symmetric, and thus both $R_{ab}t^at^b$ and
    $T_{ab}t^at^b$ effectively symmetrize both the Ricci tensor and the energy
    momentum tensor, effectively stating
    \[
      R_{(ab)} = T_{(ab)}.
    \]
    Since $R_{ab}$ and $T_{ab}$ are manifestly symmetric, $R_{(ab)}=R_{ab}$ and
    $T_{(ab)}=T_{ab}$, allowing us to continue holding equality between both
    sides of the expression when $t^at^b$ is removed.
  \end{solution}

\question Show this, using the diagrammatic notation, if you like.

  \begin{solution}
    Expanding the Ricci identity, we find
    \begin{align*}
      \nabla_{[a}{R_{bc]d}}^e &= \frac{1}{6}(\nabla_a{R_{bcd}}^e - \nabla_a{R_{cbd}}^e + \nabla_b{R_{cad}}^e - \nabla_b{R_{acd}}^e + \nabla_c{R_{abd}}^e - \nabla_c{R_{bad}}^e), \\
&= \frac{1}{3}(\nabla_a{R_{bcd}}^e - \nabla_b{R_{acd}}^e + \nabla_c{R_{abd}}^e),
    \end{align*}
    where the last line uses the antisymmetry of the first two indices of the
    Riemann tensor. If we contract $e$ with $c$, we obtain
    \begin{align*}
      \nabla_{[a}{R_{bc]d}}^c &= \frac{1}{3}(\nabla_aR_{bd} - \nabla_bR_{ad} + \nabla_c{R_{abd}}^c), \\
      &= \frac{1}{3}(\nabla_aR_{bd} - \nabla_bR_{ad} + \nabla^cR_{bacd}),
    \end{align*}
    where, now, the antisymmetry of both the first and last pairs of indices is
    utilized in the last line. Finally, raising $a$ and contracting it with $d$
    gives
    \begin{align*}
      g^{ad}\nabla_{[a}{R_{bc]d}}^c &= \frac{1}{3}(\nabla^dR_{bd} - \nabla_bR + \nabla^cR_{bc}), \\
                                    &= \frac{1}{3}(2\nabla^aR_{ab} - \nabla_bR), \\
      &= \frac{1}{6}\nabla^a(R_{ab} - \frac{1}{2}Rg_{ab}).
    \end{align*}
    Since this equals zero, we may multiply through by $6$ to arrive at
    \[
      \nabla^a(R_{ab} - \frac{1}{2}Rg_{ab}) = 0.
    \]
  \end{solution}

\question Why?

  \begin{solution}
    We can contract the suggested field equation to obtain
    \[
      R = -4\pi{G}T,
    \]
    where $T = T_a^a$. This can be multipled by $\frac{1}{2}g_{ab}$ and
    subtracted from the original, uncontracted equation to get
    \[
      R_{ab} - \frac{1}{2}Rg_{ab} = -4\pi{G}T_{ab} + 2{\pi}GTg_{ab}.
    \]
    From here, we may take the covariant derivative of both sides (remembering
    that $\nabla^aT_{ab}=0$):
    \begin{align*}
      \nabla^a(R_{ab}-\frac{1}{2}Rg_{ab}) &= -4\pi{G}\nabla^aT_{ab} + 2\pi{G}\nabla^aTg_{ab}, \\
      0 &= 2\pi{G}\nabla_bT,
    \end{align*}
    which implies that $T_a^a$ must be constant---but this is clearly not the
    case in the real world.
  \end{solution}

\question Explain the coefficient $-8\pi{G}$, as compared with $-4\pi{G}$.

\begin{solution}
  Following a similar process to that in the next exercise (which was completed
  first), we find (by considering the diagonal terms of both sides of the field
  equation) that
  \begin{align*}
    \frac{1}{2}(R_{00} + R_{11} + R_{22} + R_{33}) &= -8\pi{G}T_{00}, \\
\frac{1}{2}(R_{00} + R_{11} - R_{22} - R_{33}) &= -8\pi{G}T_{11}, \\
\frac{1}{2}(R_{00} - R_{11} + R_{22} - R_{33}) &= -8\pi{G}T_{22}, \\
\frac{1}{2}(R_{00} - R_{11} - R_{22} + R_{33}) &= -8\pi{G}T_{33}. \\
  \end{align*}
  Adding all of these together yields
  \[
    \frac{1}{2}R_{00} = -8\pi{G}(T_{00} + T_{11} + T_{22} + T_{33}).
  \]
  Since the energy density ($T_{00}$) is expected to be much larger than the
  pressure ($T_{11}, T_{22}, T_{33}$) in the Newtonian limit, this reduces to
  \[
    R_{00} = -4\pi{G}T_{00},
  \]
  which is exactly the necessary relationship needed to reproduce the Newtonian
  volume-acceleration effects, as, in an observer's frame, (where $t^0 = 1$ and
  all other components are zero), we have
  \begin{align*}
    R_{ab}t^{a}t^{b}\delta{V} &= -4\pi{G}T_{ab}t^at^b\delta{V}, \\
    R_{00} &= -4\pi{G}T_{00}.
    \end{align*}
\end{solution}

\question Why?

  \begin{solution}
    Clearly, in a vaccuum, the metric is simply the Minkowski metric. Therefore,
    for all $a\neq{b}$, the field equation collapses to $R_{ab}=0$. All that's
    left is to show that, given a vaccuum, every diagonal component of $R_{ab}$
    is also zero.

    Recognizing that $R=R_a^a=R_{00}-R_{11}-R_{22}-R_{33}$, we may rewrite the
    vaccuum field equation as
    \[
      R_{ab} - \frac{1}{2}(R_{00}-R_{11}-R_{22}-R_{33})g_{ab} = 0.
    \]
    When $a=b=0$, this simplifies to
    \[
      R_{00} + R_{11} + R_{22} + R_{33} = 0.
    \]
    When $a=b=1$, this becomes
    \[
      R_{00} + R_{11} - R_{22} - R_{33} = 0.
    \]
    Likewise, for the remaining cases, we have
    \begin{align*}
      R_{00} - R_{11} + R_{22} - R_{33} &= 0, \\
      R_{00} - R_{11} - R_{22} + R_{33} &= 0.
    \end{align*}
    By adding together the permutations of the bottom three equations, we obtain
    \[
      R_{00} = R_{11} = R_{22} = R_{33},
    \]
    and thus the first equation is only satisifed when all four components are
    equal to zero. Hence we can write
    \[
      R_{ab}=0
    \]
    in a vaccuum.
  \end{solution}

\question Why?

  \begin{solution}
    By contracting both sides of the field equation, we obtain
    \[
      R - 2R = -8\pi{G}T,
    \]
    or $R = 8\pi{G}T$ (where we have used the fact that ${g_a}^b = 4$ in a
    four-dimensional spacetime). Substituting this back into the field equation
    gives
    \[
      R_{ab} - 4\pi{G}Tg_{ab} = -8\pi{G}T_{ab},
    \]
    which may be rearranged to read
    \[
      R_{ab} = -8\pi{G}(T_{ab} - \frac{1}{2}Tg_{ab}).
    \]
    From here, we may simply add the cosmological constant (being as it is just
    that: a constant) to the right side,
\[
  R_{ab} = -8\pi{G}(T_{ab} - \frac{1}{2}Tg_{ab}) + \Lambda{g}_{ab}.
\]
  \end{solution}

\question Show that all the `traces' of $\mathbf{C}$ vanish (e.g.
  ${C_{abc}}^a=0$, etc.). Do this calculation in diagrammatic form, if you
  wish.
\end{questions}

\end{document}
