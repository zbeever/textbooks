\documentclass[../the-road-to-reality.tex]{subfiles}

\begin{document}

\section{The ladder of infinity}

\begin{questions}

	\question Show how these rules work, explaining why $p$ has to be prime.

	\begin{solution}
		Addition and subtraction are quite straight-forward, being defined as
		\[
			a_1 + a_2 \equiv b\mod{p} \iff a_1 + a_2 - b = kp, \quad k\in\mathbb{Z}
		\]
		and
		\[
			a_1 - a_2 \equiv b\mod{p} \iff a_1 - a_2 - b = kp, \quad k\in\mathbb{Z},
		\]
		i.e. where $b$ is the remainder of $a_1 + a_2$ (or $a_1 - a_2$) on
		division by $p$. The existence of such a remainder follows from the
		division theorem. The additive inverse of $a_1$ is simply $-a_1$, while
		the additive identity is $0$---both rules arise due the
		definition of modulo arithmetic via the integers.

		Multiplication is defined similarly, but here we see the need for $p$ to
		be prime. We have
		\[
			a_1a_2 \equiv b\mod{p} \iff a_1a_2 - b = kp, \quad k\in\mathbb{Z}.
		\]
		The multiplicative identity is $1$, but what about the
		multiplicative inverse? This is the number $a^{-1}$ such that
		\[
			aa^{-1} \equiv 1\mod{p} \iff aa^{-1} - 1 = kp,\quad k\in\mathbb{Z}.
		\]
		That is, $aa^{-1}$ must be coprime to $p$. In each finite field (with $p$ prime), there is
		only one element that, when multiplying $a$, returns such an element.

		If $p$ were composite, say $p=mn$, then elements $m$ and $n$ would 
		not have a multiplicative inverse. This is because, when $p$ is prime and
		$a \equiv b\mod{p}$, the smallest integer we may multiply $a$ by to leave
		it unchanged is $p$---all others less than this are not divisible by $p$,
		and thus will modify $b$. If, on the other hand, $p = mn$, then element
		$m$ and $n$ may be multiplied by each other to return to themselves,
		thereby bypassing the other elements (including the multiplicative
		identity, which cannot exist because $am$ will never be coprime to $mn$
		for any $a$).
	\end{solution}

	\question Make complete addition and multiplication tables for $\mathbb{F}_4$
	and check that the laws of algebra work (where we assume that $1 + \omega +
	\omega^2 = 0$).

	\begin{solution}
		Assuming the multiplicative and additive identities are unaltered in this case, we have

		\[
			\begin{tabular}{>{$}l<{$}|*{4}{>{$}l<{$}}}
				+  & 0  & 1 & \omega & \omega^2 \\
				\hline\vrule height 12pt width 0pt
				0  & 0  & 1 & \omega & \omega^2 \\
				1 & 1 & 0  & \omega^2  &  \omega \\
				\omega  & \omega  & \omega^2 & 0  & 1 \\
				\omega^2 & \omega^2 & \omega & 1 & 0  \\
			\end{tabular} 
		\]

		\[
			\begin{tabular}{>{$}l<{$}|*{4}{>{$}l<{$}}}
				\times  & 0  & 1 & \omega & \omega^2 \\
				\hline\vrule height 12pt width 0pt
				0  & 0  & 0 & 0 & 0 \\
				1 & 0 & 1  & \omega  &  \omega^2 \\
				\omega  & 0 & \omega & \omega^2 & 1 \\
				\omega^2 & 0 & \omega^2 & 1 & \omega \\
			\end{tabular} 
		\]
		In the case of the addition table, a brief explanation of $\omega^2 + 1$
		is in order: if we assume $\omega^2 = 1 +
		\omega$, then $\omega^2 + \omega = 1 + \omega + \omega = 1 + 0 = 1$.
	\end{solution}

	\question Show this.

	\begin{solution}
		We can return to $\S15.6$ to remind ourselves that a projective space
		$\mathbb{P}^n$ can be described by $n$ independent ratios. We can express these
		as coordinates, just as in Exercise 15.10, taking care to ensure there is no
		overlap. Then, we simply count the number of possible coordinates (each
		mapping to a specific point).

		Consider
		\begin{gather*}
			A_1 = \{1,0,\dots,0\} \\
			A_2 = \{z^0, 1, \dots, 0\} \\
			\vdots \\
			A_n = \{z^0, z^1, \dots, 1\},
		\end{gather*}
		where a projective space $\mathbb{P}^n$ can be described in coordinates with the union of the
		first $n$ such sets. Working with finite fields, each coordinate $z^i$ is
		limited to a finite number of values $q$. Thus, for
		$\mathbb{P}^n(\mathbb{F}_q)$, there are
		\[
			1 + q + q^2 + \cdots + q^n = \frac{q^{n+1} - 1}{q - 1}
		\]
		different points.
	\end{solution}

	\question Show how to construct \textit{new} magic discs, in teh cases $q=3,5$
	by starting at a particular marked point on one of the discs that I have
	given and then multiplying each of the angular distances from the other
	marked points by some fixed integer. Why does this work?

	\question The finite field $\mathbb{F}_8$ has elements $0, 1, \varepsilon,
	\varepsilon^2, \varepsilon^3, \varepsilon^4, \varepsilon^5,
	\varepsilon^6$, where $\varepsilon^7=1$ and $1+1=0$. Show that either (1)
	there is an identity of the form $\varepsilon^a + \varepsilon^b +
	\varepsilon^c=0$ whenever $a$, $b$, and $c$ are numbers on the background
	circle of Fig. 16.1a which can line up with the three spots on the disc,
	or else (2) the same holds, but with $\varepsilon^2$ in place of
	$\varepsilon$ (i.e. $\varepsilon^{3a} + \varepsilon^{3b} + \varepsilon^{3c}
	= 0$).

	\question Show that the `associator' $a(bc) - (ab)c$ is antisymmetrical in
	$a$, $b$, $c$ when these are generating elements, and deduce that this
	(whence also $a(ab) = a^2b$) holds for \textit{all} elements.
	\textit{Hint}: Make use of Fig. 16.3 and the full symmetry of the Fano
	plane.

	\question See if you can provide such an explicit procedure, by finding
	some sort of systematic way of ordering all the fractions. You may find
	the result of Exercise [16.8] helpful.

	\begin{solution}
		Perhaps the simplest procedure is to imagine a table where the rows and columns are labeled by elements of $\mathbb{N}$ and each cell's value is given by the ratio of the corresponding column label to its row label. This construction lists all possible rational numbers.

		To create a mapping between $\mathbb{N}$ and such a construction, imagine drawing a line from the origin of our table (the cell corresponding to $1:1$) up one cell, then diagonally down (skipping the original cell), then over one cell, then diagonally up, and so on. We can parameterize the line via the natural numbers, providing a mapping from $\mathbb{N}$ to $\mathbb{Q}$.
	\end{solution}

	\question Show that the function $\frac{1}{2}((a+b)^2+3a+b)$ explicitly
	provides a $1$-$1$ correspondence between the natural numbers and the
	pairs $(a, b)$ of natural numbers.

	\question Spell this out in detail.

	\begin{solution}
		Imagine we have three sets, $A$, $B$, and $C$, with respective cardinal numbers $\alpha$, $\beta$, $\gamma$. The first two inequalities in the text imply that we can set up $1$-to-$1$ correspondences between $A$ and $B$ and between $B$ and $C$. Call the first of these $T$ and the second $S$, so $T(A) = B$ and $S(B) = C$. The fact that we can write $\alpha \leq \gamma$ follows from our ability to string together successive mappings. That is, if $A$ can be mapped to $B$ in a bijective manner, then we can simply map all of \textit{these} elements into $C$ with $S$ in a similarly bijective manner.
	\end{solution}

	\question Prove this. Outline: there is a $1$-$1$ map $b$ taking $A$ to some subset $bA (=B(A))$ of $B$, and a $1$-$1$ map $a$ taking $B$ to some subset $aB$ of $A$; consider the map of $A$ to $B$ which uses $b$ to map $A$-$aB$ to $bA$-$baB$ and $abA$-$abaB$ to $babA$-$babaB$, etc. and which uses $a^{-1}$ to map $aB$-$abA$ to $B-bA$ and $abaB-ababA$ to $baB-babA$, etc., and sort out what to do with the rest of $A$ and $B$.

	\question Explain this.

	\begin{solution}
		Treating such a number as binary (with a decimal point to the far left) leads to ambiguities between numbers such as 
		\[
			1000000\cdots \qquad \text{and} \qquad 0111111\cdots
		,\] 
		as, in the limit, $1011111\cdots$ converges to $1100000\cdots$. To see this, we can convert such a number to our decimal system by expanding in base-$2$,
		\begin{align*}
			0111111\cdots &= 0\cdot{2}^{-1} + 1\cdot{2}^{-2} + 1\cdot{2}^{-3} + \cdots \\
				      &= \sum_{n=2}^{\infty}2^{-n} \\
				      &= \frac{1}{2},
		\end{align*}
		which equals $1000000\cdots = 1\cdot{2}^{-1} = \frac{1}{2}$.
	\end{solution}

	\question Exhibit one. \textit{Hint}: Look at Fig. 9.8, for example.

	\begin{solution}
		Commonly used is the $\arctan$ function, which takes the entirety of $\mathbb{R}$ to $(-\frac{\pi}{2}, \frac{\pi}{2})$.
	\end{solution}

	\question Explain why this is essentially the same argument as the one I have given here, in the case $\alpha = \aleph_0$ for showing $\alpha < 2^\alpha$.

	\question Show that this is what happens.

	\question Give a rough description of how our algorithm might be performed and explain these particular values.

	\question Show this.

	\question Can you see why this is so? \textit{Hint}: For an arbitrary Turing machine action of $\mathbf{T}$ applied to $n$, we can consider an effective Turing machine $\mathbf{Q}$ which has the property that $\mathbf{Q}(r)=0$ if $\mathbf{T}$ applied to $n$ has not stopped after $r$ computational steps, and $\mathbf{Q}(r)=1$ if it has. Take the modulo $2$ sum of $\mathbf{Q}(n)$ with $\mathbf{T}_t(n)$ to get $\mathbf{T}_s(n)$.

	\question See if you can establish this.

	\question Explain why $(A^B)^C$ may be identified with $A^{B\times{C}}$, for sets $A$, $B$, $C$.

	\begin{solution}
		Because, in the convention we are using, $(A^B)^C$ stands for the set of all mappings from $C$ into the set of all mappings from $B$ into $A$. This is associating each mapping from $B$ into $A$ with a copy of $C$, allowing us to view it as a mapping from the pair of elements in $C$ and $B$ into $A$. Applying this to the entire set of mappings yields $A^{B\times{C}}$.
	\end{solution}

\end{questions}

\end{document}
