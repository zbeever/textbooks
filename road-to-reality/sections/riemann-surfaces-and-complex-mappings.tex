\documentclass[../the-road-to-reality.tex]{subfiles}

\begin{document}

\section{Riemann surfaces and complex mappings}

\begin{questions}
	
\question Explain why.

\begin{solution}
	A finite number of sheets forming a Riemann surface is the result of a finite number of values assigned to an input of a multifunction. In the case of $z\to{z^{m/n}}$, consider the substitution $z = re^{i\theta}$. With this, we see that $$z^{\frac{m}{n}} = re^{i\frac{m}{n}\theta} = re^{i(\frac{m}{n}\theta + 2\pi} = re^{i\frac{m}{n}(\theta+2\pi{n})}$$ That is, our multifunction returns to a previous value whenever the input encircles the origin $n$ times.
\end{solution}

\question Now try $(1 - z^4)^{1/2}$.

\begin{solution}
	The four branch points of $(1 - z^4)^{1/2}$ are the four roots of unity of $z^4$, namely $1$, $i$, $-1$, and $-i$. As a reminder, these points are when our multifunction's values overlap. Being a square root function, there are two values associated with each input, with each the negative of the other. The points of overlap (branch points) are when $(1-z^4)^{1/2} = -(1-z^4)^{1/2}$, i.e. when our function evaluates to $0$. Geometrically, our multifunction has two sheets in its Riemann surface, each adjoined to the other at the branch points.

	We may stop here, noting the similarity with Fig. 8.2 in the text, but we may also be more explicit. Consider making two branch cuts in our function: from $1$ to $i$ and from $-1$ to $-i$. Our codomain is the extended complex plane, or Riemann sphere, with two cuts in it. Topologically, a sphere with two cuts is nothing but a cylinder. We can adjoin this to our second sheet to form a torus, just as we had down with $(1 - z^3)^{1/2}$.
\end{solution}

\question Can you see how this comes about? (\textit{Hint}: Think of the Riemann sphere of the variable $w(=\log{z})$; see $\S$8.3.

\begin{solution}
	This can be seen by constructing an injective mapping from the infinite sheets of the $z$-plane to the $w$-plane, and then from the $w$-plane to the Riemann sphere. The first of these mappings can be understood as a kind of change of coordinates from radial to polar. Consider $\log{z}=\log{r} + i\theta$. Fixing $r$ and changing the angle traces out the a vertical line segment in the $w$-plane, with each winding number being separated from any other by $2\pi$. On the other hand, fixing $\theta$ and changing $r$ traces out a horizontal line the $w$-plane. In this way, every point in the infinite ramp of the $z$-plane maps to exactly one point in the $w$-plane. From here, simply stereographically project the $w$-plane to obtain the Riemann sphere.
\end{solution}

\question Show this.

\begin{solution}
	Consider an arbitrary circle in the complex plane, described by $z - a = re^{i\theta}$, or, equivalently, $(z - a)(\bar{z} - \bar{a}) = |r|^2$. We are considered with how this set changes under the mapping $w = z^{-1}$, or, equivalently, $w^{-1} = z$. Substituting in the second of these gives
	
	\begin{align*}
		\Big(\frac{1}{w} - a\Big)\Big(\frac{1}{\bar{w}} - \bar{a}\Big) &= |r|^2 \\
		(1 - wa)(1 - \bar{w}\bar{a}) &= |r|^2w\bar{w} \\
		w\bar{w}|a|^2 - wa - \bar{w}\bar{a} + 1 &= |r|^2w\bar{w} \\
		\frac{1}{|r|^2 - |a|^2} &= w\bar{w} + \frac{a}{|r|^2 - |a|^2}w + \frac{\bar{a}}{|r|^2 - |a|^2}\bar{w} \\
		\alpha &= w\bar{w} + \alpha{a}w + \alpha\bar{a}\bar{w} \\
		\alpha &= (w + \alpha\bar{a})(\bar{w} + \alpha{a}) - \alpha|a|^2 \\
		|\gamma|^2 &= (w + b)(\bar{w} + \bar{b})
	\end{align*}
	
	where $\alpha = (|r|^2 - |a|^2)^{-1}$, $b = \alpha\bar{a}$, and $\gamma = \alpha(1 + |a|^2)$. This represents another set of points in the complex plain describing a circle, so $w = z^{-1}$ sends circles to circles. In the case of $|r|^2 = |a|^2$, circles in the $z$-plane get mapped to lines (circles of infinite radius) in the $w$-plane.
\end{solution}

\question Verify that the sequence of transformations $z \to Az + B$, $z \to z^{-1}$, $z \to Cz + D$ indeed leads to a bilinear map.

\begin{solution}
	Carrying out each transformation in turn gives a final result of $z \to \frac{C}{Az + B} + D$, which can be rearranged to give $\frac{(DA)z + (B + C)}{Az + B}$. This is a bilinear map.
\end{solution}

\question Check that these two stereographic projections are related by $w = z^{-1}$.

\question Show this.

\question Show that these replacements give holomorphically equivalent spaces. Find all the special values of $p$ where these equivalences lead to additional discrete symmetries of the Riemann surface.

\end{questions}
	
\end{document}
