\documentclass[../the-road-to-reality.tex]{subfiles}

\begin{document}

\section{Surfaces}

\begin{questions}

\question Explain why subtraction and divison can be constructed from these.

\begin{solution}
        Subtraction may be reframed as addition of a negative number, that is $z - w = z + (-w)$. Division by $w$ may be defined by identifying $v$ in $wv = 1$, then multiplying $z$ by $v$.
\end{solution}

\question Derive both of these.

\begin{solution}
        Replacing $z$ with $x + iy$ yields (in the case of the first expression)
        
        \begin{align*}
                z^2 + \bar{z}^2 &= (x + iy)^2 + (x - iy)^2 \\
                &= x^2 + i2xy - y^2 + x^2 - i2xy - y^2 \\
                &= 2x^2 - 2y^2
        \end{align*}
        
        For the second expression, we have
        
        \begin{align*}
                z\bar{z} &= (x + iy)(x - iy) \\
                &= x^2 - ixy + ixy + y^2 \\
                &= x^2 + y^2
        \end{align*}
\end{solution}

\question Consider the real function $f(x, y) = xy(x^2 + y^2)^{-N}$, in the respective cases $N = 2$, $1$, and $\frac{1}{2}$. Show that in each case the function is differentiable ($C^{\omega}$) with respect to $x$, for any fixed $y$-value (and that the same holds with the roles of $x$ and $y$ reversed). Nevertheless, $f$ is not smooth as a function of the pair $(x,y)$. Show this in the case $N=2$ b demonstrating that the function is not even bounded in the neighbourhood of the origin $(0, 0,)$ (i.e. it tkaes arbitrarily large values there), in the case $N = 1$ by demonstrating that the function though bounded is not actually continuous as a function of $(x, y)$, and in the case of $N = \frac{1}{2}$ by showing that though the function is now continuous, it is not smooth aong the line $x=y$. (\textit{Hint}: Examine the values of each function along straight ines through the origin in the $(x,y)$-plane.) Some readers may find it illuminating to use a suitable $3$-dimensional graph-plotting computer facility, if this is available$-$though this is by not means necessary.

\question Prove that the mixed second derivatives $\partial^2f/\partial{y}\partial{x}$ and $\partial^2f/\partial{x}\partial{y}$ are always equal if $f(x, y)$ is a polynomial. (A \textit{polynomial} in $x$ and $y$ is an expression built up from $x$, $y$, and constants by use of addition and multiplication only.)

\begin{solution}
        We can represent $f$ by $f = \sum_{m,n}c_{mn}x^my^n$. Taking the partial derivatives of this gives

        \begin{align*}
                \frac{\partial{f}}{\partial{x}} &= \sum_{m,n}mc_{mn}x^{m-1}y^n \\
                \frac{\partial{f}}{\partial{y}} &= \sum_{m,n}nc_{mn}x^my^{n-1} \\
        \end{align*}

        Taking the opposite partial derivative of each of these expressions yields

	\[
        \frac{\partial^2f}{\partial{y}\partial{x}} = \frac{\partial^2f}{\partial{x}\partial{y}} = \sum_{m,n}mnc_{mn}x^{m-1}y^{n-1}
	.\] 
\end{solution}

\question Show that the mixed second derivatives of the function $f = xy(x^2 - y^2)/(x^2 + y^2)$ are unequal at the origin. Establish directly the lack of continuity in its second partial derivatives at the origin.

\question Find the form of $F(X, Y)$ explicitly when $f(x, y) = x^3 - y^3$ when $X = x-y$, $Y=xy$. \textit{Hint}: What is $x^2 + xy + y^2$ in terms of $X$ and $Y$; what does this have to do with $f$?

\begin{solution}
        Our function can be rewritten as $f(x, y) = (x^2 + xy + y^2)(x - y)$. The last term is clearly $X$, while the first term is $X^2 + 3Y$. Putting this together, we find $F(X,Y) = X^3 + 3YX$.	
\end{solution}

\question Find $A$ and $B$ in terms of $a$ and $b$; by analogy, write down $a$ and $b$ in terms of $A$ and $B$.

\begin{solution}
	We can represent $\frac{\partial}{\partial{X}}$ by $\frac{\partial{x}}{\partial{X}}\frac{\partial}{\partial{x}} + \frac{\partial{y}}{\partial{X}}\frac{\partial}{\partial{y}}$, and similarly for the other coordinate variables. Inserting these equivalent expressions into our different representatios of $\xi$ gives us
	\begin{align*}
		A &= a\frac{\partial{X}}{\partial{x}} + b\frac{\partial{X}}{\partial{y}} \\
		B &= a\frac{\partial{Y}}{\partial{x}} + b\frac{\partial{Y}}{\partial{y}} \\
		a &= A\frac{\partial{x}}{\partial{X}} + B\frac{\partial{x}}{\partial{Y}} \\
		b &= B\frac{\partial{y}}{\partial{X}} + B\frac{\partial{y}}{\partial{Y}} \\
	\end{align*}
\end{solution}

\question Derive this explicitly. \textit{Hint}: You may use 'chain rule' expression for $\partial/\partial{X}$ and $\partial/\partial{Y}$ that are the exact analogies of the expression for $\partial/\partial{x}$ that was displayed earlier.

\begin{solution}
	Working one variable at a time, we know $\frac{\partial}{\partial{X}} = \frac{\partial{x}}{\partial{X}}\frac{\partial}{\partial{x}} + \frac{\partial{y}}{\partial{X}}\frac{\partial}{\partial{y}}$. We also know that $Y = y + x$, or $y = Y - x = Y - X$. Putting these together gives us $\partial{x}/\partial{X} = 1$ and $\partial{y}/\partial{X} = -1$, so
	
	\[
	\frac{\partial}{\partial{X}} = \frac{\partial}{\partial{x}} - \frac{\partial}{\partial{y}}
	.\] 

	Analogously, $\frac{\partial}{\partial{Y}} = \frac{\partial{x}}{\partial{Y}}\frac{\partial}{\partial{x}} + \frac{\partial{y}}{\partial{Y}}\frac{\partial}{\partial{y}}$. We have $\partial{x}/\partial{Y} = 0$ and $\partial{y}/\partial{Y} = 1$, and so
	
	\[
	\frac{\partial}{\partial{Y}} = \frac{\partial}{\partial{y}}
	.\] 
\end{solution}

\question Show this explicitly using 'chain rule' expressions that we have seen earlier.

\begin{solution}
	We know that $u = \partial\Phi/\partial{x}$ and $v = \partial\Phi/\partial{y}$. From this, it immediately follows that $\xi(\Phi) = a\frac{\partial\Phi}{\partial{x}} + b\frac{\partial\Phi}{\partial{y}} = au + bv$.
\end{solution}

\question Explain this from three different points of view: (a) intuitively, from general principles (how could a $\bar{z}$ appear?), (b) using the geometry of holomorphic maps described in $\S$8.2, and (c) explicitly, using the chain rule and the Cauchy-Riemann equations that we are about to come to.

\question Do this.

\begin{solution}
	We can expand $\partial/\partial\bar{z}$ as $\frac{\partial{x}}{\partial\bar{z}}\frac{\partial}{\partial{x}} + \frac{\partial{y}}{\partial\bar{z}}\frac{\partial}{\partial{y}}$ Furthermore, we can express $x$ and $y$ in terms of $z$ and $\bar{z}$: $x = (z + \bar{z})/2$ and $y = (z - \bar{z})/(2i)$. Quickly computing $\partial{x}/\partial\bar{z} = 1/2$ and $\partial{y}/\partial\bar{z} = i/2$, we see
	
\[
	\frac{\partial\Phi}{\partial\bar{z}} = \frac{1}{2}\frac{\partial\Phi}{\partial{x}} + i\frac{1}{2}\frac{\partial\Phi}{\partial{y}}
.\] 	

	Setting this equal to $0$ and multiplying through by $2$ gives us
	
	\[
	\frac{\partial\Phi}{\partial{x}} + i\frac{\partial\Phi}{\partial{y}} = 0
	.\] 
\end{solution}

\question Give a more direct derivation of the Cauchy-Riemann equations, from the definition of a derivative.

\begin{solution}
        Recalling that the Cauchy-Riemann equations are merely a consequence of a function holomorphicity, let us try to prove the latter fact. This boils down to showing that a given function is differentiable in the complex plane (as, if it appears to be $C^1$, it is immediately $C^\omega$). We have

        \begin{align*}
                \frac{\mathrm{d}\Phi}{\mathrm{d}z} &= \lim_{\Delta{z}\to{0}}\frac{\Phi(z + \Delta{z}) - \Phi(z)}{\Delta{z}} \\
                &= \lim_{\Delta{z}\to{0}}\frac{\alpha(x + \Delta{x}, y + \Delta{y}) - \alpha(x, y)}{\Delta{x} + i\Delta{y}} + i\frac{\beta(x + \Delta{x}, y + \Delta{y}) - \beta(x, y)}{\Delta{x} + i\Delta{y}}
        \end{align*}

        Where we have split both $\Phi$ and $z$ into their constituent real and imaginary parts. As differentiability in the complex plane must hold when approaching a point from any direction, let us consider two cases: $\Delta{y} = 0$ and $\Delta{x} = 0$.

        \begin{align*}
                \lim_{\Delta{z}\to{0}}\frac{\alpha(x + \Delta{x}, y) - \alpha(x, y)}{\Delta{x}} + i\frac{\beta(x + \Delta{x}, y) - \beta(x, y)}{\Delta{x}} = \frac{\partial\alpha}{\partial{x}} + i\frac{\partial\beta}{\partial{x}} \\
                \lim_{\Delta{z}\to{0}}\frac{\alpha(x, y + \Delta{y}) - \alpha(x, y)}{i\Delta{y}} + i\frac{\beta(x, y + \Delta{y}) - \beta(x, y)}{i\Delta{y}} = -i\frac{\partial\alpha}{\partial{y}} + \frac{\partial\beta}{\partial{y}}
        \end{align*}

        These must be equal to each other for our function to be holomorphic. Equating real and imaginary parts gives us

	\[
        \frac{\partial\alpha}{\partial{x}} = \frac{\partial\beta}{\partial{y}} \qquad \frac{\partial\alpha}{\partial{y}} = -\frac{\partial\beta}{\partial{x}}
	.\] 
\end{solution}

\question Show this.

\begin{solution}
        Assuming the Cauchy-Riemann conditions hold, we have

        \begin{align*}
                \nabla^2\alpha = \frac{\partial}{\partial{x}}\Big(\frac{\partial\alpha}{\partial{x}}\Big) + \frac{\partial}{\partial{y}}\Big(\frac{\partial\alpha}{\partial{y}}\Big) = \frac{\partial}{\partial{x}}\Big(\frac{\partial\beta}{\partial{y}}\Big) - \frac{\partial}{\partial{y}}\Big(\frac{\partial\beta}{\partial{x}}\Big) = 0 \\
                \nabla^2\beta = \frac{\partial}{\partial{x}}\Big(\frac{\partial\beta}{\partial{x}}\Big) + \frac{\partial}{\partial{y}}\Big(\frac{\partial\beta}{\partial{y}}\Big) = -\frac{\partial}{\partial{x}}\Big(\frac{\partial\alpha}{\partial{x}}\Big) + \frac{\partial}{\partial{y}}\Big(\frac{\partial\alpha}{\partial{x}}\Big) = 0
        \end{align*}

        Where the last equality in both equations comes from the equality of mixed partial derivatives.
\end{solution}

\question Show this.

\begin{solution}
        Assuming $\alpha$ is harmonic (i.e. $\nabla^2\alpha = 0$), and defining $\beta = \int\partial\alpha/\partial{x}\mathrm{d}y$, we have

        \begin{align*}
                \frac{\partial\beta}{\partial{x}} &= \frac{\partial}{\partial{x}}\int\frac{\partial\alpha}{\partial{x}}\mathrm{d}y \\
                &= \int\frac{\partial^2\alpha}{\partial{x}^2}\mathrm{d}y \\
                &= -\int\frac{\partial^2\alpha}{\partial{y}^2}\mathrm{d}y \\
                &= -\frac{\partial\alpha}{\partial{y}}
        \end{align*}

        and

        \begin{align*}
                \frac{\partial\beta}{\partial{y}} &= \frac{\partial}{\partial{y}}\int\frac{\partial\alpha}{\partial{x}}\mathrm{d}y \\
                &= \frac{\partial\alpha}{\partial{x}}
        \end{align*}

        Where the first equation's third equality comes from $\alpha$ being harmonic, and the second equations last equality comes from integration being the inverse of differentiation.
\end{solution}

\question Spell this out in the case $\Phi(u, v) = \theta(v)h(u)$, where the functions $\theta$ and $h$ are defined as $\S\S$6.1, 3. THe kidney-shaped region must avoid the non-negative $u$-axis.

\end{questions}
	
\end{document}
