\documentclass[../road-to-reality.tex]{subfiles}

\begin{document}
	\printanswers
	
	\section{The entangled quantum world}
	
	\begin{questions}
		\question Can you explain this vanishing? Recall the $4$-dimensional notion of `divergence' described in $\S19.3$; here we need the $3$-space version. \textit{Hint}: See Exercise [19.2].
		
		\begin{solution}
			Since we are concerned with the case where no sources are present, Gauss's law for the electric field is $0$. The divergence of the magnetic field is always taken to be $0$.
		\end{solution}
	
		\question If $|\!\uparrow\rangle$ and $|\!\downarrow\rangle$ are normalized, what factor does $|\Omega\rangle$ need to make it normalized? (You may assume that $\||\alpha\rangle|\beta\rangle\|=\|\alpha\|\|\beta\|$.)
		
		\begin{solution}
			The norm of $|\Omega\rangle$, as currently defined, is
			\begin{align*}
				\langle\Omega|\Omega\rangle &= \big(\langle\downarrow\!|\langle\uparrow\!| - \langle\uparrow\!|\langle\downarrow\!|\big)\big(|\!\uparrow\rangle|\!\downarrow\rangle - |\!\downarrow\rangle|\!\uparrow\rangle\big) \\
				&= \langle\downarrow\!|\langle\uparrow\!|\!\uparrow\rangle|\!\downarrow\rangle - \langle\downarrow\!|\langle\uparrow\!|\!\downarrow\rangle|\!\uparrow\rangle - \langle\uparrow\!|\langle\downarrow\!|\!\uparrow\rangle|\!\downarrow\rangle + \langle\uparrow\!|\langle\downarrow\!|\!\downarrow\rangle|\!\uparrow\rangle \\
				&= \langle\downarrow\!|\!\downarrow\rangle + \langle\uparrow\!|\!\uparrow\rangle \\
				&= 2
			\end{align*}
			and so $|\Omega\rangle$ must be multiplied by $1/\sqrt{2}$ make it normalized.
		\end{solution}
	
		\question Can you see quickly why this has spin $0$? \textit{Hint}: One way is to use the index notation to show that any such anti-symmetrical combination must essentially be a scalar, bearing in mind that the spin space is $2$-dimensional.
		
		\question Why not? Find a way of doing this, however, if $|\alpha\rangle$ and $|\beta\rangle$ are not so localized.
		
		\question Confirm this parenthetic comment, and give a direct calculational verification of the above expression for $|\Omega\rangle$. \textit{Hint}: See Exercise [22.26].
		
		\begin{solution}
			A general normalized $2$-spinor can be written in component form as
			\[
				|\!\swarrow\rangle = \begin{pmatrix}e^{i\phi/2}\cos\tfrac{\theta}{2} \\ e^{-i\phi/2}\sin\tfrac{\theta}{2}\end{pmatrix}
			\]
			where $\theta$ and $\phi$ correspond to the usual coordinates parameterizing $S^2$. In the standard up-down spin basis,
			\[
				|\!\uparrow\rangle = \begin{pmatrix}1 \\ 0\end{pmatrix} \qquad |\!\downarrow\rangle = \begin{pmatrix}0 \\ 1\end{pmatrix}
			\]
			we see 
			\[
			|\!\swarrow\rangle = a|\!\uparrow\rangle + b|\!\downarrow\rangle
			\]
			where $a = e^{i\phi/2}\cos\tfrac{\theta}{2}$ and $b = e^{-i\phi/2}\sin\tfrac{\theta}{2}$. To change this to a spinor representing a state pointing in the opposite direction, we must make the substitutions 
			\begin{gather*}
				\theta \to \pi - \theta \\
				\phi \to \pi + \phi
			\end{gather*}
			which enacts the changes
			\begin{gather*}
				a \to ie^{i\phi/2}\sin\tfrac{\theta}{2} = i\bar{b} \\
				b \to -ie^{-i\phi/2}\cos\tfrac{\theta}{2} = -i\bar{a}
			\end{gather*}
			We can ignore the common factor of $i$ here, including it in the arbitrary phase factor of the state. Therefore, we have found
			\begin{gather*}
				|\!\swarrow\rangle = a|\!\uparrow\rangle + b|\!\downarrow\rangle \\
				|\!\nearrow\rangle = \bar{b}|\!\uparrow\rangle - \bar{a}|\!\downarrow\rangle 
			\end{gather*}
			To show that the state $|\!\swarrow\rangle|\!\nearrow\rangle - |\!\nearrow\rangle|\!\swarrow\rangle$ is proportional to $|\Omega\rangle = |\!\uparrow\rangle|\!\downarrow\rangle - |\!\downarrow\rangle|\!\uparrow\rangle$, we expand out the former using what we have found above
			\begin{align*}
				|\!\swarrow\rangle|\!\nearrow\rangle - |\!\nearrow\rangle|\!\swarrow\rangle &= \big(a|\!\uparrow\rangle + b|\!\downarrow\rangle\big)\big(\bar{b}|\!\uparrow\rangle - \bar{a}|\!\downarrow\rangle\big) - \big(\bar{b}|\!\uparrow\rangle - \bar{a}|\!\downarrow\rangle\big)\big(a|\!\uparrow\rangle + b|\!\downarrow\rangle\big) \\
				&= (a\bar{b} - \bar{b}a)|\!\uparrow\rangle|\!\uparrow\rangle - (a\bar{a} + \bar{b}b)|\!\uparrow\rangle|\!\downarrow\rangle + (b\bar{b} + \bar{a}a)|\!\downarrow\rangle|\!\uparrow\rangle - (b\bar{a} - \bar{a}b)|\!\downarrow\rangle|\!\downarrow\rangle \\
				&= -(a\bar{a} + b\bar{b})\big(|\!\uparrow\rangle|\!\downarrow\rangle - |\!\downarrow\rangle|\!\uparrow\rangle\big) \\
				&\propto |\!\uparrow\rangle|\!\downarrow\rangle - |\!\downarrow\rangle|\!\uparrow\rangle
			\end{align*}
		\end{solution}
	
		\question Why?
		
		\begin{solution}
			The Majorana description calls for a state of $n$ spins to be the symmetric product of them, and so we have
			\[
				|\!\leftarrow\nearrow\rangle = |\!\leftarrow\rangle|\!\nearrow\rangle + |\!\nearrow\rangle|\!\leftarrow\rangle
			\]
		\end{solution}
	
		\question See if you can prove these. \textit{Hint}: Use the coordinate and/or geometric desicrptions of $\S22.9$.
		
		\begin{solution}
			It's easiest to rewrite our states in terms of the up and down basis states. We can do this using our results from 23.5,
			\begin{align*}
				|\!\leftarrow\rangle &= \frac{\sqrt{2}}{2}|\!\uparrow\rangle + \frac{\sqrt{2}}{2}|\!\downarrow\rangle \\
				|\!\nearrow\rangle &= \cos \alpha|\!\uparrow\rangle + \sin \alpha|\!\downarrow\rangle
			\end{align*}
			where
			\[
				\alpha = -\frac{1}{2}\Big(\frac{\pi}{2} - \cos^{-1}\frac{3}{5}\Big).
			\]
			It is useful to note that $\sin^2\alpha = 1/10$ and $\cos^2\alpha = 9/10$, and so $\sin\alpha = -1/\sqrt{10}$ and $\cos\alpha = 3/\sqrt{10}$. Using this, we can rewrite the state as
			\begin{align*}
				|\!\leftarrow\nearrow\rangle &= \Big(\frac{\sqrt{2}}{2}|\!\uparrow\rangle + \frac{\sqrt{2}}{2}|\!\downarrow\rangle\Big)\Big(\cos\alpha|\!\uparrow\rangle + \sin\alpha|\!\downarrow\rangle\Big) + \Big(\cos\alpha|\!\uparrow\rangle + \sin\alpha|\!\downarrow\rangle\Big)\Big(\frac{\sqrt{2}}{2}|\!\uparrow\rangle + \frac{\sqrt{2}}{2}|\!\downarrow\rangle\Big) \\
				&= \frac{3\sqrt{5}}{5}|\!\uparrow\rangle|\!\uparrow\rangle + \frac{\sqrt{5}}{5}|\!\uparrow\rangle|\!\downarrow\rangle + \frac{\sqrt{5}}{5}|\!\downarrow\rangle|\!\uparrow\rangle - \frac{\sqrt{5}}{5}|\!\downarrow\rangle|\!\downarrow\rangle
			\end{align*}
			This is clearly not orthogonal to $|\!\downarrow\rangle|\!\downarrow\rangle$. As for the other states, let's examine them one at a time.
			\begin{align*}
				\Big(\langle\downarrow\!|\langle\leftarrow\!|\Big)|\!\leftarrow\nearrow\rangle &= \Big(\frac{\sqrt{2}}{2}\langle\downarrow\!|\langle\uparrow\!| + \frac{\sqrt{2}}{2}\langle\downarrow|\langle\downarrow\!|\Big)|\!\leftarrow\nearrow\rangle \\
				&= \frac{3}{2}\sin\alpha + \frac{1}{2}\cos\alpha \\
				&= 0 \\
				\Big(\langle\leftarrow\!|\langle\downarrow\!|\Big)|\!\leftarrow\nearrow\rangle &= \Big(\frac{\sqrt{2}}{2}\langle\uparrow\!|\langle\downarrow\!| + \frac{\sqrt{2}}{2}\langle\downarrow|\langle\downarrow\!|\Big)|\!\leftarrow\nearrow\rangle \\
				&= \frac{3}{2}\sin\alpha + \frac{1}{2}\cos\alpha \\
				&= 0 \\
				\Big(\langle\rightarrow\!|\langle\rightarrow\!|\Big) |\!\leftarrow\nearrow\rangle &= \Big(\frac{\sqrt{2}}{2}\langle\uparrow\!| - \frac{\sqrt{2}}{2}\langle\downarrow\!|\Big)\Big(\frac{\sqrt{2}}{2}\langle\uparrow\!| - \frac{\sqrt{2}}{2}\langle\downarrow\!|\Big)|\!\leftarrow\nearrow\rangle \\
				&= \frac{1}{2}\Big(\langle\uparrow\!|\langle\uparrow\!| - \langle\uparrow\!|\langle\downarrow\!| - \langle\downarrow\!|\langle\uparrow\!| + \langle\downarrow\!|\langle\downarrow\!|\Big)|\!\leftarrow\nearrow\rangle \\
				&= \frac{\sqrt{2}}{2}\cos\alpha + \frac{\sqrt{2}}{2}\sin\alpha - \frac{\sqrt{2}}{4}\Big(2\cos\alpha + 2\sin\alpha\Big) \\
				&= 0
			\end{align*}
		\end{solution}
	
		\question Show this.
		
		\begin{solution}
			We must first normalize $|\!\leftarrow\nearrow\rangle$. Its norm is
			\begin{align*}
				\langle\leftarrow\nearrow\!|\!\leftarrow\nearrow\rangle &= 2\cos^2\alpha + 2\sin^2\alpha + (\cos\alpha+ \sin\alpha)^2 \\
				&= 3\cos^2\alpha + 2\cos\alpha\sin\alpha + 3\sin^2\alpha \\
				&= \frac{12}{5}
			\end{align*}
			and so we multiply it by $\sqrt{5/12}$. Both parties measuring $|\!\downarrow\rangle|\!\downarrow\rangle$ results in a probability of
			\[
				\Big[\Big(\langle\downarrow\!|\langle\downarrow\!|\Big)|\!\leftarrow\nearrow\rangle\Big]^2 = \frac{5}{6}\sin^2\alpha = \frac{1}{12}.
			\]
		\end{solution}
	
		\question Explain all these numbers in both the boson and fermion cases.
		
		\begin{solution}
			The fermion case is quite simple. The antisymmety condition rules out permutations, so we are left with only combinations of states,
			\[
				{{10}\choose{n}} = \frac{10!}{n!(10-n)!}
			\]
			The boson case is one of assigning $n$ indistinguishable quantities to $10$ bins, i.e. it is a variant of the famous `stars and bars' combinatorics problem, and so may be parameterized by
			\[
				{{10+n-1}\choose{n}} = \frac{(9 + n)!}{9!n!}.
			\]
		\end{solution}
	
	\question Confirm this, with appropriate conventions concerning coordinate axes etc.
	
	\question See if you can give a fuller explanation of this, using quanglement ideas or otherwise.
	
	\end{questions}
\end{document}
