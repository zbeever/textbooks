\documentclass[../the-road-to-reality.tex]{subfiles}

\begin{document}

\section{Fibre bundles and gauge connections}

\begin{questions}

  \question Explain why the dimension of $\mathcal{M}\times\mathcal{V}$ is the
  sum of the dimensions of $\mathcal{M}$ and $\mathcal{V}$.

  \begin{solution}
    An element in the product space $\mathcal{M}\times\mathcal{V}$ is specified
    by the ordered pair $(m, v)$. That is, we associate to every point in $\mathcal{M}$
    a space $\mathcal{V}$. The dimension of our product space is, then, the
    dimension of $\mathcal{M}$ augmented by the dimension of $\mathcal{V}$, or
    $\dim\mathcal{M} + \dim\mathcal{V}$.
  \end{solution}

  \question Explain this.

  \begin{solution}
    A structure preserving transformation is one such that
    \[
      \mathbf{T}(x)\mathbf{T}(y) = \mathbf{T}(xy).
    \]
    For the transformation $\mathbf{T}(x) = -x$, we have
    \[
      \mathbf{T}(x)\mathbf{T}(y) = (-x)(-y) = xy \neq \mathbf{T}(xy) = -xy,
    \]
    and so such a `flip' of $\mathbb{R}$ is not a symmetry.
  \end{solution}

  \question Spell this argument out, using the construction of $B$ from two
  patches, as indicated above.

  \begin{solution}
    Suppose, as Penrose does, that we build our M{\"o}bius from two
    semi-circular bands---one having `vertical' coordinates flipped in relation
    to the other. To smoothly glue such parts together, we must avoid
    discontinuities, and so our coordinate systems must match on the overlap in
    these two sections, but the only such coordinate that satisfies $x=-x$ is
    $x=0$. Therefore, a smooth cross-section through the M{\"o}bius bundle must
    cross $0$ if it is to pass through all patches.
  \end{solution}

  \question Carry out this argument. Can you see how to do the $S^{15}$ case?

  \question Show this. \textit{Hint}: Take the tangent vector to be
  $u\partial/\partial{v} - v\partial/\partial{u} + x\partial/\partial{y} - y\partial/\partial{x}$.

  \question Why does every such spinor field take the value zero at at least one
  point of $S^2$?

  \begin{solution}
    Consider traversing such a field around a great circle of $S^2$. Being as
    this is a rotation of $2\pi$, the spinor vectorspin we arrive at should be the negative
    of what we started with. But these are the same, and so must equal zero.
  \end{solution}

  \question Explain this in detail.

  \question Show that $\mathcal{B}'^{\mathbb{C}}$, interpreted as a real bundle
  over $S^2$ is indeed the same as $T(S^2)$. \textit{Hint}: Re-examine Exercise [15.5].

  \question Explain how to do this. \textit{Hint}: Think of Cartesian
    coordinates $(x,y,z)$. Take two at a time, with the canvas given by the
    third set to unity.

  \begin{solution}
    To cover the whole of $\mathbb{P}^2$, our three planes must capture every
    single point. Points in a $2$-dimensional projective space are represented as slopes of
    lines through the origin of a $3$-dimensional vector space. We need to find
    $2$ points on each line to calculate its slope, and so consider choosing as our
    three planes those Penrose suggests.

    Because the $3$-dimensional vector space containing our projective space is
    Euclidean, any line through its origin will intersect these planes two
    times, except when such a line is exactly perpendicular to one plane. But,
    being as this occurs for only one such line per plane, it is uniquely
    specified by such a projection. Thus, all points in our projective space are
    captured on these three planes.
  \end{solution}

  \question Explain why there are $n$ independent ratios. Find $n+1$ sets of $n$
    ordinary coordinates (constructed from the $z$s), for $n+1$ different
    coordinate spatches, which together cover $P^n$.

    \begin{solution}
      There are $n$ independent ratios because we can normalize every value with
      respect to a particlar one. This is also the key to constructing the
      necessary coordinate sets. One possible answer is
      \begin{gather*}
        \{1, z^1, \dots, z^n\} \\
        \{z^0, 1, \dots, z^n\} \\
        \vdots \\
        \{z^0, z^1, \dots, 1\}
      \end{gather*}
    \end{solution}


  \question Explain this geometry, showing that the bundle $\mathbb{R}^{n+1}-O$
    over $\mathbb{RP}^n$ can be understood as the composition of the bundle
    $\mathbb{R}^{n+1}-O$ over $S^n$ (the fibre, $\mathbb{R}^{+}$, being the
    positive reals) and of $S^n$ as a twofold cover of $\mathbb{RP}^n$.

  \question Check this.

    \begin{solution}
      We are simply asked to check this, so
      \begin{align*}
        \frac{\partial\Phi}{\partial{z}} &= \Big(\frac{\partial{B}}{\partial{z}} + \frac{\partial{\overline{B}}}{\partial{z}}\Big)e^{(B+\overline{B})} \\
&= A\Phi + \overline{\frac{\partial{B}}{\partial{\overline{z}}}}\Phi \\
&= A\Phi + \overline{\int\frac{\partial{A}}{\partial\overline{z}}\mathrm{d}z} \\
&= A\Phi,
      \end{align*}
      where the second term becomes zero by virtue of $A$'s holomorphicity.
    \end{solution}
    
\question Verify this formula.

\begin{solution}
  Expanding each nabla yields
  \begin{align*}
    (\nabla\overline{\nabla} - \overline{\nabla}\nabla)\Phi &= \Big(\frac{\partial}{\partial{z}} - A\Big)\overline{\Big(\frac{\partial}{\partial{z}} - A\Big)}\Phi - \overline{\Big(\frac{\partial}{\partial{z}} - A\Big)}\Big(\frac{\partial}{\partial{z}} - A\Big)\Phi \\
&= \Big(\frac{\partial}{\partial{z}} - A\Big)\Big(\frac{\partial}{\partial\overline{z}} - \overline{A}\Big)\Phi - \Big(\frac{\partial}{\partial\overline{z}} - \overline{A}\Big)\Big(\frac{\partial}{\partial{z}} - A)\Phi \\
&= \frac{\partial^2\Phi}{\partial{z}\partial\overline{z}} - \frac{\partial\overline{A}}{\partial{z}}\Phi - A\frac{\partial\Phi}{\partial\overline{z}} + A\overline{A}\Phi - \frac{\partial^2\Phi}{\partial\overline{z}\partial{z}} + \frac{\partial{A}}{\partial\overline{z}}\Phi + \overline{A}\frac{\partial\Phi}{\partial{z}} - \overline{A}A\Phi \\
&= \frac{\partial{A}}{\partial\overline{z}}\Phi - \frac{\partial\overline{A}}{\partial{z}}\Phi + \overline{A}\frac{\partial\Phi}{\partial{z}} - \overline{\overline{A}\frac{\partial\Phi}{\partial{z}}} \\
&= (\frac{\partial{A}}{\partial\overline{z}} - \frac{\partial\overline{A}}{\partial{z}})\Phi,
  \end{align*}
  
where the vanishing of the last two terms in the second to last line comes from
the fact that $\Phi$ is real, and anything minus its complex conjugate is purely imaginary.
\end{solution}

\question Confirm the assertions in this paragraph, finding the explicit value
  of $k$ that gives this required factor $2$.
    
\end{questions}

\end{document}