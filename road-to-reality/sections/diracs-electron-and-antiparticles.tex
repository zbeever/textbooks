\documentclass[../road-to-reality.tex]{subfiles}

\begin{document}
	\printanswers
	
	\section{Dirac's electron and antiparticles}
	
	\begin{questions}
		\question Check that this is indeed a solution.
		
		\begin{solution}
			This can be exactly solved through integration, as it can be rewritten
			\begin{align*}
				\frac{1}{\psi}\frac{\partial\psi}{\partial t} &= {-\frac{iE}{\hbar}} \\
				\int_0^t\frac{1}{\psi}\frac{\partial\psi}{\partial t'}\,\mathrm{d}t' &= \int_0^t{-\frac{iE}{\hbar}}\,\mathrm{d}t' \\
				\int_{\psi_0}^{\psi}\frac{\mathrm{d}\psi}{\psi} &= {-\frac{iE}{\hbar}}t \\
				\ln\psi &= \ln\psi_0 - \frac{iE}{\hbar}t \\
				\psi &= \psi_0e^{-iEt/\hbar}
			\end{align*}
		\end{solution}
		
		\question Explain why adding a constant $K$ to the Hamiltonian simply has the effect that all solutions of the Schr\"odinger equation are multiplied by the same factor. Find this factor. Does this substantially affect the quantum dynamics? Supose we are concerned with the gravitational effect of a qunatum system. Why can we not simply `renormalize` the energy in this way under the circumstances?
	
		\begin{solution}
			As can be seen from the previous problem, adding $K$ to the Hamiltonian will result in the phase factor $e^{-iKt/\hbar}$ multiplying every solution. Since wavefunctions are fixed up to a phase factor, this does not change the dynamics.
			
			The reason we cannot renormalize gravitational energy this way is because now a constant addition to the Hamiltonian corresponds to a change in mass. Such a change would effect how gravity couples to the system, and thus change its dynamics.
		\end{solution}
	
	\question Schr\"odinger's equation, here, is $\partial\psi/\partial t = (i\hbar/2\mu)\nabla^2\psi$. Confirming, first, that for an energy eigenstate with energy $E$ we have $-\nabla^2\psi = A\psi$, where $A = 2\mu\hbar^{-2}E$, use \textit{Green's theorem} $\int\bar{\psi}\nabla^2\psi\,\mathrm{d}^3x=-\int\nabla\bar{\psi}\cdot\nabla\psi\,\mathrm{d}^3x$ to show that $A$ must be positive for a normalizable state. (Conversely, it is in fact true that, for positive $A$, there are many solutions of $-\nabla^2\psi=A\psi$, which tail off suitably towards infinity so that the norm $\|\psi\|$ remains finite and we can normalize to $\|\psi\|=1$, if we wish.) Show how to derive Green's theorem, from the fundamental theorem of exterior calculus.
	
	\question Make some suggestions, either using Fourier transforms (\S9.4), or a power series, or contour integrals, or otherwise.
	
	\question Check.
	
	\begin{solution}
		Expanding out the wave operator, we find
		\begin{align*}
			\square &= \Big(\gamma_0\frac{\partial}{\partial t} - \gamma_1\frac{\partial}{\partial x} - \gamma_2\frac{\partial}{\partial y} - \gamma_3\frac{\partial}{\partial z}\Big)^2 \\
			&= \gamma_0^2\frac{\partial^2}{\partial t^2} - \gamma_0\gamma_1\frac{\partial^2}{\partial t\partial x} - \gamma_0\gamma_2\frac{\partial^2}{\partial t\partial y} - \gamma_0\gamma_3\frac{\partial^2}{\partial t\partial z} \\
			&\qquad-\gamma_1\gamma_0\frac{\partial^2}{\partial x\partial y} + \gamma_1^2\frac{\partial^2}{\partial x^2} - \gamma_1\gamma_2\frac{\partial^2}{\partial x\partial y} - \gamma_1\gamma_3\frac{\partial^2}{\partial x\partial z} \\
			&\qquad - \gamma_2\gamma_0\frac{\partial^2}{\partial y\partial t} - \gamma_2\gamma_1\frac{\partial^2}{\partial y\partial x} + \gamma_2^2\frac{\partial^2}{\partial y^2} - \gamma_2\gamma_3\frac{\partial^2}{\partial y\partial z} \\
			&\qquad - \gamma_3\gamma_0\frac{\partial^2}{\partial z\partial t} - \gamma_3\gamma_1\frac{\partial^2}{\partial z\partial x} - \gamma_3\gamma_2\frac{\partial^2}{\partial z\partial y} + \gamma_3^2 \frac{\partial^2}{\partial z^2} \\
			&= \frac{\partial^2}{\partial t^2} - \frac{\partial^2}{\partial x^2} - \frac{\partial^2}{\partial y^2} - \frac{\partial^2}{\partial z^2}
		\end{align*}
		where the other terms cancel by the equality of mixed partial derivatives.
	\end{solution}

	\question Show this.
	
	\begin{solution}
		The Dirac equation is
		\[
			\hbar\slashed\partial\psi = -i\mu\psi
		\]
		or, expanding out $\slashed\partial$,
		\[
			\hbar\gamma_0\frac{\partial\psi}{\partial t} - \hbar\gamma\cdot\nabla\psi = -i\mu\psi
		\]
		Moving the spatial term to the right and multiplying by $\gamma_0$ (keeping in mind that $\gamma_0^2 = 1$) gives
		\[
			i\hbar\frac{\partial\psi}{\partial t} = (i\hbar\gamma_0\gamma\cdot\nabla + \gamma_0\mu)\psi
		\]
	\end{solution}

	\question Explain this. \textit{Hint}: Generalize Exercise [22.18].
	
	\question Explain this comment in relation to the connection between quaternions and Clifford elements explained in \S11.5.
	
	\question Show why $2\times2$ matrices cannot satisfy all the conditions; find a set of $4\times4$ matrices that do.

	\question Explain why this is the standard `gauge prescription.'
	
	\end{questions}
\end{document}
