\documentclass[../road-to-reality.tex]{subfiles}

\begin{document}
	\printanswers
	
	\section{Quantum field theory}
	
	\begin{questions}
		\question Explain why this gives states with the correct symmetries for bosons and fermions, as described in $\S23.8$.
		
		\question Make sense of all this (and verify this commutation law for creation and annihilation of particles of a given type) by referring to the index notation of $\S23.8$ or the diagrammatic notation of Fig. 12.17, or both, using expressions like $\bar{\psi}_{\alpha}\psi^{[\alpha}\phi^{\beta}\cdots\chi^{\kappa]}$. Sort out all the factorial factors which preserve normalization of the state, both in the fermion and the boson case.
		
		\question Explain this Clifford-algebra structure, spelling out the role of the scalar product more explicitly. (Take the defining laws for a Clifford algebra in the form $\gamma_p\gamma_q + \gamma_q\gamma_p = -2g_{pq}\mathbf{I}$.) \textit{Hint}: $g_{pq}$ need not be diagonal.
		
		\question Explain what addition, and multiplication by a scalar constant, mean in this space.
		
		\question Show this. (Don't worry about subtleties like `fall-off conditions'.)
		
		\question Explain why we can remove a specific state in this way, despite my earlier qualifications about what an annihilation operator actually does. (\textit{Hint}: See Exercise [26.2].)
		
		\question By referring back to Exercise [26.2] and Fig. 12.18, exhibit this algebraic difference in the abstract-index or diagrammatic notation.
		
		\question Give a `physical interpretation' of the history of Fig. 26.3b, in terms of particle creation and annihilation.
		
		\question Try to make these statements more precise by referring to first-order changes in the path, using `$O$' symbols (as in $\S14.5$), and relating this to the discussion given in $\S20.1$, concerning the meaning of `stationary action`. (Assume that $S$ is large in units of $\hbar$.)
		
		\question Why?
		
		\question Explain how this singularity arises, by first rewriting $(\slashed{P} - M + i\varepsilon)^{-1}$ as a quotient for which the denominator is $P_{\alpha}P^{\alpha} - M^2 - \varepsilon^2$.
		
		\question Why not? Explain how $4$-momentum conservation at each vertex determines the $4$-momentum of the virtual photon. \textit{Hint}: All electrons have the same mass!
		
		\question What is this freedom?
		
		\question Can you see why this should be?
		
		\question Can you see why?
		
		
	\end{questions}
\end{document}