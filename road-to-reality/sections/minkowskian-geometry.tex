\documentclass[../road-to-reality.tex]{subfiles}

\begin{document}

\section{Minowskian geometry}

\begin{questions}
  
\question Find $\mathbf{C}$ explicitly for each of the three cases $\mathbb{E}^4$, $\mathbb{M}$, and $\tilde{\mathbb{M}}$. \textit{Hint}: Think of how $\mathbf{C}$ is to act on $\omega$, $\xi$, $\eta$, and $\zeta$. It is not quite the standard operation of complex conjugation in the cases of $\mathbb{M}$ and $\tilde{\mathbb{M}}$.

  \begin{solution}
    Consider the case of normal complex conjugation: real numbers are characterized by $z = \overline{z}$, while purely imaginary ones satisfy $z = -\overline{z}$. To encode this `realness' and `imaginaryness,' we may define component-wise operations that make use of the above. Explicitly, we have
    \begin{align*}
      \mathbf{C}_{\mathbb{E}^4} &: (\omega, \xi, \eta, \zeta) \to (\overline{\omega}, \overline{\xi}, \overline{\eta}, \overline{\zeta}) \\	
      \mathbf{C}_{\mathbb{M}} &: (\omega, \xi, \eta, \zeta) \to (\overline{\omega}, {-\overline{\xi}}, {-\overline{\eta}}, {-\overline{\zeta}}) \\
      \mathbf{C}_{\tilde{\mathbb{M}}} &: (\omega, \xi, \eta, \zeta) \to ({-\overline{\omega}}, \overline{\xi}, \overline{\eta}, \overline{\zeta})
    \end{align*}
    All of these operators satisfy $\mathbf{C}^2=1$ by virtue of their building blocks (ordinary complex conjugation). Furthermore, the requirement that a point $\mathbf{x}$ be real ($\mathbf{x} = \mathbf{Cx}$) allows these operators to select the necessary `real' subspaces from $\mathbb{CE}^4$.
  \end{solution}

\question Can you see why?

\question Confirm it in this case examining the $4\times{4}$ Lie algebra matrices explicitly.

  \begin{solution}
    The Lie algebra matrices are those infinitesimal elements satisfying
    \[
      \mathbf{A}^T\mathbf{gA} = \mathbf{g}
      ,\] 
    i.e. those preserving the metric (and thus inner products). By writing $\mathbf{A} = \mathbf{I} + \varepsilon\mathbf{X}$, we find
    \[
      (\mathbf{I}^T + \varepsilon\mathbf{X}^T)\mathbf{g}(\mathbf{I} + \varepsilon\mathbf{A}) = \mathbf{g} + \varepsilon(\mathbf{X}^T\mathbf{g} + \mathbf{g}\mathbf{X}) + \mathcal{O}(\varepsilon^2) = \mathbf{g}
      ,\] 
    or
    \[
      \mathbf{X}^T\mathbf{g} = -\mathbf{g}\mathbf{X} = -(\mathbf{X}^T\mathbf{g})^T
      ,\] 
    that is, $\mathbf{X}^T\mathbf{g}$ is antisymmetric. Antisymmetric matrices having the dimensions $4\times{4}$ are $6$-dimensional, completing the exercise.
    
    If we wish to continue on to find the explicit representation, we can write $\mathbf{X}^T\mathbf{g}$ in terms of the standard antisymmetric basis $\mathbf{E}_i$ (where there is exactly one $1$ to the right of the diagonal and one $-1$ to the left), we have
    \[
      \mathbf{X}^T\mathbf{g} = \sum_{i=1}^6a_i\mathbf{E}_i
      .\] 
    Postmultiplying by $\mathbf{g}$ (which is its own inverse) and taking the transpose gives us a basis for $\mathbf{X}$, whose elements are given by $\mathbf{E}_i^T\mathbf{g}$. Explicitly, we have
    \[
      \mathbf{E}_1 = \begin{pmatrix} 0 & 1 & 0 & 0 \\ 1 & 0 & 0 & 0 \\ 0 & 0 & 0 & 0 \\ 0 & 0 & 0 & 0 \end{pmatrix} \qquad
      \mathbf{E}_2 = \begin{pmatrix} 0 & 0 & 1 & 0 \\ 0 & 0 & 0 & 0 \\ 1 & 0 & 0 & 0 \\ 0 & 0 & 0 & 0 \end{pmatrix} \qquad
      \mathbf{E}_2 = \begin{pmatrix} 0 & 0 & 0 & 1 \\ 0 & 0 & 0 & 0 \\ 0 & 0 & 0 & 0 \\ 1 & 0 & 0 & 0 \end{pmatrix}
    \] 
    \[
      \mathbf{E}_4 = \begin{pmatrix} 0 & 0 & 0 & 0 \\ 0 & 0 & 1 & 0 \\ 0 & -1 & 0 & 0 \\ 0 & 0 & 0 & 0 \end{pmatrix} \qquad
      \mathbf{E}_5 = \begin{pmatrix} 0 & 0 & 0 & 0 \\ 0 & 0 & 0 & 1 \\ 0 & 0 & 0 & 0 \\ 0 & -1 & 0 & 0 \end{pmatrix} \qquad
      \mathbf{E}_6 = \begin{pmatrix} 0 & 0 & 0 & 0 \\ 0 & 0 & 0 & 0 \\ 0 & 0 & 0 & 1 \\ 0 & 0 & -1 & 0 \end{pmatrix}
    \] 
  \end{solution}

\question Explain this action of the Poincare group a little more fully.

  \begin{solution}
    The generators of the Poincare group allow us to move any point within a (future directed) null-cone to any other point within the same cone: this is what is meant by the transitive action of the group on the bundle of future-timelike directions on $\mathbb{M}$.

    Intuitively, this captures the loss of absolute simultaneity between separate observers.
  \end{solution}

\question (i) Under what circumstances is it possible for a $3$-plane element $\eta$ to contain its normal $\eta^{\perp}$, in $\mathbb{M}$. (ii) Show that there are two distinct families of $2$-planes that are the orthogonal complements of themselves in $\mathbb{CE}^4$, but neither of these families survives in $\mathbb{M}$. (These so-called `self-dual' and `anti-self-dual' complex $2$-planes will have considerable importance later; see $\S{32.2}$ and $\S{33.11}$.)

  \begin{solution}
    A hyperplane $P$ can be described as follows: a vector $\xi$ is contained within the hyperplane if it is orthogonal to a
    one-dimensional family of vectors, i.e.
    \[
      g_{\alpha\beta}\eta^\alpha\xi^\beta = 0 \implies \xi \in P.
    \]
    If a hyperplane is to contain its normal, then we must have
    \[
      g_{\alpha\beta}\eta^{\alpha}\eta^{\beta} = 0.
    \]
    This is satisfied when $\eta$ is a null-vector, and thus our hyperplane is a light cone.
    An arbitrary $2$-plane is specified by the set of vectors lying in the span of
    \[
      a\eta + b\xi + \chi,
    \] 
    where $a$ and $b$ are arbitrary complex constants. For such a plane to be orthogonal to itself, we must have
    \[
      (a\eta + b\xi + \chi)\cdot(c\eta + d\xi + \chi) = 0.
    \]
    As $a$, $b$, $c$, and $d$ are all arbitrary, we may choose convenient values to obtain the necessary conditions. From a combination of $0$s and $1$s, we find
    \[
      \eta\cdot\eta = \xi\cdot\xi = \eta\cdot\xi + \eta\cdot\chi + \xi\cdot\chi = -\chi\cdot\chi = 0.
    \]
    If we choose $\xi = \mathbf{0}$ so that our plane runs through the origin of $\mathbb{CE}^4$, then our conditions simplify to
    \[
      \eta\cdot\eta = \xi\cdot\xi = \eta\cdot\xi = 0.
    \]
    These can be satisfied by choosing
    \[
      \eta = \begin{pmatrix}1 & i & 0 & 0\end{pmatrix} \qquad \xi = \begin{pmatrix}0 & 0 & 1 & i\end{pmatrix}
    \]
    or, alternatively,
    \[
      \eta = \begin{pmatrix}1 & -i & 0 & 0\end{pmatrix} \qquad \xi = \begin{pmatrix}0 & 0 & 1 & -i\end{pmatrix}.
    \]
    In either case, the elements of our self-orthogonal $2$-planes do not have the necessary `realness' required by $\mathbb{M}$, i.e. they do not obey $\eta = \mathbf{C}\eta$ where $\mathbf{C}$ is defined in an earlier problem.
  \end{solution}

\question Show all this. \textit{Hint}: It is handy to make use of coordinates $x$, $y$, and $w$, where $w = (t - z - 1/\lambda)\sqrt{\lambda} = (1-t-z)/\sqrt{\lambda}$.

  \begin{solution}
    We will take Penrose's suggestion, as such a substitution encodes the planes we will use to take slices of our light cone. Rewriting our light cone as
    \[
      (t-z)(t+z) - x^2 - y^2 = 0,
    \]
    we can remove $t$ and $z$ by making the substitutions $t-z = (1+\sqrt{\lambda}w)/\lambda$ and $t+z=(1-\sqrt{\lambda}w)$. Doing so, we find the our slices are described by
    \[
      \frac{1}{\lambda}(1+\sqrt{\lambda}w)(1-\sqrt{\lambda}w)-x^2-y^2 = \frac{1}{\lambda} - w^2 - x^2 - y^2 = 0,
    \]
    or, alternatively,
    \[
      w^2 + x^2 + z^2 = \frac{1}{\lambda}.
    \]
    This, of course, is the metric on a sphere of radius $1/\sqrt{\lambda}$.

    These coordinates have a singularity at $\lambda = 0$. To show the parabolic nature of the metric in this case, we simply set $\lambda=0$ in the equation for our plane, obtaining $z + t = 2$. Removing $t$, our metric becomes
    \[
      x^2+y^2+4z = 4,
    \]
    which describes a paraboloid.
  \end{solution}

\question Show why the hyperbolic straight lines are repesented as straight in the `Klein' case and by circles meeting the boundary orthogonally in the `Poincare' case,, indicating, by use of a `signature flip' why this second case is indeed conformal.

  \begin{solution}
    Hyperbolic straight lines, or geodesics, are given by intersections of $\mathcal{H}^+$ with a plane through the origin. The projective plane, then, naturally encodes these structures as straight lines---the projection is simply one plane interescted with another.

    The conformal projection is considerably more complicated. To characterize geodesics, let us describe them first on the projective plane. All such structures are straight lines here, so let us rotate our coordinate system until the line of interest lies parallel to the $x$-axis, as then it takes the very simple form $t=1$, $x = t\lambda$, and $y = tk$, where $\lambda$ is a real number parameterizing our line and $k$ is a constant denoting the distance our line is from the $x$-axis.

    If we allow $t$ to change values, we obtain a vector emenating from the origin, with $t=1$ being our projective plane. This vector intersects the hyperboloid when its coordinates obey the equation
    \[
      t^2 - x^2 - y^2 = t^2 - (t\lambda)^2 - (tk)^2 = 1.
    \]
    Since the hyperboloid $\mathcal{H}^+$ exists only for positive $t$ values, we can reduce this requirement to $t = 1/\sqrt{1-\lambda^2-k^2}$, the positive square root of $t^2$. So our hyperbolic geodesic is parameterized by the vector
    \[
      \Big(\frac{1}{\sqrt{1 - \lambda^2 - k^2}},\frac{\lambda}{\sqrt{1 - \lambda^2 - k^2}},\frac{k}{\sqrt{1 - \lambda^2 - k^2}}\Big).
    \]
    The easiest way to project this conformally is to imagine stretching our vector by $1$ in the $t$ direction and normalizing $t$ (projecting to the plane $t = 1$, which is really the $t=0$ plane of our original vector), i.e. $(t, t\lambda, tk) \to (1, t\lambda/(t+1), tk/(t+1))$. Carrying out this operation gives a projected geodesic vector of 
    \[
      \Big(1, \frac{\lambda}{1 + \sqrt{1 - \lambda^2 - k^2}}, \frac{k}{1 + \sqrt{1 - \lambda^2 - k^2}}).
    \]
    Graphing such a curve gives a clear circle shifted in the $y$ direction (our last component). To explicitly illustrate this, consider $x^2 + (y - c)^2$, or
    \begin{align*}
      \frac{\lambda^2}{(1+\sqrt{1-\lambda^2-k^2})^2} + \Big(\frac{k}{1+\sqrt{1-\lambda^2-k^2}}-c\Big)^2 = &\frac{\lambda^2}{(1+\sqrt{1-\lambda^2-k^2})^2} + \frac{k^2}{(1 + \sqrt{1 - \lambda^2 - k^2})^2} \\
                                                                                                            &- \frac{2kc}{1+\sqrt{1-\lambda^2-k^2}} + c^2 \\
                                                                                                        = &\frac{\lambda^2 + k^2 - 2kc(1 + \sqrt{1 - \lambda^2 - k^2}) + c^2(1 + \sqrt{1 - \lambda^2 - k^2})^2}{(1 + \sqrt{1 - \lambda^2 - k^2})^2} \\
                                                                                                        = &\frac{2(c^2-kc)(1 + \sqrt{1 - \lambda^2 - k^2}) - (c^2 - 1)\lambda^2 - (c^2 - 1)k^2}{2(1 + \sqrt{1 - \lambda^2 - k^2}) - \lambda^2 - k^2}. \\
    \end{align*}
    This is equal to a constant when $c^2 - kc = c^2 - 1$, or $c = 1/k$. Hence, hyperbolic geodesics on $\mathcal{H}^+$ get mapped to circles contained within the unit circle when projected in this way.

    To see that these circles meet the unit circle at right angles, we must first find the condition for when two arbitrary circles meet at right angles. This is when their radii $r_1$, $r-2$ can be considered as two legs to a right triangle, and hence the distance $d$ between their center can be found via $d^2 = r_1^2 + r_2^2$. In the case of our projected circle, it has a radius of $\sqrt{1/k^2 - 1}$, while our unit circle has a radius of $1$. The two are separated by a distance $1/k$. Substituting these into our relationship between $d^2 = r_1^2 + r_2^2$ yields an identity,
    \[
      \frac{1}{k^2} = 1^2 + \frac{1}{k^2} - 1 = \frac{1}{k^2},
    \]
    showing that the circle does indeed meet the boundary of the unit circle
    orthogonally. That this projection is a conformal follows from the fact that
    a projection of the complex sphere is conformal, and taking a `real' slice
    of such a sphere maintains this property.
  \end{solution}

\question Use a `signature-flip' argument, to see why adding lengths in hyperbolic geometry should give rise to the addition formula being used here, namely $(u+v)c/(1+uv)$, for `adding' the velocities $uc$ and $vc$ in the same spatial direction. Consider adding arc lengths around a circle or sphere, the `velocity' corresponding to each arc length being the tangent of the angle it subtends at the centre.

  \begin{solution}
    If such a hyperbolic surface is taken to be the velocity space, then any
    velocity vector can be reached from another by a hyperbolic rotation. This
    is because such a surface is defined by the vectors satisfying
    $g_{ab}v^av^b=c$ for some constant $c$.
    
    Instead of jumping directly to adding angles within hyperbolic geometry and considering their tangents the velocities of interest, let us examine the same phenomenon on a circle. That is, what is the tangent of $\theta_w$ given that we know the tangents of $\theta_u$ and $\theta_v$, and $\theta_w = \theta_u + \theta_v$. By simple trigonometric identities (which may most easily be derived by considering the rotation matrix in $\mathbb{R}^2$), we have
    \begin{align*}
      \tan\theta_w &= \tan(\theta_u + \theta_v) \\
                   &= \frac{\sin(\theta_u + \theta_v)}{\cos(\theta_u + \theta_v)} \\
                   &= \frac{\sin\theta_u\cos\theta_v + \cos\theta_u\sin\theta_v}{\cos\theta_u\cos\theta_v - \sin\theta_u\sin\theta_v} \\
                   &= \frac{\tan\theta_u + \tan\theta_v}{1 - \tan\theta_u\tan\theta_v}.
    \end{align*}
    Considering these tangents to be velocities, we have
    \[
      w = \frac{u + v}{1 - uv}.
    \]
    We have been working in $\mathbb{CE}^4$ with our attention restricted to one direction. If we switch to Minkowski space, we must take all of our velocities to be imaginary (as they are $3$-velocities, and these components are purely imaginary in our scheme). Then we have
    \[
      iw = \frac{iu + iv}{1 - i^2uv},
    \]
    or
    \[
      w = \frac{u + v}{1 + uv}
    \]
    where we have set $c = 1$.
  \end{solution}

\question Justify this assertion; prove the equivalence of the above two displayed formulae.

  \begin{solution}
    For speeds $v\ll{1}$, we may expand $\rho$ in $v$ to find
    \begin{align*}
      \rho &= \frac{1}{2}\log\frac{1+v}{1-v} \\
           &= \frac{1}{2}\log(1+v) - \frac{1}{2}\log(1-v) \\
           &\approx \frac{1}{2}\Big(0 + v + \mathcal{O}(v^2)\Big) - \frac{1}{2}\Big(0 - v + \mathcal{O}(v^2)\Big) \\
           &= v + \mathcal{O}(v^2)
    \end{align*}
    To show the equivalence of the two given formulas, we may simply solve for $v$ given $\rho$,
    \begin{align*}
      \rho &= \frac{1}{2}\log\frac{1+v}{1-v} \\
      e^\rho &= \sqrt{\frac{1+v}{1-v}} \\
      (1-v)e^{2\rho} &= 1+v \\
      v(1 + e^{2\rho}) &= e^{2\rho} - 1 \\
      ve^{\rho}(e^{\rho} + e^{-\rho}) &= e^{\rho}(e^{\rho} - e^{-\rho}) \\
      v &= \frac{e^{\rho}-e^{-\rho}}{e^{\rho}+e^{-\rho}}
    \end{align*}
  \end{solution}

\question Try to fill in the details of an ingenious argument ofr this, due to the highly original and influential Irish relativity theorist John L. Synge, which requires no calculation! THe argument proceeds roughly as follows. Consider the geometrical configuration consisting o fhte past light cone $\mathcal{C}$ of an event $\mathcal{O}$ and a (timelike) 3-plane $P$ through $O$. Let $\Sigma$ be the intersection of $\mathcal{C}$ and $\mathcal{P}$. Describe the `history', as time progresses, of the respective spatial descriptions of $\mathcal{C}$, $\mathcal{P}$, and $\Sigma$, according to some particular Minkowskian reference frame. Explain why any observer at $O$ see $\Sigma$ as a circle and, moreover, that this geometrical constructon characterizes, in a frame-independent way, those bundles of rays that appear to an observer as a circle.

\question Derive this formula.

  \begin{solution}
    Geometrically, it is obvious that a great circle on our sphere will map to a line on the plane under stereographic projection. If this plane is to be identified with $\mathbb{C}$, this amounts to saying that the projection will take the form
    \[
      \zeta = R(\theta)e^{i\phi}.
    \]
    Furthermore, we must have that $R(\pi) = 0$, $R(\pi/2) = 1$, and $R(0) = \infty$. This is accomplished in a smooth manner by making the identification $R(\theta) = \cot\frac{1}{2}\theta$, and so
    \[
      \zeta = {e^{i\phi}}\cot\frac{1}{2}\theta.
    \]
  \end{solution}

\question Try to derive this formula using the spacetime geometry ideas above.

  \begin{solution}
    Consider two observers moving away from each other with a velocity $v$. We
    will concentrate on the direction of motion and only consider one spatial
    dimension. Let us denote the coordinates of the observer at rest with $t$
    and $x$, and those of the observer in motion with $t'$ and $x'$. We can
    find the intersection of a hyperbolic space-like surface with our two
    observer's time axes through the relation
    \[
      t^2-x^2 = t'^2 - x'^2
    \]
    where we set $x = vt$ and $x' = 0$. This gives the relationship
    \[
      t' = t\sqrt{1 - v^2}.
    \]
    That is, an observer sees those in motion with
    (respect to themselves) as having slower clocks.

    Now, we know that any change in velocity maintains the causal structure of
    spacetime, i.e. light cones are preserved. This means the lines described by
    \[
      \frac{t}{x} = 1 \qquad \frac{t'}{x'} = 1
    \]
    must coincide. We can substitute our found expression for $t'$ into this and further
    express $x'$ as a corrected $x$ via $x' = \gamma(v)x$. This requirement
    becomes
    \[
      \frac{t}{x} = 1 \iff \frac{t\sqrt{1-v^2}}{\gamma(v)x} =
      \frac{\sqrt{1-v^2}}{\gamma(v)} = 1.
    \]
    Ergo, $\gamma(v) = \sqrt{1-v^2}$, and so distances moving at a speed
    relative to an observer are squashed. Reinserting units of $c$, this becomes
    $x' = x\sqrt{1-(v/c)^2}$.
  \end{solution}

\question Develop this argument in detail, to show why the FitzGerald-Lorentz flattening exactly compensates for the effect arising from the path-length difference. Show that for small angular diameter, the apparent effect is a rotation of the sphere, rather than a flattening.

\question Use conservation of energy and momentum to show that if a stationary billiard ball is hit by another of the same mass, then they emerge at right angles (assuming an elastic collision, so there is no conversion of kinetic energy to heat).

  \begin{solution}
    When all masses are the same, conservation of momentum reduces to
    \[
      \mathbf{v}_1 = \mathbf{v}_2 + \mathbf{v}_3.
    \]
    Since energy is proportional to the square of the speed, we are encouraged
    to take the squared magntiude of this expression,
    \[
      |\mathbf{v}_1|^2 = |\mathbf{v}_2|^2 + 2\mathbf{v}_2\cdot\mathbf{v}_3 + |\mathbf{v}_3|^2.
    \]
    This is an elastic collision, and so
    \[
      \frac{1}{2}m|\mathbf{v}_1|^2 = \frac{1}{2}m|\mathbf{v}_2|^2 + \frac{1}{2}m|\mathbf{v_3}|^2,
    \]
    or $|\mathbf{v}_1|^2 = |\mathbf{v}_2|^2 + |\mathbf{v}_3|^2$. Taken together
    with the second equation, this implies $\mathbf{v}_2\cdot\mathbf{v}_3=0$,
    which shows that the particles depart at a right angle to each other.
  \end{solution}

\question Show all this.

\question Why do spinning skaters pull in their arms to increase their rate of rotation?

  \begin{solution}
    In order for $\mathbf{M}$ to be conserved, a reduction in $\mathbf{x}$
    necessarily implies an increase on $\mathbf{p}$. To see this explicitly, we
    can expand $\mathbf{M}$ as
    \[
      {M}^{ab} = x^ap^b - x^bp^a,
    \]
    where it is now clear that the momentum increases in a way consistent with rotation.
  \end{solution}

\question Show this. (N.B. The position vector of the mass centre is the sum of the quantities m$\mathbf{x}$ divided by the sum of the masses $m$.)

\question Show that the formula for the increases mass is $m(1 - v^2/c^2)^{-1/2}$, where $v$ is the velocity of the particle in the second frame; see below.

  \begin{solution}
    We are given that the $4$-momentum is the $4$-velocity scaled by the rest
    mass $\mu$. Let us rotate our coordinate system so that the velocity vector
    of interest points along one axis, i.e. $v^a = (1, v^1, 0, 0)$. Reexamining
    the velocity space $\mathcal{H}^+$ considered earlier, we see that $v^0$ is
    given by $\cosh\rho$ and $v^1$ is given by $\sinh\rho$. Furthermore, from
    the formula for rapidity, we know that $\sinh\rho=v_3\cosh\rho$, where $v_3$
    is the magnitude of the $3$-velocity. So our
    four-momentum is
    \[
      p^a = \mu\cosh\rho(1, v_3, 0, 0)
    \]
    From inspection, we immediately see that $\mu\cosh\rho = m$, as this
    coincides with $p^a=(E, -\mathbf{p})$. The hyperbolic cosine function can be
    expressed in terms of the $3$-velocity by a careful examination of Fig.
    18.11. Being as the $4$-velocity is normalized to unity time, $\cosh\rho$ will be
    equal to this normalization factor.

    We can find $\cosh\rho$ from the formula given in $\S18.4$,
    \begin{align*}
      \cosh\rho &= \frac{1}{2}e^{\rho} + \frac{1}{2}e^{-\rho} \\
                &= \frac{1}{2}\sqrt{\frac{1+v}{1-v}} + \frac{1}{2}\sqrt{\frac{1-v}{1+v}}
    \end{align*}
    and so
    \begin{align*}
      (\cosh\rho)^2 &= \frac{1}{4}\Big(\frac{1+v}{1-v}\Big) + \frac{1}{2}\sqrt{\frac{1+v}{1-v}\frac{1-v}{1+v}} + \frac{1}{4}\Big(\frac{1-v}{1+v}\Big) \\
                    &= \frac{1}{4}\Big(\frac{2+2v^2}{1-v^2}\Big) + \frac{1}{2}\\
                    &= \frac{1}{1-v^2}
    \end{align*}
    Putting this all together gives
    \[
      m = \frac{\mu}{\sqrt{1-v^2}}.
    \]
  \end{solution}

\question Why?

  \begin{solution}
    The first equation, $\mathbf{p} = m\mathbf{v}$, is simply the definition of
    the $3$-momentum. The second and third were found in the previous exercise.
  \end{solution}

\question Use the Taylor series of $\S6.4$ to derive $(1+x)^{1/2}=1+\frac{1}{2}x-\frac{1}{8}x^2+\frac{1}{16}x^3-\cdots$ Hence, obtain a power series expansion for the energy $E = [(c^2\mu)^2+c^2\mathbf{p}^2]^{1/2}$ of a particle of rest-mass $\mu$ and $3$-momentum $\mathbf{p}$. Show that the leading term is just Einstein's $E=mc^2$ applied to the rest energy $\mu$, and that the next term is the Newtonian expression for kinetic energy. Write down the next two terms, so as to give better approximations to the full relativistic energy.

  \begin{solution}
    If we remove $c^2\mu$ from the radical, we obtain $E = c^2\mu\sqrt{1 +
      (v/c)^2}$. When $v \ll c$, we may expand this in $(v/c)^2$ to find
    \begin{align*}
      E &= \mu{c^2}\Big(1 + \frac{1}{2}\frac{v^2}{c^2} - \frac{1}{8}\frac{v^4}{c^4} +
          \frac{1}{16}\frac{v^6}{c^6} + \mathcal{O}(\frac{v^8}{c^8})\Big) \\
        &= \mu{c^2} + \frac{1}{2}\mu{v}^2 - \frac{1}{8}\mu\frac{v^4}{c^2} + \frac{1}{16}\mu\frac{v^6}{c^4} + \mathcal{O}\Big(\frac{v^8}{c^8}\Big)
    \end{align*}
  \end{solution}

\question Why?

  \begin{solution}
    On the particle's worldline, the only nonzero components of $4$-momentum and
    $4$-position are $p^0$ and $x^0$, respectively---to itself, the particle is
    at rest and at the origin of its coordinate system. So the only possible
    component of $M^{ab}$ that might be nonzero is $M^{00}$, but this is zero by
    the antisymmetric nature of the angular momentum tensor.
  \end{solution}

\question Explain, in detail, in the relativistic case.
\end{questions}

\end{document}
