\documentclass[../feynman-lectures-on-physics.tex]{subfiles}

\begin{document}

\section{Vectors}

Generalize Exs. 6.1 through 6.4 to three dimensional motion using vector notation.

\begin{questions}
\question If two bodies have masses $m_1$ and $m_2$ and are moving at velocities $\mathbf{v}_1$ and $\mathbf{v}_2$, show that the CM system is moving at velocity
  \[
    \mathbf{v}_{\text{cm}} = \frac{m_1\mathbf{v}_1 + m_2\mathbf{v}_2}{m_1+m_2}
    .\] 

  \begin{solution}
    If we transform to the center of mass system, we have, by definition		
    \[
      m_1(\mathbf{v}_1 - \mathbf{v}_{\text{cm}}) + m_2(\mathbf{v}_2 - \mathbf{v}_{\text{cm}}) = 0
      .\] 
    Rearranging this gives
    \[
      \mathbf{v}_{\text{cm}} = \frac{m_1\mathbf{v}_1 + m_2\mathbf{v}_2}{m_1+m_2}
      .\] 
  \end{solution}

\question Show that for $N$ bodies of masses $m_i$ and velocities $\mathbf{v}_i$ the velocity of the coordinate system, in which the total momentum is zero, is given by
  \[
    \mathbf{v}_{\text{cm}} = \frac{\sum_{i=1}^Nm_i\mathbf{v}_i}{\sum_{i=1}^Nm_i}
    .\] 

  \begin{solution}
    For the same reasons given in the previous answer, we have
    \[
      \sum_{i=1}^Nm_i(\mathbf{v}_i - \mathbf{v}_{\text{cm}}) = 0
      ,\] 
    which can be rearranged to get
    \[
      \mathbf{v}_{\text{cm}} = \frac{\sum_{i=1}^Nm_i\mathbf{v}_i}{\sum_{i=1}^Nm_i}
      .\] 
  \end{solution}

\question If $T$ is the total kinetic energy of the two masses in Ex. 7.1 and $T_{\text{CM}}$ their total kinetic energy in the CM system, show that
\[
T = T_{\text{CM}} + \Big(\frac{m_1+m_2}{2}\Big)\|\mathbf{v}_{\text{cm}}\|^2
.\] 

\begin{solution}
	Following the argument used in the answer to Ex. 6.3, we have
	\[
	T_{\text{CM}} = \frac{1}{2}m_1\mathbf{v}_1'\cdot\mathbf{v}_1' + \frac{1}{2}m_2\mathbf{v}_2'\cdot\mathbf{v}_2'
	.\] 
	We can write the expression for the energy of the original system using $\mathbf{v}_1' = \mathbf{v}_1 - \mathbf{v}_{\text{CM}}$ and $\mathbf{v}_2' = \mathbf{v}_2 - \mathbf{v}_{\text{CM}}$,
	\begin{align*}
		T &= \frac{1}{2}m_1(\mathbf{v}_1'+\mathbf{v}_{\text{CM}})\cdot(\mathbf{v}_1'+\mathbf{v}_{\text{CM}}) + \frac{1}{2}m_2(\mathbf{v}_2'+\mathbf{v}_{\text{CM}})\cdot(\mathbf{v}_2'+\mathbf{v}_{\text{CM}}) \\
		&= \frac{1}{2}m_1\mathbf{v}_1'\cdot\mathbf{v}_1' + m_1\mathbf{v}_1'\cdot\mathbf{v}_{\text{CM}} + \frac{1}{2}m_1\mathbf{v}_{\text{CM}}\cdot\mathbf{v}_{\text{CM}} + \frac{1}{2}m_2\mathbf{v}_2'\cdot\mathbf{v}_2' + m_2\mathbf{v}_2'\cdot\mathbf{v}_{\text{CM}} + \frac{1}{2}m_2\mathbf{v}_{\text{CM}}\cdot\mathbf{v}_{\text{CM}} \\
		&= T_{\text{CM}} + (m_1\mathbf{v}_1' + m_2\mathbf{v}_2')\cdot\mathbf{v}_{\text{CM}} + \Big(\frac{m_1+m_2}{2}\Big)\mathbf{v}_{\text{CM}}\cdot\mathbf{v}_{\text{CM}} \\
		&= T_{\text{CM}} + \Big(\frac{m_1+m_2}{2}\Big)\|\mathbf{v}_{\text{CM}}\|^2
	\end{align*}
	where the middle term in the second to last equation disappears because the momentum of the center of mass system is zero.
\end{solution}

\question Generalize the result of Ex. 7.3 to $N$ masses. Show that
\[
T = T_{\text{CM}} + \frac{\sum_{i=1}^Nm_i}{2}\|\mathbf{v}_{\text{cm}}\|^2
.\] 

\begin{solution}
	In this case,
	\[
	T_{\text{CM}} = \sum_{i=1}^N\frac{1}{2}m_i\mathbf{v}_i'\cdot\mathbf{v}_i'
	,\] 
	and so we have
	\begin{align*}
		T &= \sum_{i=1}^N\frac{1}{2}m_i(\mathbf{v}_i' + \mathbf{v}_{\text{CM}})\cdot(\mathbf{v}_i' + \mathbf{v}_{\text{CM}}) \\
		&= \sum_{i=1}^N\frac{1}{2}m_i\mathbf{v}_i'\cdot\mathbf{v}_i' + \sum_{i=1}^Nm_i\mathbf{v}_i'\cdot\mathbf{v}_{\text{CM}} + \sum_{i=1}^N\frac{1}{2}m_i\mathbf{v}_{\text{CM}}\cdot\mathbf{v}_{\text{CM}} \\
		&= T_{\text{CM}} + \mathbf{v}_{CM}\cdot\sum_{i=1}^Nm_i\mathbf{v}_i' + \frac{\sum_{i=1}^Nm_i}{2}\|\mathbf{v}_{\text{CM}}\|^2 \\
		&= T_{\text{CM}} + \frac{\sum_{i=1}^Nm_i}{2}\|\mathbf{v}_{\text{CM}}\|^2, 
	\end{align*}
	where the middle term in the second to last equation disappears for the same reason as it did in the previous exercise.
\end{solution}

\question A particle is initially at a point $\mathbf{r}_0$, and is moving under gravity with an initial velocity $\mathbf{v}_0$. Find the subsequent motion $\mathbf{r}(t)$.

\begin{solution}
	Assuming the gravitational field remains roughly constant over the particle's trajectory, we may denote its acceleration by $\mathbf{a}$ and integrate twice to find
	\begin{align*}
		\mathbf{r}(t) &= \int\int\mathbf{a}\,\mathrm{d}t\mathrm{d}t \\
		&= \int \mathbf{a}t + \mathbf{v}_0\,\mathrm{d}t \\
		&= \frac{1}{2}\mathbf{a}t^2 + \mathbf{v}_0t + \mathbf{r}_0.
	\end{align*}
\end{solution}

\question You are given three vectors,
\begin{align*}
	\mathbf{a} &= 3\mathbf{i} + 2\mathbf{j} - \mathbf{k}, \\
	\mathbf{b} &= 2\mathbf{i} - \mathbf{j} + \mathbf{k}, \\
	\mathbf{c} &= \mathbf{i} + 3\mathbf{j} \\
\end{align*}
Find
\begin{parts}
	\part $\mathbf{a} + \mathbf{b}$
	\part $\mathbf{a} - \mathbf{b}$
	\part $\mathbf{a}_x$
	\part $\mathbf{a} \cdot \mathbf{i}$
	\part $\mathbf{a} \cdot \mathbf{b}$
	\part $(\mathbf{a}\cdot\mathbf{c})\mathbf{b} - (\mathbf{a}\cdot\mathbf{b})\mathbf{c}$
\end{parts}

\begin{solution}
	\begin{parts}
		\part $(3+2)\mathbf{i} + (2-1)\mathbf{j} + (-1+1)\mathbf{k} = 5\mathbf{i} - \mathbf{j}$
		\part $(3-2)\mathbf{i} + (2 - {-1})\mathbf{j} + (-1-1)\mathbf{k} = \mathbf{i} + 3\mathbf{j} - 2\mathbf{k}$
		\part $3$
		\part $3\cdot{1} + 0\cdot{0} + 0\cdot{0} = 3$
		\part $3\cdot{2}+2\cdot{-1}+{-1}\cdot{1} = 3$
		\part $(3\cdot{1} + 2\cdot{3} + -1\cdot{0})(2\mathbf{i}-\mathbf{j}+\mathbf{k}) - (3)(\mathbf{i} + 3\mathbf{j}) = (19 - 3)\mathbf{i} + (-9 + 9)\mathbf{j} + (9 + 0)\mathbf{k} = 16\mathbf{i} + 9\mathbf{k}$
	\end{parts}
\end{solution}

\question A particle of mass $\SI{1}{\kilo\gram}$ is moving in such a way that its position is described by the vector
\[
\mathbf{r}(t) = t\mathbf{i} + (t + t^2/2)\mathbf{j} - (4/\pi^2)\sin(\pi{t}/2)\mathbf{k}.
\]
\begin{parts}
	\part Find the position, velocity $\mathbf{v}(t)$, acceleration $\mathbf{a}(t)$, and kinetic energy $T(t)$ of the particle at $t=0$ and $t=1$ second.
	\part Find the force $\mathbf{F}(t)$ that will produce this motion.
	\part Find the radius of curvature $R(t)$ of the particle's path at $t=1$ second.
\end{parts}

\begin{solution}
	\begin{parts}
		\part The position vector is already given. Its derivative is the sought velocity vector,
		\[
		\mathbf{v}(t) = \frac{\mathrm{d}\mathbf{r}(t)}{\mathrm{d}t} = \mathbf{i} + (1 + t)\mathbf{j} - (2/\pi)\cos(\pi{t}/2)\mathbf{k}.
		\]
		The derivative of this is the acceleration,
		\[
		\mathbf{a}(t) = \frac{\mathrm{d}\mathbf{v}(t)}{\mathrm{d}t} = \mathbf{j} + \sin(\pi{t}/2)\mathbf{k}.
		\]
		The particle's kinetic energy at time $t$ is given by $m\|v(t)\|^2/2$, or
		\[
		T(t) = \frac{1}{2}\Big(1 + (1 + t)^2 + (2/\pi)^2\cos^2(\pi{t}/2)\Big) = 1 + t + t^2/2 + (2/\pi^2)\cos(\pi{t}/2).
		\]
		At $0$ and $1$ seconds, these values are
		\[
		\mathbf{r}(0) = \mathbf{0}\,\si{\meter} \qquad \mathbf{v}(0) = \mathbf{i} + \mathbf{j} - (2/\pi)\mathbf{k}\,\si{\meter\per\second} \qquad \mathbf{a}(0) = \mathbf{j}\,\si{\meter\per\second\squared} \qquad T(0) = 1 + (2/\pi^2)\,\si{\joule}
		\]
		and
		\[
		\mathbf{r}(1) = \mathbf{i} + (3/2)\mathbf{j} - (4/\pi^2)\mathbf{k}\,\si{\meter} \qquad \mathbf{v}(1) = \mathbf{i} + 2\mathbf{j}\,\si{\meter\per\second} \qquad \mathbf{a}(1) = \mathbf{j} + \mathbf{k}\si{\meter\per\second\squared} \qquad T(1) = 5/2\,\si{joule}
		\]
		\part The force capable of producing such a motion is simply the acceleration vector times the mass of the particle, or
		\[
		\mathbf{F}(t) = m\mathbf{a}(t) = \mathbf{j} + \sin(\pi{t}/2)\mathbf{k}.
		\]
		\part We can find the radius of curvature by solving $\|\mathbf{a}_{\perp}(t)\|=\|\mathbf{v}(t)\|^2/R$ for $R$, where $\mathbf{a}_{\perp}(t)$ is the acceleration perpendicular to the velocity vector. This can be found by subtracting off the part of the acceleration vector tangential to the velocity vector,
		\[
		\mathbf{a}_{\perp}(t) = \mathbf{a}(t) - \frac{\mathbf{v}(t)\cdot\mathbf{a}(t)}{\|\mathbf{v}(t)\|^2}\mathbf{v}(t).
		\]
		At $1$ second, this becomes
		\[
		\mathbf{a}_{\perp}(1) = -(2/5)\mathbf{i} + (1/5)\mathbf{j} + \mathbf{k}\,\si{\meter\per\second\squared},
		\]
		while $\|\mathbf{v}(1)\|^2 = \SI{5}{\meter\per\second}$. The magnitude of $\mathbf{a}_{\perp}(1)$ is $sqrt(6/5)$, and so the radius of curvature at $1$ second is
		\[
		R(1) = 5/\sqrt{6/5} \approx \SI{4.56}{\meter}.
		\]
	\end{parts}
\end{solution}

\question A pilot flying at an air speed of $100$ knots wishes to travel due north. He knows, from talking to the airport meteorologist, that there is a $25$ knot wind from west to east at his flight altitude.
\begin{parts}
	\part In what direction should he head his plane?
	\part What will be the duration $T$ of his flight, if his destination is $100$ land miles away? (Neglect the time for landing and take-off, and note that $1$ knot $=$ $1.15$ miles per hour.)
\end{parts}

\begin{solution}
	\begin{parts}
		\part The desired velocity vector, the planes velocity vector, and the wind's
		velocity vector will combine to make a right triangle with a hypotenuse of
		$100$ knots and a short side of $25$ knots. From north, then, the pilot will
		have to face
		\[
		\arcsin\frac{25}{100} \approx 14.48^\circ
		\]
		towards the east
		\part The pilot's northward velocity is given by $v_N = \sqrt{100^2 - 25^2}
		\approx 96.82$ knots, or $\SI{111.35}{mi\per{h}}$. So the duration of the
		flight will be
		\[
		T = \frac{d}{v_N} = \SI{53.9}{min}.
		\]
	\end{parts}
\end{solution}

\question A cyclist rides at $\SI{10}{{mi}\per{h}}$ due north and the wind, which is blowing at $\SI{6}{mi\per{h}}$ from a point between $N$ and $E$, appears to the cyclist to come from a point $15^\circ$E of N.
\begin{parts}
	\part Find the true direction of the wind.
	\part Find the direction in which the wind will appear to meet the cyclist on his return if he rides at the same speed.
\end{parts}

\begin{solution}
	\begin{parts}
		\part We can find the true direction of the wind by observing that the
		apparent velocity $\mathbf{v}_A$ is related to the true velocity
		$\mathbf{v}_T$ and the cyclist's velocity $\mathbf{v}_C$ by
		$\mathbf{v}_T - \mathbf{v}_C = \mathbf{v}_A$. In Cartesian coordinates,
		the three vectors are given by
		\begin{gather*}
			\mathbf{v}_C = 10\mathbf{j} \\
			\mathbf{v}_A = -A\sin(15^\circ)\mathbf{i} - A\cos(15^\circ)\mathbf{j} \\
			\mathbf{v}_T = -6\sin(\theta)\mathbf{i} - 6\cos(\theta)\mathbf{j}
		\end{gather*}
		If we equate the $\|\mathbf{v}_T\|^2 = 36$ to the squared
		magnitude of $\mathbf{v}_A + \mathbf{v}_C$, we find $A \approx
		\SI{15.072}{mi\per{h}}$, and so the angle the true velocity vector makes
		with respect to north is
		\[
		\angle{\mathbf{v}_T} =
		\arctan\Big(\frac{-15.072\sin(15^\circ)}{10-15.072\cos(15^\circ)}\Big)
		= 40.5^\circ.
		\]
		\part Using the relationship above, we find
		\[
		\mathbf{v}_A = \mathbf{v}_T - \mathbf{v}_C = (-6\sin(40.5^\circ))\mathbf{i} + (10-6\cos(40.5^\circ))\mathbf{j}
		\]
		which makes an angle of $35.6^\circ$E of S.
	\end{parts}
\end{solution}
	
	\question A man standing on the bank of a river $\SI{1.0}{mi}$ wide wishes to get to a point directly opposite him on the other bank. He can do this in two ways:
	\begin{parts}
		\part head somewhat upstream, so that his resultant motion is straight across,
		\part head toward the opposite bank and then walk up along the bank from the point downstream to which the current has carried him.
	\end{parts}
	If he can swim $\SI{2.5}{mi\per{h}}$ and walk $\SI{4.0}{mi\per{h}}$, and if
	the current is $\SI{2.0}{mi\per{h}}$ which is the faster way to cross, and by
	how much?
	
	\begin{solution}
		In the first case, the velocity vectors for the stream and his independent motion are
		given by
		\begin{gather*}
			\mathbf{v}_S = -2\hat{\mathbf{i}} \\
			\mathbf{v}_M = 2.5\cos\theta\hat{\mathbf{i}} + 2.5\sin\theta\hat{\mathbf{j}},
		\end{gather*}
		where $\theta$ is such that the $\hat{\mathbf{i}}$ component of the vector sum
		of these velocities is zero, i.e. $\theta = \cos^{-1}(4/5)$. In this case,
		the man's overall velocity is given by
		\[
		\mathbf{v}_T = 2.5\sin\cos^{-1}\frac{4}{5}\hat{\mathbf{j}} = 1.5\hat{\mathbf{j}}.
		\]
		The total time taken to cross the river is then $t =
		\|\mathbf{d}\|/\|\mathbf{v}_T\|$, or $40$ minutes.
		
		In the second case, the velocity vectors for the stream and the man are
		\begin{gather*}
			\mathbf{v}_S = -2\hat{\mathbf{i}} \\
			\mathbf{v}_M = 2.5\hat{\mathbf{j}}.
		\end{gather*}
		The man will cross the stream once he has gone a distance of one mile in the
		$\hat{\mathbf{j}}$ direction, which occurs at the $24$ minute mark. At this point, he will have moved a distance
		$-0.8$ miles downstream. At a speed of $\SI{4}{mi\per{h}}$, he will cover
		this distance in $12$ minutes to arrive at his destination, giving a total
		travel time of $36$ minutes. Hence, method two is faster by approximately
		$4$ minutes.
	\end{solution}
	
	\question A motorboat that runs at a constant speed $V$ relative to the water
	is operated in a straight river channel where the water is flowing smoothly
	with a constant speed $R$. The boat is first sent on a round trip from its
	anchor point to a point a distance $d$ directly upstream. It is then sent on
	a round trip from its anchor point to a point a distance $d$ away directly
	across the stream. For simplicity assume that the boat runs the entire
	distance in each case at full speed and that no time is lost in reversing
	course at the end of the outward lap. If $t_V$ is the time the boat took to
	make the round trip in line with the stream flow, $t_A$ the time the boat
	took to make the round trip across the stream, and $t_L$ the time the boat
	would take to go a distance $2d$ on a lake,
	\begin{parts}
		\part What is the ratio $t_V$/$t_A$?
		\part What is the ratio $t_A/t_L$?
	\end{parts}
	
	\begin{solution}
		\begin{parts}
			\part The first case is particularly simple, as the magnitude of the boat's
			total velocity moving upstream is $V - R$, while the magnitude of its
			velocity downstream is $V + R$. This gives a total time of
			\[
			t_V = \frac{d}{V-R} + \frac{d}{V+R} = \frac{2d}{V^2-R^2}V.
			\]
			For the second case, if the boat is to move in a straight line, then the
			boat's must be at an angle to the channel's edge such that $V\cos\theta
			= R$, which, after geometric considerations, gives a cross-stream
			velocity of
			\[
			V\sin\theta = V\sin\cos^{-1}\frac{R}{V} = \sqrt{V^2-R^2}.
			\]
			By symmetry, this velocity is the same for both legs of the trip. The
			total time taken is
			\[
			t_A = \frac{2d}{\sqrt{V^2-R^2}},
			\]
			and thus the desired ratio is
			\[
			\frac{t_V}{t_A} = \frac{V}{\sqrt{V^2-R^2]}}.
			\]
			This makes sense, as if $R$ were to be $0$, the two times would be equal.
			\part The time taken to go a distance of $2d$ on a lake is simply
			\[
			t_L = \frac{2d}{V}
			\]
			and from this we see that $t_A/t_L=t_V/t_A$.
		\end{parts}
	\end{solution}
	
	\question Use vectors to find the great circle distance $D$ between two points
	on the earth (radius = $r_{\varEarth}$), whose latitudes and longitudes are
	$(\lambda_1,\phi_1)$ and $(\lambda_2,\phi_2)$.
	
	\textit{Note}: Use a system of rectangular coordinates with the origin at the
	center of the earth, one axis along the earth's axis, another pointed toward
	$\lambda=0$, $\phi=0$, and the third axis pointed toward $\lambda=0$, $\phi =
	90^\circ$ W. Let longitudes vary from $0^\circ$ westward to $360^\circ$.
	
	\question What is the magnitude and direction of the acceleration $\mathbf{a}$
	of the moon at
	\begin{parts}
		\part New moon?
		\part Quarter moon?
		\part Full moon?
	\end{parts}
	\textit{Note}:
	\begin{align*}
		R_{\varEarth\Sun} &= \SI{1.50e8}{\kilo\meter} \\
		R_{\varEarth\leftmoon} &= \SI{3.85e5}{\kilo\meter} \\
		M_{\Sun} &= \SI{3.33e5}{M_{\varEarth}}
	\end{align*}
	
	\question Two identical $45^\circ$ wedges $M_1$ and $M_2$, with smooth faces
	and $M_1=M_2=\SI{8}{\kilo\gram}$, are used to move a smooth-faced mass $M =
	\SI{384}{\kilo\gram}$, as shown in Fig. 7-1. Both wedges rest upon a smooth
	horizontal plane; one wedge is butted against a vertical wall, and to the
	other wedge a force $F=\SI{592}{kg-wt}$ is applied horizontally.
	\begin{parts}
		\part What is the magnitude and direction of the acceleration
		$\mathbf{a}_1$ of the movable wedge $M_1$?
		\part What is the magnitude and direction of the acceleration
		$\mathbf{a}$ of the larger wedge $M$?
		\part What force $F_2$ does the stationary wedge $M_2$ exert on the
		heavy mass $M$?
	\end{parts}
	Neglect friction.
	
	\question A mass $m$ is suspended from a frictionless pivot at the end of a
	string of arbitrary length, and is set to whirling in a horizontal circular
	path whose plane is a distance $H$ below the pivot point, as shown in Fig.
	7-2. Find the period of revolution $T$ of the mass in its orbit.
	
	\begin{solution}
		There are only two forces acting on the mass, friction and tension. If the
		mass is to move in a plane, the vertical components of both forces must
		cancel, and so
		\[
		A\sin\theta = mg,
		\]
		where $A$ is the tension and $\theta$
		is the angle made between the string and the horizontal. The requirement
		that the mass move in a circular manner introduces the additional
		restriction of
		\[
		A\cos\theta = \frac{mv^2}{R}.
		\]
		We may divide the first equation by the second to find
		\[
		\tan\theta = \frac{gR}{v^2},
		\]
		and, using $\tan\theta = H/R$, we obtain
		\[
		\frac{H}{R} = \frac{gR}{v^2}.
		\]
		The period of the mass's revolution is given by the total distance it
		traverses in one revolution divided by its velocity, or $T = 2\pi{R}/v$.
		Substituting this into the above equation and simplifying gives
		\[
		T = 2\pi\sqrt{\frac{H}{g}};
		\]
		the greater the distance between the ceiling and the mass, the longer its period.
	\end{solution}
	
	\question Two small, sticky, putty balls $a$ and $b$, each of mass $1$ gram,
	travel under the influence of gravity with acceleration
	$-9.8\hat{\mathbf{k}}\,\si{\meter\per\second\squared}$. Given the initial
	condition at $t=0$,
	\begin{align*}
		\mathbf{r}_a(0) &= 7\hat{\mathbf{i}} + 4.9\hat{\mathbf{k}}, \\
		\mathbf{v}_a(0) &= 7\hat{\mathbf{i}} + 3\hat{\mathbf{j}}, \\
		\mathbf{r}_b(0) &= 49\hat{\mathbf{i}} + 4.9\hat{\mathbf{k}}, \\
		\mathbf{v}_b(0) &= -7\hat{\mathbf{i}} + 3\hat{\mathbf{j}},
	\end{align*}
	find $\mathbf{r}_a(t)$ and $\mathbf{r}_b(t)$ for all times $t>0$.
	
	\begin{solution}
		By simple integration, one can immediately find
		\begin{align*}
			\mathbf{r}_a(t) &= (7 + 7t)\hat{\mathbf{i}} + (3t)\hat{\mathbf{j}} + (4.9 - 4.9t^2)\hat{\mathbf{k}} \\
			\mathbf{r}_b(t) &= (49 - 7t)\hat{\mathbf{i}} + (3t)\hat{\mathbf{j}} + (4.9-4.9t^2)\hat{\mathbf{k}}
		\end{align*}
		which, from inspection, breaks down at $t = 3$ seconds, as this is when
		both position vectors are equal: the two balls will stick together and
		continue on as one. To find the resulting position, we may use
		conservation of momentum at the instant of impact to obtain
		\[
		m\mathbf{v}_a(3) + m\mathbf{v}_b(3) = 2m\mathbf{v_c}(3),
		\]
		or
		\[
		\mathbf{v}_c(3) = 3\hat{\mathbf{j}} - 29.4\hat{\mathbf{k}}.
		\]
		Then, for $t>3$, we have
		\begin{align*}
			\mathbf{r}_c(t) &= \mathbf{r}_a(3) + \mathbf{v}_c(3)(t - 3) + \frac{1}{2}\mathbf{a}(3)(t-3)^2 \\
			&= (28)\hat{\mathbf{i}} + (9 + 3(t-3))\hat{\mathbf{j}} + (-39.2 - 29.4(t-3) - 4.9(t-3)^2)\hat{\mathbf{k}} \\
			&= (28)\hat{\mathbf{i}} + (3t)\hat{\mathbf{j}} + (4.9-4.9t^2)\hat{\mathbf{k}}.
		\end{align*}
	\end{solution}
	
	\question You are on a ship traveling steadily east at $15$ knots. A ship
	o a steady course whose speed is known to be $26$ knots is observed
	$6.0$ miles due south of you; it is later observed to pass behind you,
	its distance of closest approach being $3.0$ miles.
	\begin{parts}
		\part What was the course of the other ship?
		\part What was the time $T$ between its position south of you and its
		position of closest approach?
	\end{parts}

\end{questions}

\end{document}
