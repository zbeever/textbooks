\documentclass[../feynman-lectures-on-physics.tex]{subfiles}

\begin{document}

\section{Newton's Laws}

\begin{questions}

\question Two blocks of mass $m_1=\SI{1}{\kilo\gram}$, $m_2=\SI{2}{\kilo\gram}$
on a horizontal surface, connected by a string, are being pulled by another
string which is attached to a mass $m_3=\SI{2}{\kilo\gram}$ hanging ove ra
pulley, as shown in Fig. 5-1. Neglect friction and the masses of the pulley and
the strings.
\begin{parts}
  \part Sketch free-body diagrams for all masses, showing the forces acting.
  \part Find the acceleration $a$ of the masses.
  \part Find the tensions $T_1$ and $T_2$ in the strings.
\end{parts}

\begin{solution}
\begin{parts}
\part The hanging mass has two forces acting upon it: gravity in the downward
direction and $T_2$ in the upward one. The mass $m_2$ has four forces: gravity
acting downward, the normal force acting upward, $T_2$ acting to the left, and
$T_1$ acting to the right. Finally, mass $m_3$ has three forces: gravity acting
downward, the normal force acting upward, and $T_1$ acting to the left.
\part We can find the acceleration by noting that it is the same throughout the
system. From $m_3$, we see $m_3a = m_3g - T_2$, from $m_2$, we see $m_2a = T_2 -
T_1$, and from $m_1$ we see $m_1a = T_1$. Putting everything in terms of the
masses of our system gives
\[
m_3a = m_3g - (m_2a + T_1) = m_3g - m_2a - m_1a.
\]
Solving for $a$ yields
\[
a = g\frac{m_3}{m_1 + m_2 + m_3} = \frac{2}{5}g.
\]
\part Substituting in our above answer for $a$ gives us
\begin{align*}
  T_1 &= \frac{m_1m_3}{m_1 + m_2 + m_3} = \frac{2}{5}\,\si{\newton} \\
  T_2 &= \frac{m_2m_3 + m_1m_3}{m_1 + m_2 + m_3} = \frac{6}{5}\,\si{\newton}
\end{align*}
\end{parts}
\end{solution}

\question A mass $m$ (kg) hangs on a cord suspended from an elevator which is
descending with an acceleration $0.1g$. What is the tension $T$ in the cord in newtons?

\begin{solution}
  We are given the acceleration, and we know that
  \[
    ma = mg - T.
  \]
  Solving for $T$ yields $T = \frac{9}{10}mg = 8.82m\,\si{\newton}$.
\end{solution}

\question Two objects of mass $m=\SI{1}{\kilo\gram}$ each, connected by a taut
string of length $L=\SI{2}{\meter}$, move in a circular orbit with constant
speed $V=\SI{5}{\meter\per\second}$ about their common center $C$ in a zero-$g$
environment, as shwon in Fig. 5-2. What is the tension $T$ in the string in
newtons?

\begin{solution}
The acceleration felt by the masses is in the inward radial direction and it is
given by
\[
a = \frac{v^2}{r}.
\]
The only force that can provide such a centripetal acceleration is the tension
in the string, and so we must have
\[
m\frac{v^2}{r} = T,
\]
or $T = \SI{25}{\newton}$. 
\end{solution}

\question Referring to Fig. 5-3: What horizontal force $F$ must be constantly
applied to $M$ so that $M_1$ and $M_2$ do not move relative to $M$? Neglect
friction.

\begin{solution}
  When the force is applied, it will accelerate the system as whole, and so
  \[
    F = (M+M_1+M_2)a.
  \]
  Meanwhile, both $M_1$ and $M_2$ can undergo acceleration in addition to that
  from the force. Specifically,
  \begin{align*}
    M_1(a+a_{M_2}) &= T \\
    M_2a_{M_2} &= M_2g - T \\
    M_2a &= F_N.
  \end{align*}
  The condition that our subsystem does not move relative to $M$ is $a_{M_2}$
  (the additional acceleration) is zero, or $M_2g = T$. Putting this into the
  first of the above three equations, substituting in $a$, and solving for $F$
  yields
  \[
    F = g\frac{M_2}{M_1}(M+M_1+M_2).
  \]
\end{solution}

\question Reffering to Fig. 5-4: What horizontal force $F$ must be constantly
applied to $M = \SI{21}{\kilo\gram}$ so that $m_1=\SI{5}{\kilo\gram}$ does not
move relative to $m_2=\SI{4}{\kilo\gram}$? Neglect friction.

\begin{solution}
  In order for $m_1$ and $m_2$ to not move relative to each other, the system as
  a whole must undergo uniform acceleration. From this fact, we can write
  \begin{align*}
    F &= (M + m_1 + m_2)a \\
    T &= m_1a \\
    T_x &= m_2a \\
    T_y - m_2g &= 0.
  \end{align*}
  We may use the last two equations to eliminate $T$ from the second one,
  finding
  \[
    \sqrt{m_2^2a^2 + m_2^2g^2} = m_1a,
  \]
  or
  \[
    a = \frac{m_2g}{\sqrt{m_1^2-m_2^2}}.
  \]
  Feeding this into our first equations gives us a force of
  \[
    F = (M+m_1+m_2)\frac{m_2g}{\sqrt{m_1^2-m_2^2}} = \SI{392}{\newton}.
  \]
\end{solution}

\question In the system shown in Fig. 5-5, $M_1$ slides without friction on the
inclined plane. $\theta=30^\circ$, $M_1=\SI{400}{\gram}$, $M_2=\SI{200}{\gram}$.
Find the acceleration $\mathbf{a}$ of $M_2$ and the tension $T$ in the cords.

\begin{solution}
  Drawing a force diagram for the entire system, we find that
  \begin{gather*}
    M_1a_1 = T_2 - M_1g\sin\theta \\
    T_1 = 2T_2 \\
    M_2a_2 = M_2g - T_2
  \end{gather*}
  where positive $a_1$ and $a_2$ corresponds to the situation where the
  hanging mass is falling. Here, $T_1$ corresponds to the rightmost cord and
  $T_2$ to the leftmost.

  These relations leave us with three unknowns, $T_1$, $a_1$, and $a_2$. To
  eliminate one more, observe that when $M_2$ falls a distance $\Delta{x}$, the
  pulley attached to $M_1$ rises by the same amount: this shortens the cord
  wrapped around the pulley, causing the mass $M_1$ to rise by an additional
  $\Delta{x}$. That is,
  \[
    2\Delta{x}_{M_2} = \Delta{x}_{M_1}.
  \]
  Differentiating both sides twice gives us the same relationship between their
  respective accelerations, or $2a_{M_2}=a_{M_1}$.

  Solving for $a_2$ yields
  \[
    a_2 = \frac{1}{1+4\frac{M_1}{M_2}}\Big(g - 2\frac{M_1}{M_2}g\sin\theta\Big),
  \]
  which comes out to be $-g/9$, or an acceleration of $g/9$ upward. Substituting
  this value into the equation relating $M_2$'s acceleration to its force yields
  a tension of
  \[
    T_2 = M_2(g - a_2) = \frac{10}{9}M_2g = \SI{2.18}{\newton}.
  \]
\end{solution}

\question A simple crane is made of two parts, ``A'' with mass $M_A$, length
$D$, height $H$, and distance $D/2$ between wheels of radius $r$; and part
``B,'' a uniform rod or boom of length $L$ and mass $M_B$. The crane is shown
assembled in Fig. 5-6, with the pivot point $P$ at midpoint of top of $A$. The
center of gravity of $A$ is midway between the wheels.
\begin{parts}
\part With the rod or boom $B$ set at angle $\theta$ with teh horizontal, what
is the maximum mass $M_{\text{max}}$ that the crane can lift without tipping
over?
\part If there is a mass $M' = (4/5)M_{\text{max}}$ at the end of the rope, what
is the minimum time $t$ necessary to raise this load $M'$ a distance
$(L\sin\theta)$ from the ground? (The angle $\theta$ remains fixed, and the mass
of the rope may be neglected.)
\end{parts}

\begin{solution}
\begin{parts}
\part The crane will tip over if its center of mass is in front of its wheels.
That is,
\[
  \frac{M_A\cdot\frac{D}{2} + M_B\cdot(\frac{D}{2} + \frac{L}{2}\cos\theta) +
  M'\cdot(\frac{D}{2} + L\cos\theta)}{M_A+M_B+M'} \leq \frac{3}{4}D.
\]
Solving for $M'$ and taking the edge case where the center of mass is exactly on
the front wheel gives a maximum mass of
\[
  M_{\text{max}} = \frac{(M_A+M_B)D - 2M_BL\cos\theta}{4L\cos\theta - D}.
\]
\part The maximum force our crane can exert when lifting an object must be equal
to the force generated by the maximum weight (due to action and reaction),
\[
T_{\text{max}} = M_{\text{max}}g.
\]
The largest acceleration $M'$ can undergo, then, is
\[
M'a = \frac{4}{5}M_{\text{max}}a = T_{\text{max}} - \frac{4}{5}M_{\text{max}}g = \frac{1}{5}M_{\text{max}}g,
\]
or $a = g/4$. Assuming $M'$ starts from rest, the total time it takes to raise
$M'$ to $L\sin\theta$ is given by solving $\Delta{x} = \frac{1}{2}at^2$ for $t$
and substituting in our value for $a$, i.e.
\[
t = \sqrt{\frac{2L\sin\theta}{a}} = \sqrt{\frac{8L\sin\theta}{g}}.
\]
\end{parts}
\end{solution}

\question An early arrangement for measuring the acceleration of gravity, called
Atwood's Machine, is shown in Fig. 5-7. The pulley $P$ and cord $C$ have
negligible mass and friction. The system is balanced with equal masses $M$ on
each side as shown (solid line), and then a small rider $m$ is added to one
side. The combined masses accelerate through a certain distance $h$, the rider
is caught on a ring and the two equal masses then move on with constant speed,
$v$. Find the value of $g$ that corresponds to the measured valeus of $m$, $M$,
$h$, and $v$.

\begin{solution}
  We can find the acceleration of the system by drawing force diagrams for each
  weight, coming up with the relations
  \begin{gather*}
    Ma = T - Mg \\
    (M+m)a = (M+m)g - T.
  \end{gather*}
  Solving for $g$ gives us
  \[
    g = \frac{m+2M}{m}a.
  \]
  Now, as the system starts from rest, the final velocity can be related to the
  acceleration by $v = a\Delta{t}$, and the total distance (up to striking the
  ring) can be related to the
  acceleration by $\frac{1}{2}a\Delta{t}^2 = h$. Removing $\Delta{t}$ from the
  second equations reveals
  \[
    a = \frac{v^2}{2h},
  \]
  and so
  \[
    g = \frac{m+2M}{2mh}v^2
  \]



\end{solution}

\question An elevator of mass $M_2$ has hanging from its ceiling a mass $M_1$,
as shown in Fig. 5-8. The elevator is being accelerated upward by a constant
force $F$. ($F$ is greater than $(M_1+M_2)/g$.) The mass $M_1$ is initially a
distance $s$ above the elevator floor.
\begin{parts}
  \part Find the acceleration $a_0$ of the elevator.
  \part What is the tension $T$ in the string connecting the mass $M_1$ to the
  elevator?
  \part If the string suddenly breaks, what is the acceleration $a$ of the
  elevator immediately after, and wha tis the acceleration $a'$ of mass $M_1$?
  \part How much time $t$ does it take for $M_1$ to hit the bottom of the elevator?
\end{parts}

\begin{solution}
 \begin{parts}
   \part The elevator total acceleration can be found by solving
   \[
     (M_1+M_2)a_0 = F - (M_1+M_2)g,
   \]
   giving a value of
   \[
     a_0 = \frac{F - (M_1+M_2)g}{M_1+M_2)}.
   \]
   \part Because $M_1$ is undergoing the same acceleration as the elevator, we
   have
   \[
     M_1a_0 = T - M_1g,
   \]
   or
   \[
     T = \frac{M_1}{M_1 + M_2}F.
   \]
   \part If the string breaks, the elevator will continue upward with an
   acceleration of
   \[
     a = \frac{F}{M_2} - g,
   \]
   while $M_1$ will only feel the force of gravity, giving an acceleration of
   \[
     a' = g.
   \]
   \part The distance the elevator will have moved upward in a time $t$ is given
   by
   \[
     \frac{1}{2}\Big(\frac{F}{M_2}-g\Big)t^2 = \Delta{s}_2,
   \]
   while the distance the hanging mass will have moved (relative to the
   elevator) is given by
   \[
     \frac{1}{2}gt^2 = \Delta{s}_1.
   \]
   The two will meet when $\Delta{s}_1 + \Delta{s}_2 = s$, so
   \begin{align*}
     s &= \Delta{s}_1 + \Delta{s}_2 \\
       &= \frac{1}{2}gt^2 + \frac{1}{2}\Big(\frac{F}{M_2}-g\Big)t^2 \\
     &= \frac{1}{2}\frac{F}{M_2}t^2.
   \end{align*}
   Solving for $t$ gives a value of
   \[
     t = \sqrt{\frac{2M_2s}{F}}.
   \]
 \end{parts}
\end{solution}

\question Given the system shown in FIg. 5-9, consider all surfaces
frictionless. If $m=\SI{150}{\gram}$ is released when it is $d = \SI{1}{\meter}$
above the base of $M=\SI{1650}{\gram}$, how long after release, $\Delta{t}$,
will $m$ strike the base of $M$?

\question None of the identical gondolas on teh Martian canal Rimini is quite
able to support the load of both Paolo and Francesca, two affectionate marsupials
who refuse to go in separate boats. The enterprising gondolier, Giuseppe,
collects their fare by rigging them up from the mast as shown in Fig. 5-10,
using the massless ropes and massless, frictionless pulleys characteristic of
Martian construction. Giuseppe ferries them across before they hit either the
mast or the deck. Assuming Paulo's mass is $\SI{90}{\kilo\gram}$ and Francesca's
is $\SI{60}{\kilo\gram}$, how much load $W$ does Giuseppe save?

\textit{Hint}: Remember that the tension in a massless cord htat passes over a
massless, frictionless pulley is the same on both sides of the pulley.

\question A painter working from a ``bosun's'' chair is hung down the side of a
tall building, as shown in Fig. 5-11. Wishing to move in a hurry, the
$\SI{180}{lb}$ painter pulls down on the fall rope so hard that he presses
against the chair with a force of only $\SI{100}{lb}$. The chair itself weighs
$\SI{30.0}{lb}$.
\begin{parts}
  \part What is the acceleration $\mathbf{a}$ of the painter and the chair?
  \part What is the total force $F$ supported by the pulley?
\end{parts}

\question A space traveler about to leave for the moon has a spring balance and
a $\SI{1.0}{\kilo\gram}$ mass $A$, which when hung on teh balance on the Earth
gives the reading of $\SI{9.8}{\newton}$. Arriving at the moon at a place where
the acceleration of gravity is not known exactly but has a value of about one
sixth the acceleration of gravity at the Earth's surface, he picks up a stone
$B$ which gives a reading of $\SI{9.8}{\newton}$ when weighed on teh spring
balance. He then hangs $A$ and $B$ ove ra pulley as shown in Fig. 5-12 and
observes that $B$ falls with an acceleration of
$\SI{1.2}{\meter\per\second\squared}$. What is the mass $m_B$ of stone $B$?

\question A mass usspended from a spring hangs motionless, and is then given an
upward blow such that it moves initially at unit speed. If the mass and spring
constant are such that the equation of motion is $\ddot{x} = -x$, find the
maximum height $x_{\text{max}}$ attained by numerical integration of the
equation of motion.

\question A particle of mass $m$ moves along a straight line. Its motion is
resisted by a force proportional to its velocity, $F=-kv$. It starts with speed
$v=v_0$ at $x=0$ and $t=0$.
\begin{parts}
  \part Find $x$ as a function of $t$ by numerical integration.
  \part Find the time $t_{1/2}$ required to lose half its speed, and the maximum
  distance $x_{\text{max}}$ attained.
\end{parts}
Notes:
\begin{parts}
\part Adjust he scales of $x$ and $t$ so that the equation of motion has simple
numerical coefficients.
\part Invent a scheme to attain good accuracy with a relatively coarse interval
for $\Delta{t}$.
\part Use dimensional analysis to deduce how $t_{1/2}$ and $x_{\text{max}}$
shoudl depend upon $v_0$, $k$, and $m$, and solve for the actual motion only for
a single convenient value of $v_0$, say $v_0=1.00$ (in the modified $x$ and $t$
units).
\end{parts}

\question A certain charged particle moves in an electric and magnetic field
according to the equations,
\begin{align*}
  \frac{\mathrm{d}v_x}{\mathrm{d}t} &= -2v_y, \\
  \frac{\mathrm{d}v_y}{\mathrm{d}t} &= 1 + 2v_x.
\end{align*}
At $t=0$ the particle starts at $x=0$, $y=0$ with velocity $v_x=1.00$, $v_y=0$.
Determine th enature of the motion by numerical integration.

\question A shell is fired with a muzzle velocity $v = \SI{1000}{ft\per\second}$
at an angle of $45^\circ$ with teh horizontal. Its motion is resisted by a force
proporitonal to the cube of its velocity ($F = -kv^3$). The coefficient $k$ is
such that the resisting force is equal to twie the weight of the shell when $v =
\SI{1000}{ft\per\second}$. Find the approximate maximum height attained
$h_{\text{max}}$, and the horizontal range $R$ by numerical integration, and
compare these with the values expected in the absence of resistance.

\end{questions}

\end{document}