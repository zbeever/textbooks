\documentclass[../qft-for-the-gifted-amateur.tex]{subfiles}

\begin{document}

\section{Lagrangians}

\begin{questions}
	\printanswers
	
	\question For boson operators satisfying
	\[
	[{\hat{a}}_{\mathbf{p}},{\hat{a}}_{\mathbf{q}}^{\dagger}] = \delta_{\mathbf{p}\mathbf{q}},
	\]
	show that
	\[
	\frac{1}{V}\sum_{\mathbf{p}\mathbf{q}}e^{i(\mathbf{p} \cdot \mathbf{x} - \mathbf{q} \cdot \mathbf{y})}[{\hat{a}}_{\mathbf{p}},{\hat{a}}_{\mathbf{q}}^{\dagger}] = \delta^{(3)}(\mathbf{x} - \mathbf{y}),
	\]
	where $V$ is the volume of space over which the system is defined.
	Repeat this for fermion commutation operators.

	\begin{solution}
	By direct substitution we have

	\begin{align*}
		\frac{1}{V}\sum_{\mathbf{p}\mathbf{q}}e^{i(\mathbf{p} \cdot \mathbf{x} - \mathbf{q} \cdot \mathbf{y})}[{\hat{a}}_{\mathbf{p}},{\hat{a}}_{\mathbf{q}}^{\dagger}] &= \frac{1}{V}\sum_{\mathbf{p}\mathbf{q}}e^{i(\mathbf{p} \cdot \mathbf{x} - \mathbf{q} \cdot \mathbf{y})}\delta_{\mathbf{p}\mathbf{q}}, \\
		&= \frac{1}{V}\sum_{\mathbf{p}}e^{i\mathbf{p} \cdot (\mathbf{x} - \mathbf{y})}.
	\end{align*}

	The same holds for the fermion anticommutator

	\begin{align*}
		\frac{1}{V}\sum_{\mathbf{p}\mathbf{q}}e^{i(\mathbf{p} \cdot \mathbf{x} - \mathbf{q} \cdot \mathbf{y})}\{{\hat{c}}_{\mathbf{p}},{\hat{c}}_{\mathbf{q}}^{\dagger}\} &= \frac{1}{V}\sum_{\mathbf{p}\mathbf{q}}e^{i(\mathbf{p} \cdot \mathbf{x} - \mathbf{q} \cdot \mathbf{y})}\delta_{\mathbf{p}\mathbf{q}} \\
		&= \frac{1}{V}\sum_{\mathbf{p}}e^{i\mathbf{p} \cdot (\mathbf{x} - \mathbf{y})}.
	\end{align*}
	That these expressions are equivalent to the three dimensional Dirac
	delta can be seen through examining a similar expression in Fourier
	transforms. Suppose we have a function $f$ and we wish to find
	$\tilde{f}$, its Fourier transform. Then we have

	\[
	f(x) = \frac{1}{\sqrt{2\pi}}\sum_{k=-\infty}^{\infty}\tilde{f}(k)e^{ikx}.
	\]
	We can find $\tilde{f}(k)$ by noting the orthogonality of complex
	exponentials over a full period. That is

	\begin{align*}
		\frac{1}{\sqrt{2\pi}}\int_{0}^{2\pi}f(x)e^{- ixm}\,\mathrm{d}x &= \frac{1}{2\pi}\int_{0}^{2\pi}\sum_{k=-\infty}^{\infty}\tilde{f}(k)e^{i(k - m)x}\,\mathrm{d}x, \\
		&= \frac{1}{2\pi}\int_{0}^{2\pi}\tilde{f}(m)\,\mathrm{d}x, \\
		&= \tilde{f}(m)
	\end{align*}

	Now that we have an expression for each $\tilde{f}(m)$, let's
	substitute this back into our Fourier series of $f$.

	\begin{align*}
	f(x) &= \frac{1}{\sqrt{2\pi}}\sum_{k=-\infty}^{\infty}\frac{1}{\sqrt{2\pi}}\int_{0}^{2\pi}f(y)e^{-iyk}\,\mathrm{d}ye^{ikx}, \\
	     &= \int_{0}^{2\pi}f(y)\frac{1}{2\pi}\sum_{k = - \infty}^{\infty}e^{ik(x - y)}\,\mathrm{d}y
	\end{align*}
	Our last line implies
	\[
		\frac{1}{2\pi}\sum_{k=-\infty}^{\infty}e^{ik(x - y)} = \delta(x - y)
	\]
	over our length of integration. It is easy to see, then, that
	\[
	\frac{1}{V}\sum_{\mathbf{p}}e^{i\mathbf{p} \cdot (\mathbf{x} - \mathbf{y})} = \delta^{(3)}(\mathbf{x} - \mathbf{y}),
	\]
	provided our exponentials are periodic over the volume, a fact
	guaranteed by the periodic boundary conditions imposed on our momentum
	eigenstates.
\end{solution}

	\question Show that for the simple harmonic oscillator:
	\begin{gather*}
	[\hat{a},({\hat{a}}^{\dagger})^{n}] = n({\hat{a}}^{\dagger})^{n - 1} \\
	\langle 0|{\hat{a}}^{n}({\hat{a}}^{\dagger})^{m}|0\rangle = n!\delta_{nm} \\
	\langle m|{\hat{a}}^{\dagger}|n\rangle = \sqrt{n + 1}\delta_{m,n + 1} \\
	\langle m|\hat{a}|n\rangle = \sqrt{n}\delta_{m,n - 1}
	\end{gather*}

	\begin{solution}
	For (a), look at its effect on $|n\rangle$,
	\begin{align*}
		[\hat{a},({\hat{a}}^{\dagger})^{n}]|n_{0}\rangle &= \hat{a}({\hat{a}}^{\dagger})^{n}|n_{0}\rangle - ({\hat{a}}^{\dagger})^{n}\hat{a}|n_{0}\rangle \\
		 &= \hat{a}\sqrt{\frac{(n_{0} + n)!}{n_{0}!}}|n_{0} + n\rangle - ({\hat{a}}^{\dagger})^{n}\sqrt{n_{0}}|n_{0} - 1\rangle \\
		 &= (n_{0} + n)\sqrt{\frac{(n_{0} + n - 1)!}{n_{0}!}}|n_{0} + n - 1\rangle - n_{0}\sqrt{\frac{(n_{0} + n - 1)!}{n_{0}!}}|n_{0} + n - 1\rangle \\
		 &= n\sqrt{\frac{(n_{0} + n - 1)!}{n_{0}!}}|n_{0} + n - 1\rangle \\
		 &= n({\hat{a}}^{\dagger})^{n - 1}|n_{0}\rangle
	\end{align*}
	For (b), we have
	\[
	({\hat{a}}^{\dagger})^{m}|0\rangle = \sqrt{m!}|m\rangle\langle 0|{\hat{a}}^{n} = \overline{({\hat{a}}^{\dagger})^{n}|0\rangle} = \overline{\sqrt{n!}|n\rangle} = \langle n|\sqrt{n!},
	\]
	so
	\[
	\langle 0|{\hat{a}}^{n}({\hat{a}}^{\dagger})^{m}|0\rangle = \sqrt{n!m!}\langle n|m\rangle = n!\delta_{nm},
	\]
	where the Kroenecker delta appears because our states are eigenvectors
	of Hermitian operators, and hence orthogonal. Both (c) and (d) follow
	immediately from this last point.
	\begin{gather*}
	\langle m|{\hat{a}}^{\dagger}|n\rangle = \sqrt{n + 1}\langle m|n + 1\rangle = \sqrt{n + 1}\delta_{m,n + 1} \\
	\langle m|\hat{a}|n\rangle = \sqrt{n}\langle m|n\rangle = \sqrt{n}\delta_{m,n - 1}
	\end{gather*}
\end{solution}

	\question The three-dimensional harmonic oscillator is described by the
	Hamiltonian
	\[
	\hat{H} = \frac{1}{2m}({\hat{p}}_{1}^{2} + {\hat{p}}_{2}^{2} + {\hat{p}}_{3}^{2}) + \frac{1}{2}m\omega^{2}({\hat{x}}_{1}^{2} + {\hat{x}}_{2}^{2} + {\hat{x}}_{3}^{2}).
	\]
	Define the creation operators ${\hat{a}}_{1}^{\dagger}$, ${\hat{a}}_{2}^{\dagger}$, and ${\hat{a}}_{3}^{\dagger}$ so that $[{\hat{a}}_{i},{\hat{a}}_{j}^{\dagger}] = \delta_{ij}$ and show that $\hat{H} = \hslash\omega\sum_{i = 1}^{3}(\frac{1}{2} + {\hat{a}}_{i}^{\dagger}{\hat{a}}_{i})$. Angular momentum can be defined as
	\[
	{\hat{L}}^{i} = - i\hslash\epsilon^{ijk}{\hat{a}}_{j}{\hat{a}}_{k},
	\]
	so that for example ${\hat{L}}^{3} = - i\hslash({\hat{a}}_{1}^{\dagger}{\hat{a}}_{2} - {\hat{a}}_{2}^{\dagger}{\hat{a}}_{1})$. Define new creation operators
	\begin{align*}
	{\hat{b}}_{1}^{\dagger} &= -\frac{1}{\sqrt{2}}({\hat{a}}_{1}^{\dagger} + i{\hat{a}}_{2}^{\dagger}), \\
	{\hat{b}}_{0}^{\dagger} &= {\hat{a}}_{3}^{\dagger}, \\
	{\hat{b}}_{- 1}^{\dagger} &= \frac{1}{\sqrt{2}}({\hat{a}}_{1}^{\dagger} - i{\hat{a}}_{2}^{\dagger}),
	\end{align*}
	and show that $[{\hat{b}}_{i},{\hat{b}}_{j}^{\dagger}] = \delta_{ij}$, the Hamiltonian is $\hat{H} = \hslash\omega\sum_{m = - 1}^{1}(\frac{1}{2} + {\hat{b}}_{m}^{\dagger}{\hat{b}}_{m})$ and ${\hat{L}}^{3} = \hslash\sum_{m = - 1}^{1}m{\hat{b}}_{m}^{\dagger}{\hat{b}}_{m}$.
	\begin{solution}
	Let's define our \[{\hat{a}}_{i}^{\dagger}\] as a straightforward generalization to our one dimensional case: 
	\[
	{\hat{a}}_{i}^{\dagger} = \Big(\hat{x_{i}} - \frac{i}{m\omega}\hat{p_{i}}\Big).
	\]
	By virtue of $[{\hat{x}}_{i},{\hat{p}}_{j}] = i\hslash\delta_{i,j}$, we have $[{\hat{a}}_{i},{\hat{a}}_{j}^{\dagger}] = \delta_{i,j}.$ We can construct our three dimensional oscillator Hamiltonian in the same manner as our one dimensional version, then,
	\[
	\hat{H} = \hslash\omega\overset{3}{\sum_{i = 1}}(\frac{1}{2} + {\hat{a}}_{i}^{\dagger}{\hat{a}}_{i}).
	\]
	Given ${\hat{b}}_{i}^{\dagger}$, let's confirm their commutation relations:
	\begin{align*}
		[{\hat{b}}_{1},{\hat{b}}_{1}^{\dagger}] &= \frac{1}{2}[({\hat{a}}_{1} - i{\hat{a}}_{2}),({\hat{a}}_{1}^{\dagger} + i{\hat{a}}_{2}^{\dagger})] \\
							&= \frac{1}{2}([{\hat{a}}_{1},{\hat{a}}_{1}^{\dagger}] + i[{\hat{a}}_{1},{\hat{a}}_{2}^{\dagger}] - i[{\hat{a}}_{2},{\hat{a}}_{1}^{\dagger}] + [{\hat{a}}_{2},{\hat{a}}_{2}^{\dagger}]) \\
							&= 1 \\
		[{\hat{b}}_{0},{\hat{b}}_{0}^{\dagger}] &= [{\hat{a}}_{3}^{,}{\hat{a}}_{3}^{\dagger}] \\
							&= 1 \\
		[{\hat{b}}_{- 1},{\hat{b}}_{- 1}^{\dagger}] &= \frac{1}{2}[({\hat{a}}_{1} + i{\hat{a}}_{2}),({\hat{a}}_{1}^{\dagger} - i{\hat{a}}_{2}^{\dagger})] \\
						      &= \frac{1}{2}([\hat{a_{1}},{\hat{a}}_{1}^{\dagger}] - i[{\hat{a}}_{1},{\hat{a}}_{2}^{\dagger}] + i[{\hat{a}}_{2}{\hat{a}}_{1}^{\dagger}] + [{\hat{a}}_{2},{\hat{a}}_{2}^{\dagger}]) \\
						      &= 1
	\end{align*}
	Clearly, both $[{\hat{b}}_{0},{\hat{b}}_{1}^{\dagger}]$ and $[{\hat{b}}_{0},{\hat{b}}_{- 1}^{\dagger}]$ are $0$. This just leaves
	\begin{align*}
		[{\hat{b}}_{1},{\hat{b}}_{- 1}^{\dagger}] &= - \frac{1}{2}[({\hat{a}}_{1} - i{\hat{a}}_{2}),({\hat{a}}_{1}^{\dagger} - i{\hat{a}}_{2}^{\dagger})] \\
							  &= - \frac{1}{2}([{\hat{a}}_{1},{\hat{a}}_{1}^{\dagger}] - i[{\hat{a}}_{1},{\hat{a}}_{2}^{\dagger}] - i[{\hat{a}}_{2},{\hat{a}}_{1}^{\dagger}] - [{\hat{a}}_{2},{\hat{a}}_{2}^{\dagger}]) \\
							  &= 0
	\end{align*}
	showing $[{\hat{b}}_{i},{\hat{b}}_{j}^{\dagger}] = \delta_{ij}.$ Now let's compute each ${\hat{b}}_{m}^{\dagger}{\hat{b}}_{m}$.

	\begin{align*}
		{\hat{b}}_{1}^{\dagger}{\hat{b}}_{1} &= \frac{1}{2}({\hat{a}}_{1}^{\dagger} + i{\hat{a}}_{2}^{\dagger})({\hat{a}}_{1} - i{\hat{a}}_{2}) \\
					 &= \frac{1}{2}({\hat{a}}_{1}^{\dagger}{\hat{a}}_{1} - i{\hat{a}}_{1}^{\dagger}{\hat{a}}_{2} + i{\hat{a}}_{2}^{\dagger}{\hat{a}}_{1} + {\hat{a}}_{2}^{\dagger}{\hat{a}}_{2}) \\
		{\hat{b}}_{0}^{\dagger}{\hat{b}}_{0} &= {\hat{a}}_{3}^{\dagger}{\hat{a}}_{3} \\
		{\hat{b}}_{- 1}^{\dagger}{\hat{b}}_{- 1} &= \frac{1}{2}({\hat{a}}_{1}^{\dagger} - i{\hat{a}}_{2}^{\dagger})({\hat{a}}_{1} + i{\hat{a}}_{2}) \\
						   &= \frac{1}{2}({\hat{a}}_{1}^{\dagger}{\hat{a}}_{1} + i{\hat{a}}_{1}^{\dagger}{\hat{a}}_{2} - i{\hat{a}}_{2}^{\dagger}{\hat{a}}_{1} + {\hat{a}}_{2}^{\dagger}{\hat{a}}_{2})
	\end{align*}
	Their sum is
	\[
	\overset{1}{\sum_{m = - 1}}{\hat{b}}_{m}^{\dagger}{\hat{b}}_{m} = \overset{3}{\sum_{i = 1}}{\hat{a}}_{i}^{\dagger}{\hat{a}}_{i},
	\]
	so we may safely rewrite our Hamiltonian as
	\[
	\hat{H} = \hslash\omega\overset{1}{\sum_{m = - 1}}\Big(\frac{1}{2} + {\hat{b}}_{m}^{\dagger}{\hat{b}}_{m}\Big).
	\]
	Similarly, it is clear that we can rewrite
	\[
	{\hat{L}}^{3} = - i\hslash({\hat{a}}_{1}^{\dagger}{\hat{a}}_{2} - {\hat{a}}_{2}^{\dagger}{\hat{a}}_{1})
	\]
	as
	\[
	{\hat{L}}^{3} = \hslash\overset{1}{\sum_{m = - 1}}m{\hat{b}}_{m}^{\dagger}{\hat{b}}_{m},
	\]
	because our cross terms in each ${\hat{b}}_{m}^{\dagger}{\hat{b}}_{m}$ already contain $i$ and $m = 0$ gets rid of our ${\hat{a}}_{3}^{\dagger}{\hat{a}}_{3}$.
	\end{solution}

	\question Since for fermions we have ${\hat{c}}_{1}^{\dagger}{\hat{c}}_{2}^{\dagger}|0\rangle = - {\hat{c}}_{2}^{\dagger}{\hat{c}}_{1}^{\dagger}|0\rangle$, then the state $|11\rangle$ depends on which fermion is added first. If we put the first fermion into the state $\psi_{i}(r_{1})$ and we put the second fermion into the state $\psi_{j}(r_{2})$, a representation for $|11\rangle$ is
	\[
	\Psi(r_{1},r_{2}) = \frac{1}{\sqrt{2}}[\psi_{1}(r_{1})\psi_{2}(r_{2}) - \psi_{2}(r_{1})\psi_{1}(r_{2})].
	\]
	Show that a generalization of this result for $N$ fermions is
	$\Psi(r_{1}\ldots r_{N})$, which can be written out as
	\[
		\frac{1}{\sqrt{N!}}\left| \begin{matrix}
	\psi_{1}(r_{1}) & \psi_{2}(r_{1}) & \ldots & \psi_{N}(r_{1}) \\
	\psi_{1}(r_{2}) & \psi_{2}(r_{2}) & \ldots & \psi_{N}(r_{2}) \\
	 \vdots & \vdots & & \vdots \\
	\psi_{1}(r_{N}) & \psi_{2}(r_{N}) & \ldots & \psi_{1}(r_{N}) \\
	\end{matrix} \right|.
	\]
	This expression is known as a Slater determinant.
	
	\begin{solution}
	Given $N$ fermions, a composite state may be arrived at $N!$ ways,
	corresponding to the number of permutations of single particle states.
	Furthermore, changing the order in which fermions are added to these
	single states introduces a minus sign to our wave function. Further
	still, having two fermions in the same state is impossible, and our wave
	function for such a composite state should be identically zero.

	These properties may be captured succinctly in the determinant of a
	matrix consisting of individual states. Put another way, systems
	representable by a Slater determinant trivially satisfy antisymmetry and
	the Pauli exclusion principal. The leading $\frac{1}{\sqrt{N!}}$ is
	simply a normalization factor.
	\end{solution}
\end{questions}

\end{document}
