\documentclass[../qft-for-the-gifted-amateur.tex]{subfiles}

\begin{document}

\section{Simple Harmonic Oscillators}

\begin{questions}
	\printanswers

	\question For the one-dimensional harmonic oscillator, show that with creation and annihilation operators defined as in eqns 2.9 and 2.10, $[\hat{a},\hat{a}] = 0$, $[\hat{a}^\dagger, \hat{a}^\dagger] = 0$, $[\hat{a},\hat{a}^\dagger]=1$, and $\hat{H}=\hbar\omega(\hat{a}^\dagger\hat{a} + \frac{1}{2})$.
	
	\begin{solution}
		Both $[\hat{a},\hat{a}]=0$ and $[\hat{a}^\dagger, \hat{a}^\dagger]=0$
		follow from the fact that any operator commutes with itself. For
		$[\hat{a}, \hat{a}^\dagger]$, we have
		\begin{align*}
		[\hat{a}, \hat{a}^\dagger] &= \frac{m\omega}{2\hbar}\Big[\hat{x} + \frac{i}{m\omega}\hat{p}, \hat{x} - \frac{i}{m\omega}\hat{p}\Big] \\
		&= \frac{m\omega}{2\hbar}\Big({- \frac{2i}{m\omega}}\Big)[\hat{x},\hat{p}] \\
		&= \frac{m\omega}{2\hbar}(\frac{2\hbar}{m\omega}) \\
		&= 1
		\end{align*}
		To answer the last part of this question, we can solve for
		$\hat{x}$ and $\hat{p}$ in terms of $\hat{a}$ and
		${\hat{a}}^{\dagger}$, then substitute these into our Hamiltonian.
		We have
		\[
			\hat{x} = \sqrt{\frac{\hbar}{2m\omega}}(\hat{a} + {\hat{a}}^{\dagger}) \qquad \hat{p} = -i\sqrt{\frac{\hbar m\omega}{2}}(\hat{a} - {\hat{a}}^{\dagger})
		\]
		so our Hamiltonian becomes
		\begin{align*}
		\hat{H} &= \frac{{\hat{p}}^{2}}{2m} + \frac{1}{2}m\omega^{2}{\hat{x}}^{2} \\
		&= - \Big(\frac{\hbar\omega}{4}\Big)(\hat{a} - {\hat{a}}^{\dagger})^{2} + \Big(\frac{\hbar\omega}{4}\Big)(\hat{a} + {\hat{a}}^{\dagger})^{2} \\
		&= \Big(\frac{\hbar\omega}{2}\Big)(\hat{a}{\hat{a}}^{\dagger} + {\hat{a}}^{\dagger}\hat{a}) \\
		&= \hbar\omega\Big({\hat{a}}^{\dagger}\hat{a} + \frac{1}{2}\Big)
		\end{align*}
		
		In the last step we made use of
		$[\hat{a},{\hat{a}}^{\dagger}] = 1$, and so
		$\hat{a}{\hat{a}}^{\dagger} = 1 + {\hat{a}}^{\dagger}\hat{a}$.
	\end{solution}
	
	\question For the Hamiltonian
	\[
		\hat{H}=\frac{\hat{p}^2}{2m}+\frac{1}{2}m\omega^2\hat{x}^2 + \lambda\hat{x}^4,
	\]
	where $\lambda$ is small, show by writing the Hamiltonian in terms of creation and annihilation operators and using perturbation theory, that the energy eigenvalues of all the levels are given by 
	\[
		E_n = \Big(n+\frac{1}{2}\Big)\hbar\omega + \frac{3\lambda}{4}\Big(\frac{\hbar}{m\omega}\Big)^2(2n^2+2n+1).
	\]
	
	\begin{solution}
		Let's consider our base Hamiltonian to be the one coinciding with our
		simple harmonic oscillator,
		\[{\hat{H}}_{0} = \frac{{\hat{p}}^{2}}{2m} + \frac{1}{2}m\omega^{2}{\hat{x}}^{2} = \hbar\omega(\hat{n} + \frac{1}{2}).
		\]
		Our perturbation, then, is
		$\hat{H_{1}} = {\hat{x}}^{4} = (\frac{\hbar}{2m\omega})^{2}(\hat{a} + {\hat{a}}^{\dagger})^{4}$.
		We can express our total Hamiltonian as $\hat{H} = {\hat{H}}_{0} + \lambda{\hat{H}}_{1}$.
		
		If $\lambda$ is sufficiently small, our perturbed Hamiltonian's energy
		eigenstate should be expressible as a Taylor series in $\lambda$
		centered around our initial one. This includes both the eigenvalue
		\emph{and} the eigenvector, as, being Hermitian, a different eigenvalue
		of our Hamiltonian necessitates a different eigenvector. Putting this
		all together gives
		\[
		({\hat{H}}_{0} + \lambda{\hat{H}}_{1})(|n^{0}\rangle + \lambda|n^{1}\rangle + \cdots) = (E_{0} + \lambda E_{1} + \cdots)(|n^{0}\rangle + \lambda|n^{1}\rangle + \cdots).
		\]
		Multiplying out and collecting factors of $\lambda$ yields
		\[
		{\hat{H}}_{0}|n^{0}\rangle + \lambda({\hat{H}}_{0}|n^{1}\rangle + \hat{H_{1}}|n^{0}\rangle) + \mathcal{O}(\lambda^{2}) = E_{0}|n^{0}\rangle + \lambda(E_{0}|n^{1}\rangle + E_{1}|n^{0}\rangle) + \mathcal{O}(\lambda^{2}).
		\]
		If $\lambda = 0$, we are left with our unperturbed Hamiltonian, and so
		$E_{0} = \hbar\omega(n + \frac{1}{2})$. To find $E_{1}$, we can
		focus on the first order part of our equation, as the coefficients of
		each power of $\lambda$ on either side must be equal. That is,
		\[
		{\hat{H}}_{0}|n^{1}\rangle + \hat{H_{1}}|n^{0}\rangle = E_{0}|n^{1}\rangle + E_{1}|n^{0}\rangle.
		\]
		Now, we may assume that our initial eigenvector is normalized,
		$\langle n^{1}|n^{1}\rangle = 1$. Imposing a normalization constraint
		on our first order approximation gives
		\[
		(\langle n^{0}| + \lambda\langle n^{1}|)(|n^{0}\rangle + \lambda|n^{1}\rangle) = 1 + \lambda(\langle n^{0}|n^{1}\rangle + \langle n^{1}|n^{0}\rangle) + \mathcal{O}(\lambda^{2}) = 1.
		\]
		In other words,
		$\langle n^{0}|n^{1}\rangle = -\langle n^{1}|n^{0}\rangle = 0$. So we
		may operate on the left by $\langle n^{0}|$ to isolate our desired
		term.
		\[
		\langle n^{0}|{\hat{H}}_{1}|n^{0}\rangle = E_{1}.
		\]
		It's easiest to carry out this computation by expanding the right-most
		term of ${\hat{H}}_{1}$ as
		$(\hat{a}\hat{a} + \hat{a}{\hat{a}}^{\dagger} + {\hat{a}}^{\dagger}\hat{a} + {\hat{a}}^{\dagger}{\hat{a}}^{\dagger})^{2}$. Let's compute
		$(\hat{a}\hat{a} + \hat{a}{\hat{a}}^{\dagger} + {\hat{a}}^{\dagger}\hat{a} + {\hat{a}}^{\dagger}{\hat{a}}^{\dagger})|n^{0}\rangle$.
		\[
		(\hat{a}\hat{a} + \hat{a}{\hat{a}}^{\dagger} + {\hat{a}}^{\dagger}\hat{a} + {\hat{a}}^{\dagger}{\hat{a}}^{\dagger})|n^{0}\rangle = (\sqrt{n(n - 1)}|n^{0} - 2\rangle + (2n + 1)|n^{0}\rangle + \sqrt{(n + 1)(n + 2)}|n^{0} + 2\rangle
		\]
		We may operate on this again with $(\hat{a}\hat{a} + \hat{a}{\hat{a}}^{\dagger} + {\hat{a}}^{\dagger}\hat{a} + {\hat{a}}^{\dagger}{\hat{a}}^{\dagger})$, but first notice that any term of the form $|n^{0} + k\rangle$ for nonzero $k \in \mathbb{Z}$ will be annihilated upon operating on the left with $\langle n^{0}|$, as all eigenvectors of the Hamiltonian are orthogonal. So the only contributing factors will be the first term, raised by ${\hat{a}}^{\dagger}{\hat{a}}^{\dagger}$, the second term, operated on by $\hat{a}{\hat{a}}^{\dagger} + {\hat{a}}^{\dagger}\hat{a}$, and the third term, lowered by $\hat{a}\hat{a}$. Carrying this all out yields
		\[
		(n(n - 1) + (2n + 1)^{2} + (n + 1)(n + 2))|n^{0}\rangle,
		\]
		which, after operating by $\langle n^{0}|$ on the left and
		simplifying, gives $3(2n^{2} + 2n + 1)$. So
		\[
		\langle n^{0}|{\hat{H}}_{1}|n^{0}\rangle = \frac{3}{4}\Big(\frac{\hbar}{m\omega}\Big)^{2}(2n^{2} + 2n + 1) = E_{1}.
		\]
		Putting this all together, we see that our Hamiltonian has eigenvalues
		of
		\[
		E_{n} = \Big(n + \frac{1}{2}\Big)\hbar\omega + \frac{3\lambda}{4}\Big(\frac{\hbar}{m\omega}\Big)^{2}(2n^{2} + 2n + 1).
		\]
	\end{solution}
	
	\question Use eqns 2.46 and 2.62 to show that
	\[
		\hat{x}_j = \frac{1}{\sqrt{N}}\Big(\frac{\hbar}{m}\Big)^{1/2}\sum_k\frac{1}{(2\omega_k)^{1/2}}[\hat{a}_ke^{ikja} + \hat{a}_k^\dagger{e^{-ikja}}].
	\]
	
	\begin{solution}
		Substituting 2.62 into 2.46 gives us
		\[
			{\hat{x}}_{j} = \frac{1}{\sqrt{N}}\sum_{k}\sqrt{\frac{\hbar}{2m\omega_{k}}}({\hat{a}}_{k} + {\hat{a}}_{- k}^{\dagger})e^{ikja}.
		\]
		By taking $\frac{\hbar}{m}$ outside the summation, distributing
		$e^{ikja}$, and reindexing the second sum, we arrive at
		
		\[
		{\hat{x}}_{j} = \frac{1}{\sqrt{N}}\Big(\frac{\hbar}{m}\Big)^{\frac{1}{2}}\sum_{k}\frac{1}{(2\omega_{k})^{\frac{1}{2}}}[{\hat{a}}_{k}e^{ikja} + {\hat{a}}_{k}^{\dagger}e^{- ijka}].
		\]
	\end{solution}
	
	\question Using $\hat{a}|0\rangle=0$ and eqns 2.9 and 2.10 together with $\langle{x}|\hat{p}|\psi\rangle=-i\hbar\frac{\mathrm{d}}{\mathrm{d}x}\langle{x}|\psi\rangle$, show that
	\[
		0 = \Big(x + \frac{\hbar}{m\omega}\frac{\mathrm{d}}{\mathrm{d}x}\Big)\langle{x}|0\rangle,
	\]
	and hence
	\[
		\langle{x}|0\rangle = \Big(\frac{m\omega}{\pi\hbar}\Big)^{1/4}e^{-m\omega{x}^2/2\hbar}.
	\]
	
	\begin{solution}
		As $\hat{a}$ annihilates the lowest eigenstate, we have
		\[
			\hat{a}|0\rangle = \sqrt{\frac{m\omega}{2\hbar}}\Big(\hat{x} + \frac{i}{m\omega}\hat{p}\Big)|0\rangle = 0,
		\]
		and so
		\[
			\Big(\hat{x} + \frac{i}{m\omega}\hat{p}\Big)|0\rangle = 0.
		\]
		Operating on the left by $\langle x|$ gives
		\begin{align*}
		\langle x|\Big(\hat{x} + \frac{i}{m\omega}\hat{p}\Big)|0\rangle &= x\langle x|0\rangle + \frac{i}{m\omega}\langle x|\hat{p}|0\rangle \\
		&= x\langle x|0\rangle + \frac{i}{m\omega} \cdot ( - i\hbar)\frac{\mathrm{d}}{\mathrm{d}x}\langle x|0\rangle \\
		&= \Big(x + \frac{\hbar}{m\omega}\frac{\mathrm{d}}{\mathrm{d}x}\Big)\langle x|0\rangle \\
		&= 0
		\end{align*}
		This is a linear, separable differential equation. Let's rewrite it in
		an easier-to-tackle form.
		\[
			\frac{1}{\langle x|0\rangle}\frac{\mathrm{d}\langle x|0\rangle}{\mathrm{d}x} = - x\frac{m\omega}{\hbar}.
		\]
		Integrating both sides with respect to $x$ gives us
		\begin{align*}
		\int\frac{1}{\langle x|0\rangle}\frac{\mathrm{d}\langle x|0\rangle}{\mathrm{d}x}\mathrm{d}x &= \int - x\frac{m\omega}{\hbar}\mathrm{d}x \\
		\ln|\langle x|0\rangle| &= - x^{2}\frac{m\omega}{2\hbar} + C \\
		\langle x|0\rangle &= Ce^{- m\omega x^{2}/2\hbar}
		\end{align*}
		We can find $C$ by normalizing our state,
		\[
			\int_{- \infty}^{\infty}C^{2}e^{- m\omega x^{2}/\hbar}\mathrm{d}x = 1.
		\]
		This is a Gaussian integral. It has the solution
		\[
			\int_{- \infty}^{\infty}e^{- m\omega x^{2}/\hbar}\text{dx} = \Big(\frac{\pi\hbar}{m\omega}\Big)^{\frac{1}{2}}.
		\]
		Substituting this into our normalization constraint yields
		\[
			C = \Big(\frac{m\omega}{\pi\hbar}\Big)^{\frac{1}{4}}.
		\]
		This gives us a final solution of
		\[
			\langle x|0\rangle = \Big(\frac{m\omega}{\pi\hbar}\Big)^{\frac{1}{4}}e^{- m\omega x^{2}/2\hbar}.
		\]
	\end{solution}
	
\end{questions}

\end{document}