\documentclass{article}

\usepackage[letterpaper, margin=1in]{geometry}
\usepackage{import}
\usepackage{xifthen}
\usepackage{pdfpages}
\usepackage{mathtools}
\usepackage{transparent}
\usepackage{amsmath}
\usepackage{isotope}
\usepackage{physics}
\usepackage{indentfirst}
\usepackage{amssymb}
\usepackage{siunitx}
\usepackage{fancyhdr}


\pagestyle{fancy}
\fancyhf{}
\rhead{Zach Beever}
\lhead{Deriving the electric field associated with an evolving current sheet}
\rfoot{Page \thepage}

\author{Zach Beever}

\title{Notes on the induced field from an evolving Harris current sheet}

\date{}

\begin{document}
	\maketitle
	
	\section{Time-evolution of the Harris model}
	
	The Harris current sheet model takes the simple form
	\[
		\mathbf{B} = B_0\tanh\Big(\frac{z}{L}\Big)\hat{\mathbf{i}} + 0\hat{\mathbf{j}} + 0\hat{\mathbf{k}}.
	\]
	If $L$ changes with time there will be an induced electric field described by 
	\[
		\curl{\mathbf{E}} = -\frac{\partial\mathbf{B}}{\partial t}.
	\]
	Signaling the dependence of $L$ on $t$ via $L(t)$, we find
	\[
		\frac{\partial\mathbf{B}}{\partial t} = B_0\sech^2\Big(\frac{z}{L(t)}\Big)\cdot\Big({-\frac{z}{L^2(t)}}\Big)\cdot\frac{\mathrm{d}L(t)}{\mathrm{d}t}\,\hat{\mathbf{i}}.
	\]
	The relevant Maxwell equation then becomes
	\[
		\frac{\partial E_z}{\partial y} - \frac{\partial E_y}{\partial z} = -B_0\sech^2\Big(\frac{z}{L(t)}\Big)\cdot\frac{z}{L^2(t)}\cdot\frac{\mathrm{d}L(t)}{\mathrm{d}t}.
	\]
	Because the Harris model extends infinitely along the $y$-axis, $E_z = 0$. If this were not so, we would see infinite growth as we moved in this direction. There is no such restriction for $E_y$, though, and so\footnote{The limits of integration are chosen so that the entire field contributes to $E_y$ at the given $z$. As we will see, we find the same result whether we set the lower limit to $\infty$ or $-\infty$.}
	\[
		E_y(z) = \int_{-\infty}^z B_0\sech^2\Big(\frac{z}{L(t)}\Big)\cdot\frac{z}{L^2(t)}\cdot\frac{\mathrm{d}L(t)}{\mathrm{d}t}\,\mathrm{d}z.
	\]
	Setting $u = z/L(t)$ and $\mathrm{d}u = \mathrm{d}z/L(t)$, this simplifies to
	\[
		E_y(z) = B_0\frac{\mathrm{d}L(t)}{\mathrm{d}t}\int_{-\infty}^{z/L(t)}\sech^2(u)\cdot u\,\mathrm{d}u.
	\]
	We can integrate this by parts to get
	\[
		E_y(z) = B_0\frac{\mathrm{d}L(t)}{\mathrm{d}t}\Big(u\tanh(u)\Big|_{-\infty}^{z/L(t)} - \int_{-\infty}^{z/L(t)}\tanh(u)\,\mathrm{d}u\Big).
	\]
	This remaining integral can be rewritten as
	\[
		\int_{-\infty}^{z/L(t)}\tanh(u)\,\mathrm{d}u = \int_{-\infty}^{z/L(t)}\frac{\sinh(u)}{\cosh(u)}\,\mathrm{d}u = \int_{-\infty}^{z/L(t)}\mathrm{d}(\ln{\cosh(u)}) = \ln\cosh u\Big|_{-\infty}^{z/L(t)}.
	\]
	Using the above, $E_y(z)$ becomes
	\[
		E_y(z) = B_0\frac{\mathrm{d}L(t)}{\mathrm{d}t}\Big[\frac{z}{L(t)}\tanh\Big(\frac{z}{L(t)}\Big)- \ln\cosh\Big(\frac{z}{L(t)} \Big) + C\Big]
	\]
	where $C$ is given by
	\[
		C = \lim_{u\to{-\infty}}\ln\cosh u - u\tanh u.
	\]
	Since, by defintion,
	\[
		\cosh u = \frac{e^u + e^{-u}}{2} \qquad \text{and} \qquad \tanh u = \frac{e^u - e^{-u}}{e^u + e^{-u}}
	\]
	we can rewrite $C$ as
	\begin{align*}
		C &= \lim_{u\to{-\infty}}\ln\frac{e^{-u}}{2} + u\frac{e^{-u}}{e^{-u}} \\
		&= \lim_{u\to-\infty}-u + u - \ln 2\\
		&= {-\ln 2}
	\end{align*}
	The final expression for the induced field is given by
	\[
		\mathbf{E}(z, t) = B_0\frac{\mathrm{d}L(t)}{\mathrm{d}t}\Big[\frac{z}{L(t)}\tanh\Big(\frac{z}{L(t)}\Big)- \ln\cosh\Big(\frac{z}{L(t)} \Big) - \ln 2\Big]\,\hat{\mathbf{j}}
	\]
	
	One major question remains: the Harris model is formulated as an equilibrium solution to the Vlasov equation. Is a time-dependent model physical? Certainly the above result captures the real $y$-directed current found in the magnetosphere, so it is at least on the right track.
	
\end{document}