\documentclass[../calculus-of-variations-and-optimal-control-theory.tex]{subfiles}

\begin{document}
\printanswers

\section{Introduction}

\begin{questions}

\question Suppose that $f$ is a $C^2$ function and $x^*$ is a point of its domain at which we have $\nabla f(x^*)\cdot d \geq 0$ and $d^T\nabla^2f(x^*)d > 0$ for every nonzero feasible direction $d$. Is $x^*$ necessarily a local minimum of $f$? Prove or give a counterexample.

\begin{solution}
	Recall that a local minimum $x^*$ of $f$ is defined as a point in the domain $D$ where there exists an $\varepsilon>0$ such that for all $x\in D$ satisfying $|x - x^*| < \varepsilon$ we have
	\[
		f(x^*) \leq f(x).
	\]
	Let us take the function given in the problem statement and expand it in a Taylor series about $x^*$,
	\[
		f(x) = f(x^* + \alpha d) = f(x^*) + \alpha\nabla f(x^*)\cdot d + \frac{1}{2}\alpha^2d^T\nabla^2 f(x^*)d + o(\alpha^2).
	\]
	Here we have chosen $|d| = 1$, and so, with $x = x^* + \alpha {d}$, we have
	\[
		|x - x^*| = |x^* + \alpha{d} - x^*| = |\alpha d| = \alpha.
	\]
	Choose an $\alpha$ such that $|o(\alpha^2)| \leq \frac{1}{2}\alpha^2d^T\nabla^2 f(x^*)d$. With this choice, $f(x)$ has the minimum value 
	\[
		f(x) \geq f(x^*) + \Big(\frac{1}{2}\alpha^2d^T\nabla^2 f(x^*)d - |o(\alpha)^2|\Big)
	\]
	which satisfies $f(x^*) < f(x)$ by virtue of the positive nature of the parenthesized term. And so, with $\varepsilon < \alpha$, $x^*$ is necessarily a local minimum of $f$.
\end{solution}

\question Give an example where a local minimum $x^*$ is not a regular point and the above necessary condition is false (be sure to justify both of these claims).

\begin{solution}
	Consider the system
	\begin{align*}
		f(x, y, z) &= (x - 1)^2 + (y - 1)^2 + z^2 \\
		h_1(x, y, z) &= z \\
		h_2(x, y, z) &= x^2 + y^2 + (z - 1)^2 - 1
	\end{align*}
	The point $(0, 0, 0)$ is trivially a local minimum of this system (it is the only point that lies in $D$) yet the condition (1.24) is false. This is because the gradients of the above functions at $(0, 0, 0)$ are
	\begin{align*}
		\nabla f(0, 0, 0) &= \begin{bmatrix}
			-2 \\ -2 \\ 0
		\end{bmatrix} \\
		\nabla h_1(0, 0, 0) &= \begin{bmatrix}
			0 \\ 0 \\ 1
		\end{bmatrix} \\
		\nabla h_2(0, 0, 0) &= \begin{bmatrix}
			0 \\ 0 \\ -2
		\end{bmatrix}
	\end{align*}
	and so no solution exists for the equation
	\[
		\nabla f(0, 0, 0) + \lambda_1\nabla h_1(0, 0, 0) + \lambda_2\nabla h_2(0, 0, 0) = 0.
	\]
\end{solution}

\question Generalize the previous argument to an arbitrary number $m \geq 1$ of equality constraints (still assuming that $x^*$ is a regular point).

\begin{solution}
	Assume that $x^*$ is a local minimum of $f$ over $D$ where $D$ is defined by $m$ equality constraints. Consider $m+1$ arbitrary vectors $d_1,\dots,d_{m+1}\in\mathbb{R}^n$ and the following map,
	\[
		F : (\alpha_1,\dots,\alpha_{m+1}) \to \big(f(x^* + \textstyle\sum_{k=1}^{m+1}\alpha_kd_k), h_1(x^* + \sum_{k=1}^{m+1}\alpha_kd_k), \dots, h_{m}(x^* + \sum_{k=1}^{m+1}\alpha_kd_k)\big).
	\]
	The Jacobian of this map is given by
	\[
		\begin{pmatrix}
			\nabla f(x^*)\cdot d_1 & \cdots & \nabla f(x^*)\cdot d_{m+1} \\
			\nabla h_1(x^*)\cdot d_1 & \cdots & \nabla h_1(x^*)\cdot d_{m+1} \\
			\vdots & \ddots & \vdots \\
			\nabla h_m(x^*)\cdot d_1 & \cdots & \nabla h_m(x^*)\cdot d_{m+1} \\
		\end{pmatrix}
	\]
	and, using the same argument as in the book, it must be singular. In brief: if it were not, the inverse function theorem guarantees we could find a point satisfying the constraints where $f$ attains a smaller value than $f(x^*)$---but this would contradict the fact that $x^*$ is a local minimum.
	
	Choose each $d_k$ such that $\nabla h_k(x^*) \cdot d_k \neq 0$ for $k=1,\dots,m$ (this is possible because every $\nabla h_k(x^*)$ is nonzero by the condition of regularity). Since the matrix is singular and every row except the first contains a nonzero element, it must be that the first row is some linear combination of the remaining rows, i.e. each entry in the top row takes the form
	\[
		\nabla f(x^*) \cdot d_k = -\big(\lambda^*_1\nabla h_1(x^*) + \cdots + \lambda^*_m\nabla h_m(x^*)\big)\cdot d_k.
	\]
	In particular, since $d_{m+1}$ can be arbitrarily chosen, this must hold for all possible $d_{m+1}$, implying
	\[
		\nabla f(x^*) + \lambda^*_1\nabla h_1(x^*) + \cdots + \lambda^*_m\nabla h_m(x^*) = 0.
	\]
\end{solution}

\question Consider a curve $D$ in the plane described by the equation $h(x) = 0$, where $h: \mathbb{R}^2 \to \mathbb{R}$ is a $C^1$ function. Let $y$ and $z$ be two fixed points in the plane, lying on the same side with respect to $D$ (but not on $D$ itself). Suppose that a ray of light emanates from $y$, gets reflected off $D$ at some point $x^* \in D$, and arrives at $z$. Consider the following two statements: (i) $x^*$ must be such that the total Euclidean distance traveled by light to go from $y$ to $z$ is minimized over all nearby candidate reflection points $x\in D$ (Fermat's principle); (ii) the angles that the light ray makes with the line normal to $D$ at $x^*$ before and after the reflection must be the same (the law of reflection). Accepting the first statement as a hypothesis, prove that the second statement follows from it, with the help of the first-order necessary condition for constrained optimality.

\begin{solution}
	The function we need to minimize is the Euclidean distance between $y$ and $z$ via reflection about $x$,
	\[
		f(x) = \sqrt{(x_1 - y_1)^2 + (x_2 - y_2)^2} + \sqrt{(x_1 - z_1)^2 + (x_2 - z_2)^2}.
	\]
	The gradient of this function is
	\[
		\nabla f(x) = \frac{1}{\sqrt{(x_1-z_1)^2 + (x_2 - z_2)^2}}\begin{bmatrix}
			x_1 - z_1 \\ x_2 - z_2
		\end{bmatrix} + \frac{1}{\sqrt{(x_1-y_1)^2 + (x_2 - y_2)^2}}\begin{bmatrix}
		x_1 - y_1 \\ x_2 - y_2
		\end{bmatrix}
	\]
	and we can see that it consists of two vectors: the normalized vector from $z$ to $x$, hereby labeled $\hat{\mathbf{v}}_{zx}$, and the normalized vector from $y$ to $x$, $\hat{\mathbf{v}}_{yx}$.
	
	Using the first-order necessary condition for constrained optimality, we see that this must be proportional to the gradient of $h(x)$, i.e. $\nabla f(x) = \hat{\mathbf{v}}_{zx} + \hat{\mathbf{v}}_{yx}$ is normal to the line described by $h(x)$. Geometrically, the only way this can occur is if the normalized vectors flank $\nabla h(x)$ at equal angles.
\end{solution}

\question Consider the space $V = C^0([0, 1], \mathbb{R})$, let $\varphi : \mathbb{R} \to \mathbb{R}$ be a $C^1$ function, and define the functional $J$ on $V$ by $J(y) = \int_0^1\phi(y(x))\,\mathrm{d}x$. Show that its first variation exists and is given by the formula $\delta J|_y(\eta) = \int_0^1\varphi'(y(x))\eta(x)\,\mathrm{d}x$.

\begin{solution}
	Let us start with (1.33),
	\[
		\delta J|_y (\eta) = \lim_{\alpha \to 0}\frac{\int_0^1\varphi(y(x) + \alpha\eta(x))\,\mathrm{d}x - \int_0^1\varphi(y(x))\,\mathrm{d}x}{\alpha}.
	\]
	As $\alpha$ approaches $0$, we may replace $\varphi(y(x) + \alpha\eta(x))$ with its first order linear approximation about $y(x)$ (which is possible because $\varphi$ is once differentiable). In particular, we have
	\begin{align*}
		\delta J|_y (\eta) &= \lim_{\alpha \to 0}\frac{\int_0^1\varphi(y(x)) + \alpha\varphi'(y(x))\eta(x)\,\mathrm{d}x - \int_0^1\varphi(y(x))\,\mathrm{d}x}{\alpha} \\
		&= \lim_{\alpha \to 0}\frac{\alpha\int_0^1\varphi'(y(x))\eta(x)\,\mathrm{d}x}{\alpha} \\
		&= \int_0^1\varphi'(y(x))\eta(x)\,\mathrm{d}x
	\end{align*}
\end{solution}

\end{questions}

\end{document}
