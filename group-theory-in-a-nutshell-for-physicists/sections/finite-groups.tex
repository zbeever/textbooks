\documentclass[../group-theory-in-a-nutshell-for-physicists.tex]{subfiles}

\begin{document}
\printanswers

\section{Finite Groups}

\begin{questions}

\question Show that for $2$-cycles $(1a)(1b)(1a) = (ab)$.

\begin{solution}
We can show this by brute force by seeing where $(1a)(1b)(1a)$ takes
$a$ and $b$. Using arrows to represent the alterations made by
successive cycles (where we multiply right to left), we see

\[
a \rightarrow 1 \rightarrow b \rightarrow b \quad \text{and} \quad
b \rightarrow b \rightarrow 1 \rightarrow a.
\]

That is, $(1a)(1b)(1a)$ exchanges $a$ and $b$, and so is equivalent to
$(ab)$.
\end{solution}

\question Show that $A_{n}$ for $n \geq 3$ is generated by $3$-cycles, that
is, any element can be written as the product of $3$-cycles.

\begin{solution}
We know that all permutations may be broken into a product of
$2$-cycles. In the case of $A_{n}$, these products always consist of
an even number of such cycles (being as they form the group of even
permutations).

If there are two $2$-cycles that share both numbers, i.e. there exists
a term like $(ab)(ba)$, they can be removed from the product (being as
they form the identity permutation). For $2$-cycles sharing one
element, such as $(ab)(cb)$, we can simply combine these into
$(abc)$. Lastly, $2$-cycles sharing no elements may be rewritten by
use of the identity,

\[
(ab)(cd) = (ab)(bc)(cb)(cd) = (abc)(cdb).
\]

All pairs in our $2$-cycle representation of $A_{n}$ have thus been
converted into products of $3$-cycles. $A_{1}$ is the identity and
$A_{2}$ is not a group, being as there are no even permutations that
make only one exchange by definition. Thus our above result holds only
for $n \geq 3$.
\end{solution}

\question Show that $S_{n}$ is isomorphic to a subgroup of $A_{n + 2}$. Write
down explicitly how $S_{3}$ is a subgroup of $A_{5}$.

\begin{solution}
In order to show that $S_{n}$ is isomorphic to a subgroup of
$A_{n + 2}$, we must produce a bijective homomorphism from one to the
other.

It is clear that all even permutations within $S_{n}$ can be mapped to
themselves, as $A_{n + 2}$ contains these elements. For odd
permutations, consider the mapping
$\sigma \rightarrow \phi(\sigma) = \sigma(n + 1, n + 2)$. When applied to an even permutation $\tau$, we define $\phi(\tau) = \tau$.

To check if this is a homomorphism, look at $\phi(\sigma\tau)$, where
$\sigma,\tau \in S_{n}$. If $\sigma$ and $\tau$ are both even,
their product is even and we have
$\phi(\sigma\tau) = \sigma\tau = \phi(\sigma)\phi(\tau)$. If they are
both odd, their product is also even. In that case, we have
\[
\phi(\sigma\tau) = \sigma\tau = \sigma(n + 1,n + 2)\tau(n + 1,n + 2) = \phi(\sigma)\phi(\tau).
\]
In the case where one permutation is odd and the other even, their
product is also odd. Their mapping becomes
\[
\phi(\sigma\tau) = \sigma\tau(n + 1,n + 2) = \phi(\sigma)\phi(\tau).
\]

Being as a homomorphism preserves the group structure, our image
(contained within $A_{n + 2}$) is a group.

Using the above, we see $S_{3}$ can be mapped to

\begin{align*}
I & \rightarrow I \\
(12) & \rightarrow (12)(45) \\
(13) & \rightarrow (13)(45) \\
(23) & \rightarrow (23)(45) \\
(123) & \rightarrow (123) \\
(132) & \rightarrow (132)
\end{align*}

where we have denoted the identity permutation by $I$.
\end{solution}

\question List the partitions of $5$. (We will need this later.)

\begin{solution}
The seven partitions of $5$ are given by
\begin{gather*}
1 + 1 + 1 + 1 + 1 \\
1 + 1 + 1 + 2 \\
1 + 2 + 2 \\
1 + 1 + 3 \\
2 + 3 \\
1 + 4 \\
5
\end{gather*}
\end{solution}

\question Count the number of elements with a given cycle structure.

\begin{solution}
We'll use a specific cycle structure to keep track of everything.
The methods used are easily generalized to any cycle structure.

Consider the structure given on page 59,
\[
(xxxxx)(xxxxx)(xxxx)(xx)(xx)(xx)(x)(x)(x)(x),
\]
or $n_{5} = 2$, $n_{4} = 1$, $n_{3} = 0$, $n_{2} = 3$, and
$n_{1} = 4$ (with $n = 24$). How many cycles can represented using a
similar structure?

There are $24 \cdot 23 \cdot 22 \cdot 21 \cdot 20$ ways to populate
the first cycle, but this is over counting by a factor of $5$ (being
as cycles such as $(12345)$ and $(23451)$ are really the same). The
possibilities for the second cycle are found similarly: there are
$(19 \cdot 18 \cdot 17 \cdot 16 \cdot 15)/5$ ways to fill it.

If we continued on and began counting the possible $(xxxx)$ cycles, we'd be missing another source of error: the first two cycles can be exchanged without altering our permutation. Indeed, if we continue without changing
anything we'll be over counting each group of $j$ elements by a factor
of $n_{j}!$ Fixing this, we see there are
\[
\frac{24 \cdot 23\cdots 15}{5^{2} \cdot 2!}
\]
different ways of choosing the first two cycles. We can continue this to
find the expression for the total number of elements with a given cycle
structure is
\[
\frac{n!}{\prod_{j}j^{n_{j}} \cdot n_{j}!}.
\]
\end{solution}

\question List the possible cycle structures in $S_{5}$ and count the number of elements with each structure.

\begin{solution}
Our answer to question $4$ comes in handy here, as the partitions of
$n$ are related to the cycle structures of $S_{n}$. There are seven
possible cycle structures,
\begin{gather*}
(x)(x)(x)(x)(x) \\
(x)(x)(x)(xx) \\
(x)(xx)(xx) \\
(x)(x)(xxx) \\
(xx)(xxx) \\
(x)(xxxx) \\
(xxxxx)
\end{gather*}
Using the above formula, we see that these have, respectively, \(1\),
\(60\), \(15\), \(40\), \(20\), \(30\), and \(24\) associated
permutations.
\end{solution}

\question Show that $\mathcal{Q}$ forms a group.

\begin{solution}
Clearly, by Hamilton's multiplication rules our set is closed. It also
has the identity, \(1\). Each element's inverse is given by
\begin{align*}
1^{- 1} & = 1 \\
{- 1}^{- 1} & = {- 1}^{} \\
i^{- 1} & = {- i}^{} \\
{- i}^{- 1} & = i \\
j^{- 1} & = {- j}^{} \\
{- j}^{- 1} & = j \\
k^{- 1} & = {- k}^{} \\
{- k}^{- 1} & = k \\
\end{align*}
Being as $\mathcal{Q}$ obeys these three properties, it forms a group.
\end{solution}

\question Show that $A_{4}$ is not simple.

\begin{solution}
To show that $A_{4}$ is not simple, we need to find an invariant (or
normal) subgroup. The subgroup $Z_{2} \otimes Z_{2}$ is given as an
example of one in the text, so we need only verify this. Explicitly,
this subgroup takes the form

\[
\{ I,(12)(34),(13)(24),(14)(23)\} = Z_{2} \otimes Z_{2} \subset A_{4},
\]

i.e. it is the identity paired with all combinations of disjoint
$2$-cycles. The remaining elements in $A_{4}$ take the (disjoint)
form of individual $3$-cycles. Obviously, $g^{- 1}Ig = I$ for all
$g \in A_{4}$, so let's concentrate on the nontrivial elements.

Label $Z_{2} \otimes Z_{2}$'s $2$-cycles by $\sigma_{i}$ and
consider a $3$-cycle in $A_{4}$ (denoted by $\tau$). We must show
that

\[
	\tau^{- 1}\sigma_{i}\sigma_{j}\tau \subset Z_{2} \otimes Z_{2}.
\]

Inserting the identity between $\sigma$'s leaves us with
$\tau^{- 1}\sigma_{i}\sigma_{j}\tau = \tau^{- 1}\sigma_{i}\tau\tau^{- 1}\sigma_{j}\tau$.
Consider just $\tau^{- 1}\sigma_{i}\tau$. If $\sigma_{i}:j \rightarrow k$, then
\[
\tau^{- 1}\sigma_{i}\tau:\tau^{- 1}(j) \rightarrow \tau^{- 1}(k).
\]
This is because
\[
\tau^{- 1}\sigma_{i}\tau(\tau^{- 1}(j)) = \tau^{- 1}\sigma_{i}(j) = \tau^{- 1}(k).
\]

Because $\tau$ is injective, $\tau^{- 1}$ defines a unique mapping.
The result is that $\tau^{- 1}\sigma_{i}\tau$ is disjoint from
$\tau^{- 1}\sigma_{j}\tau$ when $i \neq j$. But the product of two
disjoint $2$-cycles is a defining feature of $Z_{2} \otimes Z_{2}$,
and so $Z_{2} \otimes Z_{2}$ is normal. This, of course, implies (by
definition) that $A_{4}$ is not simple.
\end{solution}

\question Show that $A_{4}$ is an invariant subgroup (in fact, maximal) of
$S_{4}$.

\begin{solution}
The same argument can be made as above for general cycle structures. In
particular, we can think of transformations like $g^{- 1}hg$ for
$g \in S_{4}$ and $h \in A_{4}$ as changes of basis (or relabeling
procedures). That is, $g$ renames element $i$ to $j$, which is
acted upon by $h$, which is then taken back to its original set. The
elements of $A_{4}$ remain even permutations no matter what is fed to
them.
\end{solution}

\question Show that the kernel of a homomorphic map of a group $G$ into itself
is an invariant subgroup of $G$.

\begin{solution}
There are two parts to this. Given a set $\{ g \in G|\phi(g) = e\}$,
we must first show that the elements $g$ form a group. Secondly, we
must show that the given group is normal.

The set given contains the identity. Consider $\phi(e) = h$ for some
$h \in G$. Then
\[
e = \phi(e)\phi(e)^{- 1} = \phi(e)\phi(e^{- 1}) = \phi(e)\phi(e) = \phi(e \cdot e) = \phi(e) = h,
\]
where $\phi(g)^{- 1} = \phi(g^{- 1})$ can be seen from the fact that
$\phi(gg^{- 1}) = \phi(g)\phi(g^{- 1}) = e$. Furthermore, it is closed
under multiplication: given $g,h$ in our subset,
\[
\phi(gh) = \phi(g)\phi(h) = e \cdot e = e.
\]
Taken together, we see our set forms a group, and so is a subgroup of
$G$. Now consider performing a similarity transformation on it by an
element $h \in G$ that's \emph{not} in our subgroup. Is this still in
our subgroup? We have
\[
\phi(h^{- 1}gh) = \phi(h^{- 1})\phi(g)\phi(h) = \phi(h)^{- 1} \cdot e \cdot \phi(h) = \phi(h)^{- 1}\phi(h) = e.
\]
\end{solution}

\question Calculate the derived subgroup of the dihedral group.

\begin{solution}
As detailed in the text,
\[
D_{n} = \{ I,R,R^{2},\cdots,R^{n - 1},r,Rr,R^{2}r,\cdots,R^{n - 1}r\}.
\]

The derived subgroup of this is given by all elements of the form
$\langle a,b\rangle = a^{- 1}b^{- 1}ab$ where $a,b \in D_{n}$, as
well as products of these elements.

When $a,b$ are pure rotations, $\langle a,b\rangle = I$. This is
because
\[
\langle R^{i},R^{j}\rangle = R^{n - i}R^{n - j}R^{i}R^{j} = R^{2n - i - j + i + j} = R^{2n} = I.
\]

When $a$ is a rotation and $b$ is a rotation and reflection, we have
\[
\langle R^{i},R^{j}r\rangle = R^{n - i}rR^{n - j}R^{i}R^{j}r = R^{n - i}rR^{i}r = R^{2(n - i)},
\]
using the fact that $rRr = R^{- 1}$. Swapping the order of these, we
see
\[
\langle R^{j}r,R^{i}\rangle = rR^{n - j}R^{n - i}R^{j}rR^{i} = rR^{- i}rR^{i} = R^{2i}.
\]

Finally,
\[
\langle R^{i}r,R^{j}r\rangle = rR^{n - i}rR^{n - j}R^{i}rR^{j}r = R^{i}R^{n - j}R^{i}R^{n - j} = R^{2(n - j + i)},
\]
so all elements of the form $\langle a,b\rangle$ are rotations.
Clearly, products of these elements are also rotations. So our derived
subgroup is given by
\[
D = \{ I,R,R^{2},\cdots,R^{n - 1}\}.
\]
\end{solution}

\question Given two group elements $f$ and $g$, show that, while in general
$fg \neq gf$, $fg$ is equivalent to $gf$ (That is, they are in the
same equivalence class).

\begin{solution}
We have defined our equivalence classes as objects that can be related
by similarity transformations, i.e. $g^{\prime} \sim g$ if
$g^{\prime} = h^{- 1}gh$ for some $h \in G$. In the case of $fg$,
we see
\[
gf \sim g^{- 1}gfg = fg.
\]
\end{solution}

\question Prove that groups of even order contain at least one element (which is
not the identity) that squares to the identity.

\question Using Cayley's theorem, map $V$ to a subgroup of $S_{4}$. List the
permutation corresponding to each element of $V$. Do the same for
$Z_{4}$.

\question Map a finite group $G$ with $n$ elements into $S_{n}$ a la Cayley.
The map selects $n$ permutations, known as ``regular permutations,''
with various special properties, out of the $n!$ possible permutations
of $n$ objects.

(a) Show that no regular permutations besides the identity leaves an object
untouched.

(b) Show that each of the regular permutations takes object $1$ (say) to a
different object.

(c) Show that when a regular permutation is resolved into cycles, the cycles
all have the same length. Verify that these properties hold for what you
got in exercise $14$.

\question In a Coxeter group, show that if $n_{ij} = 2$, then $a_{i}$ and
$a_{j}$ commute.

\question Show that for an invariant subgroup $H$, the left coset $gH$ is
equal to the right coset $Hg$.

\begin{solution}
A normal subgroup $H$ obeys $g^{- 1}Hg = H$ for all $g \in G$
(where $H \subset G$). Multiplying this condition by $g$ from the
left shows
\[
Hg = gH.
\]
\end{solution}

\question In general, a group $H$ can be embedded as a subgroup into a larger
group $G$ in more than one way. For example, $A_{4}$ can be
naturally embedded into $S_{6}$ by following the route
$A_{4} \subset S_{4} \subset S_{5} \subset S_{6}$. Find another way of
embedding $A_{4}$ into $S_{6}$. Hint: Think geometry!

\question Show that the derived subgroup of $S_{n}$ is $A_{n}$. (In the text,
with the remark about even permutations we merely showed that it is a
subgroup of $S_{n}$.)

\question A set of real-valued functions $f_{i}$ of a real variable $x$ can
also define a group if we define multiplication as follows: given
$f_{i}$ and $f_{j}$, the product $f_{i} \cdot f_{j}$ is defined as
the function \(f_{i}(f_{j}(x))\). Show that the functions $I(x) = x$
and $A(x) = (1 - x)^{- 1}$ generate a three-element group.
Furthermore, including the function $C(x) = x^{- 1}$ generates a
six-element group.

\begin{solution}
	With multiplication defined this way, it is clear that $I(x)$ is the identity, as
	\[
		I(f(x)) = f(x) \quad \text{and} \quad f(I(x)) = f(x)
	\]
	for an arbitrary $f(x)$. So, given $I(x)$ and $A(x)$, only $A(x)$ acts meaningfully change a group element. Composing $A(x)$ with itself gives
	\[
		A(A(x)) = \frac{1}{1 - \frac{1}{1 - x}} = \frac{1 - x}{1 - x - 1} = \frac{x - 1}{x}.
	\]
	Denoting this by $B(x)$ and composing it with $A(x)$ gives
	\[
		B(A(x)) = \frac{\frac{1}{1 - x} - 1}{\frac{1}{1 - x}} = \frac{1 - (1 - x)}{1} = x = I(x)
	\]
	while composing $B(x)$ with itself gives
	\[
		B(B(x)) = \frac{\frac{x - 1}{x} - 1}{\frac{x - 1}{x}} = \frac{x - 1 - x}{x - 1} = \frac{1}{1 - x} = A(x).
	\]
	Finally, composing $A(x)$ with $B(x)$ results in
	\[
		A(B(x)) = \frac{1}{1 - \frac{x - 1}{x}} = \frac{x}{x - (x - 1)} = x = I(x)
	\]
	The complete multiplication table is
	\[
	\begin{tabular}{>{$}l<{$}|*{3}{>{$}l<{$}}}
		~    & I(x) & A(x) & B(x) \\
		\hline\vrule height 12pt width 0pt
		I(x) & I(x) & A(x) & B(x) \\
		A(x) & A(x) & B(x) & I(x)    \\
		B(x) & B(x) & I(x) & A(x) \\
	\end{tabular} 
	\]
	and it is consistent with the properties of a $3$ element group.
	
	Let us investigate the effect of the inclusion of $C(x)=x^{-1}$ in the original set of functions. Clearly, $C(x)$ is its own inverse. Other possible compositions are
	\begin{align*}
		A(C(x)) &= \frac{1}{1 - \frac{1}{x}} = \frac{x}{x - 1} = \frac{1}{B(x)} \equiv D(x) \\
		B(C(x)) &= \frac{\frac{1}{x} - 1}{\frac{1}{x}} = 1 - x = \frac{1}{A(x)} \equiv E(x) \\
		C(A(x)) &= \frac{1}{A(x)} = E(x) \\
		C(B(x)) &= \frac{1}{B(x)} = D(x)
	\end{align*}
	All that is left is to check that the inclusion of the newly defined functions $D(x)$ and $E(x)$ leave the set closed under composition. With $D(x)$ on the left, we have
	\begin{align*}
		D(A(x)) &= \frac{\frac{1}{1 - x}}{\frac{1}{1 - x} - 1} = \frac{1}{1 - (1 - x)} = \frac{1}{x} = C(x) \\
		D(B(x)) &= \frac{\frac{x - 1}{x}}{\frac{x - 1}{x} - 1} = \frac{x - 1}{x - 1 - x} = 1 - x = E(x) \\
		D(C(x)) &= \frac{\frac{1}{x}}{\frac{1}{x} - 1} = \frac{1}{1 - x} = A(x) \\
		D(D(x)) &= \frac{\frac{x}{x - 1}}{\frac{x}{x - 1} - 1} = \frac{x}{x - (x - 1)} = x = I(x) \\
		D(E(x)) &= \frac{1 - x}{1 - x - 1} = \frac{x - 1}{x} = B(x) 
	\end{align*}
	While a similar analysis of $E(x)$ reveals
	\begin{align*}
		E(A(x)) &= 1 - \frac{1}{1 - x} = \frac{x}{x - 1} = D(x) \\
		E(B(x)) &= 1 - \frac{x - 1}{x} = \frac{1}{x} = C(x) \\
		E(C(x)) &= 1 - \frac{1}{x} = \frac{x - 1}{x} = B(x) \\
		E(D(x)) &= 1 - \frac{x}{x - 1} = \frac{1}{1 - x} = A(x)\\
		E(E(x)) &= 1 - (1 - x) = x = I(x) \\
	\end{align*}
	With $D(x)$ and $E(x)$ as the inner function, we find
	\begin{align*}
		A(D(x)) &= \frac{1}{1 - \frac{x}{x - 1}} = \frac{x - 1}{x - 1 - x} = 1 - x = E(x) \\
		A(E(x)) &= \frac{1}{1 - (1 - x)} = \frac{1}{x} = C(x) \\
		B(D(x)) &= \frac{\frac{x}{x - 1} - 1}{\frac{x}{x - 1}} = \frac{x - (x - 1)}{x} = \frac{1}{x} = C(x) \\
		B(E(x)) &= \frac{(1 - x) - 1}{1 - x} = \frac{x}{x - 1} = D(x)\\
		C(D(x)) &= \frac{1}{\frac{x}{x - 1}} = B(x) \\
		C(E(x)) &= \frac{1}{1 - x} = A(x) \\
	\end{align*}
	The complete multiplication table of this new group is
	\begin{align*}
	\begin{tabular}{>{$}l<{$}|*{6}{>{$}l<{$}}}
	~    & I(x) & A(x) & B(x) & C(x) & D(x) & E(x) \\
	\hline\vrule height 12pt width 0pt
	I(x) & I(x) & A(x) & B(x) & C(x) & D(x) & E(x) \\
	A(x) & A(x) & B(x) & I(x) & D(x) & E(x) & C(x) \\
	B(x) & B(x) & I(x) & A(x) & E(x) & C(x) & D(x) \\
	C(x) & C(x) & E(x) & D(x) & I(x) & B(x) & A(x) \\
	D(x) & D(x) & C(x) & E(x) & A(x) & I(x) & B(x) \\
	E(x) & E(x) & D(x) & C(x) & B(x) & A(x) & I(x) \\
	\end{tabular} 
	\end{align*}
\end{solution}

\end{questions}

\end{document}
