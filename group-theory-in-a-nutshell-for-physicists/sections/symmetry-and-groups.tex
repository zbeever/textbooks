\documentclass[../group-theory-in-a-nutshell-for-physicists.tex]{subfiles}

\begin{document}
\printanswers

\section{Symmetry and Groups}

\begin{questions}

\question The center of a group $G$ (denoted by $Z$) is defined to be the set
of elements $\{ z_{1},z_{2},\cdots\}$ that commute with all elements
of $G$, that is $z_{i}g = gz_{i}$ for all $g$. Show that $Z$ is
an abelian subgroup of $G$.

\begin{solution}
We must show that $Z$ contains the identity, is closed under
multiplication, and contains the inverses of all its elements.

Because the left inverse and right inverse are the same, we have
$Ig = gI$, and so, by the definition of $Z$, $I \in Z$.

Assume that a product of two elements in $Z$ produces an element not
in $Z$. That is,

\[
	z_{i}z_{j} = g_{l} \notin Z.
\]

This would imply $g_{l}g_{k} \neq g_{k}g_{l}$, or
$z_{i}z_{j}g_k \neq g_kz_{i}z_{j}$. But, by definition of the elements of
$Z$, $z_{i}z_{j}g_{k} = z_{i}g_kz_{j} = g_kz_{i}z_{j}$, which
contradicts the previous result. We see $Z$ is closed.

Now, assume there exists an out-of-set inverse $z_{i}^{- 1}$ of an
element $z_{i}$. Then $z_{i}^{- 1}g_{j} \neq g_{j}z_{i}^{- 1}$.
Right multiplying by $z_{i}$ gives us
\[
z_{i}^{- 1}g_{j}z_{i} \neq g_{j}z_{i}^{- 1}z_{i} = g_{j}I = g_{j},
\]

or $z_{i}^{- 1}g_{j}z_{i} \neq g_{j}$. But, by virtue of the nature of
$Z$,
\[
z_{i}^{- 1}g_{j}z_{i} = z_{i}^{- 1}z_{i}g_{j} = Ig_{j} = g_{j},
\]
and so our above result becomes $g_{j} \neq g_{j}$. By contradiction, $Z$ must be
abelian subgroup of $G$.
\end{solution}

\question Let $f(g)$ be a function of the elements in a finite group $G$, and
consider the sum $\sum_{g \in G}f(g)$. Prove the identity
$\sum_{g \in G}f(g) = \sum_{g \in G}f(gg^{\prime}) = \sum_{g \in G}f(g^{\prime}g)$
for $g^{\prime}$ an arbitrary element of $G$. We will need this
identity again and again in chapters II.1 and II.2.

\begin{solution}
This identity holds if both $gg^{\prime}$ and $g^{\prime}g$ cycle
through all elements of $G$ as $g$ goes through all available
values. Imagine creating a multiplication table for our group. Cycling
through $g$, we see the products $gg^{\prime}$ and $g^{\prime}g$
move one-by-one through a row or column. As each row and column contains
every element of $G$, the products $gg^{\prime}$ and $g^{\prime}g$
touch upon every element in the group. That
\[
\sum_{g \in G}f(g) = \sum_{g \in G}f(gg^{\prime}) = \sum_{g \in G}f(g^{\prime}g)
\]
follows.
\end{solution}

\question Show that \(Z_{2} \otimes Z_{4} \neq Z_{8}\).

\begin{solution}
From the discussion preceding this problem set, we know that the
possibility of $Z_{2} \otimes Z_{4}$ being isomorphic to $Z_{8}$
would require $2$ and $4$ to be coprime (which they clearly aren't).
Geometrically, we can think of \(Z_{8}\) of lying on a circle, while
\(Z_{2} \otimes Z_{4}\) lies on a torus. These shapes are not
homeomorphic, and so it is not a surprise that our groups aren't
isomorphic.

To give an explicit example of their difference, consider the elements
that square to the identity. In $Z_{8}$, we have $( - 1)^{2} = 1$
and $(1)^{2} = 1$, whereas $Z_{2} \otimes Z_{4}$ gives us
\[
(1,1)^{2} = (1,{- 1}^{})^{2} = ({- 1}^{},1)^{2} = ({- 1}^{},{- 1}^{})^{2}.
\]
\end{solution}

\question Find all groups of order $6$.

\begin{solution}
By Lagrange's theorem, all subgroups of a group of order \(6\) must be
$Z_{2}$, $Z_{3}$, or $Z_{6}$. Clearly, if we take $Z_{6}$ to be
a subgroup, then that subgroup \emph{is} the group. What if we take
$Z_{2}$ to be a subgroup? Forming the direct product of this with
$Z_{3}$ is the only possible construction that gives us $6$
elements: $Z_{3} \otimes Z_{3}$ is of order $9$ and
$Z_{2} \otimes Z_{2} \otimes Z_{2}$ is of order $8$.

To restate our findings more concisely: there are two groups of order
$6$, $Z_{6}$ and $Z_{2} \otimes Z_{3}$.
\end{solution}

\end{questions}

\end{document}
