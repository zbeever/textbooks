\documentclass[../group-theory-in-a-nutshell-for-physicists.tex]{subfiles}

\begin{document}
\printanswers

\section{Schur's Lemma and the Great Orthogonality Theorem}

\begin{questions}

\question Show that in the $3$-dimensional vector space, the three vectors $\begin{pmatrix}1\\1\\1\end{pmatrix}$, $\begin{pmatrix}1\\ \omega \\ \omega^*\end{pmatrix}$, $\begin{pmatrix}1 \\ \omega^* \\ \omega\end{pmatrix}$ (where $\omega = e^{j2\pi/3}$) are orthogonal to one another. Furthermore, a vector $\begin{pmatrix}u \\ v \\ w\end{pmatrix}$ orthogonal to all three must vanish. Prove that in $d$-dimensional complex vector space there can be at most $d$ mutually orthogonal vectors.

\question Determine the multiplication table of the class algebra for $D_5 = C_{5v}$.

\question Show that $f(c, d, I) = \delta_{\bar{c}d}/n_c$.

\question Show that $f(c, d, e) = f(\bar{c}, \bar{d}, \bar{e})$.

\question Prove (30).

\question Use Schur's lemma to prove the almost self-evident fact that all irreducible representations of an abelian group are $1$-dimensional.

\end{questions}

\end{document}
